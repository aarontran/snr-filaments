\documentclass[manuscript]{aastex}  % one-column, double-spaced
% \documentclass[12pt,preprint]{aastex}  % one-column, single-spaced
% \documentclass[preprint2]{aastex}  % double-column, single-spaced
% \documentclass[preprint2,longabstract]{aastex}  % if abstract doesn't fit

% \documentclass[apjl]{emulateapj}
% \documentclass[apj]{emulateapj}

\shorttitle{Synchrotron Rims in Tycho's SNR}  % <~ 44 char
\shortauthors{XXX et al.}  % Max three
\slugcomment{Not to appear in Nonlearned J., 45.}  % short title pg comment

%% ==================================================================== %%
%% README for track changes                                             %%
%% To add/remove text or add comments, use the following commands:      %%
%%                                                                      %%
%%       \note[editor]{The note}                                        %%
%%     \annote[editor]{Text to annotate}{The note}                      %%
%%        \add[editor]{Text to add}                                     %%
%%     \remove[editor]{Text to remove}                                  %%
%%     \change[editor]{Text to remove}{Text to add}                     %%
%%                                                                      %%
%% Using trackchanges.sty (http://trackchanges.sourceforge.net/)        %%
%%                                                                      %%
%% If you prefer to just edit directly, that works as well.             %%
%% ==================================================================== %%

\usepackage[inline]{trackchanges}

% Up to 5 editors
\addeditor{Rob}
\addeditor{Sean}
\addeditor{Steve}
\addeditor{Aaron}
\addeditor{Brian}

%% ============
%% Bibliography
%% ============
%\biboptions{semicolon}  % natbib options

%% ===============
%% Custom packages
%% ===============

% aas.org/authors/author-names-non-roman-alphabets
% \usepackage{CJK}

% Manuscript formatting
%\usepackage{lineno}

% Math, symbols, links, inline typesetting
%\usepackage{amsthm, amsmath, amssymb}
%\usepackage{amsmath}
%\usepackage{enumerate}  % For use of (a), (b), etc.
%\usepackage{hyperref}

% Figures, tables, floats, page layout (see also deluxetable)
%\usepackage[pdftex]{graphicx}
\usepackage{booktabs}
%\usepackage[labelfont=bf, labelsep=period]{caption}  % Custom float captions
%\usepackage{pdflscape}  % rotate pages (Texlive)

%% ==============
%% Extra commands
%% ==============
\newcommand{\mt}{\mathrm}
\newcommand{\unit}[1]{\;\mt{#1}}  % http://vemod.net/typesetting-units-in-latex
\newcommand{\abt}{\mathord{\sim}} % http://tex.stackexchange.com/q/55701

% Sets, operators
\newcommand{\ints}{\mathbb{Z}}
\newcommand{\ptl}{\partial}
\newcommand{\del}{\nabla}

%% =====================
%% AASTeX utilities etc.
%% =====================

% Object linking: https://aas.org/object-linking/introduction-object-linking
% \objectname[SNR G120.2+01.4]{Tycho's SNR}
% \objectname[M1]{Crab nebula}
% \object{NGC 6397}

% Dataset linking: https://aas.org/data-sets/data-set-linking
% \dataset{ADS/Sa.ASCA\#X/86008020}
% \dataset[ADS/Sa.HST#Y0Q70101T]{HST FOS spectrum}

% Possibly useful items:
% \footnote{footnote}
% \notetoeditor{note}  % Appears in manuscript footnotes, not in preprints
% \begin{mathletters}  % Add lettering to -eqnarrays- align, eqns
% \end{mathletters}  % e.g., number (4a), (4b) vs. (4), (5)
% \url{http://google.com}

% AASTeX symbol macros
% A 90\degr~angle, an ion: \ion{Ca}{2}, \slantfrac{40}{59} (why slantfrac?).

% Figures, tables
% AASTeX provides \plotone{z.eps}, \plottwo{a.eps}{b.eps}, \plotfiddle (?!)

% Tables -- refer to author guide / sample.tex when needed.


\begin{document}

% ========================
% Title, authors, abstract
% ========================
\title{Energy Dependent X-Ray Rim Width in Tycho's Supernova Remnant}

%\begin{CJK*}{UTF8}{gbsn}
\author{
Robert Petre\altaffilmark{1},
Sean M. Ressler\altaffilmark{2},
Stephen P. Reynolds\altaffilmark{3},
Aaron Tran\altaffilmark{1,4},
Brian J. Williams\altaffilmark{1,5}
}
%\end{CJK*}

\altaffiltext{1}{NASA Goddard Space Flight Center, Greenbelt, MD 20771, USA}
\altaffiltext{2}{Dept. Physics, University of California,
    Berkeley, CA 94720, USA}
\altaffiltext{3}{Dept. Physics, North Carolina State University,
    Raleigh, NC 27695, USA}
\altaffiltext{4}{CRESST, University of Maryland, College Park, MD 20742}
\altaffiltext{5}{NASA Postdoctoral Program Fellow}

\begin{abstract}
\note[Aaron]{Copy pasted from NASA abstract, not reviewed}
Young supernova remnants may exhibit thin (~1--10\% of shock radius) X-ray rims
of synchrotron radiation from forward shock-accelerated electrons that travel
downstream of the shock and quickly cease to radiate. Rim widths limited by
radiative energy losses should decrease with energy and require magnetic field
amplification $10$--$100\times$ that expected from adiabatic shock compression.
Damped magnetic fields behind rims may produce thin rims without strong field
amplification but require energy-independent rim widths. We measured rim widths
around Tycho's supernova remnant in 5 energy bands using a 750 ks Chandra
observation. Rims narrow with increasing energy, favoring loss-limited
radiation over magnetic damping. Observed widths are best fit by electron
transport models requiring amplified magnetic fields $\abt0.1$--$1$ mG and
particle diffusion $\abt1$--$10\times$ Bohm values, consistent with prior work.
Inferred magnetic fields, diffusion coefficients, and diffusion-energy scaling
may constrain models for cosmic ray acceleration in supernova remnants and
plasma turbulence in astrophysical shocks.
\note[Aaron]{Keywords from Sean's paper}
\end{abstract}

\keywords{acceleration of particles ---
    ISM: individual objects (Tycho's SNR) ---
    ISM: magnetic fields ---
    ISM: supernova remnants ---
    X-rays: ISM}

% ============
% Introduction
% ============
\note[Aaron]{a lot of preliminary text copy-pasted from poster or my own
notes, so explanations may be a bit simplistic}

\section{Introduction}

\note[Aaron]{most text below is copy-pasted from my own notes -- explanation
may be too simplistic}

Accelerating ISM/CSM particles in the forward shocks of supernovae emit
synchrotron radiation.  Emission is strong in the shock's immediate wake, but
dies off quickly downstream as particles radiate and lose energy. The radiation
from a spherical blast wave, then, is shell-like with bright X-ray and radio
rims/filaments due to line-of-sight projection.  These filaments have finite
and energy-dependent widths, set by:
\begin{itemize}
  \item \emph{synchrotron losses}.  At the shock, high energy electrons
  efficiently radiate harder x-ray photons.  As electrons are advected
  downstream while radiating, they radiate at lower energies and lose energy
  more slowly.  Synchrotron losses depend on (1) the initial electron energy
  distribution, and (2) the gradual decrease of electron energies downstream of
  the shock.
  \item \emph{diffusion}.  Random motion with respect to bulk plasma advection.
  Higher energy electrons diffuse and travel further downstream than would be
  expected from pure advection; hence, higher energy radiation may be seen
  farther downstream of the shock than expected.
  Higher energy particles may also diffuse \emph{ahead} of the shock, possibly
  giving rise to a so-called ``cosmic-ray shock precursor''.  Note that this is
  distinct from radiative precursors \citep[e.g.,][]{ghavamian2000}.
\end{itemize}

\subsection{Abbreviated review of prior work for context}

Basic observables \citep[e.g.,][]{bamba2003, bamba2005-hist, bamba2005-vela,
parizot2006} and consequences.

Testable model predictions of magnetic damping \citep{pohl2005}.

\citet{ressler2014} found rim narrowing in the remnant of SN 1006,

Equivocal results by \citet{araya2010} for Cas A.

\subsection{Our work}

We make measurements to distinguish between these two end member scenarios.
\note[Aaron]{paper roadmap} First we review our measurements,
explain the relevant model physics and governing equations,
explain our fitting procedure.

Then we discuss the implications of our fits and models

% ============
% Observations
% ============
\section{Observations}



We used archival \textit{Chandra} observations of Tycho
(\dataset[ADS/Sa.CXO\#obs/10093--10097]{ObsIDs 10093--10097},
\dataset[ADS/Sa.CXO\#obs/10902]{10902--10906})
obtained with the ACIS-I CCD between 2009 Apr 11 and 2009 May 5.
The total exposure time is $734 \unit{ks}$.
The data are divided into five energy bands:
0.7--1 keV, 1--1.7 keV, 2--3 keV, 3--4.5 keV, and 4.5--7 keV.
We excise counts between 1.7--2 keV to avoid \ion{Si}{13} (He$\alpha$) line
emission prevalent in the remnant's thermal ejecta.

Thirteen regions were selected with the following criteria in mind:
\note[Aaron]{itemized lists are just for outlining}
\begin{itemize}
    \item Nonthermal filaments should be singular and localized (i.e., either
        no overlapping or completely overlapping rims -- this rules out parts
        of NE limb)
    \item Nonthermal filaments should have sufficient signal-to-noise ratio
        that we can identify a clear FWHM in all bands (rules out a lot of
        faint southern wisps)
    \item Avoid plumes of thermal emission: identify spatially from images,
        esp. in lower energy bands (0.7--1 keV, 1--2 keV)
        (e.g.,~\ref{fig:snr}, eastern side), identify spectrally from emission
        lines.
\end{itemize}

\begin{figure}
    \plotone{figures/f0-snr.eps}
    \caption{RGB image of Tycho with region selections overlaid.  Bands are
    0.7--1 keV (red), 1--2 keV (green) and 2--7 keV (blue)}
    \label{fig:snr}
\end{figure}

Discuss point-spread function if that may be a point of contention.
Show that it is not a problem for us (simply say that all rims are at
least ~1 arcsec and hence are spatially resolved? Compare,
\citet{bamba2005-hist} measured upstream/downstream scale widths separately?)

\subsection{Filament width measurements}
\begin{figure*}
    \plotone{figures/f0-prfs.pdf}
    \caption{Some semblance of a caption.}
    \label{fig:profiles}
\end{figure*}

\subsection{Filament spectra}

And we shall purify the remnant's rims
that they may show us how they do thin
discriminating damped models.

\begin{figure}
    \plotone{figures/f0-spec.pdf}
    \caption{Another caption \citep{fermi1949}}
    \label{fig:spec}
\end{figure}

Finally, here are some numbers!

%\begin{table}
%\end{table}

% =======================
% Filament width modeling
% =======================
\section{Filament width models}

Cite \citet{ressler2014} vigorously here.  Just present the governing transport
equations (with variables), namely that from:
\[
  \frac{\ptl f}{\ptl t} + \del \cdot \left( f \vec{v} \right)
  = C + \del \cdot \left( D \del f \right)
\]
we obtain (1-D time-independent flow, neglect compressible flow?!,
space-independent diffusion coefficient):
\[
    v \frac{\ptl f}{\ptl x} - D \frac{\ptl^2 f}{\ptl x^2} = C
\]
and by random selection of the source/sink term $C$ we obtain either a
simplified model:
\begin{equation}
    v \frac{\ptl f}{\ptl x} - D \frac{\ptl^2 f}{\ptl x^2} +
    \frac{f}{\tau_{\mt{synch}}} = 0
\end{equation}
where this is equation (6) of \citet{ressler2014}; we also look at
\begin{equation}
    v \frac{\ptl f}{\ptl x} - D \frac{\ptl^2 f}{\ptl x^2} =
    K_0 E^{-s} e^{-E/E_{\mt{cut}}} \delta(x) + \frac{\ptl}{\ptl E}
      \left(bB^2E^2f\right)
\end{equation}
which is equation (12) of \citet{ressler2014}.


% ===================
% Results (best fits)
% ===================

\section{Results}

Let's just say if any fit runs to $\eta_2 = 10^5$ or $B_0 = 10$ mG (for Tycho),
we deem it effectively unconstrained.

\begin{table}
% Here I'm following the AASTeX sample...
%\tabletypesize{\scriptsize}
\caption{Full model best fit parameters, region 1.\label{tab:fit1}}
\begin{tabular}{@{}rllr@{}}
\toprule
{} & \multicolumn{3}{c}{Region 1} \\
\cmidrule(l){2-4}
$\mu$ (-) & $\eta_2$ (-) & $B_0$ ($\mu$G) & $\chi^2$\\
\midrule
0.00 & ${19.1}^{+19}_{-7.1}$ & ${784}^{+1.2 \times 10^{2}}_{-72}$ & 57.5590\\
0.33 & ${56}^{+75}_{-29}$ & $930^{+190}_{-140}$ & 34.6006\\
0.50 & ${92}^{+160}_{-54}$ & ${1010}^{+260}_{-180}$ & 25.5737\\
1.00 & ${370}^{+3300}_{-310}$ & ${1260}^{+4500}_{-440}$ & 9.7750\\
1.50 & ${23}^{+32}_{-10}$ & ${623}^{+130}_{-67}$ & 8.7937\\
2.00 & ${11.4}^{+5.2}_{-3.3}$ & ${512}^{+34}_{-26}$ & 9.5862\\
\bottomrule
\end{tabular} 
\end{table}

Table reference goes here~\ref{tab:fit1}.

% ==========
% Discussion
% ==========
\section{Discussion}

What do all these numbers with their errors mean?


% ==========
% Conclusion
% ==========
\section{Conclusions}

In conclusion, crazy B field amplification is not that weird.  Magnetic damping
can be ruled out and our result is robust throughout the remnant.

\acknowledgments

The scientific results reported in this article are based on data obtained from
the Chandra Data Archive.

{\it Facilities:} \facility{CXO (ACIS-I)}

\appendix

\section{Appendix material}

Appendix text if so desired.

\bibliographystyle{apj}  % AASTeX journal macros (for abbrevs) supplied by ADS
\bibliography{refs-snr}

\end{document}
