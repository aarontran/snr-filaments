\documentclass[manuscript]{aastex}  % one-column, double-spaced
% \documentclass[12pt,preprint]{aastex}  % one-column, single-spaced
% \documentclass[iop, apj, twocolappendix]{emulateapj}

\shorttitle{Synchrotron Rims in Tycho's SNR}  % <~ 44 char
\shortauthors{XXX et al.}  % Max three
\slugcomment{Draft, \today}  % short title pg comment

%% ==================================================================== %%
%% README for track changes                                             %%
%% To add/remove text or add comments, use the following commands:      %%
%%                                                                      %%
%%       \note[editor]{The note}                                        %%
%%     \annote[editor]{Text to annotate}{The note}                      %%
%%        \add[editor]{Text to add}                                     %%
%%     \remove[editor]{Text to remove}                                  %%
%%     \change[editor]{Text to remove}{Text to add}                     %%
%%                                                                      %%
%% ==================================================================== %%

\usepackage[inline]{trackchanges}  % trackchanges.sourceforge.net
\addeditor{Rob}
\addeditor{Sean}
\addeditor{Steve}
\addeditor{Aaron}
\addeditor{Brian}

\usepackage{amsmath}  % amsthm, amssymb
\usepackage{hyperref}
% \usepackage{CJK}  % aas.org/authors/author-names-non-roman-alphabets
\usepackage{booktabs}
%\usepackage[labelfont=bf, labelsep=period]{caption}  % Custom float captions
%\usepackage{pdflscape}  % rotate pages (Texlive)

\newcommand{\mt}{\mathrm}
\newcommand{\unit}[1]{\;\mt{#1}}  % http://vemod.net/typesetting-units-in-latex
\newcommand{\abt}{\mathord{\sim}} % http://tex.stackexchange.com/q/55701
\newcommand{\ptl}{\partial}
\newcommand{\del}{\nabla}

% This paper
\newcommand{\tsup}{\textsuperscript}
\newcommand{\Chandra}{\textit{Chandra}\ }
\defcitealias{ressler2014}{R14}

\begin{document}

\title{Energy Dependent X-Ray Rim Width in Tycho's Supernova Remnant}

%\begin{CJK*}{UTF8}{gbsn}
\author{
Robert Petre\altaffilmark{1},
Sean M. Ressler\altaffilmark{2},
Stephen P. Reynolds\altaffilmark{3},
Aaron Tran\altaffilmark{1,4},
Brian J. Williams\altaffilmark{1,5}
}
%\end{CJK*}

\affil{
\tsup{1}NASA Goddard Space Flight Center, Greenbelt, MD 20771, USA \\
\tsup{2}Dept. Physics, University of California, Berkeley, CA 94720, USA \\
\tsup{3}Dept. Physics, North Carolina State University, Raleigh, NC 27695, USA
}

%\altaffiltext{1}{NASA Goddard Space Flight Center, Greenbelt, MD 20771, USA}
%\altaffiltext{2}{Dept. Physics, University of California,
%    Berkeley, CA 94720, USA}
%\altaffiltext{3}{Dept. Physics, North Carolina State University,
%    Raleigh, NC 27695, USA}
\altaffiltext{4}{CRESST, University of Maryland, College Park, MD 20742}
\altaffiltext{5}{NASA Postdoctoral Program Fellow}

\begin{abstract}
\note[Aaron]{Copy pasted from NASA abstract, not reviewed}
Young supernova remnants may exhibit thin (~1--10\% of shock radius) X-ray rims
of synchrotron radiation from forward shock-accelerated electrons that travel
downstream of the shock and quickly cease to radiate. Rim widths limited by
radiative energy losses should decrease with energy and require magnetic field
amplification $10$--$100\times$ that expected from adiabatic shock compression.
Damped magnetic fields behind rims may produce thin rims without strong field
amplification but require energy-independent rim widths. We measured rim widths
around Tycho's supernova remnant in 5 energy bands using a 750 ks \Chandra
observation. Rims narrow with increasing energy, favoring loss-limited
radiation over magnetic damping. Observed widths are best fit by electron
transport models requiring amplified magnetic fields $\abt0.1$--$1$ mG and
particle diffusion $\abt1$--$10\times$ Bohm values, consistent with prior work.
Inferred magnetic fields, diffusion coefficients, and diffusion-energy scaling
may constrain models for cosmic ray acceleration in supernova remnants and
plasma turbulence in astrophysical shocks.
\end{abstract}

% Six keywords, alphabetical order
\keywords{acceleration of particles ---
    ISM: individual objects (Tycho's SNR) ---
    ISM: magnetic fields ---
    ISM: supernova remnants ---
    shock waves ---
    X-rays: ISM}

% ============
% Introduction
% ============
\section{Introduction}

% -------------------
% Background / review
% -------------------
\subsection{What are thin rims?}

Define thin rims (hereafter, thin rims).
\note[Aaron]{most text below is copy-pasted from my own notes -- explanation
may be too simplistic}

Somewhere here, define upstream/downstream of shock (makes sense in reference
frame of shock).

Accelerating ISM/CSM particles in the forward shocks of supernovae emit
synchrotron radiation.  Emission is strong in the shock's immediate wake, but
dies off quickly downstream as particles radiate and lose energy. The radiation
from a spherical blast wave, then, is shell-like with bright X-ray and radio
rims/filaments due to line-of-sight projection.  These filaments have finite
and energy-dependent widths, set by:
(1) \emph{synchrotron losses}.  At the shock, high energy electrons efficiently
radiate harder x-ray photons.  As electrons are advected downstream while
radiating, they radiate at lower energies and lose energy more slowly.
Synchrotron losses depend on the initial electron energy distribution and the
gradual decrease of electron energies downstream of the shock.
(2) \emph{diffusion}.  Random motion with respect to bulk plasma advection.
Higher energy electrons diffuse and travel further downstream than would be
expected from pure advection; hence, higher energy radiation may be seen
farther downstream of the shock than expected. Higher energy particles may also
diffuse \emph{ahead} of the shock, possibly giving rise to a so-called
``cosmic-ray shock precursor''.  Note that this is distinct from radiative
precursors \citep[e.g.,][]{ghavamian2000}.

Mention relevance to: cosmic ray acceleration, turbulent field amplification,
(note assumptions in re proton/electron spectra).  Cut-off energy of injected
electrons?  Broadly, all astrophysical shock processes: AGN jets, galaxy
clusters \citep{van-weeren2010}, PWNe, et cetera.  Usual background,
utility spiel.

% ------------------------
% Measurements and meaning
% ------------------------
\subsection{Prior measurements and inferred physical parameters}

Models for particle transport.  Bohm, sub-Bohm diffusion.
Testable model predictions of magnetic damping \citep{pohl2005}.

Previous measurements of basic observables \citep[e.g.,][]{bamba2003,
bamba2005-hist, bamba2005-vela, parizot2006} and consequences.
Conclusion that magnetic fields are amplified.

Equivocal results by \citet{araya2010} for Cas A.
\citet{ressler2014} found rim narrowing in the remnant of SN 1006.

% -----------
% Tycho's SNR
% -----------
\subsection{Tycho's SNR}
Why Tycho's supernova remnant (hereafter, Tycho)?

Like SN 1006, Tycho has well defined synchrotron rims associated with expansion
into a low-density ISM \note[Aaron]{justification: high density ISM would
prevent the forward shock from running ahead of the thermal ejecta?}.
\citet{williams2013} favor mean pre-shock ISM density $\abt 0.2 \unit{cm^{-3}}$
from \textit{Spitzer} observations of shocked ISM dust emission,
\note[Aaron]{is this proton or electron density?  Or are they equal pre-shock?
Text mentions $n_e = 1.2 n_p$ \emph{post}-shock.}
consistent with Tycho's X-ray expansion \citep{katsuda2010}.
Compare w/ SN 1006 pre-shock density $n_p \abt 0.15$--$0.3 \unit{cm^{-3}}$
\citep{raymond2007, heng2007, winkler2013}.

Like all historical SNe, Tycho is close enough that we
can spatially resolve these rims with \Chandra and say something useful.

Any other remarks? (comparison to core-collapse SNe? no compact object to mess
things up? or is that flatly trivial)

% -------------
% Paper roadmap
% -------------
\subsection{Our work (paper roadmap)}

We make measurements to distinguish between the two end member scenarios.
Our procedure, in both measurement and rim width modeling, follows that of
\citet{ressler2014} (hereafter, \citetalias{ressler2014}) with only slight
modifications.
We first select regions around Tycho's forward shock for analysis, measure
rim widths, and verify that rim spectra are free of thermal line emission.
Using two models for particle transport, we fit measured widths...

We discuss the implications of our fits and models for magnetic shock
amplification (lending credence to previous estimates, in disfavoring magnetic
field amplification), and discuss particle diffusion/acceleration at the shock.

Any further results (constraints on precursors, etc).

Consequences, implications, why is this worthwhile?

% =============================
% Transport models, observables
% =============================
\section{Transport models}

We first briefly review the models we consider for particle transport; a fuller
exposition is given by \citetalias{ressler2014}.

Just present the governing transport equations.  From:
\[
  \frac{\ptl f}{\ptl t} + \del \cdot \left( f \vec{v} \right)
  = C + \del \cdot \left( D \del f \right)
\]
we obtain (1-D time-independent flow, neglect compressible flow?!,
space-independent diffusion coefficient):
\[
    v \frac{\ptl f}{\ptl x} - D \frac{\ptl^2 f}{\ptl x^2} = C
\]
and by selection of the source/sink term $C$ we obtain either a
simplified model:
\begin{equation} \label{eq:simp-mod}
    v \frac{\ptl f}{\ptl x} - D \frac{\ptl^2 f}{\ptl x^2} +
    \frac{f}{\tau_{\mt{synch}}} = 0
\end{equation}
where this is equation (6) of \citetalias{ressler2014}; we also look at
\begin{equation} \label{eq:full-mod}
    v \frac{\ptl f}{\ptl x} - D \frac{\ptl^2 f}{\ptl x^2} =
    K_0 E^{-s} e^{-E/E_{\mt{cut}}} \delta(x) + \frac{\ptl}{\ptl E}
      \left(bB^2E^2f\right)
\end{equation}
which is equation (12) of \citetalias{ressler2014}.

\note[Aaron]{EXPLAIN what the equations mean.  How do we eventually back out a filament
shape, filament width?  Just a few sentences.}

Bibdesk -- mark up all the papers I've really gone through and understood.
Then, add the relevant citations and notes, to be sure I haven't missed or
misunderstood anything.

% ============
% Observations
% ============
\section{Observations}

% TODO go through and double check tenses.

We measured synchrotron rim widths from an archival \Chandra
ACIS-I observation of Tycho
(RA: 00\tsup{h}25\tsup{m}19\fs0, dec: +64\arcdeg08\arcmin10\farcs0; J2000)
between 2009 Apr 11 and 2009 May 5 (PI: John P. Hughes;
\dataset[ADS/Sa.CXO\#obs/10093--10097]{ObsIDs: 10093--10097},
\dataset[ADS/Sa.CXO\#obs/10902--10906]{10902--10906}).
The total exposure time is $734 \unit{ks}$.
Level 1 \Chandra data were reprocessed with CIAO 4.6 and CALDB 4.6.1.1 and kept
unbinned with ACIS spatial resolution $0.492\arcsec$.
Merged and corrected events were divided into five energy bands:
0.7--1 keV, 1--1.7 keV, 2--3 keV, 3--4.5 keV, and 4.5--7 keV;
we excluded 1.7--2 keV counts to avoid \ion{Si}{13} (He$\alpha$) line
emission, prevalent in the remnant's thermal ejecta, which could contaminate
our nonthermal profile measurements.

We selected 13 regions in 5 distinct filaments around Tycho's shock
(Figure~\ref{fig:snr}) based on the following criteria:
(1) filaments should be singular and localized; multiple filaments must not
overlap or completely overlap (rules out parts of NE limb);
(2) filaments should have clear FWHMs in all bands, compared to background
signal or downstream thermal emission (rules out some faint southern wisps);
(3) filaments should be clear of spatial plumes of thermal ejecta in \Chandra
images; this rules out, e.g., areas of strong nonthermal and thermal emission
on Tycho's eastern limb.

\begin{figure}
    \plotone{figures/f0-snr.eps}
    \caption{RGB image of Tycho with region selections overlaid.  Bands are
    0.7--1 keV (red), 1--2 keV (green) and 2--7 keV (blue).
    \note[Aaron]{Temporary figure; regions need re-numbering.}}
    \label{fig:snr}
\end{figure}

Discuss point-spread function if that may be a point of contention.
Show that it is not a problem for us (simply say that all rims are at
least ~1 arcsec and hence are spatially resolved? Compare,
\citet{bamba2005-hist} measured upstream/downstream scale widths separately?)

% --------------------------
% FWHM measurement procedure
% --------------------------
\subsection{Filament width measurements}

We obtained radial intensity profiles by integrating along the shock in each
region.  Intensity profiles peak sharply at/behind the shock, demarcating our
thin rims, \note[Aaron]{abt 2--3 arcsec; measure this or report $w_u$?}
and then fall off until thermal emission from ejecta picks up at Tycho's
``contact discontinuity'' \citep{warren2005}.

We fit rim profiles, obtained by integrating intensity along the shock in each
region, to a piecewise two-exponential model:
\begin{equation} \label{eq:prof}
    h(x) =
    \begin{cases}
        A_u \exp \left(\frac{x_0 - x}{w_u}\right) + C_u, &x \geq x_0 \\
        A_d \exp \left(\frac{x - x_0}{w_d}\right) + C_d, &x < x_0
    \end{cases}
\end{equation}
where $h(x)$ is profile height and $x$ is radial distance from remnant center.
The rim model, which we emphasize is strictly empirical, has 6 free parameters
$A_u, x_0, w_u, w_d, C_u, C_d$ with $A_d = A_u + (C_u - C_d)$ enforcing
continuity at $x=x_0$. Our model is similar to that of \citet{bamba2003,
bamba2005-hist} and differs slightly from that of \citetalias{ressler2014}.
To fit only the nonthermal rim in each intensity profile, we smoothed profiles
with a 21-point ($\abt 10\arcsec$) Hanning window and bounded the fit
domain at the first local data minimum downstream of rim peak; the fit domain
extends upstream to the region's outer edge.

\begin{figure}
    \plotone{figures/f0-prfs.pdf}
    \caption{Profile fits for Region 1}
    \label{fig:profiles}
\end{figure}

From the fitted profiles we extracted a full width at half maximum (FWHM) for
each region and each energy band.
\note[Aaron]{We did NOT subtract out $\min(C_u, C_d)$ before measuring FWHM
(fix this?).  May mention effect of using slightly different models for FWHMs}
We could not resolve a FWHM in regions 2, 6, 8, 9, and 11 \note[Aaron]{KEEP
UPDATED} at 0.7--1 keV (Table~\ref{tab:flmt2} \note[Aaron]{UPDATE depending on
how we present/tabulate FWHMs}); in these regions, the downstream FWHM bound
would extend outside the fit domain.  We were able to resolve FWHMs for all
regions at higher energy bands (1--7 keV).

To estimate FWHM uncertainties, we horizontally stretched each best-fit
profile by mapping radial coordinate $x$ to
$x'(x) = x (1 + \xi (x-x_0)/(50\arcsec-x))$ with $\xi$ an arbitrary stretching
parameter; this yields a new profile $h'(x) = h(x'(x))$.
We varied $\xi$ (and hence rim FWHM) to vary each profile fit $\chi^2$ by 2.7,
yielding confidence limits on our measured FWHMs

% ----------------
% Filament spectra
% ----------------
\subsection{Filament spectra}

As a ``back-verification'', we check that our rims' X-ray spectra
are clear of thermal line emission.

In each radial profile, we select an interval containing FWHM bounds from all
energy bands.  We 

Each rim is divided into an upstream

From our model fits and we divide the rim into an upstream and downstream region;
the 

And we shall purify the remnant's rims\\
that they may show us how they do thin\\
discriminating damped models.

\begin{figure}
    \plotone{figures/f0-spec.pdf}
    \caption{Spectra from Region 1 (I think?)}
    \label{fig:spec}
\end{figure}

This also gives a post-hoc / further filter on intermediate results --
we can quantify the thermal emission...

We further constrain thermal emission by checking that spectral fits
favor pure continuum emission over continuum with Gaussian components for
\ion{Si}{13} and \ion{S}{15} He$\alpha$ lines at $\abt 1.8$, $2.4$ keV.

HERE I should mention any upper bounds on spectral lines (Equivalent width or
what have you).  Since this is not a ``result'', being ancillary to our work.

% -------------
% Model fitting
% -------------
\subsection{Filament model fitting}

Explain why we are fixing $\mu$.
Explain why we also give results with $\eta_2$ fixed at $1$.

To perform fits, we tabulated a large grid for fixed values of $\mu$ and shock
velocity $v_s$.

If any fit runs to $\eta_2 = 10^5$ or $B_0 = 10 \unit{mG}$ (for Tycho),
we deem it effectively unconstrained.


A few remnant-specific parameters enter into the model calculations.  We take
electron spectral index $s = 2.3$ (from radio spectral index $\alpha = 0.65$,
\citet{kothes2006} \note[Aaron]{not using $0.58$ from Green's catalog?!}),
remnant distance $3 \unit{kpc}$ \citep[cf.][]{hayato2010}, and
shock radius $1.08 \times 10^{19} \unit{cm}$ from angular radius $240\arcsec$
\citep{green2009}.  Tycho's forward shock velocity varies azimuthally by up to
a factor of 2; we interpolate the data of \citet{williams2013} to estimate
individual shock velocities for each of our regions.

As the full continuous loss model must be numerically solved, its predicted
rim widths are subject to resolution error in the numerical integrals.  We
chose our integration resolutions such that the fractional error associated
with halving/doubling our resolution is less than $1\%$ for the
parameter space relevant to our filaments. \note[Aaron]{rewrite -- unclear.
Need to verify $1\%$ claim + check Pacholczyk table resolution...}

% =======================
% Results, FWHMs and fits
% =======================
\section{Results}

We present results here.  Explain a bit about how we divvy up / present the
results.  Also, for each region, give the relevant shock velocity value.  Or, a
range of values.
We do X to the FWHMs (geometric/arithmetic average) within each filament.
We also consider global FWHMs average.

% --------------------
% FWHM results, tables
% --------------------
\subsection{Measured FWHMs, energy dependence, etc}

\begin{table*}
\scriptsize
\centering
\caption{Filament 2 full width at half max measurements.
Exponent $m_E$ estimated point to point.\label{tab:flmt2}}
\begin{tabular}{@{}l ccccc r@{ $\pm$ }l r@{ $\pm$ }l r@{ $\pm$ }l r@{ $\pm$ }l @{}}

\toprule
{} & \multicolumn{5}{c}{FWHM (arcsec)} & \multicolumn{8}{c}{$\mE$ (-)} \\
\cmidrule(lr){2-6} \cmidrule(l){7-14}
Region & Band 1 & Band 2 & Band 3 & Band 4 & Band 5
       & \multicolumn{2}{c}{Bands 1--2} & \multicolumn{2}{c}{Bands 2--3}
       & \multicolumn{2}{c}{Bands 3--4} & \multicolumn{2}{r}{Bands 4--5} \\ [0.2em]
{} & (0.7--1 keV) & (1--1.7 keV) & (2--3 keV) & (3--4.5 keV) & (4.5--7 keV)
   & \multicolumn{2}{c}{(1 keV)} & \multicolumn{2}{c}{(2 keV)}
   & \multicolumn{2}{c}{(3 keV)} & \multicolumn{2}{r}{(4.5 keV)} \\
\midrule
1 & {} & ${8.80}^{+0.18}_{-0.15}$ & ${6.34}^{+0.26}_{-0.21}$ & ${7.40}^{+0.30}_{-0.23}$ & ${5.57}^{+0.47}_{-0.42}$
  & \multicolumn{2}{c}{} & $-0.47$ & $0.06$ & $0.38$ & $0.13$ & $-0.70$ & $0.22$ \\ [0.5em]
2 & {} & ${4.22}^{+0.12}_{-0.09}$ & ${2.36}^{+0.12}_{-0.09}$ & ${3.00}^{+0.16}_{-0.12}$ & ${4.11}^{+0.34}_{-0.30}$
  & \multicolumn{2}{c}{} & $-0.84$ & $0.08$ & $0.59$ & $0.16$ & $0.77$ & $0.23$ \\ [0.5em]
3 & {} & ${2.47}^{+0.08}_{-0.07}$ & ${1.78}^{+0.09}_{-0.07}$ & ${2.10}^{+0.11}_{-0.11}$ & ${1.32}^{+0.10}_{-0.09}$
  & \multicolumn{2}{c}{} & $-0.47$ & $0.08$ & $0.41$ & $0.17$ & $-1.15$ & $0.22$ \\

\cmidrule{1-14}
4 & ${5.85}^{+0.37}_{-0.33}$ & ${4.35}^{+0.09}_{-0.08}$ & ${3.26}^{+0.11}_{-0.09}$ & ${3.69}^{+0.12}_{-0.11}$ & ${3.20}^{+0.21}_{-0.18}$
  & $-0.83$ & $0.18$ & $-0.41$ & $0.05$ & $0.31$ & $0.11$ & $-0.35$ & $0.17$ \\ [0.5em]
5 & {} & ${4.52}^{+0.11}_{-0.12}$ & ${3.06}^{+0.11}_{-0.11}$ & ${3.25}^{+0.15}_{-0.13}$ & ${3.04}^{+0.21}_{-0.18}$
  & \multicolumn{2}{c}{} & $-0.56$ & $0.06$ & $0.15$ & $0.14$ & $-0.17$ & $0.19$ \\ [0.5em]
6 & ${2.48}^{+0.18}_{-0.18}$ & ${2.32}^{+0.05}_{-0.06}$ & ${2.98}^{+0.11}_{-0.09}$ & ${2.05}^{+0.08}_{-0.09}$ & ${2.21}^{+0.15}_{-0.14}$
  & $-0.19$ & $0.21$ & $0.36$ & $0.06$ & $-0.92$ & $0.13$ & $0.18$ & $0.19$ \\ [0.5em]
7 & ${2.69}^{+0.20}_{-0.17}$ & ${2.33}^{+0.05}_{-0.05}$ & ${2.31}^{+0.08}_{-0.08}$ & ${1.81}^{+0.09}_{-0.07}$ & ${1.83}^{+0.11}_{-0.08}$
  & $-0.39$ & $0.20$ & $-0.01$ & $0.06$ & $-0.60$ & $0.14$ & $0.02$ & $0.17$ \\ [0.5em]
8 & ${2.33}^{+0.21}_{-0.20}$ & ${2.72}^{+0.08}_{-0.08}$ & ${2.38}^{+0.10}_{-0.09}$ & ${2.10}^{+0.10}_{-0.09}$ & ${2.37}^{+0.20}_{-0.17}$
  & $0.43$ & $0.26$ & $-0.19$ & $0.07$ & $-0.30$ & $0.15$ & $0.29$ & $0.22$ \\ [0.5em]
9 & ${2.16}^{+0.24}_{-0.23}$ & ${2.35}^{+0.07}_{-0.06}$ & ${2.47}^{+0.11}_{-0.11}$ & ${1.91}^{+0.09}_{-0.09}$ & ${2.20}^{+0.17}_{-0.16}$
  & $0.24$ & $0.31$ & $0.07$ & $0.07$ & $-0.63$ & $0.16$ & $0.34$ & $0.22$ \\ [0.5em]
10 & ${2.38}^{+0.24}_{-0.23}$ & ${1.99}^{+0.07}_{-0.06}$ & ${1.76}^{+0.09}_{-0.08}$ & ${1.59}^{+0.09}_{-0.08}$ & ${1.58}^{+0.13}_{-0.12}$
  & $-0.50$ & $0.29$ & $-0.18$ & $0.08$ & $-0.24$ & $0.18$ & $-0.02$ & $0.23$ \\

\cmidrule{1-14}
11 & {} & ${3.23}^{+0.15}_{-0.13}$ & ${2.52}^{+0.16}_{-0.13}$ & ${1.90}^{+0.14}_{-0.13}$ & ${3.09}^{+0.45}_{-0.38}$
  & \multicolumn{2}{c}{} & $-0.36$ & $0.10$ & $-0.70$ & $0.22$ & $1.21$ & $0.37$ \\ [0.5em]
12 & {} & ${3.86}^{+0.17}_{-0.16}$ & ${2.61}^{+0.15}_{-0.13}$ & ${3.02}^{+0.22}_{-0.21}$ & ${2.23}^{+0.21}_{-0.17}$
  & \multicolumn{2}{c}{} & $-0.56$ & $0.10$ & $0.36$ & $0.22$ & $-0.74$ & $0.27$ \\ [0.5em]
13 & ${2.85}^{+0.22}_{-0.17}$ & ${2.43}^{+0.05}_{-0.05}$ & ${2.36}^{+0.08}_{-0.05}$ & ${1.95}^{+0.09}_{-0.10}$ & ${1.84}^{+0.11}_{-0.14}$
  & $-0.45$ & $0.20$ & $-0.04$ & $0.05$ & $-0.47$ & $0.13$ & $-0.15$ & $0.20$ \\

\cmidrule{1-14}
14 & ${2.86}^{+0.17}_{-0.16}$ & ${2.42}^{+0.06}_{-0.04}$ & ${2.23}^{+0.08}_{-0.07}$ & ${2.38}^{+0.10}_{-0.08}$ & ${2.19}^{+0.12}_{-0.10}$
  & $-0.47$ & $0.17$ & $-0.12$ & $0.06$ & $0.17$ & $0.12$ & $-0.20$ & $0.15$ \\ [0.5em]
15 & ${2.71}^{+0.17}_{-0.16}$ & ${1.99}^{+0.05}_{-0.04}$ & ${1.80}^{+0.06}_{-0.05}$ & ${1.87}^{+0.07}_{-0.05}$ & ${1.52}^{+0.09}_{-0.08}$
  & $-0.85$ & $0.18$ & $-0.15$ & $0.05$ & $0.09$ & $0.11$ & $-0.51$ & $0.16$ \\ [0.5em]
16 & ${1.87}^{+0.14}_{-0.13}$ & ${1.73}^{+0.04}_{-0.03}$ & ${1.52}^{+0.06}_{-0.05}$ & ${1.25}^{+0.06}_{-0.04}$ & ${1.23}^{+0.08}_{-0.06}$
  & $-0.22$ & $0.21$ & $-0.18$ & $0.06$ & $-0.49$ & $0.13$ & $-0.04$ & $0.17$ \\ [0.5em]
17 & ${1.65}^{+0.13}_{-0.12}$ & ${1.92}^{+0.05}_{-0.05}$ & ${1.54}^{+0.06}_{-0.07}$ & ${1.45}^{+0.07}_{-0.06}$ & ${2.05}^{+0.16}_{-0.14}$
  & $0.43$ & $0.22$ & $-0.31$ & $0.07$ & $-0.16$ & $0.15$ & $0.86$ & $0.21$ \\

\cmidrule{1-14}
18 & {} & ${4.45}^{+0.13}_{-0.12}$ & ${3.18}^{+0.17}_{-0.16}$ & ${2.96}^{+0.20}_{-0.19}$ & ${1.65}^{+0.21}_{-0.16}$
  & \multicolumn{2}{c}{} & $-0.49$ & $0.09$ & $-0.17$ & $0.21$ & $-1.45$ & $0.32$ \\ [0.5em]
19 & {} & ${2.30}^{+0.08}_{-0.06}$ & ${2.28}^{+0.11}_{-0.08}$ & ${2.16}^{+0.12}_{-0.11}$ & ${1.60}^{+0.17}_{-0.14}$
  & \multicolumn{2}{c}{} & $-0.02$ & $0.08$ & $-0.13$ & $0.17$ & $-0.74$ & $0.27$ \\ [0.5em]
20 & ${4.81}^{+0.31}_{-0.31}$ & ${1.84}^{+0.06}_{-0.03}$ & ${1.87}^{+0.08}_{-0.06}$ & ${1.56}^{+0.07}_{-0.06}$ & ${2.14}^{+0.23}_{-0.23}$
  & $-2.68$ & $0.19$ & $0.02$ & $0.07$ & $-0.44$ & $0.14$ & $0.77$ & $0.28$ \\

\midrule
Mean & $2.89 \pm 0.35$ & $3.11 \pm 0.37$ & $2.53 \pm 0.23$ & $2.47 \pm 0.30$ & $2.35 \pm 0.23$
  & $-0.46$ & $0.24$ & $-0.25$ & $0.06$ & $-0.14$ & $0.10$ & $-0.09$ & $0.15$ \\

\bottomrule
\end{tabular}
\tablecomments{Mean values computed for all regions; mean $\mE$ values are
averages for region $\mE$ values (i.e., not computed from mean FWHMs).  Errors
on mean values are standard errors of the mean.  Horizontal rules group
individual regions into filaments.}

\tablecomments{Mean FWHMs computed by arithmetic mean, errors from blah}
\end{table*}

Table~\ref{tab:flmt2} lists some numbers from Filament 2.  As previously noted
we could not resolve FWHMs for some regions in the 0.7--1 keV band.

Error estimates / magnitudes.

Mention the number of counts.  This might impact our decision to go back and
slice regions smaller (if we do that).

% -------------------------
% Model fit results, tables
% -------------------------
\subsection{Model}

Table reference goes here (Table~\ref{tab:fit1}).
Explain error values on parameters.

\begin{table}
\centering
\caption{Full model best fit parameters, region 1.\label{tab:fit1}}
\begin{tabular}{@{}rllr@{}}
\toprule
{} & \multicolumn{3}{c}{Region 1} \\
\cmidrule(l){2-4}
$\mu$ (-) & $\eta_2$ (-) & $B_0$ ($\mu$G) & $\chi^2$\\
\midrule
0.00 & ${19.1}^{+19}_{-7.1}$ & ${784}^{+1.2 \times 10^{2}}_{-72}$ & 57.5590\\
0.33 & ${56}^{+75}_{-29}$ & $930^{+190}_{-140}$ & 34.6006\\
0.50 & ${92}^{+160}_{-54}$ & ${1010}^{+260}_{-180}$ & 25.5737\\
1.00 & ${370}^{+3300}_{-310}$ & ${1260}^{+4500}_{-440}$ & 9.7750\\
1.50 & ${23}^{+32}_{-10}$ & ${623}^{+130}_{-67}$ & 8.7937\\
2.00 & ${11.4}^{+5.2}_{-3.3}$ & ${512}^{+34}_{-26}$ & 9.5862\\
\bottomrule
\end{tabular} 

\end{table}


% ==========
% Discussion
% ==========
\section{Discussion}

What do all these numbers with their errors mean?

Magnetic field amplification numbers -- how do they compare with previous
studies?  What is the azimuthal variation (and significance thereof)?

Diffusion -- sub-Bohm diffusion?

\subsection{Comparisons to Cas A (pulled from Sean's \LaTeX\ source)}

\note[Aaron]{for my own reference/reading/thinking (temporary)}

A detailed application of our results to other SNRs such as Cas A will
require much more extensive analysis, but we can use the published
filament widths of Araya et al. (2010) for Cas A to get preliminary
estimates of the magnetic field strength and diffusion coefficient by
applying our model. In their data, it appears that the filaments in
Cas A shrink by a factor of $\sim 0.8$ between 0.3 and 3 keV, while the
filament widths appear to be energy-independent between 3 and 6 keV.
Qualitatively, this is consistent with the loss-limited model, as our
parameter $m_{\rm E}$ is predicted to decrease with energy.  For the
lower energy range of 0.3--3 keV, reproducing $m_{\rm E}\sim -0.1$
(equivalent to the factor of 0.8 drop in size) requires magnetic
fields on the order of 200-500 $\mu$G and diffusion coefficients about
$5 \times D_{\rm Bohm}$(3 keV), about an order of magnitude higher
than the values one obtains by neglecting the energy dependence. One
can also see directly from Figure~\textbf{?!} that $\mu < 1$ models
of the diffusion coefficient are excluded for $m_{\rm E}\sim -0.1$.


To quickly compare how our results will differ from those of other authors, we
applied the analytic approximation of Equation 22 and Equation 9 to the FWHMs
presented by Araya et al. (2010) for Cas A. Here we used $m_{\rm E} = \log(
{\rm FWHM}(3keV)/{\rm FWHM}(.3keV))/\log(10)$ and fit the filaments at a photon
energy of 3 keV.  Hence $\eta_3 \equiv D/D_{\rm Bohm}(3 keV)$. Using the same
parameters, our results are shown in Table~\ref{tab:Araya}. To fit the energy
dependence of the FWHM (i.e. $m_{\rm E}$), we required higher diffusion
coefficients and, consequently, magnetic fields about an order of magnitude
higher than their results. We note here that values of $\mu<1$ were unable to
reproduce the data and that the diffusion coefficients cited by Araya et al.
(2010) are below the minimally allowed Bohm value ($\eta_3 = 1$).

We intended this comparison solely as a qualitative analysis, and thus did not
rigorously keep track of uncertainties or extend the comparison to the full
numerical calculation. The conclusion is clear, however, which is that
considering the measured value of $m_{\rm E}$ can have a dramatic effect.

\begin{table*}[h]
\caption{Best fit parameters for the filaments of Cas A based on data from
Araya et. al (2010) in varying values of $\mu$ (Approximate Analytic Results)
Dashes denote places where fits were unobtainable.}
\centering
\begin{tabular}{c c c c c c c c c}
\hline \hline
  &\multicolumn{2}{c}{$\mu = 1$} & \multicolumn{2}{c}{$\mu = 1.5$} & \multicolumn{2}{c}{$\mu = 2$} & \multicolumn{2}{c}{Araya et. al} \\
  & & & & & &  &\multicolumn{2}{c}{($\mu = 1$)} \\
$Filament$ &$\eta_{3}$ & $B_{0}$ &$\eta_{3}$ & $B_{0}$ & $\eta_{3}$ & $B_{0}$& $\eta_{3}$ & $B_{0}$ \\ [.5ex]
1 & 6.5  & 750 $\mu$G & 2 & 567 $\mu$G & 1.1 & 506 $\mu$G & 0.12 & 72 $\mu$G \\
2 & 3.6 & 710 $\mu$G & 1.4 & 582 $\mu$G & - & - & 0.02 & 37 $\mu$G\\
3 & 3.2 &  710 $\mu$G& 1.3 & 589 $\mu$G& - & - & 0.02 & 53 $\mu$G\\
4 & 3.1& 515 $\mu$G & 1.3 & 430 $\mu$G & -& - & 0.02 & 40 $\mu$G \\
5 & 10.7 & 1163 $\mu$G & 2.5 & 809 $\mu$G & 1.3&  706 $\mu$G & 0.025 & 52 $\mu$G \\
6 & 5.7 & 844 $\mu$G & 1.9 & 650 $\mu$G  & 1 & 583 $\mu$G & 0.1 & 56 $\mu$G \\
7 & 8.3 & 872 $\mu$G & 2.2 & 635 $\mu$G& 1.2 & 560 $\mu$G & 0.15 & 66 $\mu$G\\
8 & 13 & 1010 $\mu$G & 2.7 & 738 $\mu$G & 1.4 & 639 $\mu$G & 0.02 & 35 $\mu$G\\
9 & 4.7 & 605 $\mu$G & 1.7 & 479 $\mu$G & - & -& 0.02 & 29 $\mu$G\\

\hline
\hline
\end{tabular}
\label{tab:Araya}
\end{table*}

\subsection{End of Sean's Cas A analysis}

Lalala.

% ==========
% Conclusion
% ==========
\section{Conclusions}

In conclusion, crazy B field amplification is not that weird.
Magnetic damping can be ruled out and our result is robust throughout the remnant.

% ================
% Acknowledgements
% ================
\acknowledgments

The scientific results reported in this article are based on data obtained from
the \Chandra Data Archive.
This research has made use of NASA's Astrophysics Data System.

{\it Facilities:} \facility{CXO (ACIS-I)}

% ========
% Appendix
% ========
\appendix

For comparison to \citetalias{ressler2014}, we present our full model results
employing 2 energy bands (as in \citetalias{ressler2014}) and all 3 energy
bands (0.7--1, 1--2, 2--7 keV).  The simple model results are identical,
although the errors obtained from my procedure are larger.

\section{SN 1006 results, 2 energy bands}  % Use numberedappendix option for emulateapj

Tables tables tables tables

\section{SN 1006 results, 3 energy bands}

Tables tables tables tables

\section{Discussion}

Lalalalala.  Beautiful numbers!

% ==========
% References
% ==========
\bibliographystyle{apj}  % AASTeX journal macros are supplied in ADS entries
\bibliography{refs-snr}

\end{document}
