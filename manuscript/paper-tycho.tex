%\documentclass[manuscript]{aastex}  % one-column, double-spaced GENERATE BIB
%\documentclass[12pt,preprint]{aastex}  % one-column, single-spaced
\documentclass[iop, apj, numberedappendix, twocolappendix]{emulateapj}

\shorttitle{Synchrotron Rims in Tycho's SNR}  % <~ 44 char
\shortauthors{XXX et al.}  % Max three
\slugcomment{Draft, \today}  % short title pg comment

%% ==================================================================== %%
%% README for track changes                                             %%
%% To add/remove text or add comments, use the following commands:      %%
%%                                                                      %%
%%       \note[editor]{The note}                                        %%
%%     \annote[editor]{Text to annotate}{The note}                      %%
%%        \add[editor]{Text to add}                                     %%
%%     \remove[editor]{Text to remove}                                  %%
%%     \change[editor]{Text to remove}{Text to add}                     %%
%%                                                                      %%
%% ==================================================================== %%

\usepackage[inline]{trackchanges}  % trackchanges.sourceforge.net
\addeditor{Rob}
\addeditor{Sean}
\addeditor{Steve}
\addeditor{Aaron}
\addeditor{Brian}

\usepackage{amsmath}  % amsthm, amssymb
\usepackage{CJK}  % aas.org/authors/author-names-non-roman-alphabets
\usepackage{booktabs}
\usepackage{hyperref}

\newcommand*{\mt}{\mathrm}
\newcommand*{\unit}[1]{\;\mt{#1}}  % vemod.net/typesetting-units-in-latex
\newcommand*{\abt}{\mathord{\sim}} % tex.stackexchange.com/q/55701
\newcommand*{\ptl}{\partial}
\newcommand*{\del}{\nabla}

% This paper
\newcommand*\mean[1]{\bar{#1}}
\renewcommand{\vec}[1]{\mathbf{#1}}
\newcommand*{\tsup}{\textsuperscript}

\defcitealias{ressler2014}{R14}
\newcommand*{\Chandra}{\textit{Chandra}\ }
\newcommand*{\tsynch}{\tau_{\mt{synch}}}
\newcommand*{\mE}{m_\mt{E}}
\newcommand*{\Ecut}{E_{\mt{cut}}}
\newcommand*{\muG}{\unit{\mu G}}

\begin{document}
%\begin{CJK}{UTF8}{gbsn}

\title{Synchrotron X-Ray Rims in Tycho's Supernova Remnant are Energy Dependent}

\author{
Aaron Tran \altaffilmark{1,4},
% 陳 doesn't display?
% Also breaks emulateapj bibliography... toggle CJK env w/ AASTeX only
%Aaron Tran (陈宏裕)\altaffilmark{1,4},
Brian J. Williams\altaffilmark{1,5},
Robert Petre\altaffilmark{1},
Sean M. Ressler\altaffilmark{2},
Stephen P. Reynolds\altaffilmark{3}
}

\affil{
\tsup{1}NASA Goddard Space Flight Center, Greenbelt, MD 20771, USA \\
\tsup{2}Dept. Physics, University of California, Berkeley, CA 94720, USA \\
\tsup{3}Dept. Physics, North Carolina State University, Raleigh, NC 27695, USA
}

\altaffiltext{4}{CRESST, University of Maryland, College Park, MD 20742}
\altaffiltext{5}{NASA Postdoctoral Program Fellow}

\begin{abstract}
\note[Aaron]{Clone of AAS abstract}
Several young supernova remnants exhibit thin X-ray bright rims of synchrotron
radiation at their forward shocks.  Thin rims have been taken to indicate that
shock-accelerated electrons rapidly cool downstream of the shock, requiring
strong magnetic field amplification.  But, magnetic field damping immediately
behind the shock could produce similarly thin rims.  Synchrotron loss-limited
rim widths should decrease with energy whereas damping limited rims should be
relatively energy-independent.  To discriminate between models, we measured rim
widths around Tycho's supernova remnant in 5 energy bands using an archival
$750 \unit{ks}$ \Chandra observation.  Rims narrow with increasing energy,
favoring loss-limited radiation over magnetic damping and corroborating similar
observations in the remnant of SN 1006. Observed widths are best fit by
electron transport models requiring amplified magnetic fields of
$\abt200$--$1000 \muG$ and particle diffusion coefficients $\abt
1$--$100\times$ Bohm values, consistent with prior work on Tycho's SNR.
Non-negligible diffusion results in some degeneracy between magnetic field
strength and diffusion coefficient in setting observed rim widths, but strong
magnetic fields are required for all measurements.  A different approach may be
needed to better constrain diffusion at supernova remnant shocks.
\end{abstract}

% Six keywords, alphabetical order
\keywords{acceleration of particles ---
    ISM: individual objects (Tycho's SNR) ---
    ISM: magnetic fields ---
    ISM: supernova remnants ---
    shock waves ---
    X-rays: ISM}

% TODO GENERAL WARNING -- DATA ARE NOT ALL FULLY CONSISTENT AT THIS TIME.
% (spectra/profiles/fwhms/fits).

% ============
% Introduction
% ============
\section{Introduction} \label{sec:intro}

% What are these shock rims, what is this acceleration?
Forward shock accelerated electrons in young supernova remnants (SNRs) emit
synchrotron radiation strongly in the shock's immediate wake but quickly turn
off downstream, producing a shell-like morphology with bright X-ray and radio
rims/filaments due to line-of-sight projection \citep{koyama1995}.  Although
synchrotron emission due to accelerated electrons does not imply acceleration
of an unseen hadronic component, the prevailing theory of diffusive shock
acceleration (DSA) should operate on both protons and electrons.  Observations
of these synchrotron rims, with other lines of direct and indirect evidence,
have built a case for hadron and lepton acceleration up to $\unit{PeV}$
energies in young supernova remnants \citep{uchiyama2007, aharonian2004,
acero2010, ackermann2013}.  Efficient hadron acceleration in supernova remnant
shocks is a prime candidate source for galactic cosmic rays up to the
cosmic ray spectrum's ``knee'' \citep{vink2012}.
\note[Aaron]{the discussion of hadronic acceleration isn't really relevant,
just background information}

% What can we learn from these shocks?  Why are they interesting?
Spectral and spatial measurements of synchrotron rims can constrain physical
parameters of the shock acceleration process.  Many previous workers have
inferred strong magnetic field amplification at SNR shocks, as electrons must
rapidly radiate energy to account for the thinness of observed rims
\citep{bamba2003, vink2003, parizot2006}.  Furthermore, broadband synchrotron
synchrotron flux from supernova remnants steepens at X-ray energies
\citep{reynolds1999}, corresponding to a cut-off in the underlying electron
spectrum as predicted by diffusive shock acceleration \citep{webb1984}.
\note[Aaron]{question: which came first?  observation of synchrotron
steepening, or explanation/prediction of electron cut-off and hence synchrotron
roll-off?  Or, even, was the electron cut-off predicted but the observable
consequence of synchrotron roll-off not recognized until observations?} Many
fundamental questions about the shock acceleration process remain to be
answered.  Under what conditions do shocks accelerate particles efficiently?
How are magnetic fields amplified in such shocks?  \citet{reynolds2008} reviews
relevant observations and open questions to date.  The answers to these
questions are broadly relevant to many astrophysical shocks, such as Earth's
bow shock, stellar wind-blown bubbles, starburst galaxies \citep{heckman1990},
jets of active galactic nuclei, galaxy clusters \citep{van-weeren2010}, and
even cosmological shocks \citep{ryu2008}.
% Expand on these questions -- plasma turbulence, Alfven waves, wave damping
% behind the shock, upstream/downstream plasma interactions and instabilities
% These in turn affect plasma, CR dynamics -- how does the shock propagate and
% interact with the ISM?  Will we see a precursor due to diffusion and/or CR
% streaming?  How efficiently are CRs accelerated?

% What makes these shock rims?
The widths of these X-ray rims are controlled by a combination of
synchrotron losses, particle transport, and magnetic fields immediately
downstream of the shock.  Synchrotron losses depend on the initial electron
energy distribution and the gradual decrease of electron energies downstream of
the shock.  High energy electrons near the shock efficiently radiate harder
synchrotron X-ray photons, but decline in energy as they are transported away
from the shock; this may be seen in spectral softening downstream of the shock
\citep[e.g.,][]{cassam-chenai2007}.  The transport process is driven by bulk
plasma advection downstream of the shock and particle diffusion with respect to
bulk advection.  In general, diffusion allows higher energy electrons to
diffuse further upstream or downstream than would be expected from pure
advection.  Finally, the magnetic field strength may be quickly damped
downstream of the shock, so that electrons may not radiate as efficiently and
thin rims would reflect the spatial structure of the field, rather than
efficient particle acceleration (and losses) and/or efficient synchrotron
cooling \citep{pohl2005}.

% Whatever makes these shock rims also imposes energy dependence
Moreover, these controlling mechanisms predict different scalings for rim width
as a function of energy.  In the limiting case of constant magnetic field and
negligible particle diffusion, the rims are said to be synchrotron loss-limited
and rim widths should narrow with increasing photon frequency $\nu$ as
$\nu^{-1/2}$. Diffusion smears out rims at all energies with stronger effect at
higher energies, yielding a slower drop-off in rim widths when diffusion
dominates.  Magnetic damping, if at a length scale comparable to filament
widths, predicts comparatively energy-independent rim widths -- intuitively, if
the magnetic field turns off, synchrotron radiation turns off regardless of
electron energy.

% Magnetic damping?
Magnetic damping immediately behind the shock offers a tantalizing alternative
explanation for filament rims that requires less extreme magnetic field
amplification, and the possibility of damping in SNR shocks has not been fully
tested \citep{pohl2005, marcowith2010}.  \citet{cassam-chenai2007} examined
radio and X-ray synchrotron rim profiles, but found that neither damping nor
synchrotron losses matched radio profiles particularly well.  \citet{araya2010}
measured rims at multiple energy bands in Cas A, but suggested that the rims
only narrow more weakly than expected and did not consider magnetic damping.
\citet{rettig2012} gave model predictions for several historical SNRs and
proposed discriminating based on filament spectra -- the expectation is that
damped spectra are softer, loss-limited harder.  Recently, \citet{ressler2014}
further extended the idea of discriminating between damped or loss-limited
rims by measuring rim width energy dependence in X-ray energies and found that
rims consistently narrowed in the remnant of SN 1006, favoring synchrotron
loss-limited rims.

% Why Tycho's SNR
To further test these models, we follow \citet{ressler2014} (hereafter,
\citetalias{ressler2014}) by measuring rim widths at multiple energies in the
remnant of Tycho's 1572 supernova (hereafter, Tycho).  Tycho exhibits an
extensive shell of synchrotron-dominated thin rims around its periphery; the
rims show very little thermal emission, consistent with expansion into a low
density ISM \citet{williams2013}.  Tycho, being a nearby galactic SNR, is close
enough that its rims are easily resolved by \textit{Chandra} spatial
resolution. A deep $750 \unit{ks}$ exposure of the entire remnant from 2009
(PI: J. Hughes) also allows us to sample the remnant rims quite finely.
Furthermore, Tycho has been a prime target of study for determining whether
hadrons are being accelerated in historic SNRs \citep[and references
therein]{morlino2012}; understanding the acceleration process will also bear
upon our understanding of CR acceleration in Tycho.
\note[Aaron]{can reword/cut?}

% Paper roadmap
Our procedure closely follows that of \citetalias{ressler2014}.
We first briefly review relevant transport models which fully account for
electron advection and diffusion with continuous synchrotron losses, as
considered by, e.g., \citet[and references therein]{berezhko2004} and
\citet{rettig2012}.  We selected and measured rim widths at several locations
around Tycho's forward shock, verifying that selected rims are free of thermal
line emission.  Using our model for particle transport (with DSA included, to
first order, in a simple exponentially cut-off injected electron spectrum), we
obtain estimates on magnetic field strength and diffusion coefficient
magnitude. We discuss the implications of our model fit results for
particle transport in Tycho, despite a significant degeneracy between
diffusion and synchrotron losses when diffusion is non-negligible.

% =============================
% Transport models, observables
% =============================
\section{Nonthermal rim modeling}\label{sec:models}

We consider two particle transport models that we fit to measurements of
Tycho's thin rim widths.  We measure the rim full widths at half maximum
(FWHMs) (Section~\ref{sec:fwhms}) and use the measurements to directly assess
width-energy dependence.  The measurements are also fit to the transport
models' predictions of rim FWHMs at multiple photon energies.  The models
accept three free parameters: magnetic field strength $B_0$, a scaled diffusion
coefficient $\eta_2$, and a diffusion-energy scaling exponent $\mu$.

We use CGS (Gaussian) units throughout.  Our presentation is somewhat
abbreviated; please refer to \citetalias{ressler2014} for a fuller review and
exposition of the transport models and relevant parameters' effect on rim
widths.

%Early workers measured filament widths of several historical supernova remnants
%and considered ...  Assuming that thin rim widths are limited by synchrotron
%energy losses (loss-limited rims), you have $l_{\mt{ad}} \propto v_d \tsynch$
%gives a magnetic field...  \citet{bamba2003, vink2003, yamazaki2004,
%ballet2006} More sophisticated models have followed, but the basic result is
%consistent -- magnetic fields are amplified, with some amount of diffusion.
%\citet{parizot2006, araya2010} consider diffusion and advection simultaneously
%with the catastrophic dump equation.  \citet{berezhko2003, berezhko2004,
%cassam-chenai2007, morlino2010, rettig2012} consider diffusion and advection
%simultaneously with the continuous loss equation instead.  DSA may be
%incorporated by direct calculation, or by assumption of a resultant electron
%spectrum and cut-off \citep[e.g.,][]{zirakashvili2007}.
%\citet{morlino2010}, and references therein, consider particle transport in the
%context of nonlinear DSA neglecting particle energy losses (as would be
%relevant for accelerated protons).  All the same principles.

\subsection{Particle transport}\label{sec:transport}

% TODO Choose names for these models and refer to them consistently, throughout
% TODO define ALL symbols, synchrotron constants, etc
\note[Aaron]{\citet{blandford1987} and other papers take a much more thorough
tack in deriving the relevant transport equations, drawing specifically from
plasma physics / DSA assumptions -- I suppose that much material is not
relevant or condensed into our diffusion coefficient $D$ (w/ stated Bohm
diffusion).  Is na\"ively citing generalized ``transport'' is sufficient here?
E.g., there is often a term of form $(E/3) (\del\cdot\vec{v}) (\ptl f/\ptl E)$,
neglected for steady downstream velocity (i.e., $\del\cdot\vec{v}=0$), which
would not arise out of our equations as stated.
See, e.g., equation (9) of \citet{skilling1975}.}
% Transport fundamentals
The energy and space distribution of electrons at a supernova remnant
controls the synchrotron rims we see in X-ray and radio.  We model electron
advection and diffusion from initial injection at the remnant's forward shock
assuming 1-D steady-state plane transport, with spatially homogeneous and
isotropic diffusion, as following the equation:
\begin{equation}
    v_d \frac{\ptl f}{\ptl x} - D \frac{\ptl^2 f}{\ptl x^2} = C .
\end{equation}
Here $f = f(E,x)$ is electron distribution as a function of electron energy $E$
and downstream distance $x$ (where the forward shock is at $x=0$ and $x>0$
towards the remnant's interior); $C$ is an arbitrary source/sink term, $D$ is
the diffusion coefficient, and $v_d$ is plasma velocity downstream of the
shock.  We consider only particle transport downstream of the shock ($x \geq
0$).  The source/sink term $C$ and diffusion coefficient $D$ conveniently
encapsulate all assumptions about the accelerated electron spectrum, shock
and magnetic field structure, and time/space evolution of the particle
distribution.  More sophisticated treatments may explicitly treat magnetic
fields and diffusion across the shock, plasma streaming instabilities and
particle losses, anisotropic diffusion, the DSA process and efficiency, and
other details in great depth \citep[e.g.,][and references therein]{reville2013,
bykov2014, ferrand2014}.

% Simple model
A first approximation is to let electrons have constant energy that is dumped
catastrophically after a synchrotron timescale $\tsynch = 1/(b B^2 E)$, where
$b = 1.57 \times 10^{-3}$ in CGS units and $B$ is magnetic field (which may
vary in space as $B=B(x)$).  The sink term is $C = -f / \tsynch$, yielding:
\begin{equation} \label{eq:simp-mod}
    v_d \frac{\ptl f}{\ptl x} - D \frac{\ptl^2 f}{\ptl x^2} +
    \frac{f}{\tau_{\mt{synch}}} = 0 .
\end{equation}
No assumptions on the injected electron source term are necessary.
\citet{parizot2006} employed this model to derive estimates of magnetic fields,
diffusion coefficients, and electron spectrum cut-offs for several historical
SNRs.  Conveniently, equation~\eqref{eq:simp-mod} has analytic solution $f
\propto \exp(-x/a)$ where $a$ is a characteristic scale length.  The thin rim
FWHM predicted by equation~\eqref{eq:simp-mod} is:
\begin{equation} \label{eq:simp-fwhm}
    w = \frac{2\beta D / v_d}{\sqrt{1 +\frac{4D}{v_d^2 \tsynch}} - 1}
\end{equation}
with $\beta$ a projection factor relating $w$ to $a$ as $w = \beta a$
\citep{berezhko2004, parizot2006}.  We take $\beta = 4.6$ assuming a spatially
exponential $f(E,x)$ and projection from a spherical shell, as derived by
\citet{ballet2006}.

% Full model
A more accurate model accounts for continuous synchrotron losses as electrons
are injected from an exponentially cut-off power-law electron distribution
(normalization $K_0$, cut-off energy $\Ecut$) and transported downstream:
\begin{align} \label{eq:full-mod}
    v_d \frac{\ptl f}{\ptl x} - D \frac{\ptl^2 f}{\ptl x^2} =\;
    &K_0 E^{-s} e^{-E/\Ecut} \delta(x) \nonumber \\
    &+ \frac{\ptl}{\ptl E} \left(bB^2E^2f\right) ;
\end{align}
see \citet{webb1984}, \citet{berezhko2004}, \citet{cassam-chenai2007},
\citet{morlino2010}, \citet{rettig2012}, and references therein.
\citet{zirakashvili2007} have derived a different electron spectrum with a
steeper $\exp\left(- \left(E/\Ecut\right)^2\right)$ cut-off, but we follow
\citetalias{ressler2014} in using $E^{-s} e^{-E/\Ecut}$ for conceptual
simplicity.  The cut-off energy is determined by the diffusion coefficient $D$,
which we address in Section~\ref{sec:diffcoeff}.  To determine rim profiles and
widths, we first numerically solve for electron distribution $f(E,x)$ using
Green's function solutions by \citet{lerche1980} and \citet{rettig2012}; these
solutions are given by \citetalias{ressler2014} using notation similar to that
used herein \note[Aaron]{wording}.  The electron distribution may then be
integrated to obtain synchrotron emissivity:
\begin{equation} \label{eq:emissivity}
    j_{\nu}(x) \propto \int_0^\infty G(y) f(E,x) dE
\end{equation}
where $y \equiv \nu/(c_1 E^2 B)$ is a scaled synchrotron frequency and
$G(y) = y \int_y^\infty K_{5/3}(z) dz$ is one-particle synchrotron emissivity
with $K_{5/3}(z)$ a modified Bessel function of the second kind
\citep{pacholczyk1970}.  Integrating emissivity over lines of sight for a
spherical remnant yields intensity as a function of radial coordinate $r$:
\begin{equation} \label{eq:intensity}
    I_{\nu}(r) = 2 \int_0^{\sqrt{r_s^2 - r}}
                    j_{\nu} \left( r_s - \sqrt{s^2 + r^2} \right) ds
\end{equation}
where $s$ is the line-of-sight coordinate.  A model FWHM may then be calculated
from the resulting intensity profile.

% Diffusion
\subsection{Diffusion energy-dependence} \label{sec:diffcoeff}

% Bohm, Bohm-like, and MHD/turbulence-inspired diffusion
Much previous work has assumed Bohm-like diffusion in plasma downstream of SNR
shocks.  Bohm diffusion assumes that the particle mean free path $\lambda$ is
equal to the gyroradius $r_g = E/(eB)$, yielding diffusion coefficient
$D_{\mt{B}} = \lambda c / 3 = c E / (3 e B)$; here $c$ is the speed of light,
$E$ is particle energy, $e$ is the elementary charge, and $B$ is magnetic
field.  Bohm-like diffusion introduces a free prefactor $\eta$ such that
$\lambda = \eta r_g$ allows for varying diffusion; however, Bohm diffusion at
$\eta = 1$ is commonly considered a lower limit on the diffusion coefficient at
all energies.
\note[Aaron]{for self-edification, any justification / derivation of Bohm
limit?  All the papers seem to take it for granted}
Different types of MHD turbulence may give rise to a non-linear energy
scaling for diffusion coefficient \citep{giacalone1999, reynolds2004}.
Power-wavenumber spectra for Kolmogorov and Kraichnan turbulence, with a
quasi-linear approximation for particle motion,
\note[Aaron]{I don't really know what this means}
predict $D \propto E^{1/3}$ and $D \propto E^{1/2}$ respectively.
Particle gyroradii larger than field coherence length may to $D \propto E^{2}$
\note[Aaron]{Source?? Parizot mentions this in one sentence and cites nothing.
I have no idea what this means.}

% Define our diffusion coefficient, introduce fiducial energy
In our models, we consider a generalized diffusion coefficient with arbitrary
power-law dependence upon energy following, e.g., \citet{parizot2006}:
\begin{equation} \label{eq:diffcoeff}
    D(E) = \frac{\eta C_d E^\mu}{B}
         = \eta_h D_{\mt{B}}\left(E_h\right) \left(\frac{E}{E_h}\right)^\mu
\end{equation}
where $\mu$ parameterizes diffusion-energy scaling, and $\eta$ now has units
of $\mathrm{erg}^{1-\mu}$.
The right-hand side of equation~\eqref{eq:diffcoeff} introduces $\eta_h$, a
dimensionless diffusion coefficient scaled to the Bohm value at a fiducial
particle energy $E_h$.  This energy $E_h$ may be related to the emitted
synchrotron photons by taking the $\delta$-function approximation that an
electron of energy $E$ radiates at a singular frequency $\nu \propto E^2 B$ in
a magnetic field.  Note that $\eta_h$ and $\eta$ are related as
$\eta = \eta_h (E_h)^{1-\mu}$, and $\eta = \eta_h$ for Bohm-like diffusion
($\mu = 1$).

% Set choice of fiducial energy, eta_2
For subsequent analysis and model fitting, we take fiducial electron energy
$E_2 = E_h$ corresponding to a $2 \unit{keV}$ synchrotron photon and report all
model fit results in terms of $\eta_2 = \eta_h$.  Although the value of $E_2$
depends on magnetic field $B$ and may vary within the remnant, tying $\eta_h$
to a fixed observation energy gives a convenient and intuitive sense of
diffusion strength regardless of the underlying electron energies.
We also note that $\eta = \eta_2 (E_2)^{1-\mu}$.

\subsection{Electron cut-off}

% Electron cut-off energy
With a diffusion coefficient in hand, we determine the electron spectrum
cut-off energy by equating synchrotron loss and diffusive acceleration
timescales.  For low energies and small synchrotron losses (cooling time longer
than acceleration time), electrons may be efficiently accelerated; near or
above the cut-off energy, electrons will radiate or escape too rapidly to be
accelerated, so the energy spectrum drops off steeply.  The cut-off energy
adapted from \citet{parizot2006} is given as:
\begin{align}
    \Ecut =
        &\left(8.3\unit{TeV}\right)^{2/(1+\mu)}
        \left(\frac{B_0}{100 \muG}\right)^{-1/(1+\mu)} \nonumber \\
        &\times \left(\frac{v_s}{10^8 \unit{cm\;s^{-1}}}\right)^{2/(1+\mu)}
        \eta^{-1 / (1+\mu)} .
\end{align}
Note that this expression implicitly assumes (1) a strong shock with
compression ratio 4, and (2) isotropic magnetic turbulence both upstream and
downstream; see \citet{parizot2006} for a full derivation.

As the electron cut-off imposes a roll-off on SNR synchrotron flux,
estimates of the synchrotron roll-off frequency may provide an independent
observable to characterize shock conditions.
\note[Aaron]{todo: explore this?} % TODO FILL IN LATER -- NOT IMPORTANT NOW
\citet{miceli2013, yamazaki2014} looked at the synchrotron roll-off as a
diagnostic for electron spectrum cut-off behavior (whether loss or age limited)
and also infer various characteristics (particularly, $\eta > 1$ in SN 1006).
\citet{miceli2014-cloud} use azimuthal variation in synchrotron cut-off, with
other lines of evidence, to argue for the presence of a cloud interacting w/
the SW limb of SN 1006.

Speculation: could cloud interaction could be related to the sub-Bohm diffusion
observation in Sean's Filament 4?  The filament has many profiles, and looking
at Sean's data it looks like all the selections show a consistently strong drop
in FWHM.  \citet{miceli2013} argue that the concave distortion possibly
associated with the cloud is linked to a dip in synchrotron cut-off energies,
which they explain as being tied to a drop in shock velocity at the
indentation, due to the cloud.  If shock velocities are depressed, would more
diffusion / weaker B fields be necessary to explain the observed rim widths?
But we must still account for the quite strong width-energy drop-off!


\subsection{Magnetic damping}

% TODO REVISIT THIS AFTER SETTING UP CALCULATIONS
We also consider (not really) a magnetically damped field of form
\begin{equation}
    B(x) = B_{\mt{min}} + \left(B_0 + B_{\mt{min}}\right) \exp\left(-x / a_b\right) ,
\end{equation}
following \citet{pohl2005}.
In the full continuous energy loss model (equation~\eqref{eq:full-mod}), it
suffices to substitute $B = B(x)$ throughout; in the simpler model, $B(x)$
enters through the diffusion coefficient $D = D(E,x)$ and the analytic
expression (equation~\eqref{eq:simp-fwhm}) may no longer be valid.
Again, \citetalias{ressler2014} detail how to integrate a damped field into the
full model solutions.

(DRAFT) We currently do not present model fits with magnetic damping, although
in principle they are doable.  Set up the calculation w/ a range of scale
lengths (with $B_0$, $\eta_2$ chosen so that damping is the dominant control
for $0.7$--$7 \unit{keV}$), and compute FWHMs and $\mE$ values?

\subsection{Width-energy dependence and some intuition} \label{sec:energydep}

As discussed in Section~\ref{sec:intro}, width-energy dependence is key to
differentiating between loss/diffusion-limited rims and magnetically damped
rims.  \citetalias{ressler2014} parameterize rim width-energy dependence in
terms of a scaling exponent $\mE$ defined by:
\begin{equation}
    w(\nu) \propto \nu^{\mE}
\end{equation}
where $w(\nu)$ is filament FWHM as a function of observed photon frequency
$\nu$.  We may refer to observed photons interchangeably by energy or frequency
$\nu$, but $E$ is reserved for electron energy.  The exponent $\mE = \mE(\nu)$
is generally energy dependent.

We may consider advective and diffusive lengthscales for electron transport,
which naturally give rise to scaling relations:
\begin{equation} \label{eq:lad}
    l_{\mt{ad}} = v_d \tsynch \propto v_d B_0^{-3/2} \nu^{-1/2}
\end{equation}
and
\begin{equation} \label{eq:ldiff}
    l_{\mt{diff}} = \sqrt{D \tsynch} \propto \eta^{1/2} B_0^{-3/2} \nu^{(\mu-1)/4}
\end{equation}
If both diffusion and magnetic field damping are negligible (whether at a small
range of, or all, electron energies) and electrons are only loss-limited as
they advect downstream, $\mE$ attains a minimum value $\mE = -1/2$; this is the
strongest energy-dependence possible in our models.  Diffusion increases $\mE$
monotonically towards a value between $-1/4$ and $1/4$, depending on the value
of $\mu$.  In practice, $\mE$ appears to plateau at values between $-0.4$ and
$0$, with larger $\mu$ values yielding $\mE$ closer to $0$ (weaker
energy-dependence) \citepalias[Figure 3]{ressler2014}.  This may be attributed
in part to the electron energy cut-off, which limits the available electrons at
higher energies and thus thins the rims slightly, shifting $\mE$ negative
\citepalias[Figure 5]{ressler2014}.

% TODO REVISIT THIS AFTER SETTING UP CALCULATIONS
If magnetic damping is a primary control on rim widths, we expect to see $\mE
\sim -0.1$ \citepalias{ressler2014}.
\note[Aaron]{elaborate after filling in preceding section}

% ============
% Observations
% ============
\section{Observations}
\label{sec:observations}

% NOTE go through and double check tenses.
We measured synchrotron rim full widths at half maximum (FWHMs) from an
archival \Chandra ACIS-I observation of Tycho
(RA: 00\tsup{h}25\tsup{m}19\fs0, dec: +64\arcdeg08\arcmin10\farcs0; J2000)
between 2009 Apr 11 and 2009 May 5 (PI: J. Hughes;
\dataset[ADS/Sa.CXO\#obs/10093--10097]{ObsIDs: 10093--10097},
\dataset[ADS/Sa.CXO\#obs/10902--10906]{10902--10906}).
The total exposure time was $734 \unit{ks}$.
Level 1 \Chandra data were reprocessed with CIAO 4.6 and CALDB 4.6.1.1 and kept
unbinned with ACIS spatial resolution $0.492\arcsec$.
Merged and corrected events were divided into five energy bands:
0.7--1 keV, 1--1.7 keV, 2--3 keV, 3--4.5 keV, and 4.5--7 keV.
We excluded the 1.7--2 keV energy range to avoid any \ion{Si}{13} (He$\alpha$)
emission at 1.85 keV, prevalent in the remnant's thermal ejecta, which
might contaminate our nonthermal profile measurements.

% KEEP REGION NUMBER UPDATED
We selected 20 regions in 5 distinct filaments around Tycho's shock
(Figure~\ref{fig:snr}) based on the following criteria: (1) filaments should be
clear of spatial plumes of thermal ejecta in \Chandra images, which rules out,
e.g., areas of strong thermal emission on Tycho's eastern limb; (2) filaments
should be singular and localized; multiple filaments should either not
overlap or completely overlap (rules out parts of NE limb); (3) filament
peaks should be evident above the background signal or downstream thermal
emission (rules out faint southern filaments).  We accepted several
regions with poor quality peaks in the lowest energy band (0.7--1 keV) so long
as peaks in all higher energy bands were clear and well-fit.
\note[Aaron]{vague, imprecise}
We grouped regions into filaments by visual inspection of the remnant.
Within each filament, we chose region widths to obtain comparable counts
at the thin rim peak.
All measured rim widths are at least $1\arcsec$ and hence are resolved by
\textit{Chandra}'s point-spread function (PSF), which has FWHM of
$\abt 1\arcsec$ at $4\arcmin$ off-axis.

% TODO final figure should be higher quality, use vector overlay if possible
\begin{figure}
    \centering
    \plotone{figures/f0-snr-inv.png}
    \caption{RGB image of Tycho with region selections overlaid.  Image bands
    are 0.7--1 keV (red), 1--2 keV (green) and 2--7 keV (blue).  Bold region
    labels (1, 16) indicate region selections shown in
    Figures~\ref{fig:spec},~\ref{fig:profiles}.
    \note[Aaron]{Figure colors inverted to save ink.} \label{fig:snr}}
\end{figure}

% ----------------
% Filament spectra
% ----------------
\subsection{Filament spectra}
\label{sec:spec}

% NOTE be consistent in usage of word 'sections'
We extracted spectra for all regions to check that rim width measurements are
not affected by contaminating thermal line emission.  In each region, we
selected upstream and downstream sections from which to extract spectra.  The
upstream section is the smallest sub-region that contains the measured FWHM
bounds from all energy bands (see Section~\ref{sec:fwhms}); i.e., this section
overlies the thin rim.  The downstream section extends from the back of the rim
(end of the upstream section) to the rim model fit's (equation~\eqref{eq:prof})
downstream limit.  We extracted spectra from all \Chandra ObsIDs and grouped
the data into bins of $\geq 15$ counts.  Figure~\ref{fig:spec} plots an example
profile ($4.54$--$7 \unit{keV}$) with the downstream and upstream sections
highlighted.  We extracted background spectra from circular regions (radius
$\abt 30\arcsec$) around the remnant's exterior; each thin rim region's spectra
subtracted the closest background region's spectrum.
% Background region size is valid for data-tycho/bkg-2/ selections

% NOTE must match fig:snr, fig:profiles!
% Keep region numbers updated and consistent.
\begin{figure*}
    \plotone{figures/fig-specs_01.pdf}
    \plotone{figures/fig-specs_16.pdf}
    \caption{Spectra and fits from Regions 1 (top) and 16 (bottom) show varying
    rim morphology and example of rim where 0.7--1 keV peak could not be fit.
    Left: $4.5$--$7 \unit{keV}$ profile with highlighted downstream (blue) and
    upstream (grey) sections.  Intensity in arbitrary units (a.u.).  Middle:
    downstream spectrum with absorbed power law fit; Si and S lines at $1.85$,
    $2.45 \unit{keV}$ are clearly visible.  Right: upstream spectrum with
    absorbed power law fit shows that each region is likely free of thermal
    line
emission.}
    \label{fig:spec}
\end{figure*}

We fit spectra from each region's upstream and downstream sections to an
absorbed power law model (XSPEC 12.8.1, \texttt{phabs*po}) between $0.5$--$7
\unit{keV}$ with photon index $\Gamma$, hydrogen column density $N_{\mt{H}}$,
and a normalization as free parameters.  Table~\ref{tab:spec} lists best fit
parameters and reduced $\chi^2$ values for all regions.  Spectra from thin rims
(upstream sections) are well-fit by the power law model alone; spectra from
sections downstream of the rims are generally poorly-fit, reflecting thermal
contamination primarily from \ion{Si}{13} and \ion{S}{15} He$\alpha$ line
emission at $1.85$ and $2.45$ keV.  Our best-fit photon indices ($2.4$--$3$)
and column densities ($0.6$--$0.8 \times 10^{22} \unit{cm^{-2}}$) from upstream
spectra are consistent with previous spectral fits to Tycho's nonthermal rims
\citep{hwang2002, cassam-chenai2007}.
% NOTE keep stated ranges of spectral indices, column densities up to date
% NOTE keep example region numbers up to date

% NOTE keep region numbers updated (in discussion of good/bad downstream spectra)
\begin{table}
    \scriptsize
    \centering
    \caption{Region spectra fit parameters\label{tab:spec}}
    \begin{tabular}{@{}lcccccr@{}}
\toprule
{} & \multicolumn{3}{c}{Downstream spectra}
   & \multicolumn{3}{c}{Upstream spectra} \\
\cmidrule(lr){2-4} \cmidrule(l){5-7}
Region & $n$ & $n_H$ & $\chi^2_{\mathrm{red}}$ (dof)
       & $n$ & $n_H$ & $\chi^2_{\mathrm{red}}$ (dof) \\
{} & (-) & ($\mt{cm}^{-2}$) & {}
   & (-) & ($\mt{cm}^{-2}$) & {} \\
\midrule
1 & 2.97 & 0.68 & 2.27 (272) & 2.77 & 0.78 & 0.92 (239) \\
2 & 2.91 & 0.64 & 3.55 (163) & 2.54 & 0.67 & 1.05 (232) \\
3 & 3.00 & 0.60 & 3.41 (181) & 2.75 & 0.66 & 1.13 (245) \\
4 & 2.94 & 0.45 & 1.35 (199) & 2.88 & 0.66 & 0.92 (224) \\
5 & 2.90 & 0.50 & 1.15 (224) & 2.83 & 0.68 & 0.96 (246) \\
6 & 2.85 & 0.43 & 1.87 (194) & 3.00 & 0.63 & 1.20 (222) \\
7 & 2.80 & 0.37 & 1.43 (100) & 2.79 & 0.61 & 1.11 (243) \\
8 & 2.79 & 0.55 & 2.36 (183) & 2.83 & 0.71 & 1.22 (285) \\
9 & 2.99 & 0.68 & 4.43 (239) & 2.77 & 0.70 & 1.01 (252) \\
10 & 2.89 & 0.58 & 1.31 (186) & 2.86 & 0.71 & 1.27 (301) \\
11 & 2.89 & 0.61 & 4.92 (231) & 2.91 & 0.72 & 1.16 (281) \\
12 & 3.03 & 0.65 & 1.02 (156) & 2.91 & 0.78 & 0.97 (271) \\
13 & 2.97 & 0.76 & 1.39 (181) & 2.72 & 0.75 & 0.96 (217) \\
\bottomrule
\end{tabular}

\end{table}

Table~\ref{tab:spec} confirms that our region selections are practically free
of thermal line emission, as already suggested by visual inspection
(Figure~\ref{fig:snr}).  Our exclusion of all $1.7$--$2 \unit{keV}$ photons
further limits thermal contamination as $1.85 \unit{keV}$ Si line emission is
over a third of Tycho's thermal flux as detected by \Chandra \citep{hwang2002}.

% --------------------------
% FWHM measurement procedure
% --------------------------
\subsection{Filament width measurements}
\label{sec:fwhms}

% NOTE keep mentioned number of regions updated!
% Check these (arcsec) numbers before final version.
We obtained radial intensity profiles from $\abt 10$--$20\arcsec$ behind the
shock to $\abt 5$--$10\arcsec$ in front for each of our regions in all five
energy bands.  To increase signal-to-noise, we integrate along the shock
($5$--$23\arcsec$) in each region.  Plotted and fitted profiles are reported in
vignetting and exposure-corrected intensity units; error bars are computed from
raw counts assuming Poisson statistics.  Intensity profiles peak sharply
within $\abt 2$--$3\arcsec$ behind the shock, demarcating our thin rims,
then fall off gradually until thermal emission picks up further behind the
shock.

We fitted rim profiles to a piecewise two-exponential model:
\begin{equation} \label{eq:prof}
    h(x) =
    \begin{cases}
        A_u \exp \left(\frac{x_0 - x}{w_u}\right) + C_u, &x \geq x_0 \\
        A_d \exp \left(\frac{x - x_0}{w_d}\right) + C_d, &x < x_0
    \end{cases}
\end{equation}
where $h(x)$ is profile height and $x$ is radial distance from remnant center.
The rim model, which is strictly empirical, has 6 free parameters:
$A_u, x_0, w_u, w_d, C_u$, and $C_d$; $A_d = A_u + (C_u - C_d)$ enforces
continuity at $x=x_0$. Our model is similar to that of \citet{bamba2003,
bamba2005-hist} and differs slightly from that of \citetalias{ressler2014}.
To fit only the nonthermal rim in each intensity profile, we selected the fit
domain for each profile as follows.  The downstream bound was set at the first
local data minimum downstream of the rim peak, identified by smoothing the
profiles with a 21-point ($\abt 10\arcsec$) Hanning window.  The upstream bound
was set at the profile's outer edge (i.e., no data were removed).

\begin{figure*}[ht]
    \plotone{figures/fig-profiles_01.pdf}
    \plotone{figures/fig-profiles_16.pdf}
    \caption{Best fit profiles with measured FWHMs demarcated for each energy
        band in Region 1 (top) and Region 16 (bottom).  Energy bands increase
        from left to right.  Data points in red were excluded from profile
        fitting domains as described in text.}
    \label{fig:profiles}
\end{figure*}

% NOTE keep blacklisted/excluded region count updated!
From the fitted profiles we extracted a full width at half maximum (FWHM) for
each region and each energy band after subtracting a constant background term
$\min(C_u, C_d)$.  We could not resolve a FWHM in 4 of 20 regions at 0.7--1 keV
(Table~\ref{tab:fwhms}); in these regions, either the downstream FWHM bound
would extend outside the fit domain or we could not find an acceptable fit to
equation~\eqref{eq:prof}.  We were able to resolve FWHMs for all regions at
higher energy bands (1--7 keV).
% NOTE review the FWHM rejection criterion, on the final set of regions
% We will be blacklisting by hand, and we have to explain how we chose the
% blacklisted ones.

To estimate FWHM uncertainties, we horizontally stretched each best-fit
profile by mapping radial coordinate $x$ to
$x'(x) = x (1 + \xi (x-x_0)/(50\arcsec-x_0))$ with $\xi$ an arbitrary stretching
parameter and $x_0$ the best-fit rim center from equation~\eqref{eq:prof};
this yields a new profile $h'(x) = h(x'(x))$.
We varied $\xi$ (and hence rim FWHM) to vary each profile fit $\chi^2$ by 2.7
and took the stretched or compressed FWHMs as upper/lower bounds on our
reported FWHMs.

% -------------
% Model fitting
% -------------
\subsection{Filament model fitting}
\label{sec:fits}

% Fit data to models
We fit the two transport models given by Equations~\eqref{eq:simp-mod} and
\eqref{eq:full-mod} to our measured rim widths as a function of energy.
We mapped each width measurement to the lower energy limit of its energy band;
e.g., $0.7$--$1 \unit{keV}$ is assigned to $0.7 \unit{keV}$ and fitted to
model profile widths at $0.7 \unit{keV}$.  Errors in least squares fits
average the positive and negative errors on each FWHM measurement.

% Model knobs and how we twiddled them
We varied three physical parameters: diffusion-energy scaling exponent $\mu$,
normalized diffusion coefficient $\eta_2$, and magnetic field strength $B_0$.
To make nonlinear fitting tractable, we fixed $\mu$ in all fits and considered
$\mu = 0$, $1/3$, $1/2$, $1$, $1.5$, and $2$; recall that $\mu = 1/3$ and $1/2$
correspond to Kolmogorov and Kraichnan turbulence.  Our fits do not consider
a damped magnetic field with the additional parameter of scale length $a_b$,
and so magnetic field strength $B_0$ is assumed constant downstream of the
shock.  Below we shall show that magnetic damping cannot explain our observed
filament widths, so we present fits assuming loss-limited rim widths only.
% TODO Edit if I add magnetic damping material

% Tycho parameters TODO change green2009 to green2014 once published in BASI
A few remnant-specific parameters enter into the model calculations.  We
adopted electron spectral index $s = 2.3$ from radio spectral index $\alpha =
0.65$ \citep{kothes2006},
\note[Aaron]{will change to $s= 2.16$ from $\alpha=0.58$ following
\citet{sun2011} on next iteration}
remnant distance $3 \unit{kpc}$ \citep[cf.][]{hayato2010}, and shock radius
$1.08 \times 10^{19} \unit{cm}$ from angular radius $240\arcsec$
\citep{green2009}.  Tycho's forward shock velocity varies with azimuth by up to
a factor of 2; we interpolated shock velocities reported by
\citet{williams2013} (rescaled to $3 \unit{kpc}$ rather than $2.3 \unit{kpc}$
to estimate individual shock velocities for each region.

% How to do the fit
Equation~\eqref{eq:simp-fwhm} was directly fit to our measurements to obtain
best-fit values of $\eta_2$ and $B_0$.  The full model was numerically computed
as detailed in Section~\ref{sec:transport}, yielding intensity profiles and
hence model FWHMs in each energy band.  To assist the nonlinear
fitting, we tabulated a large grid of model FWHM values for various values of
$\mu$, $\eta_2$, $B_0$, and shock velocity $v_s$.  From a best initial grid
value, we used a Levenberg-Marquardt fitting routine to determine the best
fits.  Errors on best fit parameters were computed by varying each fit
parameter and obtaining a new best fit model, with one less degree of freedom,
to find the parameter limit such that $\Delta \chi^2 = 1$ for a roughly
1-$\sigma$ error.

We required $\eta_2$ to be positive in all fits.  We also deemed
best-fit and error bound values with $\eta_2 \geq 10^5$ and $B_0 \geq 10
\unit{mG}$ to be effectively unconstrained. % Change this if needed.

% TODO Come back after checking Pacholczyk resolution
As the full model must be numerically solved, its predicted rim widths are
subject to resolution error in the numerical integrals (discretization over
radial coordinate, line-of-sight coordinate, electron distribution, Green's
function integrals).  We chose our integration resolutions such that the
fractional error in model FWHMs associated with halving/doubling integration
resolution is less than $1\%$ for the parameter space relevant to our
filaments.

% =======================
% Results, FWHMs and fits
% =======================
\section{Results}

% --------------------
% FWHM results, tables
% --------------------
\subsection{Rim widths}
\label{sec:fwhm-results}

Table~\ref{tab:fwhms} reports FWHM measurements for all of our regions.
We also report $\mE$ values for all but the lowest energy band, computed
point-to-point between discrete energy bands as:
\begin{equation}
    \mE(E_2) = \frac{\ln(w_2/w_1)}{\ln(E_2/E_1)}
\end{equation}
where $w_1, w_2$ and $E_1, E_2$ are FWHMs and lower energy values for each
energy band (e.g., $\mE$ at $1 \unit{keV}$ is computed using FWHMs from
$0.7$--$1 \unit{keV}$ and $1$--$1.7 \unit{keV}$, with $E_2 = 1
\unit{keV}$ and $E_1 = 0.7 \unit{keV}$.  Errors on $\mE$ are propagated in
quadrature from adjacent FWHM measurements.

\begin{table*}[ht]
    % Use \tiny for manuscript; \scriptsize for 2-column
    %\tiny
    \scriptsize
    \centering
    \caption{Measured full widths at half max (FWHMs) for all regions.
             \label{tab:fwhms}}
    \begin{tabular}{@{}l ccccc r@{ $\pm$ }l r@{ $\pm$ }l r@{ $\pm$ }l r@{ $\pm$ }l @{}}

\toprule
{} & \multicolumn{5}{c}{FWHM (arcsec)} & \multicolumn{8}{c}{$\mE$ (-)} \\
\cmidrule(lr){2-6} \cmidrule(l){7-14}
Region & Band 1 & Band 2 & Band 3 & Band 4 & Band 5
       & \multicolumn{2}{c}{Bands 1--2} & \multicolumn{2}{c}{Bands 2--3}
       & \multicolumn{2}{c}{Bands 3--4} & \multicolumn{2}{r}{Bands 4--5} \\ [0.2em]
{} & (0.7--1 keV) & (1--1.7 keV) & (2--3 keV) & (3--4.5 keV) & (4.5--7 keV)
   & \multicolumn{2}{c}{(1 keV)} & \multicolumn{2}{c}{(2 keV)}
   & \multicolumn{2}{c}{(3 keV)} & \multicolumn{2}{r}{(4.5 keV)} \\
\midrule
1 & {} & ${8.80}^{+0.18}_{-0.15}$ & ${6.34}^{+0.26}_{-0.21}$ & ${7.40}^{+0.30}_{-0.23}$ & ${5.57}^{+0.47}_{-0.42}$
  & \multicolumn{2}{c}{} & $-0.47$ & $0.06$ & $0.38$ & $0.13$ & $-0.70$ & $0.22$ \\ [0.5em]
2 & {} & ${4.22}^{+0.12}_{-0.09}$ & ${2.36}^{+0.12}_{-0.09}$ & ${3.00}^{+0.16}_{-0.12}$ & ${4.11}^{+0.34}_{-0.30}$
  & \multicolumn{2}{c}{} & $-0.84$ & $0.08$ & $0.59$ & $0.16$ & $0.77$ & $0.23$ \\ [0.5em]
3 & ${1.72}^{+0.13}_{-0.10}$ & ${2.47}^{+0.08}_{-0.07}$ & ${1.78}^{+0.09}_{-0.07}$ & ${2.10}^{+0.11}_{-0.11}$ & ${1.32}^{+0.10}_{-0.09}$
  & $1.01$ & $0.21$ & $-0.47$ & $0.08$ & $0.41$ & $0.17$ & $-1.15$ & $0.22$ \\

\cmidrule{1-14}
4 & ${5.85}^{+0.37}_{-0.33}$ & ${4.35}^{+0.09}_{-0.08}$ & ${3.26}^{+0.11}_{-0.09}$ & ${3.69}^{+0.12}_{-0.11}$ & ${3.20}^{+0.21}_{-0.18}$
  & $-0.83$ & $0.18$ & $-0.41$ & $0.05$ & $0.31$ & $0.11$ & $-0.35$ & $0.17$ \\ [0.5em]
5 & ${9.12}^{+0.45}_{-0.40}$ & ${4.52}^{+0.11}_{-0.12}$ & ${3.06}^{+0.11}_{-0.11}$ & ${3.25}^{+0.15}_{-0.13}$ & ${3.04}^{+0.21}_{-0.18}$
  & $-1.97$ & $0.15$ & $-0.56$ & $0.06$ & $0.15$ & $0.14$ & $-0.17$ & $0.19$ \\ [0.5em]
6 & ${2.48}^{+0.18}_{-0.18}$ & ${2.32}^{+0.05}_{-0.06}$ & ${2.98}^{+0.11}_{-0.09}$ & ${2.05}^{+0.08}_{-0.09}$ & ${2.21}^{+0.15}_{-0.14}$
  & $-0.19$ & $0.21$ & $0.36$ & $0.06$ & $-0.92$ & $0.13$ & $0.18$ & $0.19$ \\ [0.5em]
7 & ${2.69}^{+0.20}_{-0.17}$ & ${2.33}^{+0.05}_{-0.05}$ & ${2.31}^{+0.08}_{-0.08}$ & ${1.81}^{+0.09}_{-0.07}$ & ${1.83}^{+0.11}_{-0.08}$
  & $-0.39$ & $0.20$ & $-0.01$ & $0.06$ & $-0.60$ & $0.14$ & $0.02$ & $0.17$ \\ [0.5em]
8 & ${2.33}^{+0.21}_{-0.20}$ & ${2.72}^{+0.08}_{-0.08}$ & ${2.38}^{+0.10}_{-0.09}$ & ${2.10}^{+0.10}_{-0.09}$ & ${2.37}^{+0.20}_{-0.17}$
  & $0.43$ & $0.26$ & $-0.19$ & $0.07$ & $-0.30$ & $0.15$ & $0.29$ & $0.22$ \\ [0.5em]
9 & ${2.16}^{+0.24}_{-0.23}$ & ${2.35}^{+0.07}_{-0.06}$ & ${2.47}^{+0.11}_{-0.11}$ & ${1.91}^{+0.09}_{-0.09}$ & ${2.20}^{+0.17}_{-0.16}$
  & $0.24$ & $0.31$ & $0.07$ & $0.07$ & $-0.63$ & $0.16$ & $0.34$ & $0.22$ \\ [0.5em]
10 & ${2.38}^{+0.24}_{-0.23}$ & ${1.99}^{+0.07}_{-0.06}$ & ${1.76}^{+0.09}_{-0.08}$ & ${1.59}^{+0.09}_{-0.08}$ & ${1.58}^{+0.13}_{-0.12}$
  & $-0.50$ & $0.29$ & $-0.18$ & $0.08$ & $-0.24$ & $0.18$ & $-0.02$ & $0.23$ \\

\cmidrule{1-14}
11 & ${2.30}^{+0.36}_{-0.26}$ & ${3.23}^{+0.15}_{-0.13}$ & ${2.52}^{+0.16}_{-0.13}$ & ${1.90}^{+0.14}_{-0.13}$ & ${3.09}^{+0.45}_{-0.38}$
  & $0.95$ & $0.40$ & $-0.36$ & $0.10$ & $-0.70$ & $0.22$ & $1.21$ & $0.37$ \\ [0.5em]
12 & {} & ${3.86}^{+0.17}_{-0.16}$ & ${2.61}^{+0.15}_{-0.13}$ & ${3.02}^{+0.22}_{-0.21}$ & ${2.23}^{+0.21}_{-0.17}$
  & \multicolumn{2}{c}{} & $-0.56$ & $0.10$ & $0.36$ & $0.22$ & $-0.74$ & $0.27$ \\ [0.5em]
13 & ${2.85}^{+0.22}_{-0.17}$ & ${2.43}^{+0.05}_{-0.05}$ & ${2.36}^{+0.08}_{-0.05}$ & ${1.95}^{+0.09}_{-0.10}$ & ${1.84}^{+0.11}_{-0.14}$
  & $-0.45$ & $0.20$ & $-0.04$ & $0.05$ & $-0.47$ & $0.13$ & $-0.15$ & $0.20$ \\

\cmidrule{1-14}
14 & ${2.86}^{+0.17}_{-0.16}$ & ${2.42}^{+0.06}_{-0.04}$ & ${2.23}^{+0.08}_{-0.07}$ & ${2.38}^{+0.10}_{-0.08}$ & ${2.19}^{+0.12}_{-0.10}$
  & $-0.47$ & $0.17$ & $-0.12$ & $0.06$ & $0.17$ & $0.12$ & $-0.20$ & $0.15$ \\ [0.5em]
15 & ${2.71}^{+0.17}_{-0.16}$ & ${1.99}^{+0.05}_{-0.04}$ & ${1.80}^{+0.06}_{-0.05}$ & ${1.87}^{+0.07}_{-0.05}$ & ${1.52}^{+0.09}_{-0.08}$
  & $-0.85$ & $0.18$ & $-0.15$ & $0.05$ & $0.09$ & $0.11$ & $-0.51$ & $0.16$ \\ [0.5em]
16 & ${1.87}^{+0.14}_{-0.13}$ & ${1.73}^{+0.04}_{-0.03}$ & ${1.52}^{+0.06}_{-0.05}$ & ${1.25}^{+0.06}_{-0.04}$ & ${1.23}^{+0.08}_{-0.06}$
  & $-0.22$ & $0.21$ & $-0.18$ & $0.06$ & $-0.49$ & $0.13$ & $-0.04$ & $0.17$ \\ [0.5em]
17 & ${1.65}^{+0.13}_{-0.12}$ & ${1.92}^{+0.05}_{-0.05}$ & ${1.54}^{+0.06}_{-0.07}$ & ${1.45}^{+0.07}_{-0.06}$ & ${2.05}^{+0.16}_{-0.14}$
  & $0.43$ & $0.22$ & $-0.31$ & $0.07$ & $-0.16$ & $0.15$ & $0.86$ & $0.21$ \\

\cmidrule{1-14}
18 & {} & ${4.45}^{+0.13}_{-0.12}$ & ${3.18}^{+0.17}_{-0.16}$ & ${2.96}^{+0.20}_{-0.19}$ & ${1.65}^{+0.21}_{-0.16}$
  & \multicolumn{2}{c}{} & $-0.49$ & $0.09$ & $-0.17$ & $0.21$ & $-1.45$ & $0.32$ \\ [0.5em]
19 & ${7.41}^{+0.62}_{-0.46}$ & ${2.30}^{+0.08}_{-0.06}$ & ${2.28}^{+0.11}_{-0.08}$ & ${2.16}^{+0.12}_{-0.11}$ & ${1.60}^{+0.17}_{-0.14}$
  & $-3.27$ & $0.22$ & $-0.02$ & $0.08$ & $-0.13$ & $0.17$ & $-0.74$ & $0.27$ \\ [0.5em]
20 & ${4.81}^{+0.31}_{-0.31}$ & ${1.84}^{+0.06}_{-0.03}$ & ${1.87}^{+0.08}_{-0.06}$ & ${1.56}^{+0.07}_{-0.06}$ & ${2.14}^{+0.23}_{-0.23}$
  & $-2.68$ & $0.19$ & $0.02$ & $0.07$ & $-0.44$ & $0.14$ & $0.77$ & $0.28$ \\
\midrule
Mean & $3.45 \pm 0.55$ & $3.11 \pm 0.37$ & $2.53 \pm 0.23$ & $2.47 \pm 0.30$ & $2.35 \pm 0.23$
  & $-0.55$ & $0.30$ & $-0.25$ & $0.06$ & $-0.14$ & $0.10$ & $-0.09$ & $0.15$ \\

\bottomrule
\end{tabular}
\tablecomments{Mean values computed for all regions; mean $\mE$ values are
averages for region $\mE$ values (i.e., not computed from mean FWHMs).  Errors
on mean values are standard errors of the mean.  Horizontal rules group
individual regions into filaments.}
\tablecomments{(DRAFT) In the next iteration of the data pipeline, I will cull
the FWHMs by hand. 8 of 22 regions (1--3, 5, 11--12, 18--19) will not have
FWHMS at 0.7--1 keV.}

\end{table*}

Measured rim widths decrease with energy in most regions and energy bands.
Although the measurement variation between measurements is quite large, shown
dramatically in the point-wise computed $\mE$ values, the mean rim width
decreases consistently with increasing energy.  Furthermore, mean $\mE$ values
are consistently negative and tend smoothly towards $0$ (weaker
energy-dependence) with increasing energy.

The errors on FWHM measurements are typically $\abt 10\%$ or smaller,
reflecting the high quality of underlying \Chandra data as also seen in
Figure~\ref{fig:profiles}.  Scatter in FWHM measurements may be attributed in
part to (1) our profile fitting procedure, which depends on an empirically
chosen profile fit function which may also fit the rims non-uniquely, and (2)
variation in Tycho's rim morphology, also demonstrated clearly in
comparison of rims from Regions 1 and 16 in Figure~\ref{fig:profiles}.
\note[Aaron]{review paragraph wording.  This may belong in the discussion?}

% discussion of uncertainties/caveats in our results
% \note[Aaron]{text blob}
% One way to address uncertainty in the fitting function is to consider other
% functions -- for example, a gently smoothed spline fit, a manually capped
% two-exponential function, a Gaussian sitting on a ramp or logistic curve, etc.
% Explain that nothing happens -- not much (but should reaffirm this or give
% some kind of numbers).
% Another is simply our sampling -- we have already sought to circumvent this by
% taking multiple regions in each filament, as many as possible without overly
% compromising the quality of our radial profiles.  This also makes no major
% difference to our results.

% -------------------------
% Model fit results, tables
% -------------------------
\subsection{Model fit results}
\label{sec:fit-results}

\begin{table*}[ht]
    \tiny
    \centering
    \caption{Full model best fit for individual regions, Filaments 1--2.
    \label{tab:fits}}
    \begin{tabular}{@{} l rrr rrrr @{}}
\toprule
{} & \multicolumn{3}{c}{Loss-limited}
   & \multicolumn{4}{c}{Damped, $a_b \leq 0.01$} \\
\cmidrule(lr){2-4} \cmidrule(l){5-8}
Region & $\eta_2$ (-) & $B_0$ ($\mu$G) & $\chi^2$
       & $\eta_2$ (-) & $B_0$ ($\mu$G) & $\chi^2$ & $a_b$ \\
\midrule
 1 & 3.84 & 211  & 25.1 & 1.63  & 25.4 & 24.2 & 0.008 \\
 2 & 0.87 & 309  & 80.8 & 19.02 & 16.8 & 71.4 & 0.003 \\
 3 & 1.30 & 437  & 21.2 & 183.8 & 14.1 & 20.9 & 0.003 \\
\cmidrule{1-8}
 4 & 4.85 & 343  & 33.3 & 1.72  & 25.2 & 29.4 & 0.004 \\
 5 & 1.51 & 300  & 18.4 & 0.49  & 27.6 & 14.3 & 0.003 \\
 6 & 663  & 1374 & 69.9 & 0.00  & 114  & 52.7 & 0.003 \\
 7 & 29.3 & 705  & 11.3 & 0.02  & 242  & 10.4 & 0.004 \\
 8 & 448  & 1212 & 10.0 & 0.15  & 52.2 & 9.67 & 0.003 \\
 9 & 670  & 1411 & 15.6 & 0.02  & 46.8 & 12.9 & 0.002 \\
10 & 14.2 & 682  & 1.26 & 0.36  & 36.1 & 1.32 & 0.002 \\
\cmidrule{1-8}
11 & 0.70 & 345  & 11.8 & 0.20  & 31.0 & 11.6 & 0.002 \\
12 & 1.07 & 320  & 9.60 & 0.05  & 35.0 & 8.53 & 0.002 \\
13 & 75.9 & 834  & 10.1 & 0.00  & 243  & 8.26 & 0.004 \\
\cmidrule{1-8}
14 & 456  & 1226 & 17.0 & 0.01  & 89.9 & 9.74 & 0.003 \\
15 & 76.8 & 932  & 19.8 & 4.48  & 31.3 & 19.6 & 0.003 \\
16 & 3.81 & 583  & 4.25 & 0.22  & 341  & 3.58 & 0.004 \\
17 & 237  & 1300 & 29.2 & 0.46  & 98.4 & 29.0 & 0.003 \\
\cmidrule{1-8}
18 & 0.00 & 254  & 10.0 & 4103  & 8.7  & 12.7 & 0.006 \\
19 &  629 & 1384 & 8.28 & 0.02  & 45.8 & 7.27 & 0.002 \\
20 & 51.1 & 891  & 94.0 & 1.70  & 47.3 & 93.7 & 0.003 \\
\bottomrule
\end{tabular}

\end{table*}

\begin{table*}[ht]
    \tiny
    \centering
    \caption{Full model best fits for individual regions, Filaments 3--5.
    \label{tab:fits-pt2}}
    \begin{tabular}{@{}rllr llr@{}}

\toprule
\multicolumn{7}{c}{Filament 2} \\
\cmidrule{1-7}
{} & \multicolumn{3}{c}{Region 2\tablenotemark{a}}
   & \multicolumn{3}{c}{Region 3} \\
\cmidrule(lr){2-4} \cmidrule(l){5-7}
$\mu$ (-) & $\eta_2$ (-) & $B_0$ ($\mu$G) & $\chi^2_{\mt{red}}$
          & $\eta_2$ (-) & $B_0$ ($\mu$G) & $\chi^2_{\mt{red}}$ \\
\cmidrule{1-7}
0.00 & ${0.01}^{+0.1}_{-0.01}$ & ${294.4}^{+8.9}_{-4.8}$ & 26.05
     & ${19.6}^{+19.8}_{-9.1}$ & ${720}^{+120}_{-90}$    & 15.57 \\
0.33 & ${0.01}^{+0.1}_{-0.01}$ & ${292.7}^{+6.8}_{-3.9}$ & 25.91
     & ${54}^{+110}_{-34}$ & ${850}^{+230}_{-170}$       & 10.59 \\
0.50 & ${0.01}^{+0.2}_{-0.01}$ & ${292.7}^{+6}_{-4.5}$   & 25.91
     & ${86}^{+290}_{-63}$ & ${910}^{+360}_{-230}$       & 9.09 \\
1.00 & ${0.002}^{+0.2}_{-0.002}$ & ${293}^{+2.8}_{-8.1}$ & 25.76
     & ${13}^{+23}_{-6}$ & ${550}^{+130}_{-60}$          & 8.18 \\
1.50 & ${0.09}^{+0.2}_{-0.09}$ & ${285.9}^{+7}_{-3.5}$   & 25.46
     & ${5.9}^{+2.9}_{-1.8}$ & ${448}^{+30}_{-22}$       & 8.26 \\
2.00 & ${0.06}^{+0.2}_{-0.04}$ & ${283.4}^{+4.8}_{-3.3}$ & 25.05
     & ${4.8}^{+1.6}_{-1.2}$ & ${414}^{+16}_{-14}$       & 8.59 \\

\midrule
\multicolumn{7}{c}{Filament 3} \\
\cmidrule{1-7}
{} & \multicolumn{3}{c}{Region 4}
   & \multicolumn{3}{c}{Region 5} \\
\cmidrule(lr){2-4} \cmidrule(l){5-7}
$\mu$ (-) & $\eta_2$ (-) & $B_0$ ($\mu$G) & $\chi^2_{\mt{red}}$
          & $\eta_2$ (-) & $B_0$ ($\mu$G) & $\chi^2_{\mt{red}}$ \\
\cmidrule{1-7}

0.00 & ${16.4}^{+8.3}_{-4}$ & ${756}^{+68}_{-43}$       & 70.21
     & ${16.2}^{+11}_{-4.4}$ & ${910}^{+100}_{-59}$     & 33.25 \\
0.33 & ${46}^{+28}_{-16}$ & ${889}^{+98}_{-80}$         & 47.53
     & ${47}^{+41}_{-19}$ & ${1070}^{+160}_{-110}$      & 18.00 \\
0.50 & ${74}^{+55}_{-28}$ & ${953}^{+130}_{-98}$        & 37.69
     & ${75}^{+88}_{-34}$ & ${1150}^{+220}_{-150}$      & 12.03 \\
1.00 & ${260}^{+420}_{-140}$ & ${1150}^{+290}_{-190}$   & 15.35
     & ${350}^{+1600}_{-270}$ & ${1490}^{+6000}_{-430}$ & 1.77 \\
1.50 & ${370}^{+5700}_{-280}$ & ${1130}^{+1060}_{-310}$ & 5.37
     & ${19.4}^{+15}_{-6.9}$ & ${715}^{+90}_{-59}$      & 1.58 \\
2.00 & ${26.7}^{+13}_{-7.4}$ & ${581}^{+48}_{-35}$      & 3.55
     & ${9.7}^{+2.8}_{-2.1}$ & ${587}^{+26}_{-22}$      & 2.53 \\

\midrule
\multicolumn{7}{c}{Filament 4} \\
\cmidrule{1-7}
{} & \multicolumn{3}{c}{Region 6\tablenotemark{a,b}}
   & \multicolumn{3}{c}{Region 7} \\
\cmidrule(lr){2-4} \cmidrule(l){5-7}
$\mu$ (-) & $\eta_2$ (-) & $B_0$ ($\mu$G) & $\chi^2_{\mt{red}}$
          & $\eta_2$ (-) & $B_0$ ($\mu$G) & $\chi^2_{\mt{red}}$ \\
\cmidrule{1-7}

0.00 & ${690}^{+3100}_{-690}$ & ${810}^{+26}_{-580}$       & 2.19
     & ${16}^{+26}_{-9}$ & ${730}^{+170}_{-120}$           & 21.54 \\
0.33 & ${0.001}^{+0.2}_{-0.001}$ & ${238}^{+3.5}_{-5.4}$   & 2.20
     & ${18}^{+260}_{-13}$ & ${710}^{+580}_{-170}$         & 19.88 \\
0.50 & ${0.001}^{+0.2}_{-0.001}$ & ${237.8}^{+3.5}_{-5.9}$ & 2.20
     & ${6.3}^{+20}_{-3.3}$ & ${560}^{+190}_{-70}$         & 19.74 \\
1.00 & ${0.0}^{+0.14}_{-0.0}$ & ${238.6}^{+3.2}_{-7.1}$        & 2.25
     & ${2.6}^{+1.5}_{-0.8}$ & ${451}^{+31}_{-23}$         & 19.40 \\
1.50 & ${0.0}^{+0.12}_{-0.0}$ & ${238.7}^{+3.0}_{-10.0}$       & 2.25
     & ${2.1}^{+0.8}_{-0.6}$ & ${418}^{+17}_{-13}$         & 19.16 \\
2.00 & ${0.0}^{+0.10}_{-0.0}$ & ${238.7}^{+3.2}_{-11.6}$       & 2.25
     & ${2.1}^{+0.6}_{-0.5}$ & ${403}^{+12}_{-10}$         & 19.04 \\

\midrule
\multicolumn{7}{c}{Filament 5} \\
\cmidrule{1-7}
{} & \multicolumn{3}{c}{Region 8\tablenotemark{a}}
   & \multicolumn{3}{c}{Region 9\tablenotemark{c}} \\
\cmidrule(lr){2-4} \cmidrule(l){5-7}
$\mu$ (-) & $\eta_2$ (-) & $B_0$ ($\mu$G) & $\chi^2_{\mt{red}}$
          & $\eta_2$ (-) & $B_0$ ($\mu$G) & $\chi^2_{\mt{red}}$ \\
\cmidrule{1-7}

0.00 & ${19.7}^{+28}_{-9.8}$ & ${329}^{+67}_{-43}$     & 12.92
     & ${24}^{+241}_{-241}$ & ${730}^{+1540}_{-1540}$  & 86.66 \\
0.33 & ${54}^{+160}_{-39}$ & ${386}^{+140}_{-93}$      & 7.93
     & ${61}^{+990}_{-990}$ & ${840}^{+3000}_{-3000}$       & 76.14 \\
0.50 & ${69}^{+550}_{-57}$ & ${400}^{+250}_{-120}$          & 6.90
     & ${110}^{+1800}_{-1800}$ & ${925}^{+3500}_{-3500}$    & 72.08 \\
1.00 & ${6.6}^{+5.8}_{-2.6}$ & ${223}^{+27}_{-17}$          & 6.46
     & ${400}^{+8000}_{-8000}$ & ${1100}^{+5000}_{-5000}$   & 65.19 \\
1.50 & ${4.2}^{+1.8}_{-1.2}$ & ${197}^{+10}_{-8}$           & 6.24
     & ${28}^{+18}_{-18}$ & ${570}^{+75}_{-75}$             & 64.84 \\
2.00 & ${3.8}^{+1.2}_{-0.9}$ & ${186}^{+6}_{-5}$            & 6.17
     & ${13.6}^{+4.4}_{-4.4}$ & ${470}^{+26}_{-26}$           & 65.12 \\

\bottomrule
\end{tabular}
\tablenotetext{1}{Fits for regions 2, 6, 8 have 2 dof; rest have 3}
\tablenotetext{2}{Reported $\eta_2$ values of $0$ are $\lesssim 10^{-6}$.}
\tablenotetext{3}{Errors are standard errors, not reliable! (correct ones must
be recalculated}

\end{table*}

% NOTE need better subplot display in final version, if we keep this
%\begin{figure}[ht]
%    \centering
%    %\epsscale{0.45}  % For manuscript layout only!
%    \plottwo{figures/fits-sep13/reg01.png}{figures/fits-sep13/reg02.png} \\
%    \plottwo{figures/fits-sep13/reg03.png}{figures/fits-sep13/reg04.png} \\
%    \plottwo{figures/fits-sep13/reg05.png}{figures/fits-sep13/reg06.png} \\
%    \plottwo{figures/fits-sep13/reg07.png}{figures/fits-sep13/reg08.png} \\
%    \plottwo{figures/fits-sep13/reg09.png}{figures/fits-sep13/reg10.png} \\
%    \plottwo{figures/fits-sep13/reg11.png}{figures/fits-sep13/reg12.png} \\
%    \plottwo{figures/fits-sep13/reg13.png}{figures/f0-box.png}
%    \caption{Fitted full model widths as a function of energy for all regions,
%    with measured data; best fit parameters are given in
%    Tables~\ref{tab:fits},~\ref{tab:fits-pt2}.  Plots are ordered by region
%    number.  NOTE: figures for Regions 9, 11 are incorrect.}
%    \label{fig:fits-all}
%\end{figure}

% Individual region fit results
% NOTE keep updated -- for which region are we giving simple and full fits?
Table~\ref{tab:fits} presents full model fit results for all regions
considered.  The first row of Table~\ref{tab:fits} includes both simple and
full model fits for Region 1 for comparison.  We found that fits to the simpler
catastrophic dump model (equation~\eqref{eq:simp-mod}) reproduced qualitative
trends seen in the full continuous energy loss model, as expected.  In general,
simple model fits yielded slightly smaller $\eta_2$ and/or slightly larger
$B_0$ values; this is consistent with the expectation that electrons travel
farther in the simple model than in the full model. Since electrons are not
subjected to continuous synchrotron cooling, smaller $\eta_2$ and larger $B_0$
values are needed to generate thinner rim widths as compared to the full model.
Simple model fits were also more likely to reach manual fitting limits (i.e.,
allow for unconstrained parameter values) than full model fits.

For the rest of our results and discussion, we consider only full model
fit results in Tycho for simplicity.  We do reference the simple model to build
intuition and give scaling relations, as was done in depth by
\citet{parizot2006} and \citetalias{ressler2014}.

% Averaged fit results
In each filament, we averaged best fit parameters $B_0$, $\eta_2$ from each
constituent region to obtain filament-wide estimates on $B_0$ and $\eta_2$,
reported in Table~\ref{tab:fits-avg}.  Stated uncertainties are standard errors
of the mean.  The averaged parameters and corresponding errors are strongly
affected by outliers from extreme best fit values, as can be seen in
Tables~\ref{tab:fits} and \ref{tab:fits-pt2} -- especially as the typical
number of regions per filament is quite small.  We also averaged best fit
parameters from all regions in the remnant, but the presence of outliers
renders the averages relatively unusable.
% TODO
\note[Aaron]{we could also report one fit to the global FWHM averages, as was
done by Sean?}

\begin{table*}[ht]
    \scriptsize
    \centering
    \caption{Filament averages of best-fit parameters for constituent regions
    \label{tab:fits-avg}}
    \begin{tabular}{@{}rr@{ $\pm$ }lr@{ $\pm$ }lr@{ $\pm$ }lr@{ $\pm$ }lr@{ $\pm$ }lr@{ $\pm$ }l@{}}
\toprule
{} & \multicolumn{4}{c}{Filament 1}
   & \multicolumn{4}{c}{Filament 2}
   & \multicolumn{4}{c}{Filament 3} \\
\cmidrule(lr){2-5} \cmidrule(lr){6-9} \cmidrule(lr){10-13}
$\mu$ (-) & \multicolumn{2}{c}{$\eta_2$ (-)} & \multicolumn{2}{c}{$B_0$ ($\mu$G)}
          & \multicolumn{2}{c}{$\eta_2$ (-)} & \multicolumn{2}{c}{$B_0$ ($\mu$G)}
          & \multicolumn{2}{c}{$\eta_2$ (-)} & \multicolumn{2}{c}{$B_0$ ($\mu$G)} \\
\cmidrule{1-13}
0.00 & $15.6$ & $7.6$ & $503.5$ & $176.9$
     & $16.6$ & $2.9$ & $649.3$ & $78.0$
     & $102.3$ & $92.1$ & $696.3$ & $194.8$ \\
0.33 & $47.6$ & $25.4$ & $583.0$ & $236.5$
     & $42.8$ & $9.6$ & $741.5$ & $103.1$
     & $18.5$ & $17.8$ & $499.1$ & $174.1$ \\
0.50 & $105.3$ & $73.1$ & $662.5$ & $314.9$
     & $65.6$ & $16.5$ & $778.4$ & $120.1$
     & $29.0$ & $28.4$ & $516.2$ & $197.2$ \\
1.00 & $8.2$ & $4.8$ & $391.5$ & $134.9$
     & $126.5$ & $74.7$ & $688.3$ & $147.6$
     & $4.6$ & $4.1$ & $389.0$ & $81.1$ \\
1.50 & $3.9$ & $1.7$ & $333.8$ & $95.3$
     & $772.1$ & $763.6$ & $658.7$ & $227.8$
     & $2.3$ & $1.8$ & $350.6$ & $49.2$ \\
2.00 & $3.2$ & $1.2$ & $314.3$ & $84.0$
     & $9.3$ & $4.3$ & $419.9$ & $43.9$
     & $2.0$ & $1.4$ & $336.7$ & $39.3$ \\

\midrule
{} & \multicolumn{4}{c}{Filament 4}
   & \multicolumn{4}{c}{Filament 5}
   & \multicolumn{4}{c}{Global mean} \\
\cmidrule(lr){2-5} \cmidrule(lr){6-9} \cmidrule(lr){10-13}
$\mu$ (-) & \multicolumn{2}{c}{$\eta_2$ (-)} & \multicolumn{2}{c}{$B_0$ ($\mu$G)}
          & \multicolumn{2}{c}{$\eta_2$ (-)} & \multicolumn{2}{c}{$B_0$ ($\mu$G)}
          & \multicolumn{2}{c}{$\eta_2$ (-)} & \multicolumn{2}{c}{$B_0$ ($\mu$G)} \\
\cmidrule{1-13}
0.00 & $15.5$ & $0.4$ & $794.7$ & $41.2$
     & $13.6$ & $3.5$ & $619.6$ & $102.2$
     & $27.3$ & $12.4$ & $655.5$ & $47.9$ \\
0.33 & $43.7$ & $0.8$ & $938.3$ & $54.3$
     & $33.0$ & $13.2$ & $660.9$ & $133.6$
     & $38.1$ & $5.7$ & $704.3$ & $61.8$ \\
0.50 & $164.6$ & $91.9$ & $1159.5$ & $183.0$
     & $94.7$ & $56.7$ & $742.3$ & $169.1$
     & $90.6$ & $23.4$ & $787.9$ & $83.7$ \\
1.00 & $85.6$ & $51.7$ & $834.1$ & $106.0$
     & $538.8$ & $523.4$ & $770.3$ & $292.0$
     & $180.0$ & $119.7$ & $652.2$ & $87.4$ \\
1.50 & $14.0$ & $7.6$ & $561.0$ & $26.2$
     & $6.0$ & $2.6$ & $417.8$ & $59.9$
     & $250.4$ & $243.0$ & $499.9$ & $76.5$ \\
2.00 & $6.9$ & $2.3$ & $487.3$ & $10.3$
     & $3.9$ & $1.5$ & $380.8$ & $46.8$
     & $5.8$ & $1.5$ & $397.5$ & $23.1$ \\
\bottomrule
\end{tabular}
\tablecomments{Reported errors are standard errors of the mean.
Filaments 1, 3 have only 3 samples; Filament 2 has 7; Filament 4, 4; Filament
5, 5.}

\end{table*}

% ==========
% Discussion
% ==========
\section{Discussion}

\subsection{Energy dependence rules out magnetic damping for X-ray rims}

% Set-up the argument.  Claim that B damping is untenable.  Explain the
% necessary observational consequences of B damping
% NOTE keep numbers up to date -- after running Tycho calculations
Our observations of strong energy-dependence, particularly at low energies
(0.7--1 keV, 1--1.7 keV, 2--3 keV), are incompatible with magnetic damping as a
primary remnant-wide control on filament widths.  \note[Aaron]{calculations on
reasonable magnetic damping values of $\mE$ go here}  % TODO FILL IN
\citetalias{ressler2014}, using the same formalism we applied to our model
fits, stated that magnetic damping generally requires $|\mE| \lesssim 0.1$.
Furthermore, if synchrotron rims are magnetically damped at some observation
energy, the rims must be damped at all lower observation energies as well.
Thus damping requires not only $\mE \sim -0.1$, but also that $\mE \sim
-0.1$ must hold for all energies below some threshold damping energy.  Above
this energy, advective or diffusive lengthscales may shrink below the
damping scale length so that synchrotron losses become more restrictive on rim
widths.

% Now turn to our data and say, this is not what we see.
% NOTE keep numbers up to date
We observe the contrapositive: mean $\mE$ values with errors in
Table~\ref{tab:fwhms} favor $|\mE| \gtrsim 0.25$ at $\abt 1 \unit{keV}$ and
$|\mE| \gtrsim 0.19$ at $\abt 2 \unit{keV}$, disqualifying magnetic damping at
energies above $\sim 1 \unit{keV}$.  If thin rims are \emph{not} magnetically
damped at our lowest energy bands, the rims will not be damped at any higher
energies. \note[Aaron]{belaboring the point, can make more concise}
Within the framework of our three-parameter model ($\mu$, $\eta_2$, $B_0$), the
only workaround would be to find $\mu > 1$ diffusion that is able to produce
$\mE > 0$; i.e., if rims widen at higher energies they may run up against any
presumed damping scale length.  But this is unlikely as the electron energy
cut-off shifts $\mE$ slightly negative at high energies, and diffusion only
becomes comparable to or stronger than advection around $E \gtrsim \Ecut$.

% More quantification -- give scale lengths, field strength values.
Elaborate.  How robust is this result, quantitatively?
What is the smallest damping scale length that is (roughly) consistent with all
of our observations?
\note[Aaron]{See Sean's section 3.2.1 and appendix for explication of the
magnetic damping disfavoring etc.}

% Move on to address the broader trend in mE, where energy dependence weakens
The observed decrease in $|\mE|$ with increasing energy may be attributed to
diffusion, which weakens width-energy dependence at higher energies.
As discussed in Section~\ref{sec:energydep}, where advection dominates we
expect $w \sim l_{\mt{ad}} \propto \nu^{-1/2}$ or $\mE \approx -0.5$.  As
diffusion outpaces advection for energies near and above the electron cut-off
energy, $\mE$ will approach values closer to $0$.
The increasing importance of diffusion at X-ray energies matches the
synchrotron spectrum roll-off at X-ray energies as compared to radio
observations; the criterion for synchrotron roll-off, or electron cut-off, is
equivalent to requiring that advective and diffusive lengthscales are
comparable.  Thus, a weakening energy dependence in X-rays is consistent with
prior work.
\note[Aaron]{I think the observation of rim narrowing with decreasing $|\mE|$
is a nice result of its own -- the measurements and $\mE$ don't draw on any
assumptions about the mechanism producing these thin rims.  The observation of
a spectral roll-off depends on the intensity at various energies, how much flux
comes off the rim, whereas this $|\mE|$ "relaxing" is independent of intensity.
But I'm not sure if it's worth noting/discussing in particular.}

% Addressing the variability in width measurements
The large variation in filament widths and point-wise computed $\mE$ values
renders our interpretation of $\mE$ more conjectural.  Multiple regions show both
increases and decreases in filament widths between multiple energy bands (e.g.,
Region 9).  But, our averaged results should mitigate this natural measurement
spread.

\subsection{Modeled diffusion and magnetic field amplification}

% Discuss the degeneracy
We now turn to our model fit parameters: magnetic field strength $B_0$ and
scaled diffusion coefficient $\eta_2$.  Our model fit results are poorly
constrained as the parameters $B_0$ and $\eta_2$ may covary without strongly
altering fit quality.  Equations~\eqref{eq:lad} and $\eqref{eq:ldiff}$
illustrate this clearly.  Varying $B_0$ changes both $l_{\mt{ad}}$ and
$l_{\mt{diff}}$ evenly; varying $\eta_2$ (and hence $\eta$) changes only
$l_{\mt{diff}}$, in principle.  But, if diffusion is the primary control on rim
width -- i.e., $l_{\mt{diff}} > l_{\mt{ad}}$ and thus rim width $w \sim
l_{\mt{diff}}$) -- we see that $\eta_2$ and $B_0$ become almost entirely
degenerate.  Ramping up particle diffusion and hence rim widths can be
perfectly counteracted by increasing magnetic fields and forcing electrons to
cool faster, decreasing rim widths.  If advection is the primary control on rim
width (widths narrow rapidly with energy, at $\mE \sim -0.5$), then $\eta_2$
becomes unimportant and the fit is better behaved with effectively one free
parameter.  The degeneracy of $B_0$ and $\eta_2$ is generally stronger for
smaller values of $\mu$, although there appears no clear trend as to when a
particular fit will be ill-constrained.
\note[Aaron]{a bit wordy}

% B0 lower bounds, etc.
A minimum magnetic field strength $B_0 \gtrsim 179 \muG$, within stated
errors, is required throughout the remnant.  The minimum values of $B_0$ for
each filament (1--5) are $179$, $254$, $287$, $458$, and $243 \muG$.
We observe the smallest $B_0$ values for $\eta_2$ approaching $0$ and $\mu =
2$; intuitively, $\mu = 2$ favors diffusion at higher energies ($>2\unit{keV}$)
where it is most ``needed'' to reproduce the leveling off of filament widths
and so $\mu = 2$ permits a smaller best-fit $\eta_2$ and hence smaller $B_0$.
\note[Aaron]{$\mu=2$ bit not very important}
Our minimum values of $B_0$ are consistent with prior estimates of
$\abt 200$--$300 \muG$ for primarily advective transport
\citep{volk2005, parizot2006, morlino2012} \note[Aaron]{could elaborate and
review previous work; deferring for now}.  A larger range of values has been
obtained ($130$--$500 \muG$) when considering additional effects, such as
varying compression ratio and nonlinear diffusion-energy scaling
\citep{parizot2006, cassam-chenai2007}.

Consistent with previous studies, Tycho requires strong magnetic field
amplification -- at least $\abt 20\times$ the expected value of $\abt 10 \muG$
from a strong shock with compression ratio $r=4$ and typical galactic magnetic
field strength of $\abt 2$--$3 \muG$ \citep{lyne1989, han2006}.

% More detail on relevant B0 estimates follows

% \citet{parizot2006} give $200 \muG$ for pure advection and $230 \muG$ for pure
% diffusion; with both effects, they obtain $300$--$530 \muG$ depending on the
% assumed compression ratio and diffusion-energy scaling $\mu$.
% \citet{volk2005} combine a prediction from nonlinear acceleration with actual
% data.  They too suggest something like $300 \muG$.
% \citet{volk2002} give $240 \muG$ from a multiwavelength study.  Too many papers
% from Russian theorists, seriously.
% \citet{morlino2012} do a similar thing with newer data and favor $B_0 > 200
% \muG$ strong magnetic fields -- it also is favored by the hadronic scenario of
% pion decay to help explain gamma-ray observations.
% \citet{cassam-chenai2007} take values $215 \muG$ and $130 \muG$ for different
% particle injection efficiencies to describe the X-ray rims of Tycho; note that
% they find an overall compression ratio $r \approx 6$ and treat the effect of CR
% acceleration and escape on the shock hydrodynamics.

% \citet{rettig2012} used a measured filament lengthscale and considered
% loss-limited and magnetic damping models separately for several historical
% SNRs.  For Tycho: they report $B = 310 \muG$ with $\Ecut = 24
% \unit{TeV}$; since $\Ecut$ is set by balancing advective/diffusive
% lengthscales, this is equivalent to our reporting of a diffusion coefficient
% (?).  With magnetic damping, they report $B = 150 \muG$ and a larger
% $\Ecut = 34 \unit{TeV}$.  Taken from \citet{rettig2012}: \citet{acciari2011}
% give $80$, $230 \muG$ for leptonic/hadronic models for gamma-ray
% emission in Tycho.

% 1-2 paragraphs on eta2
The values of $\eta_2$ in Tables~\ref{tab:fits} and \ref{tab:fits-pt2} are very
poorly constrained.  We observe no clear trend of $\eta_2$ with $\mu$, and the
values and error bounds on $\eta_2$ are often unreasonably large.  The extreme
outliers limit the usefulness of the average fit parameters presented in
Table~\ref{tab:fits-avg}.

Our model fits do generally favor $\eta_2 \geq 1$, respecting the Bohm limit on
diffusion.  The lowest error bound on $\eta_2$ in the averaged best-fit values
is $1.2$, with minimum error bounds in each filament of order $1$--$10$.  For
individual regions' best fits, only Regions 5 and 18 have best-fit $\eta_2$
values falling strictly below $\eta_2 = 1$.  \note[Aaron]{I think Region 5 is
skewed by the unusually large 0.7--1 keV FWHM.  Region 18 is a weird place, a
smushed-out and somewhat faint filament in the NE where Bohm just bombs and rim
widths narrow strongly.}

% A quick discussion of mu
Our fits at multiple values of $\mu$ are not especially insightful either.
The $\chi^2_{\mt{red}}$ values for individual fits are variable and quite
large, so they may not be reliable for favoring or disfavoring fits at
different $\mu$ values.  But, if we ignore this caveat, we see that values
of $\mu \geq 1$ are qualitatively favored by $\chi^2_{\mt{red}}$ in almost all
regions.  This trend may be an partially an artifact of the correlation between
$B_0$ and $\eta_2$ as they are not entirely independent parameters, but they
are less correlated for larger $\mu$.  However, our computed
$\chi^2_{\mt{red}}$ values have simply assigned a single degree of freedom to
each parameter.  \citetalias{ressler2014} also found better model fits for $\mu
\geq 1$ in the remnant of SN 1006 and noted that \citet{reynolds2004} favored
$\mu \geq 1$ on grounds of better modeled remnant morphology, but it remains
uncertain whether $\mu \geq 1$ diffusion is truly preferred.

% Data quality
We emphasize that our measurements are not data limited, as additional counts
from averaging measurements or selecting larger regions will not improve our
ability to constrain $B_0$ and $\eta_2$.  This is easily seen from reduced
$\chi^2$ values in Tables~\ref{tab:fits} and \ref{tab:fits-pt2}, which
reflect relatively tight errors in our FWHM profile measurements
(Table~\ref{tab:fwhms}).


\subsection{Tactics for constraining diffusion}

A major independent observable is the synchrotron roll-off frequency.  If
Bohm-like ($\mu = 1$) diffusion holds, the synchrotron roll-off is
independent of magnetic field strength and thus grants an external
constraint on diffusion coefficient.
In this manner, \citet{parizot2006} estimated diffusion coefficients
$\eta = 10, 5$ for compression ratios $4, 10$ in Tycho.

Speculative: we could pull out (or ourselves measure) the roll-off frequency
from \texttt{srcut} around the remnant, then estimate a diffusion coefficient
in each region and then get single parameter fits for magnetic field strengths?

\subsection{Tycho distance estimates}

In our modeling we adopted a distance $d$ to Tycho of $3 \unit{kpc}$, but
various estimates for Tycho's distance range between $2.3$--$4 \unit{kpc}$
\citep{hayato2010}.  A larger remnant distance would increase both physical
filament widths and shock velocity estimates from proper motion (as used in our
models).

% lad/ldiff scalings give eta/B0 scalings
For both simple and full models, we find roughly that $\eta_2 \propto d^2$
while $B_0$ is approximately constant.  The advective lengthscale,
equation~\eqref{eq:lad}, may be rearranged to obtain:
\begin{equation}
    B_0 = (317 \muG) \left(\frac{v_d}{10^8 \unit{cm/s}}\right)^{2/3}
                     \left(\frac{l_{\mt{ad}}}{0.01 \unit{kpc}}\right)^{-2/3}
                     \left(\frac{h\nu}{1 \unit{keV}}\right)^{-1/3}
\end{equation}
or, more simply,
\begin{equation}
    B_0 \propto \left(v_d\right)^{2/3}
                \left(l_{\mt{ad}}\right)^{-2/3} \nu^{-1/3} .
\end{equation}
Both $l_{\mt{ad}}$ and $v_d$ scale linearly with remnant distance $d$ and thus
their effects cancel in determining the magnetic field.  If diffusion is the
primary control on filament lengthscales, equation~\eqref{eq:ldiff} yields:
\begin{equation}
    \eta \propto \left(l_{\mt{diff}}\right)^2 B_0^{3} \nu^{-(\mu - 1)/2}
\end{equation}
Recall that $\eta \propto \eta_2$ as $E_h = E_2$ is constant, from
equation~\eqref{eq:diffcoeff}.
Similar results, primarily for magnetic field $B_0$, were previously given by
\citet{parizot2006}.

In practice, both advection and diffusion are important to our filament widths,
but we concurrently observe the scalings $\eta_2 \propto d^2$ and $B_0$
constant in model fits with varying distance.  When comparing full model fits
with remnant distances of $3 \unit{kpc}$ and $4 \unit{kpc}$, the deviation from
the idealized scaling is $\lesssim 1 \%$ for $B_0$ and $\abt 1$--$5\%$ for
$\eta_2$.

One explanation is that as varying remnant distance $d$ leaves the width-energy
scaling $\mE$ invariant, the relative contributions of $l_{\mt{ad}}$ and
$l_{\mt{diff}}$ should also be invariant.  Then both lengthscales should
scale simultaneously with $d$, yielding the observed behavior.
\note[Aaron]{need a logic check here}

\subsection{Nonlinear DSA, CR-modified shocks, precursors (misc things)}

Compared to our baseline assumptions for a standard strong shock ($r=4$) with
constant magnetic field advected downstream, a treatment that permits higher
compression ratios (from whatever instabilities or what have you) will
generally permit weaker magnetic fields.  The reason for this, I think
(speculating at this point due to lack of knowledge) lies in the resulting
lower plasma velocity, which depresses the advective length scale and permits
thinner rims more naturally (assuming diffusion not too strong).

Could the gradual increase of compression ratio leading up to the shock matter,
if CR streaming is important?  Maybe strengthen a diffusive precursor?

Electron energy spectrum functional form by \citet{zirakashvili2007}, see also
\citet{morlino2009}.

Rim position variation (none obvious from preliminary look), steepness of the
filament rise?
% see Figure~\ref{fig:peak-pos}

%\begin{figure}[ht]
%    \centering
%    \plotone{figures/f0-peak-pos.pdf}
%    \caption{Best-fit rim peak positions ($x_0$ in equation~\eqref{eq:prof})
%        for all regions as a function of energy band, normalized to the $2
%        \unit{keV}$ peak position.  Red line plots best linear fit to all data
%        with slope of $-0.025 \unit{arcsec/keV}$.
%        \label{fig:peak-pos}}
%\end{figure}


% ==========
% Conclusion
% ==========
\section{Conclusions}

A priori, we are confident that the rims are nonthermal and meaningful proxies
for remnant magnetic fields and diffusion, as evidenced by our spectral fits
and region selections.
A posteriori, our multiple measurements when modeled (without damping) give
diffusion and magnetic fields consistent with those derived by considering
other observables, specifically synchrotron roll-off as derived from radio flux
and spectral modeling from gamma ray data.

% Is rim narrowing universal?  Further work on more remnants
From these checks, and the knowledge that our data are not count limited, we
argue that the observation of rim narrowing in Tycho is sufficient to
rule out magnetic field damping at the lengthscales of observed X-ray rims.
Our observations also corroborate rim narrowing as observed by
\citet{ressler2014} in the remnant of SN 1006.  We will investigate thin rims
in Kepler's SNR and Cas A \citep[but cf.][]{araya2010} as well.  If significant
remnant narrowing is a common feature in young supernova remnants, strong
magnetic fields must be sustained some distance behind the forward shock -- and
this will need to be understood or characterized (the question becomes, why
aren't various putative damping mechanisms relevant to SNRs?).
\note[Aaron]{I am tempted to toss in Kepler results, to prevent duplication of
text/exposition and maybe start making some case for loss-limited rims being a
consistent observation.  That would be a lot of tables, though.}

% Modeling limitations
Filaments widths, as might be expected, do not match model predictions too
consistently.  This may be attributed in part to spherical cow assumptions on
Tycho's nonthermal emission and shock structure, as well as its surrounding
medium.  Streaming and diffusing particles may modify the shock structure;
fluid instabilities and ISM interactions may induce wave-like instabilities in
the forward shock's projected shape; multiple and fragmented filaments
\citep{caprioli2013},
even in relatively ``clean'' remnants of Type Ia SNe such as SN 1006, Tycho's
SNR, and Kepler's SNR. \note[Aaron]{needs fact-check and reading}

% Further work -- radio
As has been noted by many workers, observations of radio synchrotron rims would
provide strong and immediate evidence for magnetic field damping (at the
lengthscales of said radio rims).  Previous workers have certainly tried
\citep{cassam-chenai2007, morlino2012} but it seems not easy.
A new epoch of VLA and \Chandra observations in this year (PIs: J. Hewitt, B.
Williams) will offer a rich dataset of sub-arcsecond observations that may
reveal more about rim morphology and perhaps even temporal variation over a 10+
year baseline (compare to \citet{katsuda2010-sn1006}).

% ================
% Acknowledgements
% ================
\acknowledgments

A.T. thanks CRESST for support (or, any relevant grants?)
B.J.W. NPP?
The scientific results reported in this article are based on data obtained from
the \Chandra Data Archive.
This research has made use of NASA's Astrophysics Data System.

{\it Facilities:} \facility{CXO (ACIS-I)}

% ========
% Appendix
% ========
\clearpage
\appendix

\setcounter{table}{0}
\renewcommand{\thetable}{A\arabic{table}}
\setcounter{figure}{0}
\renewcommand{\thefigure}{A\arabic{figure}}

\section{Derivation for synchrotron roll-off and diffusion}

% This is NOT optimized for 2-column layout

Assume that a DSA model explains the observed e-/synchrotron spectra
cut-off/roll-off.  We estimate e- cut-off energy by equating
$\tau_{\mt{accel}} = \tau_{\mt{synch}}$.  This gives (Parizot et al., 2006):
\[
    \frac{3r}{r-1} \frac{r D_d + D_u}{v_s^2} = \frac{1}{b B^2 \Ecut}
\]
(\emph{I don't know if this equality was designed to exactly obtain the electron
cutoff energy, or is just a rough/scaling relation -- pull up Drury (1983) to
check this})
Following through the arguments/assumptions in Parizot et al. (2006) yields
(using $13.3 \unit{erg} = 8.3 \unit{TeV}$):
\[
    \Ecut =
        \left( 13.3 \unit{erg} \right)^{\frac{2}{1+\mu}}
        \left( \frac{B_0}{100 \muG} \right)^{-\frac{1}{1+\mu}}
        \left( \frac{v_s}{10^8 \unit{cm\;s^{-1}}} \right)^{\frac{2}{1+\mu}}
        \eta^{-\frac{1}{1+\mu}}
\]
Now, we can convert this cut-off e- energy to a roll-off synchrotron energy by
invoking the delta function assumption $\nu_{\mt{cut}} = c_m \Ecut^2 B$ with
$c_m = 1.822 \times 10^{18}$.
(\emph{Again, I don't know if this is an equality or just a scaling.  We'd need
to relate the electron spectrum cut-off to the derived synchrotron spectrum
cut-off, as is done (I think) by Zirakashvili and Aharonian (2007).})
\[
    \nu_{\mt{cut}} = c_m
        \left( 13.3 \unit{erg} \right)^{\frac{4}{1+\mu}}
        \left( \frac{B}{100\muG} \right)^{-\frac{1-\mu}{1+\mu}}
        \left( \frac{v_s}{10^8 \unit{cm/s}} \right)^{\frac{4}{1+\mu}}
        \eta^{-\frac{2}{1+\mu}}
\]
In the case of $\mu = 1$ this yields:
\[
    \nu_{\mt{cut}} = (0.133 \unit{keV} / h)
        \left( \frac{v_s}{10^8 \unit{cm/s}} \right)^{2}
        \eta^{-1}
\]
For $\mu \neq 1$ this is no longer independent of magnetic field.  But,
there is hope.  Rewrite this in terms of $\eta_2$ using
$\eta = \eta_2 E_2^{1-\mu}$ and write:
\[
    \nu_{\mt{cut}} = c_m
        \left( 13.3 \unit{erg} \right)^{\frac{4}{1+\mu}}
        \left( E_2 \right)^{-\frac{2(1-\mu)}{1+\mu}}
        \left( \frac{B}{100\muG} \right)^{-\frac{1-\mu}{1+\mu}}
        \left( \frac{v_s}{10^8 \unit{cm/s}} \right)^{\frac{4}{1+\mu}}
        \left( \eta_2 \right)^{-\frac{2}{1+\mu}}
\]
With $E_2 = \left( (2 \unit{keV}/h) / (c_m B) \right)^{1/2}$, or
$E_2 = \left( 0.2657 \unit{erg^2\;G} / B \right)^{1/2}$, we obtain:
\[
    \nu_{\mt{cut}} = c_m
        \left( 13.3 \unit{erg} \right)^{\frac{4}{1+\mu}}
        \left( \frac{0.2657 \unit{erg^2\;G}}{B} \right)^{-\frac{1-\mu}{1+\mu}}
        \left( \frac{B}{100\muG} \right)^{-\frac{1-\mu}{1+\mu}}
        \left( \frac{v_s}{10^8 \unit{cm/s}} \right)^{\frac{4}{1+\mu}}
        \left( \eta_2 \right)^{-\frac{2}{1+\mu}}
\]
And, the magnetic field terms cancel!  Hey presto.
Our final usable result is:
\[
    \nu_{\mt{cut}} = c_m
        \left( 13.3 \unit{erg} \right)^{\frac{4}{1+\mu}}
        \left( 0.2657 \unit{erg^2\;G} \right)^{-\frac{1-\mu}{1+\mu}}
        \left( 100 \muG \right)^{\frac{1-\mu}{1+\mu}}
        \left( \frac{v_s}{10^8 \unit{cm/s}} \right)^{\frac{4}{1+\mu}}
        \left( \eta_2 \right)^{-\frac{2}{1+\mu}}
\]

% --------------
% SN 1006 tables
% --------------
%\clearpage
%\section{Full model validation, SN 1006 (DRAFT ONLY)}
%
%For comparison to \citetalias{ressler2014}, we reproduce Sean's full model fit
%table (Table~\ref{tab:sean}) and present full model fits with 2 and 3 energy
%bands (Tables~\ref{tab:sn1006-2band} and \ref{tab:sn1006-3band}, respectively).
%We don't compare simple model results, as they are identical (only the error
%calculations differ).
%The procedures of Tables~\ref{tab:sean},~\ref{tab:sn1006-2band} are not quite
%the same.  Sean used $\mE$ at 2 keV instead of the $1$--$2$ keV width for
%fitting, whereas I used both widths.  But, it should not matter much as
%the fit often has $\chi^2 \sim 0$, either way (depending on the exact
%measurements and values of $\mu$).
%
%Please take the errors with a grain of salt.  They are automatically
%generated and \textbf{have not been manually validated}.  Some values may be
%invalid, where my error-finding code failed and gave a best, conservative guess
%of the error.
%
%\begin{table}[ht]
%    \tiny
%    \centering
%    \caption{Sean's SN 1006 best fit parameters \citepalias[Table 8]{ressler2014}.
%    \label{tab:sean}}
%    \begin{tabular}{@{} l c c c c c c @{}}
\toprule
{}&\multicolumn{2}{c}{Filament 1} & \multicolumn{2}{c}{Filament 2} & \multicolumn{2}{c}{Filament 3} \\
\midrule
$\mu$ &$\eta_{2}$ & $B_{0}$ &$\eta_{2}$ & $B_{0}$ & $\eta_{2}$ & $B_{0}$ \\
0 & 7.5 $\pm$ 2 & 142 $\pm$ 5 & - & - & $\lesssim$ 0.1& 77 $\pm$ .8\\
1/3 & 4 $\pm$ 1.3 & 120 $\pm$ 5 & - & - & $\lesssim$ 0.1 & 76 $\pm$ 1.4 \\
1/2 & 3 $\pm$ 1.1 & 112 $\pm$ 4 & - & - & $\lesssim$ 0.1 & 75 $\pm$ 1.0 \\
1 & 2 $\pm$ 1.0 & 100 $\pm$ 3 & 22 $\pm$ 3 & 214 $\pm$ 4 & $\lesssim$ 0.1 & 74 $\pm$ 1.1 \\
1.5 & 1.9 $\pm$ 1.2 & 95 $\pm$ 3 & 9 $\pm$ 1.2 & 167 $\pm$ 4 & $\lesssim$ 0.1 & 74 $\pm$ 1.2   \\
2 & 2 $\pm$ 1.0 & 92 $\pm$ 4  & 7 $\pm$ 1.1 & 152 $\pm$ 4 & $\lesssim$ 0.1 & 73 $\pm$ 1.2 \\
\midrule
& &\multicolumn{2}{c}{Filament 4} && \multicolumn{2}{c}{Filament 5} \\
\midrule
$\mu$ &&$\eta_{2}$ & $B_{0}$ &&$\eta_{2}$ & $B_{0}$\\
0 &&  $\lesssim$ 0.2 & 113 $\pm$ 2 && -& -\\
1/3 &&  $\lesssim$ 0.2 & 112 $\pm$2 && -& -\\
1/2 && $\lesssim$ 0.2  & 111 $\pm$2 && - & -\\
1 &&  $\lesssim$ 0.2 & 109 $\pm$ 2 && 80 $^{+\infty}_{-4}$   & 206 $\pm$ 3\\
1.5 && $\lesssim$ 0.2 & 108 $\pm$ 2 && 19 $\pm$ 2  & 140 $\pm$ 2\\
2 && $\lesssim$ 0.2 & 107 $\pm$ 2 && 12 $\pm$ 1.0  & 120 $\pm$ 2\\
\bottomrule
\end{tabular}

%\end{table}
%
%\begin{table*}[ht]
%    \scriptsize
%    \centering
%    \caption{SN 1006 best fit parameters, 2 highest energy bands (full model).
%    \label{tab:sn1006-2band}}
%    %\renewcommand{\arraystretch}{1.5}
\begin{tabular}{@{}rllr llr llr @{}}

\toprule
{} & \multicolumn{3}{c}{Filament 1} & \multicolumn{3}{c}{Filament 2} &
     \multicolumn{3}{c}{Filament 3} \\
\cmidrule(lr){2-4}
\cmidrule(lr){5-7}
\cmidrule(lr){8-10}
$\mu$ (-) & $\eta_2$ (-) & $B_0$ ($\mu$G) & $\chi^2$
          & $\eta_2$ (-) & $B_0$ ($\mu$G) & $\chi^2$
          & $\eta_2$ (-) & $B_0$ ($\mu$G) & $\chi^2$ \\

\midrule
0.00 & ${21}^{\,+290}_{\,-20}$ & ${176}^{\,+108}_{\,-80}$ & 0.0086
     & ${21}^{\,+51}_{\,-14}$ & ${262}^{\,+77}_{\,-53}$ & 7.2668
     & ${0.004}^{\,+6800}_{\,-0.004}$ & ${74.75}^{\,+183}_{\,-0.77}$ & 0.8180\\[1.5pt]
0.33 & ${3.4}^{\,+100000}_{\,-3}$ & ${117}^{\,+552}_{\,-24}$ & 0.0000
     & ${68}^{\,+282}_{\,-56}$ & ${316}^{\,+136}_{\,-98}$ & 2.5848
     & ${0.02}^{\,+0.11}_{\,-0.02}$ & ${74.45}^{\,+1.04}_{\,-0.61}$ & 0.8037\\[1.5pt]
0.50 & ${2.5}^{\,+100000}_{\,-2.2}$ & ${109}^{\,+635}_{\,-17}$ & 0.0000
     & ${106}^{\,+792}_{\,-93}$ & ${337}^{\,+204}_{\,-122}$ & 1.0919
     & ${0.024}^{\,+0.11}_{\,-0.024}$ & ${74.16}^{\,+0.87}_{\,-0.53}$ & 0.7946\\[1.5pt]
1.00 & ${1.8}^{\,+6.6}_{\,-1.4}$ & ${98.4}^{\,+23}_{\,-9.2}$ & 0.0000
     & ${21}^{\,+100000}_{\,-15}$ & ${213}^{\,+1160}_{\,-42}$ & 0.0000
     & ${0.012}^{\,+0.11}_{\,-0.012}$ & ${73.68}^{\,+0.21}_{\,-0.88}$ & 0.5364\\[1.5pt]
1.50 & ${1.7}^{\,+3.6}_{\,-1.3}$ & ${94}^{\,+12}_{\,-6.4}$ & 0.0000
     & ${8.5}^{\,+11}_{\,-4.1}$ & ${166}^{\,+24}_{\,-14}$ & 0.0000
     & ${0.007}^{\,+0.13}_{\,-0.007}$ &
     $\left({73.4}^{\,+0}_{\,-1.2}\right)$\tablenotemark{a} & 0.5062\\[1.5pt]
2.00 & ${1.9}^{\,+3.1}_{\,-1.4}$ & ${91.6}^{\,+7.8}_{\,-4.9}$ & 0.0000
     & ${7.0}^{\,+4.9}_{\,-2.8}$ & ${152.4}^{\,+11}_{\,-7.7}$ & 0.0000
     & ${0.026}^{\,+0.10}_{\,-0.026}$ & ${72.21}^{\,+0.23}_{\,-0.53}$ & 0.4443\\

\midrule
{} & \multicolumn{3}{c}{Filament 4} & \multicolumn{3}{c}{Filament 5} \\
\cmidrule(lr){2-4}
\cmidrule(lr){5-7}
$\mu$ (-) & $\eta_2$ (-) & $B_0$ ($\mu$G) & $\chi^2$
          & $\eta_2$ (-) & $B_0$ ($\mu$G) & $\chi^2$ \\

\cmidrule(lr){1-7}
0.00 & ${4300}^{\,+2500}_{\,-4300}$ & ${374.7}^{\,+3.7}_{\,-266}$ & 0.1896
     & ${23}^{\,+33}_{\,-13}$ & ${192}^{\,+40}_{\,-31}$ & 17.9802\\[1.5pt]
0.33 & ${0.011}^{\,+0.16}_{\,-0.011}$ & ${109.2}^{\,+2.1}_{\,-0.9}$ & 0.2605
     & ${68}^{\,+167}_{\,-48}$ & ${230}^{\,+72}_{\,-54}$ & 8.7407\\[1.5pt]
0.50 & ${0.0025}^{\,+0.17}_{\,-0.0025}$ & ${109.2}^{\,+1.2}_{\,-1.3}$ & 0.1959
     & ${107}^{\,+416}_{\,-82}$ & ${246}^{\,+104}_{\,-67}$ & 5.3185\\[1.5pt]
1.00 & ${0.041}^{\,+0.15}_{\,-0.041}$ & ${107.48}^{\,+1.2}_{\,-0.72}$ & 0.2759
     & ${447}^{\,+16000}_{\,-424}$ & ${311}^{\,+389}_{\,-151}$ & 0.1152\\[1.5pt]
1.50 & ${0.023}^{\,+0.17}_{\,-0.023}$ & ${106.8}^{\,+0.54}_{\,-0.41}$ & 0.1114
     & ${20}^{\,+49}_{\,-10}$ & ${142}^{\,+42}_{\,-18}$ & 0.0000\\[1.5pt]
2.00 & ${0.013}^{\,+0.18}_{\,-0.013}$ & ${106.57}^{\,+0.33}_{\,-1.4}$ & 0.0457
     & ${12.1}^{\,+9.2}_{\,-4.7}$ & ${121.6}^{\,+12}_{\,-8.5}$ & 0.0000\\

\bottomrule
\end{tabular}
\tablenotetext{1}{Seems unlikely that error is $0$, probably bad calculation}

%\end{table*}
%
%\begin{table*}[ht]
%    \scriptsize
%    \centering
%    \caption{SN 1006 best fit parameters, 3 energy bands (full model).
%    \label{tab:sn1006-3band}}
%    %\renewcommand{\arraystretch}{1.5}
\begin{tabular}{@{}rllr llr llr @{}}

\toprule
{} & \multicolumn{3}{c}{Filament 1} & \multicolumn{3}{c}{Filament 2} &
     \multicolumn{3}{c}{Filament 3} \\
\cmidrule(lr){2-4}
\cmidrule(lr){5-7}
\cmidrule(lr){8-10}
$\mu$ (-) & $\eta_2$ (-) & $B_0$ ($\mu$G) & $\chi^2$
          & $\eta_2$ (-) & $B_0$ ($\mu$G) & $\chi^2$
          & $\eta_2$ (-) & $B_0$ ($\mu$G) & $\chi^2$ \\

\midrule
0.00 & $25^{\,+170}_{\,-23}$ & ${183}^{\,+89}_{\,-70}$ & 0.1162
     & $\left({0.00}^{\,+0.26}_{\,-0.00}\right)$\tablenotemark{a} & ${132.18}^{\,+320}_{\,-0.72}$ & 53.8334
     & ${0.012}^{\,+0.18}_{\,-0.012}$ & ${75.25}^{\,+180}_{\,-0.9}$ & 1.7278\\[1.5pt]
0.33 & $730^{\,+6100}_{\,-730}$ & $350^{\,+130}_{\,-130}$ & 0.0102
     & ${0.082}^{\,+0.23}_{\,-0.082}$ & ${132.5}^{\,+3.5}_{\,-1}$ & 53.4559
     & ${0.01}^{\,+0.18}_{\,-0.01}$ & ${74.86}^{\,+1.4}_{\,-0.65}$ & 1.7621\\[1.5pt]
0.50 & $3.9^{\,+210}_{\,-2.9}$ & ${116}^{\,+630}_{\,-18}$ & 0.0741
     & ${0.043}^{\,+0.31}_{\,-0.043}$ & ${131.56}^{\,+3.8}_{\,-0.62}$ & 53.3622
     & ${0.0056}^{\,+0.16}_{\,-0.0056}$ & ${74.61}^{\,+0.72}_{\,-0.48}$ & 1.5513\\[1.5pt]
1.00 & ${2.6}^{\,+5.4}_{\,-1.7}$ & ${102.5}^{\,+18}_{\,-8.9}$ & 0.1420
     & ${0.18}^{\,+0.27}_{\,-0.17}$ & ${130.5}^{\,+2.3}_{\,-1}$ & 52.4123
     & ${0.016}^{\,+0.18}_{\,-0.016}$ & ${73.85}^{\,+0.72}_{\,-0.54}$ & 1.7703\\[1.5pt]
1.50 & ${2.5}^{\,+3.2}_{\,-1.5}$ & ${97.1}^{\,+9.1}_{\,-5.9}$ & 0.2221
     & ${0.2}^{\,+0.45}_{\,-0.18}$ & ${129.06}^{\,+2.6}_{\,-0.82}$ & 51.8975
     & ${0.01}^{\,+0.16}_{\,-0.01}$ & ${73.47}^{\,+0.68}_{\,-0.9}$ & 1.4024\\[1.5pt]
2.00 & ${2.7}^{\,+2.9}_{\,-1.6}$ & ${94.2}^{\,+6.2}_{\,-4.6}$ & 0.3025
     & ${0.53}^{\,+0.34}_{\,-0.44}$ & ${128.9}^{\,+1.6}_{\,-1.7}$ & 50.3784
     & ${0.015}^{\,+0.17}_{\,-0.015}$ & ${73.2741}^{\,+0.0092}_{\,-1.1}$ & 1.6535\\

\midrule
{} & \multicolumn{3}{c}{Filament 4} & \multicolumn{3}{c}{Filament 5} \\
\cmidrule(lr){2-4}
\cmidrule(lr){5-7}
$\mu$ (-) & $\eta_2$ (-) & $B_0$ ($\mu$G) & $\chi^2$
          & $\eta_2$ (-) & $B_0$ ($\mu$G) & $\chi^2$ \\

\cmidrule(lr){1-7}
0.00 & - & - & -
     & ${23}^{\,+34}_{\,-13}$ & ${192}^{\,+40}_{\,-31}$ & 18.1971 \\[1.5pt]
0.33 & - & - & -
     & ${61}^{\,+190}_{\,-43}$ & ${223}^{\,+80}_{\,-53}$ & 9.7081 \\[1.5pt]
0.50 & - & - & -
     & ${101}^{\,+470}_{\,-80}$ & ${242}^{\,+114}_{\,-71}$ & 6.8545 \\[1.5pt]
1.00 & - & - & -
     & ${35}^{\,+45000}_{\,-25}$ & ${174}^{\,+680}_{\,-39}$ & 3.7991 \\[1.5pt]
1.50 & - & - & -
     & ${10.9}^{\,+12}_{\,-4.7}$ & ${127}^{\,+18}_{\,-11}$ & 3.2257 \\[1.5pt]
2.00 & - & - & -
     & ${8.8}^{\,+5.1}_{\,-3}$ & ${115.4}^{\,+8.5}_{\,-6.3}$ & 2.7391 \\

\bottomrule
\end{tabular}
\tablenotetext{1}{Best fit value is formally $6.6\times10^{-6}$, close enough.}

%\end{table*}

% ==========
% References
% ==========
\bibliographystyle{apj}  % AASTeX journal macros are supplied in ADS entries
\bibliography{refs-snr}

%\end{CJK}
\end{document}
