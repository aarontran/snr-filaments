%\documentclass[manuscript]{aastex}  % one-column, double-spaced GENERATE BIB
\documentclass[iop, apj, numberedappendix]{emulateapj}

%\documentclass[12pt,preprint]{aastex}  % one-column, single-spaced
%\documentclass[iop, apj, numberedappendix, twocolappendix]{emulateapj}

% Customize figures/sizing for 2 column vs. manuscript printout
\usepackage{etoolbox}
\newtoggle{manuscript}
%\toggletrue{manuscript}
\togglefalse{manuscript}

\shorttitle{Tycho's Synchrotron Rims (\today)}  % <~ 44 char
\shortauthors{XXX et al. (\today)}  % Max three
\slugcomment{Draft, \today}  % short title pg comment

%% ==================================================================== %%
%% README for track changes                                             %%
%% To add/remove text or add comments, use the following commands:      %%
%%                                                                      %%
%%       \note[editor]{The note}                                        %%
%%     \annote[editor]{Text to annotate}{The note}                      %%
%%        \add[editor]{Text to add}                                     %%
%%     \remove[editor]{Text to remove}                                  %%
%%     \change[editor]{Text to remove}{Text to add}                     %%
%%                                                                      %%
%% ==================================================================== %%

\usepackage[inline]{trackchanges}  % trackchanges.sourceforge.net
\addeditor{Rob}
\addeditor{Sean}
\addeditor{Steve}
\addeditor{Aaron}
\addeditor{Brian}

\usepackage{amsmath}  % amsthm, amssymb
\usepackage{CJK}  % aas.org/authors/author-names-non-roman-alphabets
\usepackage{booktabs}
\usepackage{hyperref}

\newcommand*{\mt}{\mathrm}
\newcommand*{\unit}[1]{\;\mt{#1}}  % vemod.net/typesetting-units-in-latex
\newcommand*{\abt}{\mathord{\sim}} % tex.stackexchange.com/q/55701
\newcommand*{\ptl}{\partial}
\newcommand*{\del}{\nabla}

% This paper
\newcommand*\mean[1]{\bar{#1}}
\renewcommand{\vec}[1]{\mathbf{#1}}
\newcommand*{\tsup}{\textsuperscript}

\defcitealias{ressler2014}{R14}
\newcommand*{\Chandra}{\textit{Chandra}\ }
\newcommand*{\tsynch}{\tau_{\mt{synch}}}
\newcommand*{\mE}{m_\mt{E}}
\newcommand*{\Ecut}{E_{\mt{cut}}}
\newcommand*{\Bmin}{B_{\mt{min}}}
\newcommand*{\muG}{\unit{\mu G}}

\begin{document}
%\begin{CJK}{UTF8}{gbsn}

\title{Energy Dependence of Synchrotron X-Ray Rims in Tycho's Supernova
Remnant}

\author{
Aaron Tran \altaffilmark{1,4},
% 陳 doesn't display?
% Also breaks emulateapj bibliography... toggle CJK env w/ AASTeX only
%Aaron Tran (陈宏裕)\altaffilmark{1,4},
Brian J. Williams\altaffilmark{1,5},
Robert Petre\altaffilmark{1},
Sean M. Ressler\altaffilmark{2},
Stephen P. Reynolds\altaffilmark{3}
}

\affil{
\tsup{1}NASA Goddard Space Flight Center, Greenbelt, MD 20771, USA \\
\tsup{2}Dept. Physics, University of California, Berkeley, CA 94720, USA \\
\tsup{3}Dept. Physics, North Carolina State University, Raleigh, NC 27695, USA
}

\altaffiltext{4}{CRESST, University of Maryland, College Park, MD 20742}
\altaffiltext{5}{NASA Postdoctoral Program Fellow}

\begin{abstract}
Several young supernova remnants exhibit thin X-ray bright rims of synchrotron
radiation at their forward shocks.  Thin rims require strong magnetic field
amplification if rim widths are limited by shock-accelerated electron energy
losses.  But, magnetic field damping immediately behind the shock could produce
similarly thin rims.  We measured rim widths around Tycho's supernova remnant
in 5 energy bands using an archival $750 \unit{ks}$ \Chandra observation.  Rims
consistently narrow with increasing energy and are well-described by either
loss-limited or magnetically damped scenarios, hence X-ray rim widths alone
cannot discriminate between models.  Loss-limited rims require amplified
magnetic fields of $\abt200$--$1000 \muG$ and particle diffusion coefficients
$\abt1$--$100\times$ Bohm values, whereas magnetic damping permits magnetic
fields of $\abt 10$--$300 \muG$ and comparatively smaller diffusion
coefficients (typically $\abt 1$--$10$).  Corresponding thin rim radio rims
require magnetic damping, however.  We jointly model radio and X-ray profiles
and estimate magnetic damping length of order $1$--$5$\% of remnant radius,
with magnetic field strengths $\abt 50$--$400 \muG$ assuming Bohm diffusion.
Although the energy dependence of synchrotron rims cannot discriminate magnetic
field structure, joint radio and X-ray modeling favors magnetic damping in
Tycho's SNR.
\end{abstract}

% Six keywords, alphabetical order
\keywords{acceleration of particles ---
    ISM: individual objects (Tycho's SNR) ---
    ISM: magnetic fields ---
    ISM: supernova remnants ---
    shock waves ---
    X-rays: ISM}

% ============
% Introduction
% ============
\section{Introduction} \label{sec:intro}

% What are these shock rims, what is this acceleration?
Forward shock accelerated electrons in young supernova remnants (SNRs) emit
synchrotron radiation strongly in the shock's immediate wake but quickly turn
off downstream, producing a shell-like morphology with bright X-ray and radio
rims/filaments due to line-of-sight projection \citep{koyama1995}.  Strong and
time-variable synchrotron radiation \citep[e.g.,][]{uchiyama2007,
patnaude2007}, in conjunction with multiwavelength spectral modeling
\citep{aharonian2004, acero2010, ackermann2013}, suggests that electrons are
accelerated to PeV energies in young SNRs.  Although synchrotron emission due
to accelerated electrons does not imply acceleration of an unseen hadronic
component, the prevailing theory of diffusive shock acceleration (DSA) should
operate on both positive ions and electrons.  Efficient hadron acceleration in
supernova remnant shocks is a prime candidate source for galactic cosmic rays
up to the cosmic ray spectrum's ``knee'' \citep{vink2012}.

% What can we learn from these shocks?  Why are they interesting?
Spectral and spatial measurements of synchrotron rims can constrain physical
parameters of the shock acceleration process.  Many previous workers have
inferred strong magnetic field amplification at SNR shocks, as electrons must
rapidly radiate energy to account for the thinness of observed rims
\citep{bamba2003, vink2003, parizot2006}.  Furthermore, the synchrotron
spectrum of supernova remnants steepens at X-ray energies \citep{reynolds1999},
corresponding to a cut-off in the underlying electron spectrum as predicted by
diffusive shock acceleration \citep{webb1984}.  Many fundamental questions
about the shock acceleration process remain to be answered.  Under what
conditions do shocks accelerate particles efficiently?  How are magnetic fields
amplified in such shocks?  \citet{reynolds2008} reviews relevant observations
and open questions to date.  The answers to these questions are broadly
relevant to many astrophysical shocks, such as Earth's bow shock
\citep{ellison1990}, starburst galaxies \citep{heckman1990}, jets of active
galactic nuclei \citep{chen2014}, galaxy clusters \citep{van-weeren2010}, and
cosmological shocks \citep{ryu2008}.
% Expand on these questions -- plasma turbulence, Alfven waves, wave damping
% behind the shock, upstream/downstream plasma interactions and instabilities
% These in turn affect plasma, CR dynamics -- how does the shock propagate and
% interact with the ISM?  Will we see a precursor due to diffusion and/or CR
% streaming?  How efficiently are CRs accelerated?

% What makes these shock rims?
The widths of these X-ray rims are controlled by a combination of
synchrotron losses, particle transport, and magnetic fields immediately
downstream of the shock.  Synchrotron losses depend on the initial electron
energy distribution and the gradual decrease of electron energies downstream of
the shock.  High energy electrons near the shock efficiently radiate harder
synchrotron X-ray photons, but decline in energy as they are transported away
from the shock; this may be seen in spectral softening downstream of the shock
\citep[e.g.,][]{cassam-chenai2007}.  The transport process is driven by bulk
plasma advection downstream of the shock and particle diffusion with respect to
bulk advection.  In general, diffusion allows higher energy electrons to
diffuse further upstream or downstream than would be expected from pure
advection.  Finally, the magnetic field strength may be quickly damped
downstream of the shock, preventing electrons from radiating efficiently so
that thin rims reflect the spatial structure of the field rather than efficient
particle acceleration and synchrotron cooling \citep{pohl2005}.

% Whatever makes these shock rims also imposes energy dependence
Moreover, these controlling mechanisms predict different scalings for rim width
as a function of energy.  If the downstream magnetic field is constant and
diffusion is negligible downstream of the shock, the rims are said to be
synchrotron loss-limited and rim widths ($w$) should narrow with increasing
photon frequency ($\nu$) as $w \propto \nu^{-1/2}$. Diffusion smears out rims
at all energies with stronger effect at higher energies, yielding a slower
drop-off in rim widths when diffusion dominates.  Magnetic damping, if at a
length scale comparable to filament widths, also favors more energy-insensitive
rim widths -- intuitively, if the magnetic field turns off, synchrotron
radiation turns off regardless of electron energy.

% Magnetic damping?
Magnetic damping immediately behind the shock offers an alternative explanation
for filament rims that requires less extreme magnetic field amplification, and
the possibility of damping in SNR shocks has not been fully tested
\citep{pohl2005, marcowith2010}.  \citet{cassam-chenai2007} modeled radio and
X-ray synchrotron rim profiles but could not simultaneously reproduce the
intensities of observed X-ray and radio rims with damped or loss-limited
magnetic fields.  \citet{araya2010} measured rims at multiple energy bands in
Cas A, but suggested that the rims only narrow more weakly than expected and
did not consider magnetic damping.  \citet{rettig2012} gave model predictions
for several historical SNRs and proposed discriminating based on filament
spectra -- the expectation is that damped spectra are softer, loss-limited
harder.  Recently, \citet{ressler2014} further extended the idea of
discriminating between damped or loss-limited rims by measuring rim width
energy dependence in X-ray energies and found that rims consistently narrowed
in the remnant of SN 1006.

% Why Tycho's SNR
To further test these models, we follow \citet{ressler2014} (hereafter,
\citetalias{ressler2014}) by measuring rim widths at multiple energies in the
remnant of Tycho's 1572 supernova (hereafter, Tycho).  Tycho exhibits an
extensive shell of synchrotron-dominated thin rims around its periphery; the
rims show very little thermal emission, consistent with expansion into a low
density ISM \citep{williams2013}.  Tycho, being a nearby galactic SNR, is close
enough that its rims are easily resolved by \textit{Chandra} spatial
resolution. A deep $750 \unit{ks}$ exposure of the entire remnant from 2009
allows fine sampling of the remnant rims.  Furthermore, Tycho
is a prime target of study to determine whether hadrons are being accelerated
in historic SNRs \citep[and references therein]{morlino2012}; understanding the
acceleration process for electrons will also bear upon our understanding of CR
acceleration in Tycho.

% Paper roadmap
Our procedure closely follows that of \citetalias{ressler2014}.
We selected and measured rim widths at several locations around Tycho's forward
shock, verifying that selected rims are free of thermal line emission.  Using a
model for particle transport with an exponentially cut-off injected electron
spectrum, we fit width measurements to model predictions in the loss-limited
and magnetic damping scenarios to obtain estimates on magnetic field strength
and diffusion coefficient magnitude.  We discuss degeneracy in our model fits
and attempt to constrain damping by (1) measuring and modeling spectral variation
downstream of the forward shock, and (2) jointly modeling both radio and X-ray
rim morphology.
%and the implications for future studies of young supernova remnant shocks.

% =============================
% Transport models, observables
% =============================
\section{Nonthermal rim modeling}\label{sec:models}

\subsection{Particle transport}\label{sec:transport}

% Transport fundamentals
The energy and space distribution of electrons at a supernova remnant's forward
shock controls the synchrotron rims we see in X-ray and radio.  We assume that
diffusive shock acceleration creates a power-law distribution of electrons with
an exponential cut-off at the forward shock, located at $x = 0$ with $x > 0$
increasing downstream of the shock, given as
\begin{equation}
    f(E, x=0) = K_0 E^{-s} e^{-E/\Ecut} .
\end{equation}
Here $K_0$ is an arbitrary normalization, $E$ is electron energy, and the
cut-off energy $\Ecut$ is given below in Section~\ref{sec:diffcoeff}; the
spectral index $s$ is taken from the radio index $\alpha$ as $s = 2\alpha + 1$.
\citet{zirakashvili2007} have derived from DSA a slightly different energy
spectrum with super-exponential cut-off $e^{-(p/p_{\mt{cut}})^2}$, which may
impact our results; we use a simple exponential cut-off for simplicity and
consistency with \citetalias{ressler2014}.

All constants and equations are given in CGS (Gaussian) units.  Our
presentation is somewhat abbreviated, but a fuller exposition and literature
review are given by \citetalias{ressler2014}.

We model electron advection and diffusion downstream of the shock (i.e., for
$x>0$ only) with a 1-D steady-state plane transport equation:
\begin{equation} \label{eq:model}
    v_d \frac{\ptl f}{\ptl x}
    - \frac{\ptl}{\ptl x} \left(D\frac{\ptl f}{\ptl x}\right)
    - \frac{\ptl}{\ptl E} \left(bB^2E^2f\right)
    = K_0 E^{-s} e^{-E/\Ecut} \delta(x) ;
\end{equation}
this follows the work of \citet{webb1984, berezhko2004, cassam-chenai2007,
morlino2010, rettig2012}.  Here $f = f(E,x)$ is electron distribution as a
function of electron energy $E$ and downstream distance $x$, $D$ is the
diffusion coefficient, and $v_d$ is plasma velocity downstream of the shock.
The constant $b = 1.57 \times 10^{-3}$ in CGS units arises from synchrotron
power loss at electron energy $E$ (i.e., $\ptl E/\ptl t = b B^2 E^2$).
We have assumed isotropic diffusion and consider only processes operating
downstream of the shock.  We also do not incorporate models for self-similar
velocity and magnetic field evolution downstream of a spherical shock (e.g.,
the well-known Sedov-Taylor solution).  A plane flow approximation should be
reasonable as we only consider synchrotron emission within 10\% of the shock
radius, $r_s$, from the forward shock.  % TODO CITATION AND REWORD.
More sophisticated treatments may explicitly treat
magnetic fields and diffusion across the shock, plasma streaming instabilities
and particle losses, anisotropic diffusion, injection/acceleration efficiency,
and other details in great depth \citep[e.g.,][and references
therein]{reville2013, bykov2014, ferrand2014}.

To determine rim profiles and widths, we numerically solve for electron
distribution $f(E,x)$ using Green's function solutions by \citet{lerche1980}
and \citet{rettig2012}; the solutions are given by \citetalias{ressler2014}
using notation similar to ours.  We note, however, that the solutions for
$f(E,x)$ in equation~\eqref{eq:model} are derived assuming that $D(x) B^2(x)$
is constant \citep{lerche1980}.  The electron distribution may be integrated
over the one-particle synchrotron emissivity $G(y)$ to obtain the ``total''
emissivity:
\begin{equation} \label{eq:emissivity}
    j_{\nu}(x) \propto \int_0^\infty G(y) f(E,x) dE
\end{equation}
where $y \equiv \nu/(c_1 E^2 B)$ is a scaled synchrotron frequency and
$G(y) = y \int_y^\infty K_{5/3}(z) dz$ with $K_{5/3}(z)$ a modified Bessel
function of the second kind \citep{pacholczyk1970}.  Integrating emissivity
over lines of sight for a spherical remnant yields intensity as a function of
radial coordinate $r$:
\begin{equation} \label{eq:intensity}
    I_{\nu}(r) = 2 \int_0^{\sqrt{r_s^2 - r}}
                   j_{\nu} \left( r_s - \sqrt{s^2 + r^2} \right) ds
\end{equation}
where $s$ is the line-of-sight coordinate and $r_s$ is shock radius.  Finally,
we compute the full width at half maximum (FWHM) of the resulting intensity
profile to compare to our measurements of rim width.

% Diffusion
\subsection{Diffusion energy-dependence} \label{sec:diffcoeff}

% Bohm, Bohm-like, and MHD/turbulence-inspired diffusion
Most previous work has assumed Bohm-like diffusion in plasma downstream of SNR
shocks.  Bohm diffusion assumes that the particle mean free path $\lambda$ is
equal to the gyroradius $r_g = E/(eB)$, yielding diffusion coefficient
$D_{\mt{B}} = \lambda c / 3 = c E / (3 e B)$; here $c$ is the speed of light,
$E$ is particle energy, $e$ is the elementary charge, and $B$ is magnetic
field.  Bohm-like diffusion encapsulates diffusion scalings of $D \propto E$,
introducing a free prefactor $\eta$ such that $\lambda = \eta r_g$ allows for
varying diffusion strength.  However, Bohm diffusion at $\eta = 1$ is commonly
considered a lower limit on the diffusion coefficient at all energies.

% Define our diffusion coefficient, introduce fiducial energy
We consider a generalized diffusion coefficient with arbitrary
power-law dependence upon energy following, e.g., \citet{parizot2006}:
\begin{equation} \label{eq:diffcoeff}
    D(E) = \frac{\eta C_d E^\mu}{B}
         = \eta_h D_{\mt{B}}\left(E_h\right) \left(\frac{E}{E_h}\right)^\mu
\end{equation}
where $\mu$ parameterizes diffusion-energy scaling and $\eta$ now has units
of $\mt{erg}^{1-\mu}$.  The right-hand side of equation~\eqref{eq:diffcoeff}
introduces $\eta_h$, a dimensionless diffusion coefficient scaled to the Bohm
value at a fiducial particle energy $E_h$.  Note that $\eta_h$ and $\eta$ are
related as $\eta = \eta_h (E_h)^{1-\mu}$, and $\eta = \eta_h$ for Bohm-like
diffusion ($\mu = 1$).  Our notation matches that of \citetalias{ressler2014}.

% Set choice of fiducial energy, eta_2
For subsequent analysis and model fitting, we take fiducial electron energy
$E_2 = E_h$ corresponding to a $2 \unit{keV}$ synchrotron photon and report all
model fit results in terms of $\eta_2 = \eta_h$.  Although the value of $E_2$
varies with magnetic field $B$ as $E_2 \propto B^{-1/2}$ and may vary within
the remnant, tying $\eta_h$ to a fixed observation energy gives a convenient
and intuitive sense of diffusion strength regardless of the underlying electron
energies.

% e- cut-off energy result
With a diffusion coefficient in hand, we may determine the electron spectrum
cut-off energy by equating synchrotron loss and diffusive acceleration
timescales.  For low energies and small synchrotron losses (cooling time longer
than acceleration time), electrons may be efficiently accelerated; near or
above the cut-off energy, electrons will radiate or escape too rapidly to be
accelerated to higher energies and the energy spectrum drops off steeply.  The
cut-off energy is given as:
\begin{align} \label{eq:ecut}
    \Ecut =
        &\left(8.3\unit{TeV}\right)^{2/(1+\mu)}
        \left(\frac{B_0}{100 \muG}\right)^{-1/(1+\mu)} \nonumber \\
        &\times \left(\frac{v_s}{10^8 \unit{cm\;s^{-1}}}\right)^{2/(1+\mu)}
        \eta^{-1 / (1+\mu)} .
\end{align}
Here $B_0$ is the magnetic field immediately downstream of the shock, i.e.
$B_0 = B(x=0)$; our notation departs somewhat from typical notation in denoting
downstream field by $B_0$.
This result is derived by \citet{parizot2006} for $\mu=1$ assuming a strong
shock with compression ratio 4 and isotropic magnetic turbulence on both sides
of the shock.

% srcut constraint on diffusion
As the DSA imposed electron cut-off results in a cut-off on SNR synchrotron
flux, the synchrotron cut-off frequency may provide an independent observable
to estimate shock diffusion and is given by:
\begin{align} \label{eq:cutoff}
    \nu_{\mt{cut}} =
        &c_m \left(13.3 \unit{erg}\right)^{\frac{4}{1+\mu}}
        \left(100 \muG\right)
        \left(2657 \unit{erg^2}\right)^{-\frac{1-\mu}{1+\mu}} \nonumber \\
        &\left( \frac{v_s}{10^8 \unit{cm/s}} \right)^{\frac{4}{1+\mu}}
        \left( \eta_2 \right)^{-\frac{2}{1+\mu}} .
\end{align}
In particular, this equation is independent of magnetic field $B$ for all
values of $\mu$ (but, recall that the specific electron energies associated
with $\eta_2$ will depend on $B$).

% Magnetic damping
\subsection{Magnetic fields and damping}

We consider two scenarios for post-shock magnetic field: (1) a constant field
$B(x) = B_0$, corresponding to the loss-limited rim scenario, and (2) an
exponentially damped field of form:
\begin{equation} \label{eq:bdamp}
    B(x) = \left(B_0 - \Bmin\right) \exp\left(-x / a_b\right)
           + \Bmin
\end{equation}
where $a_b$ is an $e$-folding damping lengthscale, following \citet{pohl2005}.
A typical lengthscale for $a_b$ is $10^{16}$ to $10^{17} \unit{cm}$
\citep{pohl2005}, corresponding to $\abt 0.1$--$1\%$ of Tycho's radius.
Hereafter, we report $a_b$ in units of shock radius $r_s$.

The electron cut-off energy (equation~\eqref{eq:ecut}) depends on the magnetic
field, which varies in a damped model.  For simplicity, we assume that shock
acceleration is controlled by diffusion at the immediate shock and so
equation~\eqref{eq:ecut} stands as evaluated with magnetic field strength
$B_0$.

The solutions to equation~\eqref{eq:model} given by \citet{lerche1980} assume
$D(x) B^2(x)$ constant to render equation~\eqref{eq:model} semi-analytically
tractable.  Although unphysical, we enforce this assumption by modifying the
diffusion coefficient as, following \citet{rettig2012}:
\begin{equation} \label{eq:ddamp}
    D(E,x) = \frac{\eta C_d E^\mu}{B_0}
             \left[ \frac{\Bmin}{B_0} +
                    \frac{B_0 - \Bmin}{B_0} e^{-x/a_b} \right]^{-2} .
\end{equation}
This strengthens the spatial-dependence of the diffusion coefficient, as
compared to the expected $D(x) \propto 1/B(x)$.
% TODO
%\note[Aaron]{I am not certain whether this should be scaled by $\Bmin$ instead,
%will check shortly}

\subsection{Rim width-energy dependence} \label{sec:energydep}

% Define \mE
Following \citetalias{ressler2014}, we parameterize rim width-energy dependence
in terms of a scaling exponent $\mE$ defined as:
\begin{equation}
    w(\nu) \propto \nu^{\mE}
\end{equation}
where $w(\nu)$ is filament FWHM as a function of observed photon frequency
$\nu$.  The exponent $\mE = \mE(\nu)$ is generally energy dependent.  We may
refer to observed photons interchangeably by energy or frequency $\nu$, but $E$
is reserved for electron energy.

% Intuition: transport lengthscales in the absence of damping
To better intuit the effects of advection and diffusion on rim widths, we
consider advective and diffusive lengthscales for electron transport given as:
\begin{equation} \label{eq:lad}
    l_{\mt{ad}} = v_d \tsynch \propto v_d B_0^{-3/2} \nu^{-1/2}
% = \frac{v_d \sqrt{c_m}}{b} B_0^{-3/2} \nu^{-1/2}
\end{equation}
and
\begin{equation} \label{eq:ldiff}
    l_{\mt{diff}} = \sqrt{D \tsynch}
                  \propto \eta^{1/2} B_0^{-(\mu+5)/4} \nu^{(\mu-1)/4}
% = (C_d/b)^{1/2} c_m^{-(\mu-1)/4} \eta^{1/2} B_0^{-(\mu+5)/4} \nu^{(\mu-1)/4}
\end{equation}
The characteristic time is the synchrotron cooling time
$\tsynch = 1 / (b B^2 E)$ with $b = 1.57 \times 10^{-3}$.
For $\mu = 1$, note that $l_{\mt{diff}}$ is independent of $\nu$ and
$l_{\mt{diff}} \propto B_0^{-3/2}$ matches the scaling of $l_{\mt{ad}}$.
If both diffusion and magnetic field damping are negligible (whether at a small
range of, or all, electron energies) and electrons are only loss-limited as
they advect downstream, $\mE$ attains a minimum value $\mE = -1/2$.  Diffusion
increases $\mE$ monotonically from $-1/2$ to a value between $-1/4$ and $1/4$
for $\mu = 0$ and $2$ respectively \citepalias[Figure 3]{ressler2014}.  The
presence of an electron energy cut-off, decreases $\mE$ slightly in all cases
due to the decreased number of electrons and hence thinner rims at higher
energies \citepalias[Figure 5]{ressler2014}.

% Intuition: how damping affects rim widths
We expect magnetic damping to produce comparatively energy-independent rim
widths.  If synchrotron rim widths are set by magnetic damping
at some observation energy, then the rims must be damped at all lower
observation energies as well.  Then rim widths will be relatively constant
(small $|\mE|$) below a threshold energy and may decrease, or even increase
(if $\mu > 1$) once advection or diffusion become dominant controls on rim
widths at higher photon energies (advection: $l_\mt{ad} < a_b$;
diffusion: $l_{\mt{diff}} > l_{\mt{ad}}, a_b$).

We shall see below that this is qualitatively correct in setting model
predictions of rim shapes (within $\abt a_b$ of the shock front), but our
results are complicated by energy-dependent synchrotron losses downstream of
the thin rim.

% A comment
%It may be worth noting that rim width-energy dependence is a morphological
%manifestation of spectral softening downstream of the forward shock as
%observed by, e.g., \citet{cassam-chenai2007}.  So we are simply taking somewhat
%of a different approach to characterizing this softening.

% ============
% Observations
% ============
\section{Observations}
\label{sec:observations}

\subsection{Data and region selections}
\label{sec:regions}

We measured synchrotron rim full widths at half maximum (FWHMs) from an
archival \Chandra ACIS-I observation of Tycho
(RA: 00\tsup{h}25\tsup{m}19\fs0, dec: +64\arcdeg08\arcmin10\farcs0; J2000)
between 2009 Apr 11 and 2009 May 5 (PI: J. Hughes;
\dataset[ADS/Sa.CXO\#obs/10093--10097]{ObsIDs: 10093--10097},
\dataset[ADS/Sa.CXO\#obs/10902--10906]{10902--10906}); \citet{eriksen2011}
present additional observation information.
The total exposure time was $734 \unit{ks}$.
Level 1 \Chandra data were reprocessed with CIAO 4.6 and CALDB 4.6.1.1 and kept
unbinned with ACIS spatial resolution $0.492\arcsec$.
Merged and corrected events were divided into five energy bands:
0.7--1 keV, 1--1.7 keV, 2--3 keV, 3--4.5 keV, and 4.5--7 keV.
We excluded the 1.7--2 keV energy range to avoid \ion{Si}{13} (He$\alpha$)
emission at 1.85 keV, prevalent in the remnant's thermal ejecta, which
might contaminate our nonthermal profile measurements.

% NOTE keep region/filament numbering up to date
We selected 20 regions in 5 filaments around Tycho's shock
(Figure~\ref{fig:snr}) based on the following criteria: (1) filaments should be
clear of spatial plumes of thermal ejecta in \Chandra images, which rules out,
e.g., areas of strong thermal emission on Tycho's eastern limb; (2) filaments
should be singular and localized; multiple filaments should either not
overlap or completely overlap (rules out parts of NE limb); (3) filament
peaks should be evident above the background signal or downstream thermal
emission (rules out faint southern filaments).  We accepted several
regions with poor quality peaks in the lowest energy band ($0.7$--$1
\unit{keV}$) so long as peaks in all higher energy bands were clear and
well-fit.  We grouped regions into filaments by visual inspection of the
remnant.  Within each filament, we chose region widths to obtain comparable
counts at the thin rim peak.  All measured rim widths are at least $1\arcsec$
and hence are resolved by \textit{Chandra}'s point-spread function (PSF), which
has FWHM $\lesssim 1\arcsec$ within $4\arcmin$ of the optical axis.

\begin{figure}
    \centering
    \iftoggle{manuscript}{
        %\includegraphics[width=0.3\textwidth]{figures/snr.pdf}
        \includegraphics[width=0.3\textwidth]{figures/snr-inv.png}
    }{
        %\plotone{figures/snr.pdf}
        \plotone{figures/snr-inv.png}
    }
    \caption{RGB image of Tycho with region selections overlaid.  Image bands
    are 0.7--1 keV (red), 1--2 keV (green) and 2--7 keV (blue).  Bold region
    labels (1, 16) indicate region selections shown in
    Figures~\ref{fig:spec},~\ref{fig:profiles}.  Filament 1: Regions 1--3,
    filament 2: regions 4--10, filament 3: regions 11-13, filament 4:
    regions 14--17, filament 5, regions 18--20.
    \label{fig:snr}}
\end{figure}
% NOTE keep region # bolding updated to match displayed filaments

% ----------------
% Filament spectra
% ----------------
\subsection{Filament spectra}
\label{sec:spec}

We extracted spectra at and immediately behind thin rims in each region
(``rim'', ``downstream'' spectra respectively) to confirm that rim width
measurements are not contaminated by thermal line emission.  The two extraction
regions are determined by our empirical fits of rim profile shape
(Section~\ref{sec:fwhms}).  The rim section is the smallest sub-region
containing the measured FWHM bounds from all energy bands.  The downstream
section extends from the rim section's downstream edge to the farthest
downstream profile fit domain bound, in all energy bands (fits of
equation~\eqref{eq:prof}); i.e., the downstream section is bounded by the thin
rim FWHM and the intensity plateau/trough behind the rim.  To illustrate our
selections, Figure~\ref{fig:spec} plots example rim profiles ($4.5$--$7
\unit{keV}$) with the downstream and rim sections highlighted.

% Spectrum data, binning, background
Spectra were binned such that each datum had $\geq 15$ counts.  We extracted
background spectra from circular regions (radius $\abt 30\arcsec$) around the
remnant's exterior; each region's rim and downstream spectra subtracted the
closest background region's spectrum.
% Background region size is valid for data-tycho/bkg-2/ selections

% NOTE must match fig:snr, fig:profiles!
% Keep region numbers updated and consistent.
\begin{figure*}
    \iftoggle{manuscript}{
        \epsscale{0.7}
    }{}
    \plotone{figures/spec_01.pdf} \\
    \plotone{figures/spec_16.pdf}
    \caption{Spectra and fits from Regions 1 (top) and 16 (bottom) show varying
        rim morphology; Region 1 shows a rim where the $0.7$--$1 \unit{keV}$
        peak could not be fit.  Left: $4.5$--$7 \unit{keV}$ profiles with
        downstream (blue) and rim (grey) sections highlighted.  Intensity is in
        arbitrary units (a.u.).  Middle: downstream spectra with absorbed
        power law fit; Si and S lines at $1.85$, $2.45 \unit{keV}$ are clearly
        visible.  Right: rim spectra with absorbed power law fit show that
        rims in each region are likely free of thermal line emission.}
    \label{fig:spec}
\end{figure*}

% Upstream spectra and numbers
% NOTE keep stated ranges of spectral indices, column densities up to date
We fit each region's rim and downstream spectra to an absorbed power law model
(XSPEC 12.8.1, \texttt{phabs*powerlaw}) between $0.5$--$7 \unit{keV}$ with
photon index $\Gamma$, hydrogen column density $N_{\mt{H}}$, and a
normalization as free parameters.  Table~\ref{tab:spec} lists best fit
parameters and reduced $\chi^2$ values for all regions.  Rim spectra are
well-fit by the power law model alone; the best-fit photon indices ($2.4$--$3$)
and column densities ($0.6$--$0.8 \times 10^{22} \unit{cm^{-2}}$) are
consistent with previous spectral fits to Tycho's nonthermal rims
\citep{hwang2002, cassam-chenai2007}.

% Downstream spectra and numbers
% NOTE keep stated ranges, region numbers up to date
Downstream spectra are generally poorly-fit by the absorbed power law model,
reflecting thermal contamination from \ion{Si}{13} and \ion{S}{15} He$\alpha$
line emission at $1.85$ and $2.45$ keV.  To confirm that thermal emission is
dominated by these two lines near the shock, we also performed fits with (1)
both lines excised ($1.7$--$2.0 \unit{keV}$, $2.3$--$2.6 \unit{keV}$ counts
removed) and (2) with both lines fitted to Gaussian profiles.  Fits with lines
excised yield $\chi^2_{\mt{red}}$ values between $1$--$5$.  Fits with lines
fitted to Gaussian profiles yield $\chi^2_{\mt{red}}$ values $0.83$--$1.6$.  In
both fits (lines excised or modeled), we find somewhat smaller best fit column
densities ($0.3$--$0.8 \times 10^{22} \unit{cm^{-1}}$) but similar best fit
photon indices ($2.6$--$3.1$), compared to those of the rim spectra.  The
consistent photon indices indicate that the same synchrotron continuum is
present beneath thermal line emission.

\begin{table*}
    \scriptsize
    \centering
    \caption{Region spectra fit parameters\label{tab:spec}}
    \begin{tabular}{@{}lcccccr@{}}
\toprule
{} & \multicolumn{3}{c}{Downstream spectra}
   & \multicolumn{3}{c}{Upstream spectra} \\
\cmidrule(lr){2-4} \cmidrule(l){5-7}
Region & $n$ & $n_H$ & $\chi^2_{\mathrm{red}}$ (dof)
       & $n$ & $n_H$ & $\chi^2_{\mathrm{red}}$ (dof) \\
{} & (-) & ($\mt{cm}^{-2}$) & {}
   & (-) & ($\mt{cm}^{-2}$) & {} \\
\midrule
1 & 2.97 & 0.68 & 2.27 (272) & 2.77 & 0.78 & 0.92 (239) \\
2 & 2.91 & 0.64 & 3.55 (163) & 2.54 & 0.67 & 1.05 (232) \\
3 & 3.00 & 0.60 & 3.41 (181) & 2.75 & 0.66 & 1.13 (245) \\
4 & 2.94 & 0.45 & 1.35 (199) & 2.88 & 0.66 & 0.92 (224) \\
5 & 2.90 & 0.50 & 1.15 (224) & 2.83 & 0.68 & 0.96 (246) \\
6 & 2.85 & 0.43 & 1.87 (194) & 3.00 & 0.63 & 1.20 (222) \\
7 & 2.80 & 0.37 & 1.43 (100) & 2.79 & 0.61 & 1.11 (243) \\
8 & 2.79 & 0.55 & 2.36 (183) & 2.83 & 0.71 & 1.22 (285) \\
9 & 2.99 & 0.68 & 4.43 (239) & 2.77 & 0.70 & 1.01 (252) \\
10 & 2.89 & 0.58 & 1.31 (186) & 2.86 & 0.71 & 1.27 (301) \\
11 & 2.89 & 0.61 & 4.92 (231) & 2.91 & 0.72 & 1.16 (281) \\
12 & 3.03 & 0.65 & 1.02 (156) & 2.91 & 0.78 & 0.97 (271) \\
13 & 2.97 & 0.76 & 1.39 (181) & 2.72 & 0.75 & 0.96 (217) \\
\bottomrule
\end{tabular}

    \tablecomments{Absorbed power law fit parameters are photon index $\Gamma$
        and hydrogen column density $N_{\mt{H}}$.  Horizontal rules group
        individual regions into filaments.}
\end{table*}

% This table is still useful to look at the fit parameters (to make
% qualitative statements about excised/Gaussian line fit quality,
% or to send around to collaborators)
%\begin{table*}
%    \scriptsize
%    \centering
%    \caption{Region spectra fit parameters\label{tab:spec-pt2}}
%    \begin{tabular}{@{}lcccccclccr@{}}
\toprule
{} & \multicolumn{7}{c}{Downstream spectra, lines fit}
& \multicolumn{3}{c}{Lines excised} \\
\cmidrule(lr){2-8} \cmidrule(l){9-11}
Region & $\mt{N_H}$ & $\Gamma$ & $E_{\mt{Si}}$ & $W_{\mt{Si}}$
& $E_{\mt{S}}$ & $W_{\mt{S}}$ & $\chi^2_{\mt{red}}$ (dof)
& $\mt{N_H}$ & $\Gamma$ & $\chi^2_{\mt{red}}$ (dof) \\
{} & ($\mt{cm}^{-2}$) & (-) & (keV) & (keV) & (keV) & (keV) & {}
& ($\mt{cm}^{-2}$) & (-) & {} \\
\midrule
1 & 0.43 & 2.61 & 1.86 & 0.27 & 2.46 & 0.18 & 1.10 (180) & 0.44 & 2.62 & 2.39 (186) \\
2 & 0.51 & 2.79 & 1.86 & 0.56 & 2.43 & 0.39 & 1.22 (172) & 0.52 & 2.80 & 5.46 (178) \\
3 & 0.58 & 2.87 & 1.86 & 0.19 & 2.44 & 0.24 & 0.97 (180) & 0.59 & 2.88 & 2.11 (186) \\
\cmidrule{1-11}
4 & 0.51 & 2.87 & 1.84 & 0.15 & 2.46 & 0.14 & 1.06 (157) & 0.53 & 2.90 & 1.51 (163) \\
5 & 0.53 & 2.83 & 1.86 & 0.31 & 2.44 & 0.25 & 1.60 (259) & 0.54 & 2.84 & 5.08 (265) \\
6 & 0.60 & 2.90 & 1.86 & 0.19 & 2.44 & 0.16 & 1.17 (194) & 0.61 & 2.91 & 2.27 (200) \\
7 & 0.64 & 3.07 & 2.50 & 0.26 & 2.46 & 0.01 & 0.94 (136) & 0.64 & 3.02 & 1.01 (142) \\
8 & 0.68 & 2.87 & 1.85 & 0.09 & 2.47 & 0.10 & 1.14 (164) & 0.67 & 2.85 & 1.45 (170) \\
9 & 0.76 & 3.03 & 2.09 & 0.03 & 2.51 & 0.08 & 1.12 (151) & 0.74 & 2.99 & 1.13 (157) \\
10 & 0.56 & 2.79 & 1.85 & 0.14 & 2.47 & 0.09 & 0.83 (214) & 0.56 & 2.79 & 1.49 (220) \\
\cmidrule{1-11}
11 & 0.51 & 2.74 & 1.86 & 0.40 & 2.45 & 0.37 & 0.97 (131) & 0.52 & 2.75 & 2.76 (137) \\
12 & 0.47 & 2.62 & 1.86 & 0.41 & 2.43 & 0.36 & 1.06 (131) & 0.48 & 2.64 & 2.83 (137) \\
13 & 0.51 & 2.88 & 1.87 & 0.30 & 2.43 & 0.30 & 1.21 (192) & 0.52 & 2.88 & 3.35 (198) \\
\cmidrule{1-11}
14 & 0.42 & 2.92 & 1.82 & 0.05 & 2.27 & 0.23 & 1.21 (142) & 0.43 & 2.91 & 1.40 (148) \\
15 & 0.42 & 2.98 & 1.86 & 0.10 & 2.55 & 0.27 & 1.21 (144) & 0.42 & 2.92 & 1.34 (150) \\
16 & 0.42 & 2.91 & 1.85 & 0.20 & 2.33 & 0.31 & 1.05 (183) & 0.43 & 2.91 & 2.16 (189) \\
17 & 0.42 & 2.83 & 1.87 & 0.15 & 2.31 & 0.28 & 1.07 (182) & 0.44 & 2.82 & 1.78 (188) \\
\cmidrule{1-11}
18 & 0.40 & 2.85 & 1.86 & 0.18 & 2.43 & 0.16 & 1.17 (194) & 0.40 & 2.84 & 2.04 (200) \\
19 & 0.37 & 2.83 & 1.85 & 0.16 & 2.41 & 0.21 & 0.98 (127) & 0.38 & 2.83 & 1.35 (133) \\
20 & 0.33 & 2.69 & 1.86 & 0.49 & 2.46 & 0.32 & 1.57 (134) & 0.33 & 2.69 & 3.13 (140) \\
\bottomrule
\end{tabular}

%    \tablecomments{As described in text,
%    we consider absorbed power law fits with lines fitted to two Gaussians
%    (here characterized by equivalent width and line energy), or with lines
%    manually excised.  See comments on Table~\ref{tab:spec}.}
%\end{table*}

% TODO discuss srcutlog fitting (move text up from below)
\begin{table}
    \scriptsize
    \centering
    \caption{Synchrotron roll-off, $\eta_2$, and fits for $\mu = 1$
    \label{tab:spec-srcutlog}}
    \begin{tabular}{@{} lccr @{}}
\toprule
{} & \multicolumn{3}{c}{\texttt{srcut} fit parameters} \\
\cmidrule(lr){2-4}
Region & $N_{\mt{H}}$ & $\nu_{\mt{cut}}$ & $\chi^2_{\mathrm{red}}$ (dof) \\
{}     & ($10^{22} \mt{cm}^{-2}$) & (keV/$h$) & {} \\
\midrule
 1 & 0.62 & 0.30 & 1.17 (284) \\
 2 & 0.60 & 0.30 & 1.08 (202) \\
 3 & 0.68 & 0.33 & 1.14 (167) \\
\cmidrule{1-4}
 4 & 0.61 & 0.29 & 1.15 (278) \\
 5 & 0.64 & 0.27 & 1.11 (255) \\
 6 & 0.64 & 0.29 & 0.93 (231) \\
 7 & 0.71 & 0.23 & 1.14 (224) \\
 8 & 0.66 & 0.41 & 0.96 (198) \\
 9 & 0.73 & 0.30 & 0.88 (175) \\
10 & 0.68 & 0.36 & 0.96 (164) \\
\cmidrule{1-4}
11 & 0.61 & 0.55 & 1.07 (153) \\
12 & 0.57 & 0.88 & 0.90 (172) \\
13 & 0.59 & 0.38 & 1.09 (235) \\
\cmidrule{1-4}
14 & 0.54 & 0.23 & 0.95 (167) \\
15 & 0.57 & 0.29 & 1.03 (183) \\
16 & 0.58 & 0.32 & 1.12 (182) \\
17 & 0.59 & 0.30 & 0.94 (187) \\
\cmidrule{1-4}
18 & 0.55 & 0.19 & 1.13 (220) \\
19 & 0.57 & 0.34 & 0.99 (157) \\
20 & 0.55 & 0.31 & 1.07 (192) \\
\bottomrule
\end{tabular}

    \tablecomments{\texttt{srcut} fits performed in log-frequency coordinates;
        here $h$ is Planck's constant. $\eta_2$ values are computed from
        equation~\eqref{eq:cutoff} and held fixed in model fits.}
\end{table}

% Takeaway -- we are good to go
Our spectral fitting confirms that all selected region are practically free of
thermal line emission, as already suggested by visual inspection
(Figure~\ref{fig:snr}).  Our exclusion of $1.7$--$2 \unit{keV}$ photons in rim
width measurements further limits thermal contamination as $1.85 \unit{keV}$ Si
line emission is over a third of Tycho's thermal flux as detected by \Chandra
\citep{hwang2002}.

% --------------------------
% FWHM measurement procedure
% --------------------------
\subsection{Filament width measurements}
\label{sec:fwhms}

% NOTE keep mentioned number of regions updated!
% Check these (arcsec) numbers before final version.
We obtained radial intensity profiles in five energy bands from $\abt
10$--$20\arcsec$ behind the shock to $\abt 5$--$10\arcsec$ in front for each
region.  To increase signal-to-noise, we integrate along the shock
($5$--$23\arcsec$) in each region.  Plotted and fitted profiles are reported in
vignetting and exposure-corrected intensity units; error bars were computed from
raw counts assuming Poisson statistics.  Intensity profiles peak sharply within
$\abt 2$--$3\arcsec$ behind the shock, demarcating the thin rims, then fall off
gradually until thermal emission picks up further behind the shock.

We fitted rim profiles to a piecewise two-exponential model:
\begin{equation} \label{eq:prof}
    h(x) =
    \begin{cases}
        A_u \exp \left(\frac{x_0 - x}{w_u}\right) + C_u, &x \geq x_0 \\
        A_d \exp \left(\frac{x - x_0}{w_d}\right) + C_d, &x < x_0
    \end{cases}
\end{equation}
where $h(x)$ is profile height and $x$ is radial distance from remnant center.
The rim model, which is strictly empirical, has 6 free parameters:
$A_u, x_0, w_u, w_d, C_u$, and $C_d$; $A_d = A_u + (C_u - C_d)$ enforces
continuity at $x=x_0$. Our model is similar to that of \citet{bamba2003,
bamba2005-hist} and differs slightly from that of \citetalias{ressler2014}.
To fit only the nonthermal rim in each intensity profile, we selected the fit
domain for each profile as follows.  The downstream bound was set at the first
local data minimum downstream of the rim peak, identified by smoothing the
profiles with a 21-point ($\abt 10\arcsec$) Hanning window.  The upstream bound
was set at the profile's outer edge (i.e., no data were removed).
Figure~\ref{fig:profiles} illustrates the fit domain selection

\begin{figure*}%[ht]
    \iftoggle{manuscript}{
        \epsscale{1}
    }{}
    \plotone{figures/prfs_01.pdf}
    \plotone{figures/prfs_16.pdf}
    \caption{Best fit profiles with measured FWHMs demarcated for each energy
        band in Region 1 (top) and Region 16 (bottom).  Hereafter, we use
        Regions 1 and 16 to illustrate results for profiles of differing
        quality and absolute width.  We could not measure a $0.7$--$1
        \unit{keV}$ FWHM in Region 1, reflected in Table~\ref{tab:fwhms}.  Data
        in red were excluded from profile fit domains as described in text.}
    \label{fig:profiles}
\end{figure*}

% NOTE keep blacklisted/excluded region count updated!
From the fitted profiles we extracted a full width at half maximum (FWHM) for
each region and each energy band after subtracting a constant background term
$\min(C_u, C_d)$.  We could not resolve a FWHM in 8 of 20 regions at 0.7--1 keV
(Table~\ref{tab:fwhms}); in these regions, either the downstream FWHM bound
would extend outside the fit domain or we could not find an acceptable fit to
equation~\eqref{eq:prof}.  We were able to resolve FWHMs for all regions at
higher energy bands (1--7 keV).

To estimate FWHM uncertainties, we horizontally stretched each best-fit
profile by mapping radial coordinate $x$ to
$x'(x) = x (1 + \xi (x-x_0)/(50\arcsec-x_0))$ with $\xi$ an arbitrary stretching
parameter and $x_0$ the best-fit rim center from equation~\eqref{eq:prof};
this yields a new profile $h'(x) = h(x'(x))$.
We varied $\xi$ (and hence rim FWHM) to vary each profile fit $\chi^2$ by 2.7
and took the stretched or compressed FWHMs as upper/lower bounds on our
reported FWHMs.

% -------------
% Model fitting
% -------------
\subsection{Filament model fitting}
\label{sec:fits}

% How we ran model fits (itemize knobs, explain energy band mapping, errors,
% fitting procedure
We fit model FWHM predictions given by Equation~ \eqref{eq:model} to our
measured rim widths as a function of energy by varying several physical
parameters: magnetic field strength $B_0$, normalized diffusion coefficient
$\eta_2$, diffusion-energy scaling exponent $\mu$, and minimum field strength
$\Bmin$ and lengthscale $a_b$ for a damped magnetic field scenario.  We
mapped each width measurement to the lower energy limit of its energy band;
e.g., $0.7$--$1 \unit{keV}$ is assigned to $0.7 \unit{keV}$ and fitted to model
profile widths at $0.7 \unit{keV}$.  Width errors in our least squares fits
average the positive and negative errors on each FWHM measurement.  For a given
set of model parameters, we numerically computed intensity profiles and
hence model FWHMs as detailed in Section~\ref{sec:transport}; we then used a
Levenberg-Marquardt fitter to seek model parameters yielding best fit FWHMs.
To assist the nonlinear fitting, we tabulated model FWHM values on a large grid
of model parameters (primarily $\eta_2$ and $B_0$) and used best-fit grid
parameters as initial guesses for fitting.
% TODO this may be obsoleted if we recompute stuff...  I only computed errors
% on the loss-limited model, and those fits are now out of date too...
%Errors on best fit parameters were computed by varying each fit parameter and
%obtaining a new best fit model, with one less degree of freedom, to find the
%parameter limit such that $\Delta \chi^2 = 1$ for a roughly 1-$\sigma$ error.

% How we twiddled model knobs
We first considered a purely loss-limited model (constant downstream magnetic
field $B_0$) with three parameters $\mu$, $\eta_2$, and $B_0$.  To make
nonlinear fitting tractable, we fixed $\mu$ in all fits and considered $\mu =
0$, $1/3$, $1/2$, $1$, $1.5$, and $2$.  In particular, nonlinear
diffusion-energy scalings with $\mu = 1/3$ and $1/2$ may arise from Kolmogorov
and Kraichnan turbulent energy spectra respectively \citep{reynolds2004}.  The
remaining parameters $\eta_2$ and $B_0$ varied freely.

% Damped case
For damped magnetic field fits, we held the asymptotic field strength
$\Bmin$ inside the remnant constant at $5 \muG$, slightly higher than
typical intergalactic values of $\abt 2$--$3 \muG$ \citep{lyne1989, han2006}.
We stepped $a_b$ through $14$ different values between $0.5$ and $0.002$ (sampling
most finely between $0.01$ and $0.002$) and ran fits with $\eta_2$ and $B_0$
free.  For best fits, we report the value of $a_b$ yielding the smallest
$\chi^2$ value, noting that our $a_b$ sampling is coarse.

We required $\eta_2$ to be positive in all fits.  We also deemed
best-fit and error bound values with $\eta_2 \geq 10^5$ and $B_0 \geq 10
\unit{mG}$ to be effectively unconstrained. % Change this if needed.
% TODO "best-fit" vs. "best fit"... argh

% Tycho parameters
A few remnant-specific parameters affect the model calculations.  We adopted
electron spectral index $s = 2.16$ from radio spectral index $\alpha = 0.58$
\citep{sun2011}, remnant distance $3 \unit{kpc}$ \citep[cf.][]{hayato2010}, and
shock radius $1.08 \times 10^{19} \unit{cm}$ from angular radius $240\arcsec$
\citep{green2014}.  Tycho's forward shock velocity varies with azimuth by up to
a factor of 2; we interpolated velocities reported by \citet{williams2013}
(rescaled to $3 \unit{kpc}$) to estimate individual shock velocities for each
region.

% Model resolution error
Predicted rim widths are subject to resolution error in the numerical integrals
(discretization over radial coordinate, line-of-sight coordinate, electron
distribution, Green's function integrals).  We chose integration resolutions
such that the fractional error in model FWHMs associated with halving or
doubling each integration resolution is less than $1\%$ for the parameter space
relevant to our filaments.  In our error analysis, the maximum resolution
errors in a sample of parameter space are typically $0.1$--$1\%$, but mean and
median errors are typically an order of magnitude smaller that maximum errors.

% =======================
% Results, FWHMs and fits
% =======================
\section{Results}

% --------------------
% FWHM results, tables
% --------------------
\subsection{Rim widths}
\label{sec:fwhm-results}

% Present table of FWHM results
Measured rim widths decrease with energy in most regions and energy bands.
Table~\ref{tab:fwhms} reports FWHM measurements for all of our regions.
We also report $\mE$ values for all but the lowest energy band, computed
point-to-point between discrete energy bands as:
\begin{equation}
    \mE(E_2) = \frac{\ln(w_2/w_1)}{\ln(E_2/E_1)}
\end{equation}
where $w_1, w_2$ and $E_1, E_2$ are FWHMs and lower energy values for each
energy band -- e.g., $\mE$ at $1 \unit{keV}$ is computed using FWHMs from
$0.7$--$1 \unit{keV}$ and $1$--$1.7 \unit{keV}$, with $E_2 = 1 \unit{keV}$ and
$E_1 = 0.7 \unit{keV}$.  Errors on $\mE$ are propagated in quadrature from
adjacent FWHM measurements.

\begin{table*}
    \iftoggle{manuscript}{
        \tiny
    }{
        \scriptsize
    }
    \centering
    \caption{Measured full widths at half max (FWHMs) for all regions.
             \label{tab:fwhms}}
    \begin{tabular}{@{}l ccccc r@{ $\pm$ }l r@{ $\pm$ }l r@{ $\pm$ }l r@{ $\pm$ }l @{}}

\toprule
{} & \multicolumn{5}{c}{FWHM (arcsec)} & \multicolumn{8}{c}{$\mE$ (-)} \\
\cmidrule(lr){2-6} \cmidrule(l){7-14}
Region & Band 1 & Band 2 & Band 3 & Band 4 & Band 5
       & \multicolumn{2}{c}{Bands 1--2} & \multicolumn{2}{c}{Bands 2--3}
       & \multicolumn{2}{c}{Bands 3--4} & \multicolumn{2}{r}{Bands 4--5} \\ [0.2em]
{} & (0.7--1 keV) & (1--1.7 keV) & (2--3 keV) & (3--4.5 keV) & (4.5--7 keV)
   & \multicolumn{2}{c}{(1 keV)} & \multicolumn{2}{c}{(2 keV)}
   & \multicolumn{2}{c}{(3 keV)} & \multicolumn{2}{r}{(4.5 keV)} \\
\midrule
1 & {} & ${8.80}^{+0.18}_{-0.15}$ & ${6.34}^{+0.26}_{-0.21}$ & ${7.40}^{+0.30}_{-0.23}$ & ${5.57}^{+0.47}_{-0.42}$
  & \multicolumn{2}{c}{} & $-0.47$ & $0.06$ & $0.38$ & $0.13$ & $-0.70$ & $0.22$ \\ [0.5em]
2 & {} & ${4.22}^{+0.12}_{-0.09}$ & ${2.36}^{+0.12}_{-0.09}$ & ${3.00}^{+0.16}_{-0.12}$ & ${4.11}^{+0.34}_{-0.30}$
  & \multicolumn{2}{c}{} & $-0.84$ & $0.08$ & $0.59$ & $0.16$ & $0.77$ & $0.23$ \\ [0.5em]
3 & {} & ${2.47}^{+0.08}_{-0.07}$ & ${1.78}^{+0.09}_{-0.07}$ & ${2.10}^{+0.11}_{-0.11}$ & ${1.32}^{+0.10}_{-0.09}$
  & \multicolumn{2}{c}{} & $-0.47$ & $0.08$ & $0.41$ & $0.17$ & $-1.15$ & $0.22$ \\

\cmidrule{1-14}
4 & ${5.85}^{+0.37}_{-0.33}$ & ${4.35}^{+0.09}_{-0.08}$ & ${3.26}^{+0.11}_{-0.09}$ & ${3.69}^{+0.12}_{-0.11}$ & ${3.20}^{+0.21}_{-0.18}$
  & $-0.83$ & $0.18$ & $-0.41$ & $0.05$ & $0.31$ & $0.11$ & $-0.35$ & $0.17$ \\ [0.5em]
5 & {} & ${4.52}^{+0.11}_{-0.12}$ & ${3.06}^{+0.11}_{-0.11}$ & ${3.25}^{+0.15}_{-0.13}$ & ${3.04}^{+0.21}_{-0.18}$
  & \multicolumn{2}{c}{} & $-0.56$ & $0.06$ & $0.15$ & $0.14$ & $-0.17$ & $0.19$ \\ [0.5em]
6 & ${2.48}^{+0.18}_{-0.18}$ & ${2.32}^{+0.05}_{-0.06}$ & ${2.98}^{+0.11}_{-0.09}$ & ${2.05}^{+0.08}_{-0.09}$ & ${2.21}^{+0.15}_{-0.14}$
  & $-0.19$ & $0.21$ & $0.36$ & $0.06$ & $-0.92$ & $0.13$ & $0.18$ & $0.19$ \\ [0.5em]
7 & ${2.69}^{+0.20}_{-0.17}$ & ${2.33}^{+0.05}_{-0.05}$ & ${2.31}^{+0.08}_{-0.08}$ & ${1.81}^{+0.09}_{-0.07}$ & ${1.83}^{+0.11}_{-0.08}$
  & $-0.39$ & $0.20$ & $-0.01$ & $0.06$ & $-0.60$ & $0.14$ & $0.02$ & $0.17$ \\ [0.5em]
8 & ${2.33}^{+0.21}_{-0.20}$ & ${2.72}^{+0.08}_{-0.08}$ & ${2.38}^{+0.10}_{-0.09}$ & ${2.10}^{+0.10}_{-0.09}$ & ${2.37}^{+0.20}_{-0.17}$
  & $0.43$ & $0.26$ & $-0.19$ & $0.07$ & $-0.30$ & $0.15$ & $0.29$ & $0.22$ \\ [0.5em]
9 & ${2.16}^{+0.24}_{-0.23}$ & ${2.35}^{+0.07}_{-0.06}$ & ${2.47}^{+0.11}_{-0.11}$ & ${1.91}^{+0.09}_{-0.09}$ & ${2.20}^{+0.17}_{-0.16}$
  & $0.24$ & $0.31$ & $0.07$ & $0.07$ & $-0.63$ & $0.16$ & $0.34$ & $0.22$ \\ [0.5em]
10 & ${2.38}^{+0.24}_{-0.23}$ & ${1.99}^{+0.07}_{-0.06}$ & ${1.76}^{+0.09}_{-0.08}$ & ${1.59}^{+0.09}_{-0.08}$ & ${1.58}^{+0.13}_{-0.12}$
  & $-0.50$ & $0.29$ & $-0.18$ & $0.08$ & $-0.24$ & $0.18$ & $-0.02$ & $0.23$ \\

\cmidrule{1-14}
11 & {} & ${3.23}^{+0.15}_{-0.13}$ & ${2.52}^{+0.16}_{-0.13}$ & ${1.90}^{+0.14}_{-0.13}$ & ${3.09}^{+0.45}_{-0.38}$
  & \multicolumn{2}{c}{} & $-0.36$ & $0.10$ & $-0.70$ & $0.22$ & $1.21$ & $0.37$ \\ [0.5em]
12 & {} & ${3.86}^{+0.17}_{-0.16}$ & ${2.61}^{+0.15}_{-0.13}$ & ${3.02}^{+0.22}_{-0.21}$ & ${2.23}^{+0.21}_{-0.17}$
  & \multicolumn{2}{c}{} & $-0.56$ & $0.10$ & $0.36$ & $0.22$ & $-0.74$ & $0.27$ \\ [0.5em]
13 & ${2.85}^{+0.22}_{-0.17}$ & ${2.43}^{+0.05}_{-0.05}$ & ${2.36}^{+0.08}_{-0.05}$ & ${1.95}^{+0.09}_{-0.10}$ & ${1.84}^{+0.11}_{-0.14}$
  & $-0.45$ & $0.20$ & $-0.04$ & $0.05$ & $-0.47$ & $0.13$ & $-0.15$ & $0.20$ \\

\cmidrule{1-14}
14 & ${2.86}^{+0.17}_{-0.16}$ & ${2.42}^{+0.06}_{-0.04}$ & ${2.23}^{+0.08}_{-0.07}$ & ${2.38}^{+0.10}_{-0.08}$ & ${2.19}^{+0.12}_{-0.10}$
  & $-0.47$ & $0.17$ & $-0.12$ & $0.06$ & $0.17$ & $0.12$ & $-0.20$ & $0.15$ \\ [0.5em]
15 & ${2.71}^{+0.17}_{-0.16}$ & ${1.99}^{+0.05}_{-0.04}$ & ${1.80}^{+0.06}_{-0.05}$ & ${1.87}^{+0.07}_{-0.05}$ & ${1.52}^{+0.09}_{-0.08}$
  & $-0.85$ & $0.18$ & $-0.15$ & $0.05$ & $0.09$ & $0.11$ & $-0.51$ & $0.16$ \\ [0.5em]
16 & ${1.87}^{+0.14}_{-0.13}$ & ${1.73}^{+0.04}_{-0.03}$ & ${1.52}^{+0.06}_{-0.05}$ & ${1.25}^{+0.06}_{-0.04}$ & ${1.23}^{+0.08}_{-0.06}$
  & $-0.22$ & $0.21$ & $-0.18$ & $0.06$ & $-0.49$ & $0.13$ & $-0.04$ & $0.17$ \\ [0.5em]
17 & ${1.65}^{+0.13}_{-0.12}$ & ${1.92}^{+0.05}_{-0.05}$ & ${1.54}^{+0.06}_{-0.07}$ & ${1.45}^{+0.07}_{-0.06}$ & ${2.05}^{+0.16}_{-0.14}$
  & $0.43$ & $0.22$ & $-0.31$ & $0.07$ & $-0.16$ & $0.15$ & $0.86$ & $0.21$ \\

\cmidrule{1-14}
18 & {} & ${4.45}^{+0.13}_{-0.12}$ & ${3.18}^{+0.17}_{-0.16}$ & ${2.96}^{+0.20}_{-0.19}$ & ${1.65}^{+0.21}_{-0.16}$
  & \multicolumn{2}{c}{} & $-0.49$ & $0.09$ & $-0.17$ & $0.21$ & $-1.45$ & $0.32$ \\ [0.5em]
19 & {} & ${2.30}^{+0.08}_{-0.06}$ & ${2.28}^{+0.11}_{-0.08}$ & ${2.16}^{+0.12}_{-0.11}$ & ${1.60}^{+0.17}_{-0.14}$
  & \multicolumn{2}{c}{} & $-0.02$ & $0.08$ & $-0.13$ & $0.17$ & $-0.74$ & $0.27$ \\ [0.5em]
20 & ${4.81}^{+0.31}_{-0.31}$ & ${1.84}^{+0.06}_{-0.03}$ & ${1.87}^{+0.08}_{-0.06}$ & ${1.56}^{+0.07}_{-0.06}$ & ${2.14}^{+0.23}_{-0.23}$
  & $-2.68$ & $0.19$ & $0.02$ & $0.07$ & $-0.44$ & $0.14$ & $0.77$ & $0.28$ \\

\midrule
Mean & $2.89 \pm 0.35$ & $3.11 \pm 0.37$ & $2.53 \pm 0.23$ & $2.47 \pm 0.30$ & $2.35 \pm 0.23$
  & $-0.46$ & $0.24$ & $-0.25$ & $0.06$ & $-0.14$ & $0.10$ & $-0.09$ & $0.15$ \\

\bottomrule
\end{tabular}
\tablecomments{Mean values computed for all regions; mean $\mE$ values are
averages for region $\mE$ values (i.e., not computed from mean FWHMs).  Errors
on mean values are standard errors of the mean.  Horizontal rules group
individual regions into filaments.}

\end{table*}

% Observations of mE and rim widths, errors
Although the measurement scatter is quite large, shown dramatically in the
point-wise computed $\mE$ values, the mean rim width decreases consistently
with increasing energy.  Furthermore, mean $\mE$ values are consistently
negative and tend smoothly towards $0$ (weaker energy-dependence) with
increasing energy.  Errors on FWHM measurements are typically $\lesssim 10\%$,
reflecting the high quality of the underlying \Chandra data.  Scatter in FWHM
measurements may be attributed in part to (1) our profile fitting procedure,
which depends on a somewhat arbitrary choice of profile fit function, and (2)
variation in Tycho's rim morphology (e.g., Figure~\ref{fig:profiles}).

% -------------------------
% Model fit results, tables
% -------------------------
\subsection{Model fit results}
\label{sec:fit-results}

% TODO we must restructure this text, since we are now not emphasizing Regions
% 1/16 alone quite as much.  Just synthesize all together, use fig:fits to
% explain that mu dependence is very weak.

% TODO we could rewrite this all to say up front -- we work with eta2=1 unless
% otherwise stated.  But that should wait until I review the new chain of
% tables / figures w/ Rob and Brian.

% Explain our approach
We tabulate and plot detailed results for Regions 1 and 16, the same regions
shown in Figures~\ref{fig:spec} and \ref{fig:profiles}.  Although only a small
subset of our regions, qualitative trends in fit parameters were comparable in
most regions; our most salient observations, discussed below, are well
exemplified by Regions 1 and 16.  Best fit results for all regions are
summarized in Figure~\ref{fig:fits-all} and Table~\ref{tab:fits-all-eta2one}.
In all tables, we report values of $\chi^2$ rather
than $\chi^2_\mt{red}$ to ease comparison of different fits, as we often
manually stepped one or two parameters that were held constant in fitting.

% Best fit model results and tables for Regions 1, 16.
% Explain a bit about model shapes, and the relative similarity in shapes
The best-fit FWHM-energy curves for Regions 1 and 16 are plotted in
Figure~\ref{fig:fits} alongside measured FWHMs.
Our transport equation does not account for
diffusive transport upstream of the shock, projection effects from shock
ripples or other structure, and thermal emission downstream of the thin rim.
Fitting to multiple FWHM measurements should be robust to these effects as
compared to directly fitting entire rim profiles to model predictions.

\begin{figure}
    \centering
    %\epsscale{0.45}  % For manuscript layout only!
    \iftoggle{manuscript}{
        \includegraphics[width=0.46\textwidth]{figures/energywidth-eta2one-01.pdf}
        \includegraphics[width=0.46\textwidth]{figures/energywidth-eta2one-16.pdf}
    }{
        \includegraphics[width=0.23\textwidth]{figures/energywidth-eta2one-01.pdf}
        \includegraphics[width=0.23\textwidth]{figures/energywidth-eta2one-16.pdf}
    }
    \caption{Best fit rim width-energy predictions for
    Regions 1 (top) and 16 (bottom), with measured data.  Dotted/dashed curves
    plot pure loss-limited fits with varying $\mu$, solid curve plots estimated
    best moderately damped fit ($a_b < 0.01$) with $\mu = 1$.  All fits hold
    $\eta_2 = 1$ fixed; model parameters are given in
    Tables~\ref{tab:fits-all-eta2one}.
    Note that ordinate FWHM axes are not the same and are offset from the
    origin to better show model predictions and data variation.
    \label{fig:fits}}
\end{figure}

\begin{figure*}
    \centering
    \iftoggle{manuscript}{
        \includegraphics[width=0.8\textwidth]{figures/energywidth-subplot.pdf}
    }{
        \plotone{figures/energywidth-subplot.pdf}
    }
    \caption{Rim width predictions for loss-limited and damped fits with
    $\mu = 1$ and $\eta_2=1$ fixed for all regions.  Neither loss-limited nor
    damped fits
    are clearly favored in describing the data.  Some damped model predictions
    (e.g., Region 5) are not given at low energies if FWHMs cannot be
    calculated for model profiles (modeled intensity behind thin rim exceeds
    half-maximum of rim peak within model domain of $\abt 20\arcsec$).
    Corresponding best fit parameters are given in
    Table~\ref{tab:fits-all-eta2one}.
    As in Figure~\ref{fig:fits}, $y$-axis limits vary and are offset from the
    origin to better show model predictions and data.
    \label{fig:fits-all}}
\end{figure*}

% TODO cull/rewrite all references to tab:fits-loss, tab:fits-damp
% Since tables are removed, adjust discussion of eta2, mu effects on fits.
% and effects of varying ab value?

% Observations on qualitative trends in fit parameters
% NOTE keep numbers updated.  # of regions "favoring" mu >= 1
% KEEP B0 minimum number updated!!!!!
Loss-limited fits with $\mu \lesssim 1$ generally yield larger parameter values
and errors for both $\eta_2$ and $B_0$, and they are more likely to be
ill-constrained (e.g., $\eta_2 > 10^3$ or $B_0 > 10^3 \muG$).  Fits with $\mu
\gtrsim 1$ yield smaller $\chi^2$ values in a majority of regions ($\abt 17$ of
$20$), although we do not explore the robustness of this observation.  Fits to
the same data with varying $\mu$ can yield $\eta_2$ varying by $1$--$2$ orders
of magnitude; Region 16 illustrates this dramatically
(Table~\ref{tab:fits-loss}).

Magnetically damped rims are able to fit width-energy dependence in our
measurements at least as well as purely loss-limited rims for an acceptable
range of $a_b$ values (Table~\ref{tab:fits-damp}).  Damped model fits show some
variability in best fit parameters, mainly in $\eta_2$, but are generally
better-behaved than their loss-limited counterparts.  Smaller values of $a_b$
permit and favor smaller best fit values of $B_0$ and $\eta_2$ in all regions.

% Example loss-limited fits
%\begin{table}
%    \scriptsize
%    \centering
%    \caption{Loss-limited fits for Regions 1, 16
%    \label{tab:fits-loss}}
%    \begin{tabular}{@{} r llr llr @{}}

\toprule
{} & \multicolumn{3}{c}{Region 1}
   & \multicolumn{3}{c}{Region 16} \\
\cmidrule(lr){2-4} \cmidrule(l){5-7}
$\mu$ (-) & $\eta_2$ (-) & $B_0$ ($\mu$G) & $\chi^2$
          & $\eta_2$ (-) & $B_0$ ($\mu$G) & $\chi^2$ \\
\cmidrule{1-7}
0.00 & ${585}^{+3540}_{-570}$  & ${700}^{+380}_{-380}$   & 31.39
     & ${248}^{+2720}_{-230}$  & ${1640}^{+1200}_{-700}$ & 11.82 \\
0.33 & ${70}^{+60000}_{-62}$   & ${420}^{+1470}_{-160}$  & 26.85
     & ${131}^{+45000}_{-118}$ & ${1350}^{+3650}_{-540}$ & 4.38 \\
0.50 & ${12}^{+\infty}_{-7}$   & ${276}^{+1825}_{-45}$   & 26.32
     & ${44}^{+\infty}_{-38}$  & ${1030}^{+4920}_{-340}$ & 3.81 \\
1.00 & ${3.8}^{+2.5}_{-1.4}$   & ${211}^{+18}_{-12}$     & 25.05
     & ${3.8}^{+2.7}_{-1.4}$   & ${583}^{+56}_{-38}$     & 4.25 \\
1.50 & ${2.9}^{+1.2}_{-0.8}$   & ${194}^{+8}_{-7}$       & 23.94
     & ${2.6}^{+1.0}_{-0.7}$   & ${524}^{+24}_{-20}$     & 4.95 \\
2.00 & ${2.8}^{+0.9}_{-0.7}$   & ${186}^{+6}_{-5}$       & 23.11
     & ${2.3}^{+0.7}_{-0.6}$   & ${498}^{+15}_{-13}$     & 5.81 \\
\bottomrule
\end{tabular}

%    \tablecomments{Model fits computed at several fixed values of $\mu$ with
%    $\eta_2$, $B_0$ free and constant magnetic field $B(x) = B_0$.
%    Region 1 has 2 degrees of freedom (dofs); Region 16, 3.}
%\end{table}
%
%% Example damping fits
%\begin{table}
%    \scriptsize
%    \centering
%    \caption{Magnetic damping fits for Regions 1, 16
%    \label{tab:fits-damp}}
%    \begin{tabular}{@{}lrrrrrr@{}}
\toprule
{} & \multicolumn{3}{c}{Region 1} & \multicolumn{3}{c}{Region 16}\\
\cmidrule(lr){2-4} \cmidrule(l){5-7}
$a_b$ (-) & $\eta_2$ (-) & $B_0$ ($\mu$G) & $\chi^2$
          & $\eta_2$ (-) & $B_0$ ($\mu$G) & $\chi^2$ \\
\midrule
0.500 & 3.97  & 213.0 & 25.06 & 3.86 & 585  & 4.25 \\
0.050 & 3.17  & 200.4 & 25.48 & 4.15 & 594  & 4.23 \\
0.020 & 0.45  & 133.4 & 27.39 & 4.23 & 594  & 4.18 \\
0.010 & 37.1  &  17.8 & 25.70 & 3.05 & 544  & 4.05 \\
0.009 & 18.0  &  18.8 & 25.30 & 2.69 & 527  & 4.02 \\
0.008 & 1.63  &  25.4 & 24.21 & 2.24 & 505  & 3.98 \\
0.007 & 0.33  &  31.3 & 24.79 & 1.73 & 476  & 3.93 \\
0.006 & 0.084 &  35.4 & 25.60 & 1.18 & 438  & 3.85 \\
0.005 & 0.030 &  36.6 & 25.82 & 0.66 & 391  & 3.72 \\
0.004 & 0.008 &  37.3 & 24.47 & 0.22 & 340  & 3.58 \\
0.003 & 0.005 &  37.5 & 109.4 &  148 & 17.4 & 3.80 \\
0.002 & 0.011 &  38.2 & 1400  & 18.6 & 19.1 & 4.06 \\
\bottomrule
\end{tabular}

%    \tablecomments{Model fits computed at several fixed values of $a_b$ with
%    $\eta_2$, $B_0$ free and damped magnetic field (equation~\eqref{eq:bdamp}).
%    Fit degrees of freedom are as in Table~\ref{tab:fits-loss}}
%\end{table}

% Fit results for ALL regions, mu = 1, eta2 free, loss-limited and damped
% NOTE keep # of dofs updated
%\begin{table}
%    \iftoggle{manuscript}{
%        \scriptsize
%    }{
%        \tiny
%    }
%    \centering
%    \caption{Best model fits for all regions, $\mu = 1$
%    \label{tab:fits-all}}
%    \begin{tabular}{@{}r llr llr llr llr@{}}

\toprule
\multicolumn{13}{c}{Filament 1} \\
\cmidrule{1-13}
{} & \multicolumn{3}{c}{Region 1 (simple model)}
   & \multicolumn{3}{c}{Region 1}
   & \multicolumn{3}{c}{Region 2}
   & \multicolumn{3}{c}{Region 3} \\
\cmidrule(lr){2-4} \cmidrule(lr){5-7} \cmidrule(lr){8-10} \cmidrule(l){11-13}
$\mu$ (-) & $\eta_2$ (-) & $B_0$ ($\mu$G) & $\chi^2_{\mt{red}}$
          & $\eta_2$ (-) & $B_0$ ($\mu$G) & $\chi^2_{\mt{red}}$
          & $\eta_2$ (-) & $B_0$ ($\mu$G) & $\chi^2_{\mt{red}}$ \\
\cmidrule{1-13}
0.00 & ${87}^{+\infty}_{-75}$ & ${647}^{+\infty}_{-281}$ & 13.3
     & ${20}^{+31}_{-12}$ & ${336}^{+74}_{-58}$ & 16.2
     & ${1.0}^{+2.7}_{-0.6}$ & ${340}^{+75}_{-28}$ & 42.5
     & ${25}^{+28}_{-12}$ & ${872}^{+150}_{-114}$ & 27.5 \\
0.33 & ${5.2}^{+6.2}_{-2.4}$ & ${295}^{+62}_{-36}$ & 13.0
     & ${34}^{+244}_{-27}$ & ${354}^{+208}_{-99}$ & 13.5
     & ${1}^{+1}_{-0}$ & ${320}^{+33}_{-16}$ & 41.8
     & ${84}^{+168}_{-56}$ & ${1073}^{+290}_{-229}$ & 23.1 \\
0.50 & ${3.2}^{+2.4}_{-1.2}$ & ${264}^{+33}_{-23}$ & 12.8
     & ${11}^{+637}_{-6}$ & ${270}^{+391}_{-42}$ & 13.2
     & ${0.8}^{+1.1}_{-0.4}$ & ${314}^{+26}_{-13}$ & 41.5
     & ${173}^{+520}_{-133}$ & ${1225}^{+437}_{-343}$ & 21.8 \\
1.00 & ${1.4}^{+0.6}_{-0.4}$ & ${228}^{+12}_{-10}$ & 12.2
     & ${3.7}^{+2.4}_{-1.3}$ & ${208}^{+17}_{-12}$ & 12.5
     & ${0.8}^{+0.7}_{-0.3}$ & ${303}^{+17}_{-9}$ & 40.4
     & ${18000}^{+59000}_{-13000}$ & ${3213}^{+1070}_{-770}$ & 20.4 \\
1.50 & ${0.9}^{+0.3}_{-0.2}$ & ${214}^{+7}_{-6}$ & 11.7
     & ${2.9}^{+1.2}_{-0.8}$ & ${191}^{+8}_{-7}$ & 12.0
     & ${0.9}^{+0.7}_{-0.3}$ & ${298}^{+13}_{-7}$ & 39.3
     & ${9.8}^{+7.4}_{-3.9}$ & ${567}^{+61}_{-44}$ & 23.2 \\
2.00 & ${0.65}^{+0.17}_{-0.14}$ & ${206}^{+5}_{-5}$ & 11.3
     & ${2.8}^{+1.0}_{-0.7}$ & ${184}^{+6}_{-5}$ & 11.5
     & ${1.1}^{+0.7}_{-0.4}$ & ${294}^{+11}_{-7}$ & 38.2
     & ${6.5}^{+2.6}_{-2.0}$ & ${504}^{+26}_{-24}$ & 25.0 \\

\midrule
\multicolumn{13}{c}{Filament 2} \\
\cmidrule{1-13}
{} & \multicolumn{3}{c}{Region 4}
   & \multicolumn{3}{c}{Region 5}
   & \multicolumn{3}{c}{Region 6}
   & \multicolumn{3}{c}{Region 7} \\
\cmidrule(lr){2-4} \cmidrule(lr){5-7} \cmidrule(lr){8-10} \cmidrule(l){11-13}
$\mu$ (-) & $\eta_2$ (-) & $B_0$ ($\mu$G) & $\chi^2_{\mt{red}}$
          & $\eta_2$ (-) & $B_0$ ($\mu$G) & $\chi^2_{\mt{red}}$
          & $\eta_2$ (-) & $B_0$ ($\mu$G) & $\chi^2_{\mt{red}}$
          & $\eta_2$ (-) & $B_0$ ($\mu$G) & $\chi^2_{\mt{red}}$ \\
\cmidrule{1-13}
0.00 & ${18}^{+22}_{-9}$ & ${514}^{+96}_{-68}$ & 16.3
     & ${0.4}^{+0.6}_{-0.2}$ & ${279}^{+24}_{-11}$ & 33.9
     & ${20}^{+17}_{-6}$ & ${757}^{+107}_{-55}$ & 51.6
     & ${21}^{+19}_{-9}$ & ${782}^{+111}_{-89}$ & 13.8 \\
0.33 & ${34}^{+121}_{-23}$ & ${554}^{+222}_{-124}$ & 12.9
     & ${0.4}^{+0.4}_{-0.2}$ & ${273}^{+14}_{-8}$ & 33.5
     & ${59}^{+62}_{-25}$ & ${897}^{+155}_{-102}$ & 39.9
     & ${61}^{+90}_{-36}$ & ${926}^{+206}_{-167}$ & 7.5 \\
0.50 & ${18}^{+206}_{-11}$ & ${470}^{+352}_{-86}$ & 12.3
     & ${0.4}^{+0.3}_{-0.2}$ & ${270}^{+13}_{-7}$ & 33.3
     & ${102}^{+121}_{-50}$ & ${979}^{+188}_{-137}$ & 34.8
     & ${95}^{+233}_{-63}$ & ${993}^{+318}_{-216}$ & 5.5 \\
1.00 & ${4.7}^{+2.9}_{-1.5}$ & ${337}^{+29}_{-19}$ & 11.1
     & ${0.4}^{+0.3}_{-0.1}$ & ${263}^{+10}_{-5}$ & 32.6
     & ${434}^{+1152}_{-290}$ & ${1238}^{+434}_{-279}$ & 23.3
     & ${27}^{+57000}_{-16}$ & ${685}^{+3100}_{-111}$ & 3.8 \\
1.50 & ${3.5}^{+1.3}_{-0.9}$ & ${306}^{+14}_{-10}$ & 9.9
     & ${0.5}^{+0.3}_{-0.2}$ & ${260}^{+8}_{-5}$ & 31.9
     & ${2650}^{+62000}_{-2430}$ & ${1720}^{+1900}_{-760}$ & 18.1
     & ${7.9}^{+4.2}_{-2.5}$ & ${504}^{+39}_{-29}$ & 4.1 \\
2.00 & ${3.4}^{+1.0}_{-0.7}$ & ${293}^{+9}_{-7}$ & 9.0
     & ${0.6}^{+0.3}_{-0.2}$ & ${258}^{+6}_{-4}$ & 31.2
     & ${56}^{+83}_{-25}$ & ${648}^{+140}_{-73}$ & 19.0
     & ${5.8}^{+1.9}_{-1.4}$ & ${458}^{+19}_{-16}$ & 4.6 \\

%\midrule
%\multicolumn{13}{c}{Filament 2 (cont.)} \\
\cmidrule{1-13}
{} & \multicolumn{3}{c}{Region 8}
   & \multicolumn{3}{c}{Region 9}
   & \multicolumn{3}{c}{Region 10} \\
\cmidrule(lr){2-4} \cmidrule(lr){5-7} \cmidrule(lr){8-10}
$\mu$ (-) & $\eta_2$ (-) & $B_0$ ($\mu$G) & $\chi^2_{\mt{red}}$
          & $\eta_2$ (-) & $B_0$ ($\mu$G) & $\chi^2_{\mt{red}}$
          & $\eta_2$ (-) & $B_0$ ($\mu$G) & $\chi^2_{\mt{red}}$ \\
\cmidrule{1-10}
0.00 & ${23}^{+21}_{-11}$ & ${736}^{+110}_{-99}$ & 12.9
     & ${22}^{+19}_{-11}$ & ${784}^{+111}_{-99}$ & 18.5
     & ${19}^{+31}_{-11}$ & ${852}^{+192}_{-133}$ & 4.6 \\
0.33 & ${62}^{+106}_{-38}$ & ${863}^{+213}_{-162}$ & 7.8
     & ${61}^{+88}_{-35}$ & ${918}^{+200}_{-157}$ & 12.4
     & ${51}^{+159}_{-37}$ & ${997}^{+365}_{-241}$ & 1.9 \\
0.50 & ${98}^{+252}_{-66}$ & ${929}^{+306}_{-206}$ & 5.9
     & ${101}^{+191}_{-64}$ & ${996}^{+268}_{-201}$ & 10.0
     & ${83}^{+416}_{-67}$ & ${1079}^{+531}_{-335}$ & 1.0 \\
1.00 & ${754}^{+12300}_{-725}$ & ${1360}^{+1220}_{-720}$ & 3.3
     & ${433}^{+3308}_{-363}$ & ${1267}^{+815}_{-435}$ & 5.2
     & ${13}^{+65}_{-8}$ & ${667}^{+309}_{-97}$ & 0.4 \\
1.50 & ${15}^{+17}_{-7}$ & ${524}^{+86}_{-51}$ & 3.6
     & ${52}^{+\infty}_{-33}$ & ${714}^{+\infty}_{-131}$ & 4.7
     & ${6.1}^{+4.7}_{-2.3}$ & ${542}^{+54}_{-37}$ & 0.3 \\
2.00 & ${9.3}^{+4.5}_{-2.9}$ & ${452}^{+31}_{-25}$ & 4.0
     & ${16}^{+11}_{-6}$ & ${525}^{+53}_{-38}$ & 5.2
     & ${5.0}^{+2.4}_{-1.6}$ & ${501}^{+29}_{-23}$ & 0.2 \\

\bottomrule
\end{tabular}
\tablecomments{Model fits were computed at several fixed values of $\mu$ and
yield fit parameters $\eta_2$, the scaled diffusion coefficient at observed
photon energy $2 \unit{keV}$, and magnetic field $B_0$ assuming constant
downstream field.  For all regions, we use a full continuous energy loss
transport model to fit measured filament widths (equation~\eqref{eq:full-mod}).
For Region 1, we also present fit results from a simpler catastrophic energy
dump model (equation~\eqref{eq:simp-mod}) for comparison.}

%    \tablecomments{Loss-limited fits fix $\mu=1$ to match the damping model;
%    best damped fits select best $a_b$ value below $0.01$ from the range shown
%    in Table~\ref{tab:fits-damp}, enforcing moderate damping.  Fits for Regions
%    1, 2, 12, 18 have 2 dofs; all others have 3.  The manual selection of a
%    best $a_b$ value may be construed as removing one more dof.}
%    % NOTE UPDATE AS FWHMS UPDATED
%\end{table}

In Table~\ref{tab:fits-all-eta2one} we present best loss-limited and
damped fits with $\eta_2 = 1$ fixed.  As before, both loss-limited and damped
models are equally capable of fitting the observed data.

\begin{table}
    \scriptsize
    \centering
    \caption{Best model fits for all regions, $\mu = \eta_2 = 1$
    \label{tab:fits-all-eta2one}}
    \begin{tabular}{@{} l rr rrr @{}}
\toprule
{} & \multicolumn{2}{c}{Loss-limited}
   & \multicolumn{3}{c}{Damped, $a_b \leq 0.01$} \\
\cmidrule(lr){2-3} \cmidrule(l){4-6}
Region & $B_0$ ($\mu$G) & $\chi^2$
       & $B_0$ ($\mu$G) & $\chi^2$ & $a_b$ \\
\midrule
1 & 183.3 & 35.61 & 22.2 & 24.30 & 0.008 \\
2 & 312.9 & 80.81 & 20.1 & 73.70 & 0.003 \\
3 & 455.0 & 96.45 & 293.6 & 62.13 & 0.005 \\
\cmidrule{1-6}
4 & 284.8 & 51.30 & 25.1 & 29.44 & 0.004 \\
5 & 282.5 & 100.36 & 30.7 & 61.95 & 0.003 \\
6 & 410.9 & 174.31 & 49.1 & 64.65 & 0.004 \\
7 & 419.0 & 51.30 & 286.1 & 11.10 & 0.006 \\
8 & 388.2 & 48.05 & 160.8 & 9.80 & 0.005 \\
9 & 415.1 & 65.45 & 62.1 & 14.09 & 0.004 \\
10 & 466.3 & 17.28 & 36.9 & 1.33 & 0.002 \\
\cmidrule{1-6}
11 & 366.0 & 29.82 & 323.4 & 25.96 & 0.010 \\
12 & 318.0 & 9.60 & 18.4 & 8.63 & 0.003 \\
13 & 400.3 & 53.10 & 39.9 & 9.43 & 0.006 \\
\cmidrule{1-6}
14 & 383.9 & 103.42 & 50.6 & 12.50 & 0.004 \\
15 & 431.8 & 70.65 & 30.8 & 19.65 & 0.003 \\
16 & 493.4 & 15.13 & 438.1 & 3.85 & 0.006 \\
17 & 467.7 & 61.84 & 228.6 & 28.99 & 0.004 \\
\cmidrule{1-6}
18 & 283.1 & 27.88 & 16.5 & 16.37 & 0.003 \\
19 & 395.7 & 104.64 & 25.0 & 89.24 & 0.003 \\
20 & 463.6 & 121.99 & 47.4 & 93.70 & 0.003 \\
\bottomrule
\end{tabular}



    \tablecomments{Fits for Regions 1--3, 5, 11, 12, 18, and 19 have 3 degrees
    of freedom; all others have 4.  The choice of a best $a_b$ value may be
    construed as removing one additional dof.}
\end{table}

% Integrate srcutlog into textual discussion (!)

\begin{table}
    \scriptsize
    \centering
    \caption{Best model fits for all regions, $\eta_2$ derived from
        \texttt{srcut} fits
    \label{tab:fits-all-srcutlog}}
    \begin{tabular}{@{} l rrr rrr @{}}
\toprule
{} & {}
   & \multicolumn{2}{c}{Loss-limited}
   & \multicolumn{3}{c}{Damped} \\
\cmidrule(lr){3-4} \cmidrule(l){5-7}
Region & $\eta_2$
       & $B_0$ ($\mu$G) & $\chi^2$
       & $B_0$ ($\mu$G) & $\chi^2$ & $a_b$ \\
\midrule
 1 & 12.0 & 256 & 29.1 &  19 & 25.5 & 0.008 \\
 2 & 11.8 & 443 & 99.4 &  18 & 72.0 & 0.003 \\
 3 & 10.6 & 599 & 31.8 &  19 & 29.3 & 0.002 \\
\cmidrule{1-6}
 4 & 12.0 & 402 & 36.2 &  19 & 34.5 & 0.004 \\
 5 & 12.9 & 419 & 32.5 &  19 & 19.1 & 0.004 \\
 6 & 11.9 & 568 & 89.7 & 147 & 61.9 & 0.006 \\
 7 & 15.3 & 617 & 11.7 & 507 & 11.0 & 0.010 \\
 8 & 8.4  & 511 & 15.1 & 333 & 10.0 & 0.008 \\
 9 & 11.2 & 573 & 23.3 &  66 & 13.7 & 0.005 \\
10 & 9.4  & 631 &  1.4 & 572 &  1.4 & 0.010 \\
\cmidrule{1-6}
11 & 6.1  & 452 & 18.1 &  21 & 17.5 & 0.003 \\
12 & 3.7  & 375 & 12.6 &  21 &  9.7 & 0.003 \\
13 & 8.4  & 529 & 13.8 & 368 &  9.9 & 0.008 \\
\cmidrule{1-6}
14 & 11.9 & 547 & 28.6 & 120 & 12.4 & 0.006 \\
15 & 9.5  & 598 & 23.1 & 426 & 20.0 & 0.007 \\
16 & 8.4  & 670 &  6.3 & 666 &  5.4 & 0.020 \\
17 & 9.0  & 637 & 32.6 &  32 & 29.4 & 0.003 \\
\cmidrule{1-6}
18 & 14.6 & 432 & 72.4 &  17 & 16.6 & 0.003 \\
19 & 8.9  & 542 & 12.8 & 209 &  7.7 & 0.006 \\
20 & 9.6  & 631 & 96.0 &  29 & 94.2 & 0.003 \\
\bottomrule
\end{tabular}




    \tablecomments{$\eta_2$ values are computed from equation~\eqref{eq:cutoff}
    and held fixed in model fits.  See comments for
    Table~\ref{tab:fits-all-eta2one} as well.}
\end{table}



% ==========
% Discussion
% ==========
\section{Discussion}

% --------------------------------------------
% Commentary on data, observations, parameters
% Tycho distance estimates, maybe compression ratio stuff...
% Any constraints on parameter values go here
% --------------------------------------------
\subsection{Data and model fit interpretation}

\subsubsection{Width-energy dependence}

% First address the broad trend of leveling energy dependence in mE)
We suggest that the observed decrease in $|\mE|$ with increasing energy
(Table~\ref{tab:fwhms}) is caused by diffusion, which weakens width-energy
dependence at higher energies.  The photon energy at which $l_{\mt{ad}} =
l_{\mt{diff}}$ is exactly the DSA-predicted synchrotron cut-off energy,
$\abt0.3 \unit{keV}$ for Tycho \citep{hwang2002}.
% TODO citation if you can find it...
Diffusion should begin overtaking advection and push $\mE$ towards $0$ in soft
X-rays, as opposed to $\mE \abt -0.5$ expected for strong advection
(Section~\ref{sec:energydep}).  A decrease in $|\mE|$ is then reassuringly
consistent with the framework of diffusive shock acceleration.

% Discuss variability in width measurements
The large scatter in filament widths and hence scatter in point-wise computed
$\mE$ values (Table~\ref{tab:fwhms}) relegates our interpretation of $\mE$ to
conjecture.  Filament widths both increase and decrease between energy bands in
multiple regions, and diffusion $\eta_2$ and hence $\Ecut$ are poorly
constrained by our fit results.  Our averaged results should mitigate the
effects of measurement scatter, however.

% Can damping explain this mE trend? (a 2nd order effect)
% forward reference the "weak-field" damping discussion
Diffusion may be irrelevant if magnetic field damping can also cause the
observed decrease in $|\mE|$ with energy, independent of the details of shock
acceleration.  If damping controls rim widths at energies above the cut-off
energy, $|\mE|$ may slightly decrease or even increase with energy as rims
transition from being damping controlled to being diffusion loss-limited.  If
damping is only important well below the cut-off energy, then $|\mE|$ may
increase with energy as advection becomes important.  But, synchrotron
losses far downstream of the shock (many $e$-folding lengths $a_b$) can also
give rise to stronger rim width-energy dependence than expected for a damped
magnetic field, which we discuss further in section~\ref{sec:damp-fit-disc}.
The effect of damping upon $\mE(E)$ is somewhat uncertain.

\subsubsection{Modeled diffusion and magnetic field amplification}

% Discuss degeneracy in loss-limited model
We now turn to the model fitting parameters: magnetic field strength $B_0$ and
scaled diffusion coefficient $\eta_2$.  Loss-limited fits are poorly
constrained as $B_0$ and $\eta_2$ may covary without strongly altering fit
quality.  If diffusion is the primary control on rim width -- i.e.,
$l_{\mt{diff}} > l_{\mt{ad}}$ -- then $\eta_2$ and $B_0$ become degenerate as
the product $\eta^{1/2} B_0^{-3/2}$ exerts most control on rim width $w \sim
l_{\mt{diff}}$ (equation~\eqref{eq:ldiff}).  If advection is the primary
control on rim width (widths narrow rapidly with energy; i.e., $\mE \sim
-0.5$), then $\eta_2 \ll 1$ becomes unimportant and fits are well-behaved with
effectively one free parameter.  The degeneracy of $B_0$ and $\eta_2$ is
stronger for smaller values of $\mu$, although we observe no clear trend for
the conditions under which a model fit is ill-constrained.

% Then, degeneracy modification in the damping model
The degeneracy in $B_0$ and $\eta_2$ is modified by the presence of damping.
For small values of $a_b$ (strong damping), increased diffusion $\eta_2$ can
cause rim widths to narrow counterintuitively.  Speculatively, spatial
variation of the diffusion coefficient, which gives rise to a term $(\ptl D(x)
/ \ptl x) (\ptl f/\ptl x)$ in equation~\eqref{eq:model}, may oppose downstream
advection by imposing an effective velocity:
\begin{equation}
    v_d - \frac{\ptl D(x)}{\ptl x}
\end{equation}
thus requiring thinner rims as $\eta_2$ and $\ptl D(x) / \ptl x$ increase.

% TODO discuss this a bit further.  Explain that it doesn't matter for us since
% we are not exploring eta2 dependence of fits etc, but something useful to
% note

% B0 lower bounds in loss-limited model
In loss-limited model fits, the minimum magnetic field strength throughout the
remnant, within stated errors, is $B_0 = 179 \muG$.  The minimum values of
$B_0$ for each filament (1--5) are $179$, $254$, $287$, $458$, and $243 \muG$.
Best fit $B_0$ values are smallest for $\eta_2$ approaching $0$ and $\mu =
2$; intuitively, $\mu = 2$ yields stronger diffusion at higher energies
($>2\unit{keV}$) for fixed $\eta_2$, permitting a smaller best-fit $\eta_2$ and
hence smaller $B_0$.  Furthermore, fixing $\eta_2 = 1$ constrains $B_0$ between
$180$--$500 \muG$ (Table~\ref{tab:fits-all-eta2one}); the range of $B_0$ values
is directly correlated to the observed range of rim widths.  Our minimum
loss-limited values of $B_0$ are consistent with prior estimates of $\abt
200$--$300 \muG$ for advection-dominated transport \citep{volk2005,
parizot2006, morlino2012}.  If loss-limited, Tycho's rims require strong
magnetic field amplification -- at least $\abt 100\times$ the typical galactic
field values of $\abt 2$--$3 \muG$, versus the expected $4\times$ amplification
from a strong shock with compression ratio $r=4$.

Magnetic damping fits permit much smaller values of $B_0$, as expected.
The maximum best-fit $B_0$ value observed is $340 \muG$, with a majority of
values below $100 \muG$.  The minimum value of $B_0$ is around $10$--$20 \muG$,
which is consistent with the expected compression ratio of $4$ for a strong
shock and requires no field amplification.  Fixing $\Bmin = 2 \muG$ instead of
$5 \muG$ permits smaller fit $B_0$ values, though fit values for both $\eta_2$
and $B_0$ still display considerable scatter.

% More detail on relevant B0 estimates follows

% \citet{parizot2006} give $200 \muG$ for pure advection and $230 \muG$ for pure
% diffusion; with both effects, they obtain $300$--$530 \muG$ depending on the
% assumed compression ratio and diffusion-energy scaling $\mu$.
% \citet{volk2005} combine a prediction from nonlinear acceleration with actual
% data.  They too suggest something like $300 \muG$.
% \citet{volk2002} give $240 \muG$ from a multiwavelength study.  Too many papers
% from Russian theorists, seriously.
% \citet{morlino2012} do a similar thing with newer data and favor $B_0 > 200
% \muG$ strong magnetic fields -- it also is favored by the hadronic scenario of
% pion decay to help explain gamma-ray observations.
% \citet{cassam-chenai2007} take values $215 \muG$ and $130 \muG$ for different
% particle injection efficiencies to describe the X-ray rims of Tycho; note that
% they find an overall compression ratio $r \approx 6$ and treat the effect of CR
% acceleration and escape on the shock hydrodynamics.
% \citet{rettig2012} used a measured filament lengthscale and considered
% loss-limited and magnetic damping models separately for several historical
% SNRs.  For Tycho: $B = 310 \muG$ with $\Ecut = 24 \unit{TeV}$; since $\Ecut$
% is set by balancing advective/diffusive lengthscales, this is equivalent to
% our reporting of a diffusion coefficient (?).  With magnetic damping, they
% report $B = 150 \muG$ and a larger $\Ecut = 34 \unit{TeV}$.
% Taken from \citet{rettig2012}: \citet{acciari2011} give $80$, $230 \muG$ for
% leptonic/hadronic models for gamma-ray emission in Tycho.

% eta2
The best-fit values of $\eta_2$ for both loss-limited and damped fits are very
poorly constrained.  Loss-limited fits show no clear trend of $\eta_2$ with
$\mu$ (Table~\ref{tab:fits-loss}) except a general preference for smaller
$\eta_2$ at $\mu > 1$.  Loss-limited fits are generally consistent with the
Bohm limit on diffusion, favoring $\eta_2 \geq 1$; only Regions 5 and 18 have
best-fit $\eta_2$ values strictly below $\eta_2 = 1$ after accounting for
error.  Damped fits favor smaller values of $\eta_2$ in contrast to
loss-limited models, perhaps as stronger damping and smaller $\eta_2$ have
cancelling effects on rim widths (recalling that for moderate to strong
damping, smaller $\eta_2$ gives larger rim widths).

% A quick discussion of mu
Our loss-limited fits at multiple values of $\mu$ are not especially
insightful.  The $\chi^2$ values for individual fits are variable and large
($\gg 1$), so we cannot favor or disfavor fits at different $\mu$ values.  But,
neglecting the magnitude of our $\chi^2$ values, values of $\mu \geq 1$ are
qualitatively favored by $\chi^2$ in almost all regions.  This trend may be
partially an artifact of the correlation between $B_0$ and $\eta_2$ as they are
not entirely independent parameters, but they are less correlated for larger
$\mu$; fits at smaller $\mu$ may have fewer (non-integer) degrees of freedom,
partially offsetting our observations.  \citetalias{ressler2014} also found
better model fits for $\mu \geq 1$ in the remnant of SN 1006 and noted that
\citet{reynolds2004} favored $\mu \geq 1$ on grounds of better modeled remnant
morphology, but it remains uncertain whether $\mu \geq 1$ diffusion is truly
preferred.

We do not explore $\mu$ dependence extensively in the damped model, but
$\mu > 1$ permits smaller $B_0$ values and \emph{larger} $\eta_2$ values, in
keeping with our previous discussion of the effects of $B_0$ and $\eta_2$ on
rim width in both loss-limited and damped models.

\subsubsection{Effect of Tycho distance estimates on model fits}

Our models adopted a distance $d$ to Tycho of $3 \unit{kpc}$, but estimates for
Tycho's distance range between $2.3$--$4 \unit{kpc}$ \citep{hayato2010}.  A
larger remnant distance would increase both physical filament widths and shock
velocity estimates from proper motion; we may then derive expected scalings for
modeled $\eta_2$ and $B_0$ as functions of assumed distance $d$.  The advective
lengthscale, equation~\eqref{eq:lad}, may be rearranged to obtain:
\begin{align}
    B_0 =\; &(317 \muG) \left(\frac{v_d}{10^8 \unit{cm/s}}\right)^{2/3}
                \nonumber \\
            &\times \left(\frac{l_{\mt{ad}}}{0.01 \unit{kpc}}\right)^{-2/3}
                \left(\frac{h\nu}{1 \unit{keV}}\right)^{-1/3}
\end{align}
% TODO if we keep this equation, double check the 317 \muG carefully...
or, more simply,
\begin{equation}
    B_0 \propto \left(v_d\right)^{2/3}
                \left(l_{\mt{ad}}\right)^{-2/3} \nu^{-1/3} .
\end{equation}
Both $l_{\mt{ad}}$ and $v_d$ scale linearly with remnant distance $d$ and thus
their effects cancel in determining the magnetic field.  If diffusion is the
primary control on filament lengthscales, equation~\eqref{eq:ldiff} yields:
\begin{equation}
    \eta \propto \left(l_{\mt{diff}}\right)^2 B_0^{3} \nu^{-(\mu - 1)/2}
\end{equation}
Recall that $\eta \propto \eta_2$ since $E_h = E_2$ is constant for a given
$B_0$, from equation~\eqref{eq:diffcoeff}.  Similar results, primarily for
magnetic field $B_0$, were previously given by \citet{parizot2006}.

In practice, model fits with varying distance obey both scalings $\eta_2
\propto d^2$ and $B_0$ constant.  When comparing full model fits with remnant
distances of $3 \unit{kpc}$ and $4 \unit{kpc}$, the deviation from the
idealized scaling is $\lesssim 1 \%$ for $B_0$ and $\abt 1$--$5\%$ for
$\eta_2$.  As varying remnant distance $d$ leaves width-energy scaling $\mE$
invariant, the relative contributions of $l_{\mt{ad}}$ and $l_{\mt{diff}}$
should also be invariant.  Then both lengthscales should scale simultaneously
with $d$, yielding the observed behavior.  In the damped model, a larger
distance $d$ will require larger physical damping lengths from magnetic
turbulence.  But, fitted values of $a_b$ should remain unchanged as we report
$a_b$ in units of shock radius $r_s$.

\subsection{Lengthscale analysis}

% TODO Short blurb

\begin{figure}
    \centering
    \plotone{figures/lengthscales_fwhm.pdf} \\
    \caption{Stuff
        \label{fig:length}}
\end{figure}

\begin{figure}
    \centering
    \plotone{figures/lengthscales_fwhm_dimless.pdf} \\
    \caption{Stuff.  Dashed line plots scaled FWHM = $4.6$, the projection
        factor for a spherical shell (CITE)  % TODO
        \label{fig:length-dimless}}
\end{figure}

\begin{table*}
    \scriptsize
    \centering
    \caption{Lengthscale analysis
        \label{tab:lengthscales}}
    \begin{tabular}{@{}l cc cc ccr@{}}
\toprule
{} & \multicolumn{2}{c}{Measurements}
   & \multicolumn{2}{c}{Loss}
   & \multicolumn{3}{c}{Damped} \\
\cmidrule(lr){2-3} \cmidrule(lr){4-5} \cmidrule(l){6-8}
Region & $v_0$ & $w$ (2 keV)
       & $l_{\mt{ad}}$ & $l_{\mt{ad}}/l_{\mt{diff}}$
       & $l_{\mt{ad}}$ & $a_b$ & $l_{\mt{ad}}/a_b$ \\
{} & ($10^8 \unit{cm/s}$) & (\%$r_s$)
   & (\%$r_s$) & (-)
   & (\%$r_s$) & (\%$r_s$) & (-) \\
\midrule
 1 & 1.30 & 2.64 & 0.61 & 1.40 & 10.26 &  0.8 & 12.83 \\
 2 & 1.29 & 0.99 & 0.27 & 1.39 & 11.54 &  0.3 & 38.48 \\
 3 & 1.29 & 0.74 & 0.17 & 1.38 &  0.17 &  2.0 &  0.08 \\
\cmidrule{1-8}
 4 & 1.28 & 1.36 & 0.31 & 1.38 & 10.08 &  0.4 & 25.20 \\
 5 & 1.28 & 1.27 & 0.30 & 1.37 & 12.07 &  0.3 & 40.24 \\
 6 & 1.28 & 1.24 & 0.18 & 1.37 &  2.61 &  0.4 &  6.53 \\
 7 & 1.27 & 0.96 & 0.17 & 1.36 &  0.29 &  0.6 &  0.49 \\
 8 & 1.27 & 0.99 & 0.19 & 1.36 &  0.90 &  0.5 &  1.80 \\
 9 & 1.26 & 1.03 & 0.17 & 1.36 &  2.09 &  0.4 &  5.23 \\
10 & 1.26 & 0.73 & 0.14 & 1.35 &  9.21 &  0.2 & 46.05 \\
\cmidrule{1-8}
11 & 1.25 & 1.05 & 0.21 & 1.34 &  0.21 &  5.0 &  0.04 \\
12 & 1.24 & 1.09 & 0.25 & 1.33 &  0.25 & 50.0 &  0.01 \\
13 & 1.23 & 0.98 & 0.18 & 1.32 &  0.34 &  0.6 &  0.57 \\
\cmidrule{1-8}
14 & 1.14 & 0.93 & 0.17 & 1.23 &  2.84 &  0.4 &  7.10 \\
15 & 1.14 & 0.75 & 0.15 & 1.22 &  3.95 &  0.3 & 13.16 \\
16 & 1.13 & 0.63 & 0.12 & 1.21 &  0.14 &  0.6 &  0.24 \\
17 & 1.12 & 0.64 & 0.13 & 1.20 &  0.36 &  0.4 &  0.91 \\
\cmidrule{1-8}
18 & 1.15 & 1.32 & 0.28 & 1.23 &  0.27 &  5.0 &  0.05 \\
19 & 1.18 & 0.95 & 0.17 & 1.27 &  1.99 &  0.4 &  4.97 \\
20 & 1.17 & 0.78 & 0.14 & 1.26 &  3.08 &  0.3 & 10.28 \\
\bottomrule
\end{tabular}

    \tablecomments{All derived from fits with $\eta_2 = 1$ and such.  Ratio of
        $l_{\mt{ad}}/l_{\mt{diff}}$ is same for loss-limited and damped fits
    (see text).
    All values computed at fiducial energy $2 \unit{keV}$.}
\end{table*}



% ----------------
% Main null result
% ----------------
\subsection{Damping is poorly-constrained by X-ray width-energy dependence}
\label{sec:damp-fit-disc}

% Conclusion -- width-energy dependence is not so helpful
Width-energy dependence is not as sensitive a discriminant on damping versus
loss-limited models as may be intuitively expected.  Both loss-limited and
damped fits do not perfectly capture sharp drop-offs (especially between
$0.7$--$1 \unit{keV}$ and $1$--$2 \unit{keV}$), and the data show large
scatter.  In most regions, best fit width-energy curves have $\mE \sim -0.2$.
Only in cases where rim widths drop very sharply (data and best loss-limited
fit $\mE \sim -0.5$) is damping significantly disfavored as a control on rim
widths (e.g., Region 18).

% TODO broadly, this section must be restructured to emphasize
% that our division is VERY arbitrary.  I do NOT want the fit results to be
% construed as physically meaningful.

% The most useful result is just to explain -- damping plausibly works.

% Explain the dichotomy in energy dependence
When width-energy dependence is weak, both loss-limited and damped models give
best fits with very similar profiles -- strong fields at the shock in both models
give very sharp drop-offs in rim intensity.  When width-energy dependence is
stronger, damping profiles can only yield energy dependent rims if magnetic
fields are fairly small (say, $\lesssim 50 \muG$) and rims decline to half
maximum at a distance larger than the damping lengthscale.  Away from the
shock, the magnetic field asymptotically approaches $B_{\mt{min}} = 5 \muG$ and
emission is controlled by synchrotron losses acting differentially on
the range of electron energies we consider.  Synchrotron radiation sieves out
higher energy electrons as in the loss-limited model for rim widths, giving
rise to energy dependent emission.  Each case is shown in
Figure~\ref{fig:rims}; in Region 1 the damping profiles evolve noticeably as
compared to Region 16, with decreasing energy.


\begin{figure}
    \centering
    \iftoggle{manuscript}{
        \epsscale{0.7}
        \plotone{figures/prfs-fit-damp-01.pdf} \\
        \plotone{figures/prfs-fit-damp-16.pdf}
        \epsscale{1}
    }{
        \plotone{figures/prfs-fit-damp-01.pdf} \\
        \plotone{figures/prfs-fit-damp-16.pdf}
    }
    \caption{Model predictions illustrate weak-field (Region 1) and
        strong-field damping (Region 16) for damped best fit parameters with
        $\mu=1$ and $\eta_2=1$ fixed.  In X-ray energies ($0.7$--$4.5
        \unit{keV}$) model profiles are not strongly energy dependent, but
        weak-field damping profiles evolve and do not show a measurable FWHM at
        sufficiently low energies.
        Profiles are normalized to peak thin rim intensity.  Model parameters
        are given in Table~\ref{tab:fits-all-eta2one}.  Shaded regions on damped
        profiles indicate damping lengthscale $a_b$.
        \label{fig:rims}}
\end{figure}

% TODO discuss mE here for weak-field damping, even though we don't plot it
% anymore.

% Weak-field damping
To discriminate, we observe that the best damped fit parameters for Regions
1, 2, 4, 5, 10, 12, and 18 all predict a $1.375 \unit{GHz}$ radio emission
profile with no measurable FWHM, whereas all other regions predict a measurable
rim.  Moreover, these regions are all best fit with field $B_0 < 40 \muG$.
We refer to this model behavior as
``weak-field'' damping, associated with weak magnetic fields and stronger
emission intensity immediately downstream of the thin rim.

% Strong-field damping
All other regions predict consistently thin rims with measurable thin rims at
decreasing energy.  The width-energy dependence parameterized by $\mE$ trends
towards zero at low energy, indicating that rim widths are comparatively energy
independent.
At higher energy, rims become moderately energy dependent ($\mE \sim -0.2$) to
match the observed width-energy dependence in X-rays.  The gradual increase in
$|\mE|$ with energy is expected in the damping model as advection and/or
diffusion take
control of rim widths at increasing energy, as discussed in
Section~\ref{sec:energydep},  The best damped fits for these regions all have
larger $B_0$ values than in the ``weak-field'' damping case.  We refer to these
fits as giving rise to ``strong-field'' damping, associated with stronger
magnetic fields, weaker emission intensity behind the thin rim, and clear rims
with measurable FWHMs at low photon energies (at and below soft X-ray).

% Explanation and caveats
The distinction between weak- and strong-field damping may not
be physically meaningful given the large $\chi^2$ values on our model fits.
Several regions may be well-fit by ``weak-field'' and ``strong-field'' damping
alike.
However, ``weak-field'' damping is often associated with data best fit with
stronger energy dependence, as quantified by $\mE$ values of best loss-limited
and damped fits.  The qualitative behavior of $B_0$ is suggestive; very strong
fields at the shock ($\gtrsim 100 \muG$) do not appear compatible with
``weak-field''-like energy dependence if the magnetic field is damped.

%As ``weak-field'' and ``strong-field'' fits predict strongly differing FWHMs at
%energies below soft X-rays, observations of synchrotron rims at radio
%wavelengths may better discriminate between damping scenarios.

% SN 1006
Our modeling also allows us to reassess the significance of rim width-energy
dependence in the remnant of SN 1006 presented by \citetalias{ressler2014}.  We
fit averaged filament widths in SN 1006 measured by \citetalias{ressler2014} to
our damping model using the same procedure as for Tycho.  We fix $\Bmin = 5
\muG$, although best fit magnetic fields for SN 1006 are smaller than those of
Tycho due to the much wider filaments of SN 1006.
Damped fits are comparable to or better than loss-limited fits in 3 of 5
filaments in SN 1006.  Two filaments with strong energy dependence ($\mE \sim
-0.5$) favor a loss-limited model with sub-Bohm diffusion ($\eta_2 \ll 1$).
The width-energy dependence in SN 1006 ($\mE \sim -0.3$ to $-0.5$) is overall
slightly stronger than in Tycho, and the best damped fits for SN 1006 all fall
into the ``weak-field'' case.  The best fit $B_0$ values are less than $40
\muG$ in the damped model, compared to $\abt 100$--$200 \muG$ in the
loss-limited model.

% TODO Where does this go, where should we integrate this discussion about
% profile shape w.r.t. width-energy dep. fitting?

% TODO WRITE THIS CAPTION

\begin{figure*}%[ht]
    \plotone{figures/pargrid-eta2_1-model.pdf}
    \caption{A grid of best fit parameters, yay!  WRITE THIS CAPTION.}
    \label{fig:pargrid}
\end{figure*}


% --------
% srcutlog
% --------
\subsection{Synchrotron cut-off constrains diffusion}

% TODO move this up in discussion to EARLIER.

Our model fits have not incorporated the synchrotron cut-off energy, which
permits an estimate of diffusion coefficient $\eta_2$ for a given value of
$\mu$ (equation~\eqref{eq:cutoff}) that is independent of magnetic field
$B_0$.  For example, \citet{parizot2006} used measurements of synchrotron
cut-off (and hence electron energy cut-off) to estimate diffusion coefficients
$\eta = 10, 5$ for compression ratios $4, 10$ in Tycho.

To measure synchrotron cut-offs in Tycho, we fitted ``rim'' spectra
(Section~\ref{sec:spec}) to the XSPEC model \texttt{srcut}, modified to fit in
log-frequency units; we fix the radio spectral index $\alpha = 0.58$
\citep{sun2011} as done in our profile modeling and fit for column density
$N_{\mt{H}}$ and cut-off frequency $\nu_{\mt{cut}}$ in each region.
Table~\ref{tab:fits-all-srcutlog} lists best spectrum fit parameters for each
region.  The fitted cut-off frequency is typically $0.3 \unit{keV/h}$,
consistent with fits by \citet{hwang2002}, but Regions 11, 12 have unusually
high cut-off frequencies $0.55$, $0.88 \unit{keV/h}$ consistent with harder rim
spectra (Table~\ref{tab:spec}).

Using the derived values of $\nu_{\mt{cut}}$ ($0.19$--$0.88 \unit{keV/h}$) and
hence $\eta_2$ ($4$--$15$), we computed new loss-limited fits for all regions
with cut-off derived values of $\eta_2$ fixed (Table~\ref{tab:fits-srcutlog}).
Fitted $B_0$ values span $250$--$700 \muG$.  Variation in $\mu$ has a smaller
effect on best fits as we are able to derive different $\eta_2$ values as a
function of $\mu$; Regions 1 and 16 have best fit values of $B_0$ increasing
over $252$--$266 \muG$ and $660$--$700 \muG$ respectively as $\mu$ ranges
between $0$--$2$.  Our fits with cut-off derived values of $\eta_2$ are
comparable to fits with $\eta_2 = 1$ fixed (Table~\ref{tab:fits-all-eta2one}).
We are unable to deduce any insights from requiring that $\eta_2$ reflects
measurements of $\nu_{\mt{cut}}$.  But, diffusion coefficients computed in this
manner may give more credible derived estimates of $B_0$ if the DSA assumptions
invoked are relevant.


% ---------------------------
% Spectral variation analysis
% ---------------------------
\subsection{Spectral softening and model fits}
% Wording of spectral index changes/variation is inconsistent and unwieldy

Spectral variation may also discriminate between loss-limited and damping
models as proposed by \citet{rettig2012}.  Electron energy losses will result
in spectral softening downstream of the shock and thin rim, previously
documented in Tycho by, e.g., \citet{cassam-chenai2007}, but the precise
spectral evolution depends on the assumed post-shock magnetic field structure
(whether loss-limited or damped).  To quantify spectral softening, we consider
the change in spectral index ($\Delta \Gamma$) between spectra taken at the
thin rim and just downstream of the thin rim.

% Measured spectra
We fitted rim and downstream spectra (Section~\ref{sec:spec}) in each region
between $2$--$7 \unit{keV}$ to an absorbed power law with column density
$N_{\mt{H}}$ fixed at $0.7 \times 10^{22} \unit{cm^{-2}}$.  To account for
thermal emission in downstream spectra, we excised $2.3$--$2.6$ and
$3.0$--$3.2 \unit{keV}$ photons to remove \ion{S}{15} and \ion{Ar}{17}
He$\alpha$ lines at $2.45$ and $3.11 \unit{keV}$ respectively.  Mean rim and
downstream spectral indices were $2.94$ and $3.23$ respectively; the mean
index change $\Delta \Gamma$ was $0.3$ (stdev: $0.2$).  Different approaches to
removing thermal emission in downstream spectra -- adding a Gaussian component
to fit \ion{S}{15} line, or excising all $2$--$2.6 \unit{keV}$ photons --
yielded mean downstream spectral index $3.17$, with mean $\Delta \Gamma \abt
0.24$ (stdev: $0.22$ and $0.18$ respectively).

% Model spectra
% Needs rewrite
We also computed model photon spectral indices as a function of energy for our
best fit loss-limited and damping parameters with $\eta_2 = 1$ fixed
(Table~\ref{tab:fits-all-eta2one}); specifically, we computed the $1
\unit{keV}$ FWHM and radially integrated profiles from the forward shock to one
FWHM behind the shock to obtain rim spectra, and integrate from one to two
FWHMs behind the shock to obtain model downstream spectra.  The integration
domains were chosen to roughly match the extraction regions of actual rim
spectra (Section~\ref{sec:spec}).  We fitted the model spectra to a power law
between $2$--$7 \unit{keV}$ to compare with our measurements.
Figure~\ref{fig:specvar} plots model spectra with power law fits and spectral
indices for Regions 1 and 16 as a function of photon energy.

\begin{figure}[h]
    \centering
    \iftoggle{manuscript}{
        \includegraphics[width=0.8\textwidth]{figures/plt-specvar-1.pdf} \\
        \includegraphics[width=0.8\textwidth]{figures/plt-specvar-16.pdf}
    }{
        \plotone{figures/plt-specvar-1.pdf} \\
        \plotone{figures/plt-specvar-16.pdf}
    }
    \caption{Model spectra and spectral indices as a function of energy for
    best damping fits with $\eta_2 = 1$, Regions 1 (top) and 16 (bottom).
    Model spectra (left) plotted with best power law fits between 2--7 keV,
    shaded on plot.  Spectral indices (right) demonstrate softer downstream
    spectra and expected roll-off in the power-law synchrotron spectrum.
    The power law fits yield rim/downstream indices of 2.37/2.44 for Region 1,
    and 2.56/2.61 for Region 16; spectral index changes are 0.07 and 0.05, much
    smaller than $\Delta\Gamma \sim 0.3$ from measured spectra.
    \label{fig:specvar}}
\end{figure}

% Result
Model parameters derived from FWHM width-energy fits are unable to reproduce
the observed filament spectrum evolution.  The mean change $\Delta\Gamma$ was
$0.05$ (stdev: $0.0005$) for loss-limited model profiles and $0.05$ (stdev:
$0.02$) for damped model profiles.  Modeled spectral variation in all cases is
inconsistent with the observed $\Delta\Gamma \sim 0.2$--$0.3$.
Our damped model results disagree with those of \citet[][Fig.~4]{rettig2012},
who predict spectral index changes of up to $\sim 0.15$ for strong damping.  To
check our results, we followed the procedure of \citet{rettig2012} and
attempted to reproduce their tables and figures.  We found reasonable agreement
for loss-limited model spectra but could not reproduce their damped model
spectra for the given parameters.  A weaker shock magnetic field (e.g., $B_0 =
35 \muG$ vs. $85 \muG$, with downstream field $B_{\mt{min}} = 10 \muG$) gave
spectral index variation closer to that reported by \citet{rettig2012}.

% Interpretation
The disagreement between modeled and observed spectral evolution behind the
shock suggests that either model parameters ($B_0$, $\eta_2$, $a_b$) or our
model assumptions are incorrect, in the damped case.  Our results should be
representative of the parameter space of damped model filaments.  Damped
magnetic field fit values span $24$--$400 \muG$ with varied $a_b$ values;
varying $\eta_2$ increased the mean $\Delta\Gamma$ slightly, but not enough to
be consistent with our measurements.

The modeled spectral variation appears to disfavor loss-limited rim widths, as
the predicted $2$--$7 \unit{keV}$ spectral variation $\Delta\Gamma \sim 0.05$
is consistent for all regions considered.  Damped magnetic fields, especially
if relatively weak at the shock ($B_0 \lesssim 40 \muG$), appear associated
with stronger spectral variation ($\Delta\Gamma \sim 0.06$--$0.09$).  Spectral
variation may thus favor shock magnetic fields with little to no amplification.
But, modeled $\Delta\Gamma$ values $< 0.1$ still underpredict observations of
$\Delta\Gamma \sim 0.3$.

% ----------------------------------------
% Constraining rims via radio measurements
% ----------------------------------------
\subsection{Radio rims require magnetic damping}

% Introduction
Thin radio rims spatially coincident with X-ray synchrotron rims may help
constrain magnetic field damping.  Radio emission in Tycho arises from the same
population of shock-accelerated electrons generating nonthermal X-ray rims.
But, $\abt$GeV electrons have cooling times of order $10^5$--$10^7 \unit{yr}$
in $10$--$100 \muG$ magnetic fields; synchrotron emission in the remnant
interior may reflect the spatial structure of long-lived electrons interacting
with a turbulent field rather than the advection of recently shock-accelerated
electrons.
Despite the extended population of radio synchrotron electrons, radio rims and
decreases in emission intensity immediately behind the shock may be explained
by magnetic damping.  If remnant-wide diffuse emission near the forward shock
increases monotonically towards the center of the remnant, only magnetic field
damping can cause emission to decrease.  A loss-limited field predicts
monotonically increasing emission downstream of the forward shock in our
steady-state transport model.
% TODO Redundant

% Procedure
We extract radio profiles from a $1.375 \unit{GHz}$ image of Tycho taken
with the VLA in A configuration in March 1994 (PI: D. Moffett); see
\citet{reynoso1997} for a detailed presentation.
% TODO update
\note[Aaron]{Q for Steve: Was data from all configurations merged?  Done by
\citet{reynoso1997} it seems.  FITS header states A config. but would like to
verify}
The half-power beam width of $\abt 1.5\arcsec$ just resolves thin radio rims
and structure near the forward shock; the image is sampled at $0.5\arcsec$.
%We shift radio profiles forward following X-ray proper motion measurements by
%\citet{katsuda2010-tycho} over a 7 year baseline; the average shift is
%$\abt4.5\arcsec$.
We also extracted $4$--$7 \unit{keV}$ X-ray profiles in all regions from the
previous archival \Chandra observation to jointly model radio and X-ray
profiles, permitting somewhat firmer discrimination of plausible model
parameters.  Figure~\ref{fig:radio-snr} shows the extraction regions overlaying
the radio image and the previous X-ray profile regions of Figure~\ref{fig:snr}.

\begin{figure}
    \centering
    \iftoggle{manuscript}{
        \includegraphics[width=0.3\textwidth]{figures/radio-snr-inv.png}
    }{
        \plotone{figures/radio-snr-inv.png}
        %\plotone{figures/radio-snr.eps}
    }
    \caption{Radio image of Tycho's SNR at $1.375 \unit{GHz}$ with linear
    scaling.  Extraction regions (green) for joint radio and X-ray profile
    analysis overlay region selections for X-ray rim width analysis
    (Figure~\ref{fig:snr}). \label{fig:radio-snr}}
\end{figure}

% Procedure: model computation
We compute model radio and X-ray measured profiles from the transport model of
equation~\eqref{eq:model} for varying $B_0$ and $a_b$.  The parameters
$B_{\mt{min}} = 5 \muG$, $\eta_2 = 1$, and $\mu = 1$ are held fixed.
Diffusion, in particular, is negligible for modeled radio emission as particle
energies are 3 orders of magnitude lower than in X-ray.  Neglecting diffusion
also circumvents the unphysical assumption that $D(x) B^2(x)$ is constant,
which was invoked to obtain Green's function solutions to
equation~\eqref{eq:model}.

% Procedure: chi-by-eye "fitting"
We align model profiles to measurements by eye, varying relative amplitudes and
translations in radio and X-ray independently to best match the measured
profiles.  We manually seek values of $B_0$ and $a_b$ that best reproduce both
radio and X-ray profiles.  Although ``fitting'' by eye cannot strongly
constrain model parameters, we can find plausible values of $B_0$ and $a_b$
and qualitatively constrain magnetic damping throughout the remnant.  Results
from a more involved nonlinear fit may not be more reliable as we cannot
constrain spatially heterogeneous radio emission within the remnant.  Moreover,
our transport model neglects self-similar downstream evolution of shocked
plasma (decaying velocity, density, and magnetic field) and is unreliable far
downstream of the shock.

% Procedure: cont
The radio and X-ray profile modeling here contrasts strongly with our previous
width-energy fitting from X-ray measurements alone, which neglected profile
shape in favor of more robust FWHM measurements.  Manually fitting profiles
allows us to consider radio and X-ray filaments that do not have well-defined
FWHMs, especially as radio rims do not fall below 50\% of the peak emission.
We can also use profiles from regions not previously considered due to the lack
of an X-ray FWHM, especially in softer ($0.7$--$4$) X-rays.

% Results: morphology, individual regions, main figure
We identify three classes of radio profiles: thin rims with downstream troughs,
plateaus, and continuous rises.  Regions B, C, D (southern limb) show plateaus
in radio emission.  Regions J, L, M (around NW) show continuous rises in
emission.  All other regions have a radio rim within $15\arcsec$ of the forward
shock, where the forward shock in radio is assumed at zero intensity.
Figure~\ref{fig:radio-prfs} shows extracted radio and nonthermal X-ray
($4$--$7\unit{keV}$) profiles for three well-fit regions with our best manually
selected model profiles, illustrating each of the three radio profile types
observed.

\begin{figure}
    \centering
    \iftoggle{manuscript}{
        \epsscale{0.6}
        \plotone{figures/radio-fits-B.pdf} \\
        \plotone{figures/radio-fits-K.pdf} \\
        \plotone{figures/radio-fits-M.pdf}
        \epsscale{1}
    }{
        % Slightly larger figures than given by \plotone
        \includegraphics[width=0.46\textwidth]{figures/radio-fits-B.pdf} \\
        \includegraphics[width=0.46\textwidth]{figures/radio-fits-K.pdf} \\
        \includegraphics[width=0.46\textwidth]{figures/radio-fits-M.pdf}
    }
    \caption{Measured radio and X-ray profiles plotted with model profiles for
    varying $a_b$ and $B_0$ in each region, showing typical parameters (and
    ranges) required to reproduce radio and X-ray rims simultaneously in our
    model.  \emph{Solid black curves} in all regions plot our manually chosen best
    model profiles.  Profiles are chosen to show varying radio rim morphology,
    including plateaus (B), thin rims with troughs (K), and continuous rises
    (M); these profiles show some of the best agreement of our selected
    regions, but cf. Figure~\ref{fig:radio-prfs-meh}.  Profile radial
    coordinates are shifted arbitrarily to aid visual comparison (following
    proper motions of \citet{katsuda2010-tycho} where possible) but may not be
    physically meaningful.  Negative radio intensity is associated with
    deconvolution of raw VLA visibilities.
    \label{fig:radio-prfs}}
\end{figure}

% Results: linchpin for damping model here
A loss-limited rim model (constant magnetic field $B_0$) cannot produce thin
radio rims.  Our model requires damping length $a_b \lesssim 0.1$ to produce a
plateau or thin rim in radio emission.  For regions with thin radio rims, the
best manually selected profiles have $B_0$ between $50$--$400 \muG$, neglecting
only Region N which could not be modeled simultaneously in both X-ray and radio
(Figure~\ref{fig:radio-prfs-meh}).  The damping length $a_b$ ranges between
$0.01$--$0.03$, or $2$--$7\arcsec$.

% TODO discuss how changing a_b / B_0 affect radio/X-ray rims, qualitatively

The radio plateaus in regions B, C, D are compatible with damping lengths
between $0.01$--$0.05$.  Continuous rises in radio emission are best modeled
with $a_b \gtrsim 0.1$ and $B_0 \abt 200$--$300 \muG$.  Although damping
lengths $a_b \gtrsim 0.1$ may not be physically meaningful, such large values
of $a_b$ yield practically constant magnetic field behind the shock regardless.
Our results are also insensitive to the assumed value of $\Bmin$; only model
profiles with small magnetic fields ($B_0 \lesssim 100 \muG$) and/or small
damping length ($a_b < 0.01$) are affected if we take $\Bmin = 2 \muG$ rather
than $5 \muG$.
% TODO rewrite for clarity (!)

% Results: downstream emission should bound damping for an idealized,
% spherically symmetric remnant
The slope and depth of the radio emission decrease behind a thin rim also
gives a soft bound on the minimum magnetic field and maximum allowable damping
length, if other sources of radio emission increase monotonically downstream of
the shock and thus do not give rise to radio rims.  The maximum plausible $a_b$
is $\abt 0.01$--$0.05$ (1--5\% of shock radius, or $2$--$12\arcsec$); the
minimum magnetic field is $\abt 20 \muG$ assuming $\Bmin = 5 \muG$, at least
consistent with the expectation for a strong adiabatic shock.

\begin{figure}
    \centering
    \iftoggle{manuscript}{
        \epsscale{0.8}
        \plotone{figures/radio-fits-I.pdf} \\
        \plotone{figures/radio-fits-N.pdf}
        \epsscale{1}
    }{
        \includegraphics[width=0.47\textwidth]{figures/radio-fits-I.pdf} \\
        \includegraphics[width=0.47\textwidth]{figures/radio-fits-N.pdf}
    }
    \caption{Measured profiles poorly described by our model profiles, compared
    to Figure~\ref{fig:radio-prfs}.  Region I shows irregularly shaped radio
    rim.  Region N contains two superposed filaments that cannot be modeled by
    a single rim in both X-ray and radio; the narrow X-ray rim requires
    atypically strong magnetic fields ($\sim 800 \muG$), and the emission
    plateau behind the radio rim requires small damping lengths ($a_b \sim
    0.005$).
    \label{fig:radio-prfs-meh}}
\end{figure}

% Discussion: confounding effects (and figure)
Tycho's shock structure is more complex than assumed by our transport model.
Moreover, emission in the remnant interior clearly shows spatial structure
(Figure~\ref{fig:radio-snr}).  Figure \ref{fig:radio-prfs-meh} shows two
regions that were poorly described by our model.  A majority of our regions
have irregular rims; in at least 2--3 regions, this may be attributed to
projection of multiple distinct filaments.  Others (e.g., Region K) show rims
with slopes that cannot be matched by our models, whether too steep or
shallow.  Shape mismatch may be attributed in part to point-spread mismatch,
diffusion (e.g., $\eta_2 \neq 1$), shock precursors, or other effects.

Nevertheless, our conclusion that thin radio rims require magnetic damping is
supported by more sophisticated modeling.  Hydrodynamic models with diffusive
shock acceleration \citep{cassam-chenai2007, slane2014} also cannot produce
radio profiles with narrow rims in a purely advected magnetic field.
\citet{cassam-chenai2007} produced radio rims using a damped magnetic field,
though they could not simultaneously reconcile predicted X-ray and radio
intensities to observations.  Our model captures the essential physics for a
very young remnant such as Tycho, for which the forward shock should be
adiabatic.


\begin{table}
    \scriptsize
    \centering
    \caption{Eyeballed fits
        \label{tab:fits-eyeball}}
    \begin{tabular}{@{}rrrrr@{}}
\toprule
%\multicolumn{5}{c}{Eyeballed fits} \\
%\midrule
Region & $B_0$ & $a_b$ & $l_{\mt{ad}}(2\unit{keV})$ & $l_{\mt{ad}}/a_b$ \\
{} & ($\mu$G) & (\%$r_s$) & (\%$r_s$) & (-) \\
\midrule
A &  50 & 2.0  &  3.81 &  1.90 \\
B & 200 & 5.0  &  0.48 &  0.10 \\
C &  15 & 1.0  & 23.16 & 23.16 \\
D & 100 & 5.0  &  1.35 &  0.27 \\
E & 120 & 3.0  &  1.02 &  0.34 \\
F & 300 & 2.5  &  0.26 &  0.10 \\
G & 250 & 2.0  &  0.34 &  0.17 \\
H & 300 & 2.0  &  0.26 &  0.13 \\
I & 250 & 2.0  &  0.34 &  0.17 \\
J & 300 & $\infty$ &  0.26 &  0.01 \\
K & 400 & 1.0  &  0.17 &  0.17 \\
L & 250 & $\infty$ &  0.34 &  0.02 \\
M & 200 & $\infty$ &  0.48 &  0.00 \\
N & 800 & 1.0 &  0.06 &  0.06 \\
O & 200 & 0.50 &  0.48 &  0.95 \\
P & 150 & 1.2 &  0.73 &  0.61 \\
\bottomrule
\end{tabular}

    \tablecomments{$\infty$ indicates that we favor a loss-limited fit here
    ($a_b > 10$\% of shock radius $r_s$)}
\end{table}


% -------------------------------------------
% Model correctness - are our results valid?!
% -------------------------------------------
\subsection{Model correctness}

% Data quality
We emphasize that our measurements are not data limited, as additional counts
from averaging measurements or selecting larger regions will not improve our
ability to constrain $B_0$ and $\eta_2$.  This is easily seen from
Figure~\ref{fig:fits} as well as large $\chi^2$ values in
Tables~\ref{tab:fits-loss} that reflect relatively tight errors in our FWHM
profile measurements (Table~\ref{tab:fwhms}).

The exponential variation in $D(x)$ (equation~\eqref{eq:ddamp}) may not be
physically reasonable; as noted above, the assumption $D(x)
\propto 1 / B^2(x)$ gives rise to sharp gradients in diffusion coefficient and
even inverts the effect of $\eta_2$ on our model profiles (where larger
$\eta_2$ can cause rim widths to narrow).  Nevertheless, the modeled behavior
appears physically reasonable.  And, only X-ray profiles are impacted by the
assumption on $D(x)$ as radio profiles assume no diffusion.

% TODO question: we have been ignoring intensity (absolute comparison).  What
% if, e.g., with strong diffusion particles get out of the damping zone faster
% and don't even have a chance to radiate?
% I have no clue... check this, double check Sean's normalizations to be safe

%Expressions for $D$ in terms of unit power per log bandwidth, which is energy
%dependent as particles scatter preferentially on certain waves or whatever...
%could $D$ just be high at the shock (and that would also suggest taking $D
%\propto 1 / B_0$ rather than $D \propto 1 / \Bmin$, for our expression
%for cut-off energy $\Ecut$, could be a problem).

%Quickly remark on effect of changing compression ratio (effectively dropping
%$v_d$ downstream).  In both loss-limited and damped models, a higher
%compression ratio at the shock gives rise to a lower downstream velocity, which
%permits weaker magnetic fields and diffusion.

% ==========
% Conclusion
% ==========
\section{Conclusion}

We measured the widths of several thin synchrotron filaments around Tycho's
supernova remnant and found moderate narrowing of rim widths throughout the
remnant, corroborating rim narrowing observed by \citet{ressler2014} in the
remnant of SN 1006.  We confirmed that selected filaments are dominated by
nonthermal emission and have clearly measurable full widths at half maximum in
4--5 energy bands.

A steady-state particle transport model with constant magnetic field gives
diffusion coefficients and magnetic field strengths broadly consistent with
prior estimates from rim widths \citep[e.g.,][]{parizot2006, rettig2012} and
radio and gamma ray measurements \citep{acciari2011, morlino2012}.  The same
model with a damped magnetic field is equally capable of describing our
measured data.  At weak energy dependence the two models are indistinguishable
and magnetic damping fits still favor moderately amplified magnetic fields.  At
moderate energy dependence ($\mE \sim -0.3$), the damping model favors weaker
magnetic fields, but modeled FWHMs may not be truly representative of thin rim
widths and profiles.

Thin radio synchrotron rims, however, can only be produced at the shock by
magnetic damping.  Assuming shocked electrons account for most radio emission
immediately downstream of Tycho's forward shock, we jointly model radio and
X-ray profiles to measured profiles and find that damping lengths of $1$--$5$\%
of the shock radius are required throughout most of the remnant; only a few
(3/16) selected regions are plausibly consistent with a constant advected
magnetic field.  Typical magnetic field strengths range between $50$--$400
\muG$.  Although we cannot bound damping lengths and fields from our
qualitative profile comparisons, our results are physically reasonable and are
likely good to order-of-magnitude.

%Additional proxies for diffusion and/or magnetic field are needed to better
%constrain magnetic damping in the shocks of young supernova remnants.
%
%As our data are not count limited, we suggest that additional proxies for
%diffusion and magnetic field are needed to establish or rule out the presence
%of damping at the shocks of young supernova remnants.  More sophisticated
%modeling of shock physics \citep[e.g.,][]{zirakashvili2014} and shock/filament
%structure \citep[cf.][]{caprioli2013} may be necessary to improve modeling
%results.

% ================
% Acknowledgements
% ================
\acknowledgments

A.T. thanks CRESST for sponsorship/support at GSFC (any relevant grants?)
We thank David Moffett for providing the VLA image used in this study.
The scientific results reported in this article are based on data obtained from
the \Chandra Data Archive.
This research made extensive use of NASA's Astrophysics Data System.
This research also made use of APLpy, an open-source plotting package for
Python hosted at \href{http://aplpy.github.com}{http://aplpy.github.com}.

{\it Facilities:} \facility{CXO (ACIS-I)}

% ==========
% References
% ==========
\bibliographystyle{apj}  % AASTeX journal macros are supplied in ADS entries
\bibliography{refs-snr}

% ========
% Appendix
% ========
\clearpage
\appendix

\setcounter{table}{0}
\renewcommand{\thetable}{A\arabic{table}}
\setcounter{figure}{0}
\renewcommand{\thefigure}{A\arabic{figure}}

% ======================================
% Extra tables/fit information for Tycho
% ======================================
\section{Extra Tycho results}

% TODO EXPLAIN/DISCUSS BMIN=ZERO INLINE (not here obviously)
\begin{table}
    \scriptsize
    \centering
    \caption{Best model fits with $\Bmin = 0$ and $\mu = \eta_2 = 1$}
    \begin{tabular}{@{} l rr rrr @{}}
\toprule
{} & \multicolumn{2}{c}{Loss-limited}
   & \multicolumn{3}{c}{Damped} \\
\cmidrule(lr){2-3} \cmidrule(l){4-6}
Region & $B_0$ ($\mu$G) & $\chi^2$
       & $B_0$ ($\mu$G) & $\chi^2$ & $a_b$ \\
\midrule
 1 & 182 &  35.1 & 162 & 26.6 & 0.030 \\
 2 & 312 &  80.9 & 316 & 80.8 & 0.050 \\
 3 & 426 &  21.5 & 430 & 21.4 & 0.020 \\
\cmidrule{1-6}
 4 & 284 &  50.9 & 272 & 43.1 & 0.020 \\
 5 & 288 &  19.3 & 289 & 19.3 & 0.500 \\
 6 & 410 & 173.7 &  90 & 61.5 & 0.005 \\
 7 & 418 &  50.9 & 277 & 11.1 & 0.006 \\
 8 & 388 &  47.7 & 205 &  9.9 & 0.006 \\
 9 & 414 &  65.1 & 143 & 14.0 & 0.005 \\
10 & 466 &  17.1 & 307 &  1.6 & 0.005 \\
\cmidrule{1-6}
11 & 355 &  12.0 & 359 & 11.9 & 0.050 \\
12 & 317 &   9.6 & 318 &  9.6 & 0.500 \\
13 & 400 &  52.7 & 240 & 10.0 & 0.006 \\
\cmidrule{1-6}
14 & 383 & 102.9 &  52 &  9.9 & 0.005 \\
15 & 431 &  70.3 & 255 & 20.2 & 0.005 \\
16 & 493 &  15.0 & 430 &  3.9 & 0.006 \\
17 & 467 &  61.6 & 194 & 29.1 & 0.004 \\
\cmidrule{1-6}
18 & 283 &  28.1 & 286 & 27.2 & 0.050 \\
19 & 401 &  36.9 & 144 &  8.1 & 0.005 \\
20 & 463 & 121.8 & 140 & 94.4 & 0.004 \\
\bottomrule
\end{tabular}

    \tablecomments{See comments for Table~\ref{tab:fits-all-eta2one}.}
\end{table}

% TODO EXPLAIN/DISCUSS ETA2 INLINE
\begin{table}
    \scriptsize
    \centering
    \caption{Best model fits with $\eta_2 = 10$ fixed}
    \begin{tabular}{@{} l rr rrr @{}}
\toprule
{} & \multicolumn{2}{c}{Loss-limited}
   & \multicolumn{3}{c}{Damped} \\
\cmidrule(lr){2-3} \cmidrule(l){4-6}
Region & $B_0$ ($\mu$G) & $\chi^2$
       & $B_0$ ($\mu$G) & $\chi^2$ & $a_b$ \\
\midrule
 1 & 248 & 28.1 &  19 & 25.0 & 0.008 \\
 2 & 429 & 97.9 &  18 & 72.3 & 0.003 \\
 3 & 592 & 31.5 &  19 & 28.1 & 0.002 \\
\cmidrule{1-6}
 4 & 388 & 35.3 &  19 & 33.3 & 0.004 \\
 5 & 400 & 30.4 &  19 & 20.2 & 0.004 \\
 6 & 550 & 92.7 & 160 & 61.8 & 0.006 \\
 7 & 569 & 12.8 & 455 & 11.0 & 0.009 \\
 8 & 528 & 14.2 & 333 & 10.0 & 0.008 \\
 9 & 561 & 24.2 &  70 & 13.6 & 0.005 \\
10 & 637 &  1.4 & 638 &  1.4 & 0.500 \\
\cmidrule{1-6}
11 & 496 & 20.3 &  20 & 16.0 & 0.003 \\
12 & 446 & 16.9 &  18 &  8.6 & 0.003 \\
13 & 546 & 12.9 & 417 &  9.8 & 0.009 \\
\cmidrule{1-6}
14 & 529 & 30.7 & 133 & 12.1 & 0.006 \\
15 & 603 & 22.9 & 427 & 20.0 & 0.007 \\
16 & 692 &  7.0 &  21 &  5.2 & 0.002 \\
17 & 650 & 32.1 & 435 & 29.4 & 0.006 \\
\cmidrule{1-6}
18 & 400 & 67.6 &  17 & 17.9 & 0.003 \\
19 & 554 & 12.3 & 201 &  7.7 & 0.006 \\
20 & 636 & 95.9 &  29 & 94.3 & 0.003 \\
\bottomrule
\end{tabular}

    \tablecomments{See comments for Table~\ref{tab:fits-all-eta2one}.}
\end{table}

% TODO EXPLAIN
\begin{figure*}
    \centering
    \includegraphics[width=0.5\textwidth]{figures/specvar-dgamma.pdf}
    \caption{Spectral variation as a function of energy for damped best fit
    parameters, w/ different choice of $\eta_2$.}
\end{figure*}



% ================================
% Best fit information for SN 1006
% ================================
\clearpage
\section{Best fits, SN 1006}

Here we give best loss-limited and damped fit results for averaged filament
widths in SN 1006, taken from \citet{ressler2014}.

\begin{figure*}[h]
    \centering
    \includegraphics[width=0.7\textwidth]{figures/energywidth-subplot-sn1006.pdf}
    \caption{Rim width predictions for loss-limited (dashed) and damped fits
        (solid) with $\mu = 1$ fixed for all regions, SN 1006.}
\end{figure*}

% Fit results for ALL regions, mu = 1, loss-limited and damped
\begin{table}[h]
    \footnotesize
    \centering
    \caption{Best model fits for all regions, $\mu = 1$}
    \begin{tabular}{@{} l rrr rrrr @{}}
\toprule
{} & \multicolumn{3}{c}{Loss-limited}
   & \multicolumn{4}{c}{Damped, $a_b \leq 0.01$} \\
\cmidrule(lr){2-4} \cmidrule(l){5-8}
Region & $\eta_2$ (-) & $B_0$ ($\mu$G) & $\chi^2$
       & $\eta_2$ (-) & $B_0$ ($\mu$G) & $\chi^2$ & $a_b$ \\
\midrule
1 & 2.59 & 102.5 & 0.14 & 0.04 & 34.6 & 0.15 & 0.006 \\
2 & 0.17 & 130.3 & 52.43 & 27.58 & 16.3 & 17.31 & 0.004 \\
3 & 0.02 & 73.7 & 1.80 & 47.85 & 15.3 & 11.49 & 0.010 \\
4 & 0.03 & 107.3 & 0.47 & 1018.51 & 10.7 & 2.15 & 0.008 \\
5 & 35.16 & 174.3 & 3.80 & 2.32 & 30.1 & 3.67 & 0.010 \\
\bottomrule
\end{tabular}


    \tablecomments{Loss-limited fits have $\mu=1$ fixed;
        best damped fits select best $a_b$ value between $0.002$ and $0.01$
        (1.8--9 arcsec; $a_b$ is given in units of shock radius $r_s$),
        enforcing moderate damping.  All fits have 1 dof.  The manual selection
        of a best $a_b$ value may be construed as removing one more dof.}
\end{table}

\begin{figure*}[h]
    \centering
    \includegraphics[width=0.48\textwidth]{figures/prfs-fit-sn1006-01.pdf}
    \includegraphics[width=0.48\textwidth]{figures/prfs-fit-sn1006-02.pdf}
    \includegraphics[width=0.48\textwidth]{figures/prfs-fit-sn1006-03.pdf}
    \includegraphics[width=0.48\textwidth]{figures/prfs-fit-sn1006-04.pdf}
    \includegraphics[width=0.48\textwidth]{figures/prfs-fit-sn1006-05.pdf}
    \caption{Rim width predictions for loss-limited and damped fits with
    $\mu = 1$ fixed for all regions, SN 1006.}
\end{figure*}

% =======================
% Extra -- postage stamps
% =======================

%\clearpage
%\section{Extra}
%
%% NB this blows up the file size...
%\begin{figure*}[hb]
%    \centering
%    \plotone{figures/energywidth-subplot-stamp.pdf}
%    \caption{Philatelics of Tycho's synchrotron rims with application to
%    \Chandra publicity}
%\end{figure*}


% ====================================================
% Synchrotron cut-off energy and diffusion coefficient
% ====================================================
\clearpage
\section{Derivation for synchrotron cut-off and diffusion}

% NOT optimized for 2-column layout

Assume that a DSA model explains the observed e-/synchrotron spectra
cut-off.  We estimate e- cut-off energy by equating
$\tau_{\mt{accel}} = \tau_{\mt{synch}}$.  This gives (Parizot et al., 2006):
\begin{equation}
    \frac{3r}{r-1} \frac{r D_d + D_u}{v_s^2} = \frac{1}{b B^2 \Ecut}
\end{equation}
(\emph{I don't know if this equality was designed to exactly obtain the electron
cut-off energy, or is just a rough/scaling relation -- pull up Drury (1983) to
check this})
Following through the arguments/assumptions in Parizot et al. (2006) yields
(using $13.3 \unit{erg} = 8.3 \unit{TeV}$):
\begin{equation}
    \Ecut =
        \left( 13.3 \unit{erg} \right)^{\frac{2}{1+\mu}}
        \left( \frac{B_0}{100 \muG} \right)^{-\frac{1}{1+\mu}}
        \left( \frac{v_s}{10^8 \unit{cm\;s^{-1}}} \right)^{\frac{2}{1+\mu}}
        \eta^{-\frac{1}{1+\mu}}
\end{equation}
Now, we can convert this cut-off e- energy to a cut-off synchrotron energy by
invoking the delta function assumption $\nu_{\mt{cut}} = c_m \Ecut^2 B$ with
$c_m = 1.82 \times 10^{18}$.
(\emph{Again, I don't know if this is an equality or just a scaling.  We'd need
to relate the electron spectrum cut-off to the derived synchrotron spectrum
cut-off, as is done (I think) by Zirakashvili and Aharonian (2007).})
\begin{equation}
    \nu_{\mt{cut}} = c_m
        \left( 13.3 \unit{erg} \right)^{\frac{4}{1+\mu}}
        \left( 100 \muG \right)
        \left( \frac{B}{100\muG} \right)^{-\frac{1-\mu}{1+\mu}}
        \left( \frac{v_s}{10^8 \unit{cm/s}} \right)^{\frac{4}{1+\mu}}
        \eta^{-\frac{2}{1+\mu}}
\end{equation}
In the case of $\mu = 1$ this yields:
\begin{equation}
    \nu_{\mt{cut}} = (0.133 \unit{keV} / h)
        \left( \frac{v_s}{10^8 \unit{cm/s}} \right)^{2}
        \eta^{-1}
\end{equation}
For $\mu \neq 1$ this is no longer independent of magnetic field.  But,
there is hope.  Rewrite this in terms of $\eta_2$ using
$\eta = \eta_2 E_2^{1-\mu}$ and write:
\[
    \nu_{\mt{cut}} = c_m
        \left( 13.3 \unit{erg} \right)^{\frac{4}{1+\mu}}
        \left( 100 \muG \right)
        \left( E_2 \right)^{-\frac{2(1-\mu)}{1+\mu}}
        \left( \frac{B}{100\muG} \right)^{-\frac{1-\mu}{1+\mu}}
        \left( \frac{v_s}{10^8 \unit{cm/s}} \right)^{\frac{4}{1+\mu}}
        \left( \eta_2 \right)^{-\frac{2}{1+\mu}}
\]
With $E_2 = \left( (2 \unit{keV}/h) / (c_m B) \right)^{1/2}$, or
$E_2 = \left( 0.2657 \unit{erg^2\;G} / B \right)^{1/2}$, we obtain:
\[
    \nu_{\mt{cut}} = c_m
        \left( 13.3 \unit{erg} \right)^{\frac{4}{1+\mu}}
        \left( 100 \muG \right)
        \left( \frac{0.2657 \unit{erg^2\;G}}{B} \right)^{-\frac{1-\mu}{1+\mu}}
        \left( \frac{B}{100\muG} \right)^{-\frac{1-\mu}{1+\mu}}
        \left( \frac{v_s}{10^8 \unit{cm/s}} \right)^{\frac{4}{1+\mu}}
        \left( \eta_2 \right)^{-\frac{2}{1+\mu}}
\]
And, the magnetic field terms cancel.  Our final result
(equation~\eqref{eq:cutoff}) is:
\begin{equation}
    \nu_{\mt{cut}} = c_m
        \left( 13.3 \unit{erg} \right)^{\frac{4}{1+\mu}}
        \left( 2657 \unit{erg^2\;G} \right)^{-\frac{1-\mu}{1+\mu}}
        \left( \frac{v_s}{10^8 \unit{cm/s}} \right)^{\frac{4}{1+\mu}}
        \left( \eta_2 \right)^{-\frac{2}{1+\mu}}
\end{equation}
which depends only upon $v_s$ and $\eta_2$.

%\end{CJK}
\end{document}
