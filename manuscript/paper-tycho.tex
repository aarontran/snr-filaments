%\documentclass[manuscript]{aastex}  % one-column, double-spaced GENERATE BIB
% \documentclass[12pt,preprint]{aastex}  % one-column, single-spaced
\documentclass[iop, apj, numberedappendix, twocolappendix]{emulateapj}
% \documentclass[iop, apj, numberedappendix]{emulateapj}


\shorttitle{Synchrotron Rims in Tycho's SNR}  % <~ 44 char
\shortauthors{XXX et al.}  % Max three
\slugcomment{Draft, \today}  % short title pg comment

%% ==================================================================== %%
%% README for track changes                                             %%
%% To add/remove text or add comments, use the following commands:      %%
%%                                                                      %%
%%       \note[editor]{The note}                                        %%
%%     \annote[editor]{Text to annotate}{The note}                      %%
%%        \add[editor]{Text to add}                                     %%
%%     \remove[editor]{Text to remove}                                  %%
%%     \change[editor]{Text to remove}{Text to add}                     %%
%%                                                                      %%
%% ==================================================================== %%

\usepackage[inline]{trackchanges}  % trackchanges.sourceforge.net
\addeditor{Rob}
\addeditor{Sean}
\addeditor{Steve}
\addeditor{Aaron}
\addeditor{Brian}

\usepackage{amsmath}  % amsthm, amssymb
% \usepackage{CJK}  % aas.org/authors/author-names-non-roman-alphabets
\usepackage{booktabs}
%\usepackage[labelfont=bf, labelsep=period]{caption}  % Custom float captions
%\usepackage{pdflscape}  % rotate pages (Texlive)
\usepackage{hyperref}

\newcommand*{\mt}{\mathrm}
\newcommand*{\unit}[1]{\;\mt{#1}}  % http://vemod.net/typesetting-units-in-latex
\newcommand*{\abt}{\mathord{\sim}} % http://tex.stackexchange.com/q/55701
\newcommand*{\ptl}{\partial}
\newcommand*{\del}{\nabla}

% This paper
\newcommand*\mean[1]{\bar{#1}}
\renewcommand{\vec}[1]{\mathbf{#1}}
\newcommand*{\tsup}{\textsuperscript}

\newcommand*{\Chandra}{\textit{Chandra}\ }
\newcommand*{\tsynch}{\tau_{\mt{synch}}}
\newcommand*{\mE}{m_\mt{E}}
\newcommand*{\Ecut}{E_{\mt{cut}}}
\defcitealias{ressler2014}{R14}

\begin{document}

\title{Synchrotron X-Ray Rims in Tycho's Supernova Remnant are Energy Dependent}

%\begin{CJK*}{UTF8}{gbsn}
\author{
Aaron Tran\altaffilmark{1,4},
Brian J. Williams\altaffilmark{1,5},
Robert Petre\altaffilmark{1},
Sean M. Ressler\altaffilmark{2},
Stephen P. Reynolds\altaffilmark{3}
}
%\end{CJK*}

\affil{
\tsup{1}NASA Goddard Space Flight Center, Greenbelt, MD 20771, USA \\
\tsup{2}Dept. Physics, University of California, Berkeley, CA 94720, USA \\
\tsup{3}Dept. Physics, North Carolina State University, Raleigh, NC 27695, USA
}

%\altaffiltext{1}{NASA Goddard Space Flight Center, Greenbelt, MD 20771, USA}
%\altaffiltext{2}{Dept. Physics, University of California,
%    Berkeley, CA 94720, USA}
%\altaffiltext{3}{Dept. Physics, North Carolina State University,
%    Raleigh, NC 27695, USA}
\altaffiltext{4}{CRESST, University of Maryland, College Park, MD 20742}
\altaffiltext{5}{NASA Postdoctoral Program Fellow}

\begin{abstract}
\note[Aaron]{Copy pasted from NASA abstract, not reviewed} Young supernova
remnants may exhibit thin ($\abt1$--$10$\% of shock radius) X-ray rims of synchrotron
radiation from forward shock-accelerated electrons that travel downstream of
the shock and quickly cease to radiate. Rim widths limited by radiative energy
losses should decrease with energy and require magnetic field amplification
$10$--$100\times$ that expected from adiabatic shock compression.  Damped
magnetic fields behind rims may produce thin rims without strong field
amplification but require energy-independent rim widths. We measured rim widths
around Tycho's supernova remnant in 5 energy bands using a 750 ks \Chandra
observation. Rims narrow with increasing energy, favoring loss-limited
radiation over magnetic damping. Observed widths are best fit by electron
transport models requiring amplified magnetic fields $\abt0.1$--$1$ mG and
particle diffusion $\abt1$--$10\times$ Bohm values, consistent with prior work
on SN 1006. Inferred magnetic fields, diffusion coefficients, and
diffusion-energy scaling may constrain models for cosmic ray acceleration in
supernova remnants and plasma turbulence in astrophysical shocks.
\end{abstract}

% Six keywords, alphabetical order
\keywords{acceleration of particles ---
    ISM: individual objects (Tycho's SNR) ---
    ISM: magnetic fields ---
    ISM: supernova remnants ---
    shock waves ---
    X-rays: ISM}

% TODO GENERAL WARNING -- DATA ARE NOT ALL FULLY CONSISTENT AT THIS TIME.
% (spectra/profiles/fwhms/fits).  Will be corrected / good to go in final
% revision.

% ============
% Introduction
% ============
\section{Introduction}

\note[Aaron]{this first draft of intro structure is mostly set (in terms of
material, number of paragraphs).  Explanations/etc may be too simplistic, I
have more reading to do.}

% Intro -- shocks, DSA, thin rims, filament observations; other fields/work
Forward shock accelerated electrons in young supernova remnants (SNRs) emit
synchrotron radiation strongly in the shock's immediate wake but quickly turn
off downstream \citep{koyama1995, reynolds1996}, producing a shell-like
morphology with bright X-ray and radio rims/filaments due to line-of-sight
projection (hereafter, thin rims).  X-ray and gamma ray imaging has provided
several lines of indirect and direct evidence for hadron and lepton
acceleration in SNRs \citep{uchiyama2007, aharonian2004, acero2010,
ackermann2013}; electrons, in particular, are expected to be accelerated to
$\abt10$--$100\unit{TeV}$ energies.  Although synchrotron emission due to
accelerated electrons does not imply acceleration of an unseen hadronic
component, the prevailing explanatory mechanism of diffusive shock acceleration
(DSA) should operate on both protons and electrons; thus, observations of thin
synchrotron rims has been often taken as supporting, indirect evidence for
cosmic ray acceleration.  \note[Aaron]{first/second pass, very poor wording}

Spectral and spatial measurements of synchrotron rims can help constrain
physical parameters of the shock acceleration process.  Many previous workers
have inferred magnetic field amplification at shock fronts in SNRs; a host of
mechanisms \emph{what?!} may account for the thinness of the observed
shocks -- much thinner than $1/12$ the forward shock radius $r_s$ as would be
expected from a Sedov density profile \note[Aaron]{but Bamba et al. 2003 find
consistent scale length for SN 1006?}.  \note[Aaron]{need a reference -- I
don't have a sense of which mechanisms are most important, relevant, proven?!
even} Many plasma physics details -- Bell's cosmic ray current instability
\citep{bell2001} ties magnetic field amplification w/ maximum CR energy (push
towards and just above the PeV limit).  The processes observed in SNR shocks
are broadly relevant to many collisionless astrophysical shocks, such as
Earth's bow shock, stellar wind-blown bubbles, starburst galaxies
\citep{heckman1990}, jets of active galactic nuclei, galaxy clusters
\citep{van-weeren2010}, and even cosmological shocks \citep{miniati2000,
ryu2008}.

% What makes these shocks?
The widths of these X-ray rims are thought to be controlled by a combination of
synchrotron losses, particle transport, and magnetic fields immediately
downstream of the shock.  Synchrotron losses depend on the initial electron
energy distribution and the gradual decrease of electron energies downstream of
the shock.  High energy electrons near the shock efficiently radiate harder
synchrotron X-ray photons, but decline in energy as they are transported away
from the shock; this may be seen in spectral softening downstream of the shock
\citep[e.g.,][]{cassam-chenai2007}.  The transport process is driven by bulk
plasma advection downstream of the shock and collisionless particle diffusion
\note[Aaron]{src?! need a simple review of plasma diffusion.  What are the
relevant drivers of plasma diffusion?  Does it make sense to think of regimes
of particle interaction -- e.g., thinking of single particle random
interaction, vs. macroscopic MHD waves, or other structure?!} with respect to
bulk advection.  In general, diffusion allows higher energy electrons to
diffuse further upstream or downstream than would be expected from pure
advection.  Finally, the magnetic field strength may be quickly damped behind
the shock, so that electrons may not radiate as efficiently and thin rims would
reflect the spatial structure of the field, rather than efficient particle
acceleration (and losses) and/or efficient synchrotron cooling
\citep{pohl2005}.

% Energy dependence.
Moreover, these controlling mechanisms predict different scalings for rim width
as a function of energy.  Synchrotron loss-limited rims should scale as $w
\propto \nu^{-1/2}$.  Diffusion smears out rims at all energies with stronger
effect at higher energies, yielding a slower drop-off in rim widths when
diffusion dominates.  Magnetic damping, if at a length scale comparable to
filament widths, predicts comparatively energy-independent rim widths --
intuitively, if the magnetic field turns off, synchrotron radiation turns off
regardless of electron energy.  \note[aaron]{this would benefit from
numbers/quantification, I think this statement is stronger if the magnetic
field is stronger -- compare Figure 1 of \citet{pohl2005}}\citet{ressler2014}
give a much fuller exposition of the relative effects of advection, diffusion,
and magnetic field damping on filament rim widths.

% Which one is right?
Which mechanisms are most relevant to historical SNRs?

The primary route of attack has been to just measure the filament widths,
determine relevant advective/diffusive lengthscales, and perhaps favor one or
the other.  Assuming that thin rim widths are limited by synchrotron energy
losses (loss-limited rims), you have $l_{\mt{ad}} \propto v_d \tsynch$
gives a magnetic field...
(how did people look at $\Ecut$ or $D$?)
\citet{bamba2003, vink2003, yamazaki2004, ballet2006}
More sophisticated models have followed, but the basic result is consistent --
magnetic fields are amplified, with some amount of diffusion.
\citet{parizot2006, araya2010} consider diffusion and advection simultaneously
with the catastrophic dump equation.
\citet{berezhko2003, berezhko2004, cassam-chenai2007, morlino2010, rettig2012}
consider diffusion and advection simultaneously with the continuous loss
equation instead.

Some other approaches -- spectral softening \citep{cassam-chenai2007}, CD-FS
gap \citep{warren2005}, multi-wavelength observations \citep{acero2010,
acciari2011, morlino2012} for SN 1006 and Tycho.

Magnetic damping offers a tantalizing alternative explanation and has not
been fully tested.  Equivocal results by \citet{araya2010} for Cas A.
\citet{marcowith2010} did ???
\citet{rettig2012} gave model predictions for several historical SNRs and
proposed discriminating based on filament spectra -- the
expectation is that damped spectra are softer, loss-limited harder.
\citet{ressler2014} developed the idea of investigating energy dependence and
found rim narrowing in the remnant of SN 1006, strongly favoring a loss-limited
model.

% Why Tycho's SNR
To further test these models, we turn to the remnant of Tycho's 1572 supernova
(hereafter, Tycho).  Tycho, being a nearby galactic SNR, is close enough that
its rims are well-resolved by \textit{Chandra}'s spatial resolution.  Tycho has
an extensive and unusually contiguous shell of synchrotron rims around its
periphery, consistent with expansion into a low density ISM; ISM.
\citet{williams2013} favor mean pre-shock ISM density $\abt 0.1$--$0.2
\unit{cm^{-3}}$, consistent with Tycho's X-ray expansion rate
\citep{katsuda2010}.  Although Tycho is expanding into an ISM gradient
\citep{williams2013} and interacting with a molecular cloud on its eastern limb
\citep{reynoso1999}, Tycho's spherical morphology suggests that we might expect
more consistent shock acceleration around its periphery than in other
historical remnants, whether of Type Ia or CC SNe \note[Aaron]{this is not really
justifiable, but I think it's worth noting something to this effect?}.

% Paper roadmap
\note[Aaron]{roadmap}
We measure Tycho's rim widths, following the work of \citet{ressler2014}
(hereafter, \citetalias{ressler2014}) on the remnant of SN 1006.
We make measurements to distinguish between two main scenarios
(explain how diffusion/advection competition come into this).
Our procedure, in both measurement and rim width modeling, follows that of
\citetalias{ressler2014} with only slight modifications.
We first select regions around Tycho's forward shock for analysis, measure
rim widths, and verify that rim spectra are free of thermal line emission.
Using two models for particle transport, we fit measured widths...
We discuss the implications of our fits and models for magnetic shock
amplification (lending credence to previous estimates, in disfavoring magnetic
field amplification), and discuss particle diffusion/acceleration at the shock.
Any further misc. applications (constraints on precursors, etc).


% =============================
% Transport models, observables
% =============================
\section{Nonthermal rim modeling}\label{sec:models}

% TODO Chunk of text I don't know what to do with...
We briefly review relevant particle transport models that we use to fit our
measurements; a fuller review and exposition is given by \citetalias{ressler2014}.
In particular, the transport models
account for both advection and diffusion simultaneously, as opposed to
considering the lengthscales separately with advection dominating below a
critical energy and diffusion, above
\citep{bamba2003, vink2003, yamazaki2004, bamba2005-hist}
\note[Aaron]{clarify, be more specific}.
The effects of varying various parameters, especially in regards to filament
energy dependence, are explored in detail by \citetalias{ressler2014}.


\subsection{Particle transport}\label{sec:transport}

% TODO Choose names for these models and refer to them consistently, throughout
% TODO define ALL symbols, synchrotron constants, etc (somewhere state that CGS
%      notation is being used)
% TODO double check equations, exposition against Rettig/Pohl and others
% TODO review variable definitions when closer to final draft
The energy and space distribution of electrons at a supernova remnant
controls the synchrotron rims we see in X-ray and radio.  We model electron
transport from initial injection at the remnant's forward shock as following
the advection-diffusion equation given by:
\note[Aaron]{\citet{blandford1987} take a different and more thorough tack in
deriving the relevant transport equations, and do include the velocity
divergence term (start w/ Vlasov equation, then transform to non-inertial?
shock frame and tack on assumptions).  What is the best way to present this?
I'm not sure if na\"ively citing generalized ``transport'' is sufficient.}
% TODO fix vector operators
\begin{equation}
  \frac{\ptl f}{\ptl t} + \del \cdot \left( f \vec{v} \right)
  = C + \vec{\del} \cdot \left( D \vec{\del} f \right)
\end{equation}
where $f = f(E,x)$ is electron distribution as a function of electron energy
$E$ and downstream distance $x$ (where the forward shock is at $x=0$ and
$x>0$ towards the remnant's interior); $C$ is an arbitrary source/sink term,
$D$ is the diffusion coefficient, and $\vec{v}$ is a velocity field.
By considering 1-D steady-state plane transport at a constant downstream
velocity $v_d$ \note[Aaron]{we threw out 1. spherical coordinate terms, and 2.
compressible flow terms, I think.  Is the latter justifiable?  Makes sense for
assumption of uniform downstream velocity -- but is this a safe assumption in
practice?  Double check order-of-magnitude...} and a space-independent diffusion
coefficient, we obtain:
\begin{equation}
    v_d \frac{\ptl f}{\ptl x} - D \frac{\ptl^2 f}{\ptl x^2} = C .
\end{equation}
The source/sink term $C$ is determined by our assumptions about the injected
electron energy spectrum and its time/space evolution.

A first approximation is to let electrons have constant energy that is dumped
catastrophically after a synchrotron timescale $\tsynch = 1/(b B^2 E)$, where
$b = 1.57 \times 10^{-3}$ in CGS units and $B$ is magnetic field (which may
vary as $B=B(x)$).  The sink term is $C = -f / \tsynch$, yielding:
\begin{equation} \label{eq:simp-mod}
    v_d \frac{\ptl f}{\ptl x} - D \frac{\ptl^2 f}{\ptl x^2} +
    \frac{f}{\tau_{\mt{synch}}} = 0
\end{equation}
which is equation (5) of \citetalias{ressler2014}.
\note[Aaron]{I think there should be a delta function for e- injection, but it
probably doesn't affect the solution.  Double check.}
Conveniently, equation~\eqref{eq:simp-mod} has analytic solution $f \propto
\exp(-x/a)$ where $a$ is a characteristic scale length.  The full width half
maximum (FWHM) of a thin rim produced by this ``simple'' model is given by:
\begin{equation} \label{eq:simp-fwhm}
    w = \frac{2\beta D / v_d}{\sqrt{1 +\frac{4D}{v_d^2 \tsynch}} - 1}
\end{equation}
where $\beta$ is a projection factor relating $w$ to $a$ as $w = \beta a$.  We
take $\beta = 4.6$ assuming projection from spherical shell, as derived by
\citet{ballet2006}.

A more advanced model is given by accounting for continuous radiation energy
losses while electrons are transported downstream:
\begin{equation} \label{eq:full-mod}
    v_d \frac{\ptl f}{\ptl x} - D \frac{\ptl^2 f}{\ptl x^2} =
    K_0 E^{-s} e^{-E/\Ecut} \delta(x) + \frac{\ptl}{\ptl E}
      \left(bB^2E^2f\right)
\end{equation}
which is equation (12) of \citetalias{ressler2014}.  The injected electron
distribution is an exponentially cut-off power law with normalization $K_0$ and
cut-off energy $\Ecut$; the cut-off energy is determined by the form of the
diffusion coefficient $D$, which we discuss in
Section~\ref{sec:diffcoeff}.  To determine rim profiles and widths, we
numerically solve for electron distribution $f(E,x)$ using Green's function
solutions due to \citet{lerche1980} and \citet{rettig2012}; the equations are
given using notation close to that of this paper by \citetalias{ressler2014}.
The electron distribution may then be integrated to obtain synchrotron
emissivity:
\begin{equation} \label{eq:emissivity}
    j_{\nu}(x) \propto \int_0^\infty G(y) f(E,x) dE
\end{equation}
where $y \equiv \nu/(c_1 E^2 B)$ is a scaled synchrotron frequency and
$G(y) = y \int_y^\infty K_{5/3}(z) dz$ is one-particle synchrotron
emissivity with $K_{5/3}(z)$ a modified Bessel function of the second kind
\citep{pacholczyk1970}.  Integrating emissivity over lines of sight for a
spherical shell yields intensity:
\begin{equation} \label{eq:intensity}
    I_{\nu}(r) = 2 \int_0^{\sqrt{r_s^2 - r}}
                    j_{\nu} \left( r_s - \sqrt{s^2 + r^2} \right) ds
\end{equation}
where $s$ is the line-of-sight coordinate.  We can then compute these
integrals (equations~\eqref{eq:emissivity},~\eqref{eq:intensity}) to obtain
model predictions for rim FWHMs.

In what follows we consider both transport models, but primarily present
results from the more advanced model (equation~\eqref{eq:full-mod}) as we find
the qualitative behavior of both models is comparable.

\subsection{Diffusion energy-dependence} \label{sec:diffcoeff}

The modus operandi of previous work has been to assume Bohm diffusion, or at
least Bohm diffusion-energy scaling.  Bohm diffusion makes the
assumption that particle mean free path, and hence the diffusion coefficient,
is proportional to the particle gyroradius, yielding $D \propto E$.

\citet{reynolds2004} suggested that physically motivated models of MHD
turbulence could give rise to non-linear diffusion-energy scalings, assuming
quasi-linear turbulence \note[Aaron]{read more}.
Specifically, power-law energy-wavenumber spectra for Kolmogorov and Kraichnan
turbulence predict $D \propto E^{1/3}$ and $D \propto E^{1/2}$ respectively.

We consider a generalized diffusion coefficient that allows arbitrary
power-law dependence of diffusion upon energy:
\begin{equation} \label{eq:diffcoeff}
    D(E) = \eta C_d E^\mu / B = \eta_h D(E_h) (E/E_h)^\mu
\end{equation}
As the diffusion coefficient is energy dependent, we rewrite the diffusion
coefficient in terms of a fiducial energy $E_h = 2 \unit{keV}$ throughout, with
$\eta_2 = \eta_h$.  The constant $\eta_2$ is the diffusion coefficient at
$2 \unit{keV}$ scaled by the Bohm diffusion value.
\note[Aaron]{Also give some intuition of relation of $D$ and $\Ecut$ to relate to
previous studies.}

Explain how the transport models constrain diffusion via $\mE$, define $\mE$.
Explain the basic interplay between diffusion and magnetic field, in
controlling rim widths.

With a diffusion coefficient in hand, we determine the electron spectrum
cut-off energy by equating synchrotron loss and acceleration timescales;
the acceleration timescale is determined from the theory of diffusive shock
acceleration.  The cut-off energy, following \citet{parizot2006}, is given as:
\begin{align*}
    \Ecut =
        &\left(8.3\unit{TeV}\right)^{2/(1+\mu)}
        \left(\frac{B_0}{100 \unit{\mu G}}\right)^{-1/(1+\mu)} \\
        &\times \left(\frac{v_s}{10^8 \unit{cm/s}}\right)^{2/(1+\mu)}
        \left[
            \frac{E_h^{\mu - 1}}{\eta}
        \right]^{1 / (1 + \mu)}
\end{align*}
\note[Aaron]{This differs from Sean's presentation -- I drew this from the
Fortran code and it looks consistent w/ Parizot et al., but I haven't vetted
the prefactor $8.3 \unit{TeV}$.  And, I have an extra exponent $\mu - 1$ on the
$E_h$ ?!  Come back to this shortly.}


\subsection{Magnetic damping}

We also consider (not really) a magnetically damped field of form
\begin{equation}
    B(x) = B_{\mt{min}} + \left(B_0 + B_{\mt{min}}\right) \exp\left(-x / a_b\right) ,
\end{equation}
following \citetalias[Section 3.2]{ressler2014} and \citet{pohl2005}.
\note[Aaron]{Look over \citet{marcowith2010}.}

Set up the calculation w/ a range of scale lengths (with $B_0$,
$\eta_2$ chosen so that damping is the dominant control for $0.7$--$7
\unit{keV}$), and compute FWHMs and $\mE$ values?


\subsection{Previous studies of Tycho}

\note[Aaron]{Should we follow \citet{ressler2014} in adding a table of prior
magnetic field estimates?  I am uncertain of where to review previous
literature -- introduction gives more direct motivation and context, putting
before or in-line w/models allows us to explain who used what.  Putting
before/after model exposition, keeps all the numbers/explanation of previous
work in one place -- also easier to pull in discussion of other
methods/attacks.  Right now text is rather disjointed.}
\citet{parizot2006} derive magnetic fields of $300$--$500 \unit{\mu G}$; they
also give both diffusion coefficients (in their $k_0$?) and $\Ecut$ values.
\citet{rettig2012} used a measured filament lengthscale and considered
loss-limited and magnetic damping models separately for several historical
SNRs.  For Tycho: they report $B = 310 \unit{\mu G}$ with $\Ecut = 24
\unit{TeV}$; since $\Ecut$ is set by balancing advective/diffusive
lengthscales, this is equivalent to our reporting of a diffusion coefficient
(?).  With magnetic damping, they report $B = 150 \unit{\mu G}$ and a larger
$\Ecut = 34 \unit{TeV}$.  Taken from \citet{rettig2012}: \citet{acciari2011}
give $80$, $230 \unit{\mu G}$ for leptonic/hadronic models for gamma-ray
emission in Tycho.


% ============
% Observations
% ============
\section{Observations}
\label{sec:observations}

% NOTE go through and double check tenses.
We measured synchrotron rim full widths at half maximum (FWHMs) from an
archival \Chandra ACIS-I observation of Tycho
(RA: 00\tsup{h}25\tsup{m}19\fs0, dec: +64\arcdeg08\arcmin10\farcs0; J2000)
between 2009 Apr 11 and 2009 May 5 (PI: John P. Hughes;
\dataset[ADS/Sa.CXO\#obs/10093--10097]{ObsIDs: 10093--10097},
\dataset[ADS/Sa.CXO\#obs/10902--10906]{10902--10906}).
The total exposure time was $734 \unit{ks}$.
Level 1 \Chandra data were reprocessed with CIAO 4.6 and CALDB 4.6.1.1 and kept
unbinned with ACIS spatial resolution $0.492\arcsec$.
Merged and corrected events were divided into five energy bands:
0.7--1 keV, 1--1.7 keV, 2--3 keV, 3--4.5 keV, and 4.5--7 keV.
We excluded 1.7--2 keV energy range to avoid the \ion{Si}{13} (He$\alpha$)
emission line at 1.85 keV, prevalent in the remnant's thermal ejecta, which
\remove[Brian]{could} \remove[Aaron]{would} \add[Aaron]{may}
\note[Aaron]{since some profiles are practically uncontaminated?}
contaminate our nonthermal profile measurements.

We selected 13 regions in 5 distinct filaments around Tycho's shock
(Figure~\ref{fig:snr}) based on the following criteria:
(1) filaments should be clear of spatial plumes of thermal ejecta in \Chandra
images; this rules out, e.g., areas of strong thermal emission on Tycho's eastern limb.
(2) filaments should be singular and localized; that is, multiple filaments
should either not overlap or completely overlap (rules out parts of NE limb);
(3) filaments should have clear FWHMs bands; i.e., a peak should be evident
above the background signal or downstream thermal emission (rules out some
faint southern wisps) \note[Aaron]{wishy-washy qualitative}.
We grouped regions into filaments by visual inspection of the remnant.

% NOTE generate final, higher quality figure + newer figure for regions-5
\begin{figure}
    \centering
    \plotone{figures/f0-snr-inv.png}
    \caption{RGB image of Tycho with region selections overlaid.  Bands are
    0.7--1 keV (red), 1--2 keV (green) and 2--7 keV (blue).
    \note[Aaron]{Could show the spectrum extraction regions on here too?
    Maybe too cluttered.  Figure colors inverted to save ink.}}
    \label{fig:snr}
\end{figure}

All measured rim widths are at least $\abt 1\arcsec$ and hence are resolved by
\textit{Chandra}'s point-spread function.
Looking at proposer's guide, Figure 4.13 -- the off-axis PSF should
still have FWHM of $\abt 1\arcsec$ at $4\arcmin$ (approx. Tycho's radius).
Is that good enough for us?
(Cassam-Chena\"i et al. considered PSF effects when modeling rim widths, though
it doesn't look like it made a big difference?)
If the PSF matters, this favors some combination of stronger
magnetic fields and weaker diffusion.

% ----------------
% Filament spectra
% ----------------
\subsection{Filament spectra}
\label{sec:spec}

% NOTE be consistent in usage of word 'sections'
We extracted spectra for all regions to check that rim width measurements are
not affected by contaminating thermal line emission.  In each region, we
selected upstream and downstream sections from which to extract spectra.  The
upstream section is the smallest sub-region that contains the measured FWHM
bounds from all energy bands (see Section~\ref{sec:fwhms}); i.e., this region
captures the thin rim.  The downstream section extends from the back of the rim
(end of the upstream section) to the rim model fit's (equation~\eqref{eq:prof})
downstream limit.  We merged spectra from all \Chandra ObsIDs and grouped the
data into bins of $\geq 15$ counts.  Figure~\ref{fig:spec} plots an example
profile (4.5--7 keV) with the downstream and upstream sections highlighted; the
dividing line is set at the \textbf{???--???} energy band's downstream FWHM
limit here (see Figure~\ref{fig:profiles}).
We extracted background spectra from circular regions (radius $\abt 30\arcsec$)
around the remnant's exterior; each thin rim region was matched to the closest
background region's spectrum.
% Background region size is valid for data-tycho/bkg-2/ selections

% NOTE must match fig:profiles!  Keep region numbers updated and
% consistent.  Currently, region 1 of data-tycho/regions-4/ selections
\begin{figure*}
    \plotone{figures/fig-specs.pdf}
    \caption{Spectra and fits from Region 1. Left: $4.5$--$7 \unit{keV}$
    profile with highlighted downstream (blue) and upstream (grey) sections.
    Intensity units are arbitrary (a.u.).  Middle: downstream spectrum with
    absorbed power law fit; Si and S lines at $1.85$, $2.45 \unit{keV}$ are
    clearly visible.  Right: upstream spectrum with absorbed power law fit
    shows that the filament is likely free of thermal line emission.}
    \label{fig:spec}
\end{figure*}

We fit $0.5$--$7 \unit{keV}$ spectra from each region's upstream and downstream
sections to an absorbed power law model (XSPEC 12.8.1, \texttt{phabs*po}) with
spectral index $n$, hydrogen column density $n_H$, and a normalization as free
parameters.  Table~\ref{tab:spec} lists best fit parameters and reduced
$\chi^2$ values for all regions.  Spectra from thin rims (upstream sections)
are well-fit by the power law model alone; spectra from sections downstream of
the rims are generally poorly-fit, reflecting thermal contamination primarily
from \ion{Si}{13} and \ion{S}{15} He$\alpha$ line emission at $1.85$ and $2.45$
keV.  Although most downstream spectra show thermal contamination, some regions
(e.g., 4, 5 on Tycho's northwestern limb) are reasonably well-fit downstream by
an absorbed power law as well due to a larger distance between the forward
shock and contact discontinuity.
Upstream spectra fits spectral indices between $2.5$--$3$ are consistent with
synchrotron radiation spectra for ???? population of electrons
\note[Aaron]{citation for e- distr. indices?}. Best fit column densities from
upstream spectra between $0.61$--$0.78 \times 10^{22} \unit{cm^{-2}}$ are
consistent with previous fits to Tycho's nonthermal rims \citep{hwang2002} but
somewhat larger than $\abt0.35$--$0.5 \times 10^{22} \unit{cm^{-2}}$ from radio
and optical??  HI absorption line measurements \citep{black1984, albinson1986,
kothes2004}.
% NOTE keep stated ranges of spectral indices, column densities up to date

\note[Aaron]{On Rob's suggestion of fitting thermal model -- can we get away
with just fitting line emission and constraining from there? or is the
whole NEI apparatus needed.  If we add thermal components -- report spectral
indices and reconfirm spectral softening?}

% NOTE keep region numbers updated (in discussion of good/bad downstream spectra)
\begin{table}
    \scriptsize
    \centering
    \caption{Region spectra fit parameters\label{tab:spec}}
    \begin{tabular}{@{}lcccccr@{}}
\toprule
{} & \multicolumn{3}{c}{Downstream spectra}
   & \multicolumn{3}{c}{Upstream spectra} \\
\cmidrule(lr){2-4} \cmidrule(l){5-7}
Region & $n$ & $n_H$ & $\chi^2_{\mathrm{red}}$ (dof)
       & $n$ & $n_H$ & $\chi^2_{\mathrm{red}}$ (dof) \\
{} & (-) & ($\mt{cm}^{-2}$) & {}
   & (-) & ($\mt{cm}^{-2}$) & {} \\
\midrule
1 & 2.97 & 0.68 & 2.27 (272) & 2.77 & 0.78 & 0.92 (239) \\
2 & 2.91 & 0.64 & 3.55 (163) & 2.54 & 0.67 & 1.05 (232) \\
3 & 3.00 & 0.60 & 3.41 (181) & 2.75 & 0.66 & 1.13 (245) \\
4 & 2.94 & 0.45 & 1.35 (199) & 2.88 & 0.66 & 0.92 (224) \\
5 & 2.90 & 0.50 & 1.15 (224) & 2.83 & 0.68 & 0.96 (246) \\
6 & 2.85 & 0.43 & 1.87 (194) & 3.00 & 0.63 & 1.20 (222) \\
7 & 2.80 & 0.37 & 1.43 (100) & 2.79 & 0.61 & 1.11 (243) \\
8 & 2.79 & 0.55 & 2.36 (183) & 2.83 & 0.71 & 1.22 (285) \\
9 & 2.99 & 0.68 & 4.43 (239) & 2.77 & 0.70 & 1.01 (252) \\
10 & 2.89 & 0.58 & 1.31 (186) & 2.86 & 0.71 & 1.27 (301) \\
11 & 2.89 & 0.61 & 4.92 (231) & 2.91 & 0.72 & 1.16 (281) \\
12 & 3.03 & 0.65 & 1.02 (156) & 2.91 & 0.78 & 0.97 (271) \\
13 & 2.97 & 0.76 & 1.39 (181) & 2.72 & 0.75 & 0.96 (217) \\
\bottomrule
\end{tabular}

    \tablecomments{Absorbed power law fit parameters are photon index $\Gamma$
    and hydrogen column density $\mathrm{N_H}$.}
\end{table}

Table~\ref{tab:spec} confirms that our region selections are practically free
of thermal line emission, as already suggested by visual inspection
(Figure~\ref{fig:snr}).  Our exclusion of all $1.7$--$2
\unit{keV}$ photons should further limit any thermal contamination, as
the $1.85 \unit{keV}$ Si line emission is over a third of Tycho's thermal flux
as detected by \Chandra \citep{hwang2002}.  We may safely proceed to model our
measured rim widths as being due entirely to nonthermal radiation
\note[Aaron]{bad phrasing}.

% --------------------------
% FWHM measurement procedure
% --------------------------
\subsection{Filament width measurements}
\label{sec:fwhms}

% NOTE keep mentioned number of regions updated!
% TODO regions 2,3; 6,7 should be stretched forward a little more in regions-5
%      just to be more consistent (just from eyeballing...)
% Update these numbers before final version.
We obtained radial intensity profiles from $\abt 10$--$20\arcsec$ behind the
shock to $\abt 2$--$8\arcsec$ in front for each of our 13 regions in all five
energy bands.  To increase signal-to-noise, we integrate along the shock ($\abt
8$--$24\arcsec$) in each region.  Plotted and fitted profiles are reported in
vignetting and exposure-corrected, but arbitrary, intensity units; error bars
are computed from raw count data assuming Poisson statistics.  Intensity
profiles peak sharply (within $\abt 2$--$3\arcsec$) behind the shock,
demarcating our thin rims, and then fall off until thermal emission from ejecta
picks up at Tycho's contact discontinuity \citep{warren2005}.

We fitted rim profiles, obtained by integrating intensity along the shock in
each region, to a piecewise two-exponential model:
\begin{equation} \label{eq:prof}
    h(x) =
    \begin{cases}
        A_u \exp \left(\frac{x_0 - x}{w_u}\right) + C_u, &x \geq x_0 \\
        A_d \exp \left(\frac{x - x_0}{w_d}\right) + C_d, &x < x_0
    \end{cases}
\end{equation}
where $h(x)$ is profile height and $x$ is radial distance from remnant center.
The rim model, which we emphasize is strictly empirical, has 6 free parameters:
$A_u, x_0, w_u, w_d, C_u$, and $C_d$ with $A_d = A_u + (C_u - C_d)$ enforcing
continuity at $x=x_0$. Our model is similar to that of \citet{bamba2003,
bamba2005-hist} and differs slightly from that of \citetalias{ressler2014}.
To fit only the nonthermal rim in each intensity profile, we selected the fit
domain for each profile as follows.  The downstream bound was set at the first
local data minimum downstream of the rim peak, as identified by smoothing the
profiles with a 21-point ($\abt 10\arcsec$) Hanning window.  The upstream bound
was set at the profile's outer edge (i.e., no data were removed).

\begin{figure*}[ht]
    \plotone{figures/fig-profiles.pdf}
    \caption{Best fit profiles with measured FWHMs demarcated for each energy
             band in Region 1.  Red data points are excluded from profile fit
             domain.}
    \label{fig:profiles}
\end{figure*}

% NOTE keep blacklisted/excluded region numbers updated!
From the fitted profiles we extracted a full width at half maximum (FWHM) for
each region and each energy band.  We could not resolve a FWHM in regions 2, 6,
and 8 at 0.7--1 keV (Table~\ref{tab:fwhms}); in these regions, either
the downstream FWHM bound would extend outside the fit domain or we could not
find an acceptable fit to equation~\eqref{eq:prof}.  We were able to resolve
FWHMs for all regions at higher energy bands (1--7 keV).
% NOTE review the FWHM rejection criterion, on the final set of regions

To estimate FWHM uncertainties, we horizontally stretched each best-fit
profile by mapping radial coordinate $x$ to
$x'(x) = x (1 + \xi (x-x_0)/(50\arcsec-x_0))$ with $\xi$ an arbitrary stretching
parameter and $x_0$ the best-fit rim center from equation~\eqref{eq:prof};
this yields a new profile $h'(x) = h(x'(x))$.
We varied $\xi$ (and hence rim FWHM) to vary each profile fit $\chi^2$ by 2.7
and the stretched or compressed FWHMs as upper/lower bounds on our reported
FWHMs. \note[Aaron]{Unimportant note: chi-squared values from our
fits assume that input errors are 1-sigma.  But, taking $\Delta\chi^2 = 1$
for our FWHM errors would shrink the errors even further, and the errors
don't reflect uncertainty that may arise from, e.g., using different fit
functions or taking different profiles.}


% -------------
% Model fitting
% -------------
\subsection{Filament model fitting}
\label{sec:fits}

% Fit data to models
We fit the two transport models given by Equations~\eqref{eq:simp-mod} and
\eqref{eq:full-mod} to our measured rim widths as a function of energy.
We assigned each width measurement to the lower energy limit of its energy band
(e.g., $0.7$--$1 \unit{keV}$ is assigned to $0.7 \unit{keV}$ and fitted to
model profile widths at $0.7 \unit{keV}$) \note[Aaron]{bad wording}.
We also averaged the positive and negative error bars on each FWHM measurement
(Table~\ref{tab:fwhms}) for least squares fitting.
% Averaging/merging/synthesizing, somehow, our results
Rim widths from each region are fitted separately; we consider filament and
remnant-wide physical parameters by averaging best fit parameters for each
region in Section~\ref{sec:fit-results}.

% Model knobs and how we twiddled them
We varied three physical parameters (more if we consider magnetic damping):
diffusion-energy scaling exponent $\mu$, normalized diffusion
coefficient $\eta_2$, and magnetic field strength $B_0$.
To make nonlinear fitting tractable, we fixed $\mu$ in all fits and considered
$\mu = 0$, $1/3$, $1/2$, $1$, $1.5$, and $2$;
recall that $\mu = 1/3$ and $1/2$ correspond to Kolmogorov and Kraichnan turbulence.
As $\eta_2$ is highly variable in our fits, we also performed
fits with $\eta_2 = 1$ fixed, coincident with Bohm diffusion at $2 \unit{keV}$.

A few remnant-specific parameters enter into the model calculations.  We take
electron spectral index $s = 2.3$ (from radio spectral index $\alpha = 0.65$,
\citet{kothes2006} \note[Aaron]{not using $0.58$ from Green's catalog?!}),
remnant distance $3 \unit{kpc}$ \citep[cf.][]{hayato2010}, and
shock radius $1.08 \times 10^{19} \unit{cm}$ from angular radius $240\arcsec$
\citep{green2009}.  Tycho's forward shock velocity varies with azimuth by up to
a factor of 2; we interpolated shock velocities reported by
\citet{williams2013} (rescaled to $3 \unit{kpc}$ rather than $2.3 \unit{kpc}$
to estimate individual shock velocities for each region.
\note[Aaron]{this paragraph might belong in an earlier section?}

% How to do the fit
Equation~\eqref{eq:simp-fwhm} was directly fit to our measurements to obtain
best-fit values of $\eta_2$ and $B_0$.
The full model was numerically computed as detailed in
Section~\ref{sec:transport}, yielding intensity profiles and hence model FWHMs
in each energy band; we then varied the input parameters ($\eta_2$, $B_0$) to
obtain a least squares best fit.
To assist the nonlinear fitting, we tabulated a large grid of model FWHM values
for various values of $\mu$ and shock velocity $v_s$; from a best initial grid
value, we then used a Levenberg-Marquardt fitting routine to determine the best
fits.
For both simple and full fits, we deemed fits with $\eta_2 \geq 10^5$ and $B_0
\geq 10 \unit{mG}$ effectively unconstrained; $\eta_2$ is also constrained to
be positive .

% TODO modify tables to match what is stated here -- in regards to parameters
% being effectively unbounded.
% TODO for extreme values in tables -- go back and double check chi^2 space
As the full model must be numerically solved, its predicted rim widths are
subject to resolution error in the numerical integrals.  We chose our
integration resolutions such that the fractional error in model FWHMs
associated with halving/doubling integration resolution is less than $1\%$ for
the parameter space relevant to our filaments.
\note[Aaron]{rewrite -- unclear.  Need to verify 1\% claim and check Pacholczyk
table resolution (and possibly Bessel function numbers)...}

% Errors
Finally, we computed errors on our best fit parameters by varying each
parameter and obtaining a new best fit model, with one less degree of freedom,
to find the limit s.t. $\Delta \chi^2 = 1$ to obtain a roughly 1-sigma error.
Granted, this is not guaranteed correct for nonlinear fits (that $\Delta \chi^2
= 1$ corresponds to a $68.3$\% confidence limit), but should be good enough.

% =======================
% Results, FWHMs and fits
% =======================
\section{Results}

% --------------------
% FWHM results, tables
% --------------------
\subsection{Rim widths}
\label{sec:fwhm-results}

In Table~\ref{tab:fwhms} we report FWHM measurements for all of our regions.
As previously noted, we could not resolve FWHMs for some regions in the 0.7--1
keV band.  We also report $\mE$ values for all but the lowest energy band,
computed point-to-point as: \begin{equation} \mE(E_2) =
\frac{\ln(w_2/w_1)}{\ln(E_2/E_1)} \end{equation} where $w_1, w_2$ and $E_1,
E_2$ are FWHMs and lower energy values for each energy band (e.g., $\mE$ at $1
\unit{keV}$ is computed using FWHMs from $0.7$--$1 \unit{keV}$ and $1$--$1.7
\unit{keV}$, and taking $E_2 = 1 \unit{keV}$ and $E_1 = 0.7 \unit{keV}$.
Although the values of $\mE$ reflect a good deal of variability in the
underlying width measurements, we see that most regions and energy bands
display a decrease in rim widths as a function of energy.  NOTE (draft): we
emphasize that $\mE$ is currently not considered or utilized in our subsequent
analysis.

\begin{table*}[ht]
    \tiny
    \centering
    \caption{Measured FWHMs for all regions.\label{tab:fwhms}}
    \begin{tabular}{@{}l ccccc r@{ $\pm$ }l r@{ $\pm$ }l r@{ $\pm$ }l r@{ $\pm$ }l @{}}

\toprule
{} & \multicolumn{5}{c}{FWHM (arcsec)} & \multicolumn{8}{c}{$\mE$ (-)} \\
\cmidrule(lr){2-6} \cmidrule(l){7-14}
Region & Band 1 & Band 2 & Band 3 & Band 4 & Band 5
       & \multicolumn{2}{c}{Bands 1--2} & \multicolumn{2}{c}{Bands 2--3}
       & \multicolumn{2}{c}{Bands 3--4} & \multicolumn{2}{r}{Bands 4--5} \\ [0.2em]
{} & (0.7--1 keV) & (1--1.7 keV) & (2--3 keV) & (3--4.5 keV) & (4.5--7 keV)
   & \multicolumn{2}{c}{(1 keV)} & \multicolumn{2}{c}{(2 keV)}
   & \multicolumn{2}{c}{(3 keV)} & \multicolumn{2}{r}{(4.5 keV)} \\
\midrule
1 & {} & ${8.80}^{+0.18}_{-0.15}$ & ${6.34}^{+0.26}_{-0.21}$ & ${7.40}^{+0.30}_{-0.23}$ & ${5.57}^{+0.47}_{-0.42}$
  & \multicolumn{2}{c}{} & $-0.47$ & $0.06$ & $0.38$ & $0.13$ & $-0.70$ & $0.22$ \\ [0.5em]
2 & {} & ${4.22}^{+0.12}_{-0.09}$ & ${2.36}^{+0.12}_{-0.09}$ & ${3.00}^{+0.16}_{-0.12}$ & ${4.11}^{+0.34}_{-0.30}$
  & \multicolumn{2}{c}{} & $-0.84$ & $0.08$ & $0.59$ & $0.16$ & $0.77$ & $0.23$ \\ [0.5em]
3 & ${1.72}^{+0.13}_{-0.10}$ & ${2.47}^{+0.08}_{-0.07}$ & ${1.78}^{+0.09}_{-0.07}$ & ${2.10}^{+0.11}_{-0.11}$ & ${1.32}^{+0.10}_{-0.09}$
  & $1.01$ & $0.21$ & $-0.47$ & $0.08$ & $0.41$ & $0.17$ & $-1.15$ & $0.22$ \\

\cmidrule{1-14}
4 & ${5.85}^{+0.37}_{-0.33}$ & ${4.35}^{+0.09}_{-0.08}$ & ${3.26}^{+0.11}_{-0.09}$ & ${3.69}^{+0.12}_{-0.11}$ & ${3.20}^{+0.21}_{-0.18}$
  & $-0.83$ & $0.18$ & $-0.41$ & $0.05$ & $0.31$ & $0.11$ & $-0.35$ & $0.17$ \\ [0.5em]
5 & ${9.12}^{+0.45}_{-0.40}$ & ${4.52}^{+0.11}_{-0.12}$ & ${3.06}^{+0.11}_{-0.11}$ & ${3.25}^{+0.15}_{-0.13}$ & ${3.04}^{+0.21}_{-0.18}$
  & $-1.97$ & $0.15$ & $-0.56$ & $0.06$ & $0.15$ & $0.14$ & $-0.17$ & $0.19$ \\ [0.5em]
6 & ${2.48}^{+0.18}_{-0.18}$ & ${2.32}^{+0.05}_{-0.06}$ & ${2.98}^{+0.11}_{-0.09}$ & ${2.05}^{+0.08}_{-0.09}$ & ${2.21}^{+0.15}_{-0.14}$
  & $-0.19$ & $0.21$ & $0.36$ & $0.06$ & $-0.92$ & $0.13$ & $0.18$ & $0.19$ \\ [0.5em]
7 & ${2.69}^{+0.20}_{-0.17}$ & ${2.33}^{+0.05}_{-0.05}$ & ${2.31}^{+0.08}_{-0.08}$ & ${1.81}^{+0.09}_{-0.07}$ & ${1.83}^{+0.11}_{-0.08}$
  & $-0.39$ & $0.20$ & $-0.01$ & $0.06$ & $-0.60$ & $0.14$ & $0.02$ & $0.17$ \\ [0.5em]
8 & ${2.33}^{+0.21}_{-0.20}$ & ${2.72}^{+0.08}_{-0.08}$ & ${2.38}^{+0.10}_{-0.09}$ & ${2.10}^{+0.10}_{-0.09}$ & ${2.37}^{+0.20}_{-0.17}$
  & $0.43$ & $0.26$ & $-0.19$ & $0.07$ & $-0.30$ & $0.15$ & $0.29$ & $0.22$ \\ [0.5em]
9 & ${2.16}^{+0.24}_{-0.23}$ & ${2.35}^{+0.07}_{-0.06}$ & ${2.47}^{+0.11}_{-0.11}$ & ${1.91}^{+0.09}_{-0.09}$ & ${2.20}^{+0.17}_{-0.16}$
  & $0.24$ & $0.31$ & $0.07$ & $0.07$ & $-0.63$ & $0.16$ & $0.34$ & $0.22$ \\ [0.5em]
10 & ${2.38}^{+0.24}_{-0.23}$ & ${1.99}^{+0.07}_{-0.06}$ & ${1.76}^{+0.09}_{-0.08}$ & ${1.59}^{+0.09}_{-0.08}$ & ${1.58}^{+0.13}_{-0.12}$
  & $-0.50$ & $0.29$ & $-0.18$ & $0.08$ & $-0.24$ & $0.18$ & $-0.02$ & $0.23$ \\

\cmidrule{1-14}
11 & ${2.30}^{+0.36}_{-0.26}$ & ${3.23}^{+0.15}_{-0.13}$ & ${2.52}^{+0.16}_{-0.13}$ & ${1.90}^{+0.14}_{-0.13}$ & ${3.09}^{+0.45}_{-0.38}$
  & $0.95$ & $0.40$ & $-0.36$ & $0.10$ & $-0.70$ & $0.22$ & $1.21$ & $0.37$ \\ [0.5em]
12 & {} & ${3.86}^{+0.17}_{-0.16}$ & ${2.61}^{+0.15}_{-0.13}$ & ${3.02}^{+0.22}_{-0.21}$ & ${2.23}^{+0.21}_{-0.17}$
  & \multicolumn{2}{c}{} & $-0.56$ & $0.10$ & $0.36$ & $0.22$ & $-0.74$ & $0.27$ \\ [0.5em]
13 & ${2.85}^{+0.22}_{-0.17}$ & ${2.43}^{+0.05}_{-0.05}$ & ${2.36}^{+0.08}_{-0.05}$ & ${1.95}^{+0.09}_{-0.10}$ & ${1.84}^{+0.11}_{-0.14}$
  & $-0.45$ & $0.20$ & $-0.04$ & $0.05$ & $-0.47$ & $0.13$ & $-0.15$ & $0.20$ \\

\cmidrule{1-14}
14 & ${2.86}^{+0.17}_{-0.16}$ & ${2.42}^{+0.06}_{-0.04}$ & ${2.23}^{+0.08}_{-0.07}$ & ${2.38}^{+0.10}_{-0.08}$ & ${2.19}^{+0.12}_{-0.10}$
  & $-0.47$ & $0.17$ & $-0.12$ & $0.06$ & $0.17$ & $0.12$ & $-0.20$ & $0.15$ \\ [0.5em]
15 & ${2.71}^{+0.17}_{-0.16}$ & ${1.99}^{+0.05}_{-0.04}$ & ${1.80}^{+0.06}_{-0.05}$ & ${1.87}^{+0.07}_{-0.05}$ & ${1.52}^{+0.09}_{-0.08}$
  & $-0.85$ & $0.18$ & $-0.15$ & $0.05$ & $0.09$ & $0.11$ & $-0.51$ & $0.16$ \\ [0.5em]
16 & ${1.87}^{+0.14}_{-0.13}$ & ${1.73}^{+0.04}_{-0.03}$ & ${1.52}^{+0.06}_{-0.05}$ & ${1.25}^{+0.06}_{-0.04}$ & ${1.23}^{+0.08}_{-0.06}$
  & $-0.22$ & $0.21$ & $-0.18$ & $0.06$ & $-0.49$ & $0.13$ & $-0.04$ & $0.17$ \\ [0.5em]
17 & ${1.65}^{+0.13}_{-0.12}$ & ${1.92}^{+0.05}_{-0.05}$ & ${1.54}^{+0.06}_{-0.07}$ & ${1.45}^{+0.07}_{-0.06}$ & ${2.05}^{+0.16}_{-0.14}$
  & $0.43$ & $0.22$ & $-0.31$ & $0.07$ & $-0.16$ & $0.15$ & $0.86$ & $0.21$ \\

\cmidrule{1-14}
18 & {} & ${4.45}^{+0.13}_{-0.12}$ & ${3.18}^{+0.17}_{-0.16}$ & ${2.96}^{+0.20}_{-0.19}$ & ${1.65}^{+0.21}_{-0.16}$
  & \multicolumn{2}{c}{} & $-0.49$ & $0.09$ & $-0.17$ & $0.21$ & $-1.45$ & $0.32$ \\ [0.5em]
19 & ${7.41}^{+0.62}_{-0.46}$ & ${2.30}^{+0.08}_{-0.06}$ & ${2.28}^{+0.11}_{-0.08}$ & ${2.16}^{+0.12}_{-0.11}$ & ${1.60}^{+0.17}_{-0.14}$
  & $-3.27$ & $0.22$ & $-0.02$ & $0.08$ & $-0.13$ & $0.17$ & $-0.74$ & $0.27$ \\ [0.5em]
20 & ${4.81}^{+0.31}_{-0.31}$ & ${1.84}^{+0.06}_{-0.03}$ & ${1.87}^{+0.08}_{-0.06}$ & ${1.56}^{+0.07}_{-0.06}$ & ${2.14}^{+0.23}_{-0.23}$
  & $-2.68$ & $0.19$ & $0.02$ & $0.07$ & $-0.44$ & $0.14$ & $0.77$ & $0.28$ \\
\midrule
Mean & $3.45 \pm 0.55$ & $3.11 \pm 0.37$ & $2.53 \pm 0.23$ & $2.47 \pm 0.30$ & $2.35 \pm 0.23$
  & $-0.55$ & $0.30$ & $-0.25$ & $0.06$ & $-0.14$ & $0.10$ & $-0.09$ & $0.15$ \\

\bottomrule
\end{tabular}
\tablecomments{Mean values computed for all regions; mean $\mE$ values are
averages for region $\mE$ values (i.e., not computed from mean FWHMs).  Errors
on mean values are standard errors of the mean.  Horizontal rules group
individual regions into filaments.}
\tablecomments{(DRAFT) In the next iteration of the data pipeline, I will cull
the FWHMs by hand. 8 of 22 regions (1--3, 5, 11--12, 18--19) will not have
FWHMS at 0.7--1 keV.}

\end{table*}

% -------------------------
% Model fit results, tables
% -------------------------
\subsection{Model fit results}
\label{sec:fit-results}

% TODO do all the work, but with different FWHM calculation -- to confirm
%      that it doesn't make a big difference...

% NOTE keep updated -- for which region are we giving simple and full fits?
We find that the simpler catastrophic dump model (equation~\eqref{eq:simp-mod})
produces the same qualitative trends seen in the full continuous energy loss
model.  The first row of Table~\ref{tab:fits} compares simple and full model fits for
Region 1.  In general, simple model fits are less poorly constrained (i.e.,
fits are more likely to reach manual fitting limits).
Where both simple and full model fits are well constrained (i.e., moderate to
small $\eta_2$ values), the simple model favors slightly smaller $\eta_2$ and
slightly larger $B_0$.  But this is not always true, and it is difficult to
identify any obvious trends as both fits are quite variable across $\mu$ values
and different regions.
\note[Aaron]{from the table calculations, the simple model needed larger $B_0$
values than the full model for a fixed $\eta_2$ and mean width (roughly
speaking); this effect strengthened with larger $\eta_2$.  We're looking at
order 10--50\% effects}.
For the rest of our results and discussion, we consider only full model fits.

% Individual region fit results
Table~\ref{tab:fits} presents full model fit results for all regions
considered. Errors are computed by varying free parameters $B_0$,
$\eta_2$ to obtain $\Delta\chi^2 = 1$, corresponding to $1$-$\sigma$ confidence
limits on our fit parameters.
In Figure~\ref{fig:fits-all} we also plot the best full model fits for the same
regions tabulated in Table~\ref{tab:fits}

\begin{table}[ht]
    \tiny
    \centering
    \caption{Full model best fits for individual regions, Filament 1.
    \label{tab:fits}}
    \begin{tabular}{@{} l rrr rrrr @{}}
\toprule
{} & \multicolumn{3}{c}{Loss-limited}
   & \multicolumn{4}{c}{Damped, $a_b \leq 0.01$} \\
\cmidrule(lr){2-4} \cmidrule(l){5-8}
Region & $\eta_2$ (-) & $B_0$ ($\mu$G) & $\chi^2$
       & $\eta_2$ (-) & $B_0$ ($\mu$G) & $\chi^2$ & $a_b$ \\
\midrule
 1 & 3.84 & 211  & 25.1 & 1.63  & 25.4 & 24.2 & 0.008 \\
 2 & 0.87 & 309  & 80.8 & 19.02 & 16.8 & 71.4 & 0.003 \\
 3 & 1.30 & 437  & 21.2 & 183.8 & 14.1 & 20.9 & 0.003 \\
\cmidrule{1-8}
 4 & 4.85 & 343  & 33.3 & 1.72  & 25.2 & 29.4 & 0.004 \\
 5 & 1.51 & 300  & 18.4 & 0.49  & 27.6 & 14.3 & 0.003 \\
 6 & 663  & 1374 & 69.9 & 0.00  & 114  & 52.7 & 0.003 \\
 7 & 29.3 & 705  & 11.3 & 0.02  & 242  & 10.4 & 0.004 \\
 8 & 448  & 1212 & 10.0 & 0.15  & 52.2 & 9.67 & 0.003 \\
 9 & 670  & 1411 & 15.6 & 0.02  & 46.8 & 12.9 & 0.002 \\
10 & 14.2 & 682  & 1.26 & 0.36  & 36.1 & 1.32 & 0.002 \\
\cmidrule{1-8}
11 & 0.70 & 345  & 11.8 & 0.20  & 31.0 & 11.6 & 0.002 \\
12 & 1.07 & 320  & 9.60 & 0.05  & 35.0 & 8.53 & 0.002 \\
13 & 75.9 & 834  & 10.1 & 0.00  & 243  & 8.26 & 0.004 \\
\cmidrule{1-8}
14 & 456  & 1226 & 17.0 & 0.01  & 89.9 & 9.74 & 0.003 \\
15 & 76.8 & 932  & 19.8 & 4.48  & 31.3 & 19.6 & 0.003 \\
16 & 3.81 & 583  & 4.25 & 0.22  & 341  & 3.58 & 0.004 \\
17 & 237  & 1300 & 29.2 & 0.46  & 98.4 & 29.0 & 0.003 \\
\cmidrule{1-8}
18 & 0.00 & 254  & 10.0 & 4103  & 8.7  & 12.7 & 0.006 \\
19 &  629 & 1384 & 8.28 & 0.02  & 45.8 & 7.27 & 0.002 \\
20 & 51.1 & 891  & 94.0 & 1.70  & 47.3 & 93.7 & 0.003 \\
\bottomrule
\end{tabular}

\end{table}

\begin{table}[ht]
    \tiny
    \centering
    \caption{Full model best fits for individual regions, Filaments 2--5.
    \label{tab:fits-pt2}}
    \begin{tabular}{@{}rllr llr@{}}

\toprule
\multicolumn{7}{c}{Filament 2} \\
\cmidrule{1-7}
{} & \multicolumn{3}{c}{Region 2\tablenotemark{a}}
   & \multicolumn{3}{c}{Region 3} \\
\cmidrule(lr){2-4} \cmidrule(l){5-7}
$\mu$ (-) & $\eta_2$ (-) & $B_0$ ($\mu$G) & $\chi^2_{\mt{red}}$
          & $\eta_2$ (-) & $B_0$ ($\mu$G) & $\chi^2_{\mt{red}}$ \\
\cmidrule{1-7}
0.00 & ${0.01}^{+0.1}_{-0.01}$ & ${294.4}^{+8.9}_{-4.8}$ & 26.05
     & ${19.6}^{+19.8}_{-9.1}$ & ${720}^{+120}_{-90}$    & 15.57 \\
0.33 & ${0.01}^{+0.1}_{-0.01}$ & ${292.7}^{+6.8}_{-3.9}$ & 25.91
     & ${54}^{+110}_{-34}$ & ${850}^{+230}_{-170}$       & 10.59 \\
0.50 & ${0.01}^{+0.2}_{-0.01}$ & ${292.7}^{+6}_{-4.5}$   & 25.91
     & ${86}^{+290}_{-63}$ & ${910}^{+360}_{-230}$       & 9.09 \\
1.00 & ${0.002}^{+0.2}_{-0.002}$ & ${293}^{+2.8}_{-8.1}$ & 25.76
     & ${13}^{+23}_{-6}$ & ${550}^{+130}_{-60}$          & 8.18 \\
1.50 & ${0.09}^{+0.2}_{-0.09}$ & ${285.9}^{+7}_{-3.5}$   & 25.46
     & ${5.9}^{+2.9}_{-1.8}$ & ${448}^{+30}_{-22}$       & 8.26 \\
2.00 & ${0.06}^{+0.2}_{-0.04}$ & ${283.4}^{+4.8}_{-3.3}$ & 25.05
     & ${4.8}^{+1.6}_{-1.2}$ & ${414}^{+16}_{-14}$       & 8.59 \\

\midrule
\multicolumn{7}{c}{Filament 3} \\
\cmidrule{1-7}
{} & \multicolumn{3}{c}{Region 4}
   & \multicolumn{3}{c}{Region 5} \\
\cmidrule(lr){2-4} \cmidrule(l){5-7}
$\mu$ (-) & $\eta_2$ (-) & $B_0$ ($\mu$G) & $\chi^2_{\mt{red}}$
          & $\eta_2$ (-) & $B_0$ ($\mu$G) & $\chi^2_{\mt{red}}$ \\
\cmidrule{1-7}

0.00 & ${16.4}^{+8.3}_{-4}$ & ${756}^{+68}_{-43}$       & 70.21
     & ${16.2}^{+11}_{-4.4}$ & ${910}^{+100}_{-59}$     & 33.25 \\
0.33 & ${46}^{+28}_{-16}$ & ${889}^{+98}_{-80}$         & 47.53
     & ${47}^{+41}_{-19}$ & ${1070}^{+160}_{-110}$      & 18.00 \\
0.50 & ${74}^{+55}_{-28}$ & ${953}^{+130}_{-98}$        & 37.69
     & ${75}^{+88}_{-34}$ & ${1150}^{+220}_{-150}$      & 12.03 \\
1.00 & ${260}^{+420}_{-140}$ & ${1150}^{+290}_{-190}$   & 15.35
     & ${350}^{+1600}_{-270}$ & ${1490}^{+6000}_{-430}$ & 1.77 \\
1.50 & ${370}^{+5700}_{-280}$ & ${1130}^{+1060}_{-310}$ & 5.37
     & ${19.4}^{+15}_{-6.9}$ & ${715}^{+90}_{-59}$      & 1.58 \\
2.00 & ${26.7}^{+13}_{-7.4}$ & ${581}^{+48}_{-35}$      & 3.55
     & ${9.7}^{+2.8}_{-2.1}$ & ${587}^{+26}_{-22}$      & 2.53 \\

\midrule
\multicolumn{7}{c}{Filament 4} \\
\cmidrule{1-7}
{} & \multicolumn{3}{c}{Region 6\tablenotemark{a,b}}
   & \multicolumn{3}{c}{Region 7} \\
\cmidrule(lr){2-4} \cmidrule(l){5-7}
$\mu$ (-) & $\eta_2$ (-) & $B_0$ ($\mu$G) & $\chi^2_{\mt{red}}$
          & $\eta_2$ (-) & $B_0$ ($\mu$G) & $\chi^2_{\mt{red}}$ \\
\cmidrule{1-7}

0.00 & ${690}^{+3100}_{-690}$ & ${810}^{+26}_{-580}$       & 2.19
     & ${16}^{+26}_{-9}$ & ${730}^{+170}_{-120}$           & 21.54 \\
0.33 & ${0.001}^{+0.2}_{-0.001}$ & ${238}^{+3.5}_{-5.4}$   & 2.20
     & ${18}^{+260}_{-13}$ & ${710}^{+580}_{-170}$         & 19.88 \\
0.50 & ${0.001}^{+0.2}_{-0.001}$ & ${237.8}^{+3.5}_{-5.9}$ & 2.20
     & ${6.3}^{+20}_{-3.3}$ & ${560}^{+190}_{-70}$         & 19.74 \\
1.00 & ${0.0}^{+0.14}_{-0.0}$ & ${238.6}^{+3.2}_{-7.1}$        & 2.25
     & ${2.6}^{+1.5}_{-0.8}$ & ${451}^{+31}_{-23}$         & 19.40 \\
1.50 & ${0.0}^{+0.12}_{-0.0}$ & ${238.7}^{+3.0}_{-10.0}$       & 2.25
     & ${2.1}^{+0.8}_{-0.6}$ & ${418}^{+17}_{-13}$         & 19.16 \\
2.00 & ${0.0}^{+0.10}_{-0.0}$ & ${238.7}^{+3.2}_{-11.6}$       & 2.25
     & ${2.1}^{+0.6}_{-0.5}$ & ${403}^{+12}_{-10}$         & 19.04 \\

\midrule
\multicolumn{7}{c}{Filament 5} \\
\cmidrule{1-7}
{} & \multicolumn{3}{c}{Region 8\tablenotemark{a}}
   & \multicolumn{3}{c}{Region 9\tablenotemark{c}} \\
\cmidrule(lr){2-4} \cmidrule(l){5-7}
$\mu$ (-) & $\eta_2$ (-) & $B_0$ ($\mu$G) & $\chi^2_{\mt{red}}$
          & $\eta_2$ (-) & $B_0$ ($\mu$G) & $\chi^2_{\mt{red}}$ \\
\cmidrule{1-7}

0.00 & ${19.7}^{+28}_{-9.8}$ & ${329}^{+67}_{-43}$     & 12.92
     & ${24}^{+241}_{-241}$ & ${730}^{+1540}_{-1540}$  & 86.66 \\
0.33 & ${54}^{+160}_{-39}$ & ${386}^{+140}_{-93}$      & 7.93
     & ${61}^{+990}_{-990}$ & ${840}^{+3000}_{-3000}$       & 76.14 \\
0.50 & ${69}^{+550}_{-57}$ & ${400}^{+250}_{-120}$          & 6.90
     & ${110}^{+1800}_{-1800}$ & ${925}^{+3500}_{-3500}$    & 72.08 \\
1.00 & ${6.6}^{+5.8}_{-2.6}$ & ${223}^{+27}_{-17}$          & 6.46
     & ${400}^{+8000}_{-8000}$ & ${1100}^{+5000}_{-5000}$   & 65.19 \\
1.50 & ${4.2}^{+1.8}_{-1.2}$ & ${197}^{+10}_{-8}$           & 6.24
     & ${28}^{+18}_{-18}$ & ${570}^{+75}_{-75}$             & 64.84 \\
2.00 & ${3.8}^{+1.2}_{-0.9}$ & ${186}^{+6}_{-5}$            & 6.17
     & ${13.6}^{+4.4}_{-4.4}$ & ${470}^{+26}_{-26}$           & 65.12 \\

\bottomrule
\end{tabular}
\tablenotetext{1}{Fits for regions 2, 6, 8 have 2 dof; rest have 3}
\tablenotetext{2}{Reported $\eta_2$ values of $0$ are $\lesssim 10^{-6}$.}
\tablenotetext{3}{Errors are standard errors, not reliable! (correct ones must
be recalculated}

\end{table}

% NOTE need better subplot display in final version, if we keep this
\begin{figure}[ht]
    \centering
    %\epsscale{0.45}  % For manuscript layout only!
    \plottwo{figures/fits-sep13/reg01.png}{figures/fits-sep13/reg02.png} \\
    \plottwo{figures/fits-sep13/reg03.png}{figures/fits-sep13/reg04.png} \\
    \plottwo{figures/fits-sep13/reg05.png}{figures/fits-sep13/reg06.png} \\
    \plottwo{figures/fits-sep13/reg07.png}{figures/fits-sep13/reg08.png} \\
    \plottwo{figures/fits-sep13/reg09.png}{figures/fits-sep13/reg10.png} \\
    \plottwo{figures/fits-sep13/reg11.png}{figures/fits-sep13/reg12.png} \\
    \plottwo{figures/fits-sep13/reg13.png}{figures/f0-box.png}
    \caption{Fitted full model widths as a function of energy for all regions,
    with measured data; best fit parameters are given in
    Tables~\ref{tab:fits},~\ref{tab:fits-pt2}.  Plots are ordered by region
    number.  NOTE: figures for Regions 9, 11 are incorrect.}
    \label{fig:fits-all}
\end{figure}

% Individual region fits with eta2=1 fixed
We also computed best fits while holding $\eta_2 = 1$; i.e., requiring Bohm
diffusion at $2 \unit{keV}$.  We found that the best fit $B_0$ is relatively
insensitive to $\mu$; for all regions, $B_0$ varies by no more than XXX ($\abt 10$)\% from
the value at $\mu = 1$ (with larger $B_0$ at smaller $\mu$ and vice versa).
\note[Aaron]{compute/verify this}.
Therefore, we show only best fits with $\mu = 1$ and $\eta_2 = 1$ fixed in
Table~\ref{tab:fits-avg-eta2-fix}).
\note[Aaron]{this should be computed just for Region 1, or something similar?}

\begin{table*}[ht]
    \scriptsize
    \centering
    \caption{Best fits with $\eta_2 = 1$ fixed for filament averaged FWHMs.
    \label{tab:fits-avg-eta2-fix}}
    % Best fits with eta2 = 1 fixed, for arithmetic average of FWHMs...
\begin{tabular}{@{}rllr llr llr@{}}

\toprule
{} & \multicolumn{3}{c}{Filament 1}
   & \multicolumn{3}{c}{Filament 2}
   & \multicolumn{3}{c}{Filament 3} \\
\cmidrule(lr){2-4} \cmidrule(lr){5-7} \cmidrule(lr){8-10}
$\mu$ (-) & $\eta_2$ (-) & $B_0$ ($\mu$G) & $\chi^2$
          & $\eta_2$ (-) & $B_0$ ($\mu$G) & $\chi^2$
          & $\eta_2$ (-) & $B_0$ ($\mu$G) & $\chi^2$ \\

\midrule
0.00 & 1 & $363 \pm 14.0$ & 6.7047
     & 1 & $388 \pm 8$ & 3.5110
     & 1 & $489 \pm 24$ & 5.9779\\
0.33 & 1 & $351 \pm 13.0$ & 5.6331
     & 1 & $375 \pm 7$ & 2.8832
     & 1 & $468 \pm 23$ & 5.2188\\
0.50 & 1 & $346 \pm 13.0$ & 5.2477
     & 1 & $369 \pm 7$ & 2.6260
     & 1 & $467 \pm 23$ & 4.9331\\
1.00 & 1 & $334 \pm 12.0$ & 4.5444
     & 1 & $355 \pm 7$ & 2.0359
     & 1 & $439 \pm 22$ & 4.3860\\
1.50 & 1 & $325 \pm 12.0$ & 4.3048
     & 1 & $344 \pm 7$ & 1.6590
     & 1 & $424 \pm 21$ & 4.1910\\
2.00 & 1 & $319 \pm 12.0$ & 4.3274
     & 1 & $336 \pm 7$ & 1.4397
     & 1 & $415 \pm 21$ & 4.2184\\

\midrule
{} & \multicolumn{3}{c}{Filament 4}
   & \multicolumn{3}{c}{Filament 5} \\
\cmidrule(lr){2-4} \cmidrule(lr){5-7}
$\mu$ (-) & $\eta_2$ (-) & $B_0$ ($\mu$G) & $\chi^2$
          & $\eta_2$ (-) & $B_0$ ($\mu$G) & $\chi^2$ \\

\cmidrule(lr){1-7}
0.00 & 1 & $358 \pm 15$ & 0.8029
     & 1 & $239 \pm 37$ & 0.3256\\
0.33 & 1 & $350 \pm 15$ & 1.0612
     & 1 & $231 \pm 36$ & 0.2762\\
0.50 & 1 & $347 \pm 15$ & 1.1809
     & 1 & $228 \pm 36$ & 0.2578\\
1.00 & 1 & $338 \pm 15$ & 1.4792
     & 1 & $220 \pm 34$ & 0.2214\\
1.50 & 1 & $331 \pm 14$ & 1.6770
     & 1 & $215 \pm 34$ & 0.2046\\
2.00 & 1 & $325 \pm 14$ & 1.7851
     & 1 & $210 \pm 33$ & 0.1998\\

\bottomrule
\end{tabular} 
\tablecomments{With only one free parameter, it doesn't really make sense to
vary $B_0$ to get $\Delta\chi^2 = 1$.  Here I report fit standard
errors from the numerically estimated covariance matrix.}

\end{table*}

% Averaged fit results
In each filament, we averaged best fit parameters $B_0$, $\eta_2$ from each
constituent region to obtain filament-wide estimates on $B_0$ and $\eta_2$,
reported in Table~\ref{tab:fits-avg}.  Stated uncertainties are standard errors
of the mean \note[Aaron]{may need to scale by $\abt 2 \times$ to get 1-sigma
confidence interval since our sample sizes are so small}.  We note that both
averages and errors are strongly affected by outliers from extreme best fit
values, as can be seen in Tables~\ref{tab:fits} and \ref{tab:fits-pt2} --
especially as the typical number of regions per filament is quite small.

\begin{table*}[ht]
    \scriptsize
    \centering
    \caption{Filament averages of best-fit parameters for constituent regions
    \label{tab:fits-avg}}
    \begin{tabular}{@{}rr@{ $\pm$ }lr@{ $\pm$ }lr@{ $\pm$ }lr@{ $\pm$ }lr@{ $\pm$ }lr@{ $\pm$ }l@{}}
\toprule
{} & \multicolumn{4}{c}{Filament 1}
   & \multicolumn{4}{c}{Filament 2}
   & \multicolumn{4}{c}{Filament 3} \\
\cmidrule(lr){2-5} \cmidrule(lr){6-9} \cmidrule(lr){10-13}
$\mu$ (-) & \multicolumn{2}{c}{$\eta_2$ (-)} & \multicolumn{2}{c}{$B_0$ ($\mu$G)}
          & \multicolumn{2}{c}{$\eta_2$ (-)} & \multicolumn{2}{c}{$B_0$ ($\mu$G)}
          & \multicolumn{2}{c}{$\eta_2$ (-)} & \multicolumn{2}{c}{$B_0$ ($\mu$G)} \\
\cmidrule{1-13}
0.00 & $15.6$ & $7.6$ & $503.5$ & $176.9$
     & $16.6$ & $2.9$ & $649.3$ & $78.0$
     & $102.3$ & $92.1$ & $696.3$ & $194.8$ \\
0.33 & $47.6$ & $25.4$ & $583.0$ & $236.5$
     & $42.8$ & $9.6$ & $741.5$ & $103.1$
     & $18.5$ & $17.8$ & $499.1$ & $174.1$ \\
0.50 & $105.3$ & $73.1$ & $662.5$ & $314.9$
     & $65.6$ & $16.5$ & $778.4$ & $120.1$
     & $29.0$ & $28.4$ & $516.2$ & $197.2$ \\
1.00 & $8.2$ & $4.8$ & $391.5$ & $134.9$
     & $126.5$ & $74.7$ & $688.3$ & $147.6$
     & $4.6$ & $4.1$ & $389.0$ & $81.1$ \\
1.50 & $3.9$ & $1.7$ & $333.8$ & $95.3$
     & $772.1$ & $763.6$ & $658.7$ & $227.8$
     & $2.3$ & $1.8$ & $350.6$ & $49.2$ \\
2.00 & $3.2$ & $1.2$ & $314.3$ & $84.0$
     & $9.3$ & $4.3$ & $419.9$ & $43.9$
     & $2.0$ & $1.4$ & $336.7$ & $39.3$ \\

\midrule
{} & \multicolumn{4}{c}{Filament 4}
   & \multicolumn{4}{c}{Filament 5}
   & \multicolumn{4}{c}{Global mean} \\
\cmidrule(lr){2-5} \cmidrule(lr){6-9} \cmidrule(lr){10-13}
$\mu$ (-) & \multicolumn{2}{c}{$\eta_2$ (-)} & \multicolumn{2}{c}{$B_0$ ($\mu$G)}
          & \multicolumn{2}{c}{$\eta_2$ (-)} & \multicolumn{2}{c}{$B_0$ ($\mu$G)}
          & \multicolumn{2}{c}{$\eta_2$ (-)} & \multicolumn{2}{c}{$B_0$ ($\mu$G)} \\
\cmidrule{1-13}
0.00 & $15.5$ & $0.4$ & $794.7$ & $41.2$
     & $13.6$ & $3.5$ & $619.6$ & $102.2$
     & $27.3$ & $12.4$ & $655.5$ & $47.9$ \\
0.33 & $43.7$ & $0.8$ & $938.3$ & $54.3$
     & $33.0$ & $13.2$ & $660.9$ & $133.6$
     & $38.1$ & $5.7$ & $704.3$ & $61.8$ \\
0.50 & $164.6$ & $91.9$ & $1159.5$ & $183.0$
     & $94.7$ & $56.7$ & $742.3$ & $169.1$
     & $90.6$ & $23.4$ & $787.9$ & $83.7$ \\
1.00 & $85.6$ & $51.7$ & $834.1$ & $106.0$
     & $538.8$ & $523.4$ & $770.3$ & $292.0$
     & $180.0$ & $119.7$ & $652.2$ & $87.4$ \\
1.50 & $14.0$ & $7.6$ & $561.0$ & $26.2$
     & $6.0$ & $2.6$ & $417.8$ & $59.9$
     & $250.4$ & $243.0$ & $499.9$ & $76.5$ \\
2.00 & $6.9$ & $2.3$ & $487.3$ & $10.3$
     & $3.9$ & $1.5$ & $380.8$ & $46.8$
     & $5.8$ & $1.5$ & $397.5$ & $23.1$ \\
\bottomrule
\end{tabular}
\tablecomments{Reported errors are standard errors of the mean.
Filaments 1, 3 have only 3 samples; Filament 2 has 7; Filament 4, 4; Filament
5, 5.}

\end{table*}

% Global fits...
Extras: we could report a global average of best fit parameters from all
regions, or a fit to the global FWHM average.
Plot of best fit parameters with $\mu=1$ as a function of azimuth angle, with
error bars or something similar?

\subsection{Variations on a theme of plug \& chug}

Discuss, briefly, effects of twiddle-ables: changing Tycho distance ($2.3
\unit{kpc}$ to $4 \unit{kpc}$, see \citet{hayato2010}), compression ratio $r$,
accounting for azimuthal variation in $v_s$ (compare to taking a remnant
average?).  Sean already discussed $\Ecut$ (result: more high energy electrons
to damp out narrowing at high energy, if $\Ecut$ not included).

% ==========
% Discussion
% ==========
\section{Discussion}

\subsection{How confidently can we reject B-damping?}

Everything hinges on our error bars -- this is most important.  Our rims do
thin consistently, in most regions (might be one or two places where they
don't), but (1) the errors in our individual FWHMs are larger than I'm
currently reporting because different profile fits will give a larger spread of
FWHMs (might need to consider reduced chi-squared here, but using chi-sqr to
select profile fit models is not physically meaningful)

It might be helpful to average $\mE$ values together, under the
assumption that they should be somewhat consistent within a given filament.
Then report the values and errors observed.  Again, needs more data points.

Recall that our FWHMs (Table~\ref{tab:fwhms}) have a large range of values,
IF magnetic damping is relevant in controlling filament widths, we might, e.g.,
expect a maximum filament width at which $\mE$ sharply levels off, being now
restricted by the magnetic field turn off.  We simply do not observe this in
our data between $0.7$--$7 \unit{keV}$.  We must conclude that, as in SN 1006
\citepalias{ressler2014}, magnetic damping cannot be relevant to our
observations.

What is the smallest damping scale length that is (roughly) consistent with all
of our observations?

X-ray rims are not damped -- could damping be relevant to the radio rims?

\subsection{Interpretation of fit parameters}

Magnetic field $B_0$: despite the large range of fit quality, we can give
fairly robust lower bounds.  Give lowest bound?  Consistent with previous
studies, Tycho requires strong magnetic field amplification -- up to ??? times
the expected value of $\abt 10 \unit{\mu G}$ from a strong adiabatic shock with
compression ratio $r=4$ and typical galactic magnetic field strength of $\abt
2$--$3 \unit{\mu G}$ \citep{lyne1989, han2006}.

Diffusion coefficient $\eta_2$, on the other hand, is a bit of a mess.
Would it be useful to report advective and diffusive lengthscales at 2 keV?
Idea being that it is hard to intuit what the large numbers even mean.
(especially for the simple model -- it sits on a plateau in chi-squared space
where small $\eta_2$ may be disfavored, but beyond a certain point any $\eta_2$
will do)

Can we favor a value of $\mu$ (and hence a turbulent energy spectrum)?

\subsection{Bohm diffusion}

Fixing $\mu = 1$ and $\eta_2 = 1$ in Table~\ref{tab:fits-avg} is equivalent to
assuming Bohm diffusion, as most previous studies have done.
We should find/show that our result is very similar to theirs, in terms of
$B_0$ estimates, and this may be the easiest way to back out any estimates of
azimuthal magnetic field variation.

Magnetic field amplification numbers -- how do they compare with previous
studies?  What is the azimuthal variation (and significance thereof)?

Diffusion -- sub-Bohm diffusion in Filament 4?

\subsection{Other things}

Precursors, rim position variation (none obvious from preliminary look; see
Figure~\ref{fig:peak-pos}), steepness of the filament rise?

\begin{figure}[ht]
    \centering
    \plotone{figures/f0-peak-pos.pdf}
    \caption{Best-fit rim peak positions ($x_0$ in equation~\eqref{eq:prof})
        for all regions as a function of energy band, normalized to the $2
        \unit{keV}$ peak position.  Red line plots best linear fit to all data
        with slope of $-0.025 \unit{arcsec/keV}$.
        \label{fig:peak-pos}}
\end{figure}

% ==========
% Conclusion
% ==========
\section{Conclusions}

In conclusion, crazy B field amplification is not that weird.
Magnetic damping can be ruled out and our result is robust throughout the remnant.

% ================
% Acknowledgements
% ================
\acknowledgments

The scientific results reported in this article are based on data obtained from
the \Chandra Data Archive.
This research has made use of NASA's Astrophysics Data System.

{\it Facilities:} \facility{CXO (ACIS-I)}

\clearpage

% ========
% Appendix
% ========
\appendix

\setcounter{table}{0}
\renewcommand{\thetable}{A\arabic{table}}
\setcounter{figure}{0}
\renewcommand{\thefigure}{A\arabic{figure}}

% ---------------------------
% FWHM measurement comparison
% ---------------------------
\section{Additional calculations}

Here I present full model fit results using FWHMs measured with an artificial
cap imposed on the profile height, and the background $\min(C_u, C_d)$
subtracted out before computing the FWHM.  The fitted profile may extend above
the data maximum, but I find the full-width at one half of the data maximum.
The results appear equivocal; there may be slightly smaller chi-squared values
but it may not be statistically meaningful.

I also present results for simple model fits, and fits for a larger set of
regions (22 instead of 13).

\begin{table*}[ht]
    \tiny
    \centering
    \caption{Measured FWHMs for all regions, capped and background subtracted.}
    \begin{tabular}{@{}rlllllr@{ $\pm$ }lr@{ $\pm$ }lr@{ $\pm$ }lr@{ $\pm$ }l@{}}
\toprule
{} & \multicolumn{13}{c}{Filament 1} \\
\cmidrule(l){2-14}
{} & \multicolumn{5}{c}{FWHM (arcsec)} & \multicolumn{8}{c}{$\mE$ (-)} \\
\cmidrule(lr){2-6} \cmidrule(l){7-14}
Region & 0.7--1 keV & 1--1.7 keV & 2--3 keV & 3--4.5 keV & 4.5--7 keV
       & \multicolumn{2}{c}{1 keV} & \multicolumn{2}{c}{2 keV}
       & \multicolumn{2}{c}{3 keV} & \multicolumn{2}{c}{4.5 keV} \\
\cmidrule{1-14}
1 & ${3.05}^{+0.26}_{-0.24}$ & ${3.03}^{+0.07}_{-0.07}$ & ${2.95}^{+0.11}_{-0.10}$ & ${2.47}^{+0.11}_{-0.10}$ & ${2.21}^{+0.13}_{-0.13}$
  & $-0.02$ & $0.00$ & $-0.04$ & $0.00$ & $-0.44$ & $0.06$ & $-0.28$ & $0.05$ \\
10 & ${6.09}^{+0.33}_{-0.31}$ & ${5.04}^{+0.09}_{-0.07}$ & ${4.58}^{+0.13}_{-0.12}$ & ${4.96}^{+0.15}_{-0.15}$ & ${3.61}^{+0.20}_{-0.17}$
  & $-0.53$ & $0.08$ & $-0.14$ & $0.01$ & $0.19$ & $0.02$ & $-0.78$ & $0.12$ \\
11 & ${6.32}^{+0.38}_{-0.32}$ & ${4.17}^{+0.08}_{-0.08}$ & ${3.51}^{+0.11}_{-0.10}$ & ${3.21}^{+0.13}_{-0.11}$ & ${2.55}^{+0.15}_{-0.11}$
  & $-1.17$ & $0.19$ & $-0.25$ & $0.01$ & $-0.22$ & $0.03$ & $-0.57$ & $0.09$ \\
12 & ${3.84}^{+0.21}_{-0.19}$ & ${3.02}^{+0.06}_{-0.05}$ & ${3.30}^{+0.09}_{-0.09}$ & ${2.47}^{+0.09}_{-0.08}$ & ${2.56}^{+0.14}_{-0.11}$
  & $-0.68$ & $0.11$ & $0.13$ & $0.01$ & $-0.72$ & $0.08$ & $0.09$ & $0.01$ \\
13 & ${2.67}^{+0.23}_{-0.21}$ & ${2.98}^{+0.08}_{-0.05}$ & ${2.90}^{+0.12}_{-0.11}$ & ${2.61}^{+0.11}_{-0.10}$ & ${2.57}^{+0.19}_{-0.17}$
  & $0.30$ & $0.07$ & $-0.04$ & $0.00$ & $-0.26$ & $0.04$ & $-0.03$ & $0.01$ \\

\midrule
{} & \multicolumn{13}{c}{Filament 2} \\
\cmidrule(l){2-14}
{} & \multicolumn{5}{c}{FWHM (arcsec)} & \multicolumn{8}{c}{$\mE$ (-)} \\
\cmidrule(lr){2-6} \cmidrule(l){7-14}
Region & 0.7--1 keV & 1--1.7 keV & 2--3 keV & 3--4.5 keV & 4.5--7 keV
       & \multicolumn{2}{c}{1 keV} & \multicolumn{2}{c}{2 keV}
       & \multicolumn{2}{c}{3 keV} & \multicolumn{2}{c}{4.5 keV} \\
\cmidrule{1-14}
2 & {} & ${4.49}^{+0.13}_{-0.12}$ & ${3.29}^{+0.14}_{-0.11}$ & ${3.10}^{+0.15}_{-0.13}$ & ${2.62}^{+0.20}_{-0.16}$
  & \multicolumn{2}{c}{} & $-0.45$ & $0.03$ & $-0.15$ & $0.02$ & $-0.41$ & $0.08$ \\
3 & ${3.75}^{+0.26}_{-0.19}$ & ${3.34}^{+0.06}_{-0.06}$ & ${3.13}^{+0.10}_{-0.07}$ & ${2.31}^{+0.10}_{-0.11}$ & ${2.41}^{+0.14}_{-0.17}$
  & $-0.32$ & $0.06$ & $-0.09$ & $0.00$ & $-0.75$ & $0.10$ & $0.11$ & $0.02$ \\

\midrule
{} & \multicolumn{13}{c}{Filament 3} \\
\cmidrule(l){2-14}
{} & \multicolumn{5}{c}{FWHM (arcsec)} & \multicolumn{8}{c}{$\mE$ (-)} \\
\cmidrule(lr){2-6} \cmidrule(l){7-14}
Region & 0.7--1 keV & 1--1.7 keV & 2--3 keV & 3--4.5 keV & 4.5--7 keV
       & \multicolumn{2}{c}{1 keV} & \multicolumn{2}{c}{2 keV}
       & \multicolumn{2}{c}{3 keV} & \multicolumn{2}{c}{4.5 keV} \\
\cmidrule{1-14}
4 & ${2.96}^{+0.13}_{-0.12}$ & ${2.68}^{+0.05}_{-0.03}$ & ${2.50}^{+0.07}_{-0.05}$ & ${2.49}^{+0.08}_{-0.06}$ & ${1.83}^{+0.08}_{-0.07}$
  & $-0.29$ & $0.04$ & $-0.10$ & $0.00$ & $-0.01$ & $0.00$ & $-0.75$ & $0.09$ \\
5 & ${2.33}^{+0.11}_{-0.09}$ & ${2.23}^{+0.04}_{-0.03}$ & ${2.11}^{+0.06}_{-0.05}$ & ${1.95}^{+0.06}_{-0.06}$ & ${1.80}^{+0.09}_{-0.08}$
  & $-0.13$ & $0.02$ & $-0.08$ & $0.00$ & $-0.19$ & $0.02$ & $-0.19$ & $0.03$ \\

\midrule
{} & \multicolumn{13}{c}{Filament 4} \\
\cmidrule(l){2-14}
{} & \multicolumn{5}{c}{FWHM (arcsec)} & \multicolumn{8}{c}{$\mE$ (-)} \\
\cmidrule(lr){2-6} \cmidrule(l){7-14}
Region & 0.7--1 keV & 1--1.7 keV & 2--3 keV & 3--4.5 keV & 4.5--7 keV
       & \multicolumn{2}{c}{1 keV} & \multicolumn{2}{c}{2 keV}
       & \multicolumn{2}{c}{3 keV} & \multicolumn{2}{c}{4.5 keV} \\
\cmidrule{1-14}
6 & {} & ${4.85}^{+0.13}_{-0.13}$ & ${3.69}^{+0.19}_{-0.19}$ & ${3.83}^{+0.24}_{-0.24}$ & ${2.25}^{+0.26}_{-0.20}$
  & \multicolumn{2}{c}{} & $-0.39$ & $0.03$ & $0.09$ & $0.02$ & $-1.31$ & $0.39$ \\
7 & ${4.72}^{+0.37}_{-0.34}$ & ${3.11}^{+0.07}_{-0.06}$ & ${2.79}^{+0.10}_{-0.09}$ & ${2.42}^{+0.08}_{-0.07}$ & ${2.25}^{+0.16}_{-0.14}$
  & $-1.17$ & $0.26$ & $-0.16$ & $0.01$ & $-0.35$ & $0.04$ & $-0.18$ & $0.03$ \\

\midrule
{} & \multicolumn{13}{c}{Filament 5} \\
\cmidrule(l){2-14}
{} & \multicolumn{5}{c}{FWHM (arcsec)} & \multicolumn{8}{c}{$\mE$ (-)} \\
\cmidrule(lr){2-6} \cmidrule(l){7-14}
Region & 0.7--1 keV & 1--1.7 keV & 2--3 keV & 3--4.5 keV & 4.5--7 keV
       & \multicolumn{2}{c}{1 keV} & \multicolumn{2}{c}{2 keV}
       & \multicolumn{2}{c}{3 keV} & \multicolumn{2}{c}{4.5 keV} \\
\cmidrule{1-14}
8 & {} & ${8.52}^{+0.17}_{-0.15}$ & ${6.85}^{+0.27}_{-0.21}$ & ${7.41}^{+0.30}_{-0.23}$ & ${5.06}^{+0.44}_{-0.39}$
  & \multicolumn{2}{c}{} & $-0.32$ & $0.02$ & $0.19$ & $0.02$ & $-0.94$ & $0.21$ \\
9 & ${2.65}^{+0.15}_{-0.13}$ & ${4.03}^{+0.09}_{-0.06}$ & ${3.00}^{+0.10}_{-0.07}$ & ${3.44}^{+0.13}_{-0.10}$ & ${2.85}^{+0.18}_{-0.12}$
  & $1.18$ & $0.19$ & $-0.43$ & $0.02$ & $0.34$ & $0.04$ & $-0.46$ & $0.07$ \\

\midrule
Mean & ${3.84}^{+0.45}_{-0.45}$ & ${3.96}^{+0.45}_{-0.45}$ & ${3.43}^{+0.33}_{-0.33}$ & ${3.28}^{+0.41}_{-0.41}$ & ${2.66}^{+0.24}_{-0.24}$
  & $-0.28$ & $0.22$ & $-0.18$ & $0.05$ & $-0.18$ & $0.09$ & $-0.44$ & $0.12$ \\
\bottomrule
\end{tabular}
\tablecomments{Mean values computed for all regions; mean $\mE$ values are
averages for region $\mE$ values (i.e., not computed from mean FWHMs).  Errors
on mean values are standard errors of the mean.  DRAFT ONLY: as noted in text,
$\mE$ values are not used in subsequent analysis!}

\end{table*}

\begin{table}[ht]
    \tiny
    \centering
    \caption{Full model best fits for individual regions, Filament 1.  Capped
    and background subtracted FWHMs.}
    \begin{tabular}{@{}rllrllr@{}}

\toprule
\multicolumn{6}{c}{Filament 1} \\
\cmidrule{1-7}
{} & \multicolumn{3}{c}{Region 1, full model\tablenotemark{a}}
   & \multicolumn{3}{c}{Region 1, simple model} \\
\cmidrule(lr){2-4} \cmidrule(l){5-7}
$\mu$ (-) & $\eta_2$ (-) & $B_0$ ($\mu$G) & $\chi^2_{\mt{red}}$
          & $\eta_2$ (-) & $B_0$ ($\mu$G) & $\chi^2_{\mt{red}}$ \\
\cmidrule{1-7}
0.00 & ${20.1}^{+298.9}_{-298.9}$ & ${653}^{+2013}_{-2013}$ & 13.54
     & - & - & - \\
0.33 & ${63.2}^{+893.2}_{-893.2}$ & ${789}^{+2455}_{-2455}$ & 7.11
     & - & - & - \\
0.50 & ${107}^{+1595}_{-1595}$ & ${861}^{+2891}_{-2891}$ & 4.92
     & - & - & - \\
1.00 & ${8000}^{+12300}_{-12300}$ & ${2110}^{+712}_{-712}$ & 2.50
     & - & - & - \\
1.50 & ${9.7}^{+4.7}_{-4.7}$ & ${439.8}^{+36.8}_{-36.8}$ & 3.85
     & - & - & - \\
2.00 & ${6.6}^{+2.1}_{-2.1}$ & ${391.6}^{+17.8}_{-17.8}$ & 4.92
     & - & - & - \\

\cmidrule{1-7}
{} & \multicolumn{3}{c}{Region 10}
   & \multicolumn{3}{c}{Region 11} \\
\cmidrule(lr){2-4} \cmidrule(l){5-7}
$\mu$ (-) & $\eta_2$ (-) & $B_0$ ($\mu$G) & $\chi^2_{\mt{red}}$
          & $\eta_2$ (-) & $B_0$ ($\mu$G) & $\chi^2_{\mt{red}}$ \\
\cmidrule{1-7}
0.00 & ${19.9}^{+277}_{-277}$ & ${463}^{+1339}_{-1339}$ & 33.56
     & ${18.7}^{+289}_{-289}$ & ${530}^{+1694}_{-1694}$ & 8.53 \\
0.33 & ${59.0}^{+912.5}_{-912.5}$ & ${550}^{+1882}_{-1882}$ & 21.26
     & ${4.4}^{+4.0}_{-4.0}$ & ${376.0}^{+60.2}_{-60.2}$ & 8.08 \\
0.50 & ${99}^{+1616}_{-1616}$ & ${600}^{+2200}_{-2200}$ & 16.61
     & ${3.0}^{+1.9}_{-1.9}$ & ${343}^{+34}_{-34}$ & 8.01 \\
1.00 & ${431}^{+7474}_{-7474}$ & ${764}^{+3100}_{-3100}$ & 9.64
     & ${1.8}^{+0.7}_{-0.7}$ & ${304.4}^{+14.2}_{-14.2}$ & 7.92 \\
1.50 & ${15.3}^{+6.7}_{-6.7}$ & ${336}^{+28}_{-28}$ & 9.81
     & ${1.6}^{+0.5}_{-0.5}$ & ${289}^{+9}_{-9}$ & 8.00 \\
2.00 & ${9.3}^{+2.4}_{-2.4}$ & ${290}^{+12}_{-12}$ & 10.46
     & ${1.6}^{+0.4}_{-0.4}$ & ${280.7}^{+6.5}_{-6.5}$ & 8.22 \\

\cmidrule{1-7}
{} & \multicolumn{3}{c}{Region 12}
   & \multicolumn{3}{c}{Region 13} \\
\cmidrule(lr){2-4} \cmidrule(l){5-7}
$\mu$ (-) & $\eta_2$ (-) & $B_0$ ($\mu$G) & $\chi^2_{\mt{red}}$
          & $\eta_2$ (-) & $B_0$ ($\mu$G) & $\chi^2_{\mt{red}}$ \\
\cmidrule{1-7}
0.00 & ${22}^{+210}_{-210}$ & ${656}^{+1309}_{-1309}$ & 39.31
     & ${22}^{+269}_{-269}$ & ${678}^{+1697}_{-1697}$ & 22.62 \\
0.33 & ${56}^{+995}_{-995}$ & ${756}^{+2937}_{-2937}$ & 27.26
     & ${62}^{+1190}_{-1190}$ & ${794}^{+3370}_{-3370}$ & 14.01 \\
0.50 & ${99}^{+1680}_{-1680}$ & ${832}^{+3157}_{-3157}$ & 22.61
     & ${102}^{+2243}_{-2243}$ & ${859}^{+4260}_{-4260}$ & 10.43 \\
1.00 & ${449}^{+7312}_{-7312}$ & ${1070}^{+4068}_{-4068}$ & 15.21
     & ${426}^{+10816}_{-10816}$ & ${1085}^{+6418}_{-6418}$ & 3.18 \\
1.50 & ${18}^{+9}_{-9}$ & ${479}^{+45}_{-45}$ & 15.41
     & ${114}^{+223}_{-223}$ & ${727}^{+323}_{-323}$ & 1.74 \\
2.00 & ${10}^{+3}_{-3}$ & ${407}^{+18}_{-18}$ & 16.21
     & ${21}^{+10}_{-10}$ & ${469}^{+40}_{-40}$ & 2.51 \\
\bottomrule
\end{tabular}
\tablenotetext{1}{All reported values are full model fits unless
otherwise stated.}
\tablecomments{All fits have 3 dof; all reported errors are fit standard
errors, not reliable}

\end{table}

\begin{table}[ht]
    \tiny
    \centering
    \caption{Full model best fits for individual regions, Filaments 2--5.
    Capped and background subtracted FWHMs.}
    \begin{tabular}{@{}rllr llr@{}}

\toprule
\multicolumn{7}{c}{Filament 2} \\
\cmidrule{1-7}
{} & \multicolumn{3}{c}{Region 2\tablenotemark{a}}
   & \multicolumn{3}{c}{Region 3} \\
\cmidrule(lr){2-4} \cmidrule(l){5-7}
$\mu$ (-) & $\eta_2$ (-) & $B_0$ ($\mu$G) & $\chi^2_{\mt{red}}$
          & $\eta_2$ (-) & $B_0$ ($\mu$G) & $\chi^2_{\mt{red}}$ \\
\cmidrule{1-7}
0.00 & ${3.5}^{+6.9}_{-6.9}$ & ${374.4}^{+134}_{-134}$ & 1.23
     & ${19.2}^{+269}_{-269}$ & ${610}^{+1771}_{-1771}$ & 11.00 \\
0.33 & ${1.5}^{+1.3}_{-1.3}$ & ${311.6}^{+39.4}_{-39.4}$ & 1.11
     & ${61.5}^{+787}_{-787}$ & ${739}^{+2087}_{-2087}$ & 7.04 \\
0.50 & ${1.2}^{+0.9}_{-0.9}$ & ${299.3}^{+28.7}_{-28.7}$ & 1.05
     & ${161}^{+867}_{-867}$ & ${890.1}^{+1074}_{-1074}$ & 6.30 \\
1.00 & ${1.0}^{+0.5}_{-0.5}$ & ${280.0}^{+15.6}_{-15.6}$ & 0.91
     & ${6.4}^{+3.6}_{-3.6}$ & ${411.3}^{+40.7}_{-40.7}$ & 6.79 \\
1.50 & ${0.9}^{+0.4}_{-0.4}$ & ${270.7}^{+10.8}_{-10.8}$ & 0.82
     & ${4.0}^{+1.4}_{-1.4}$ & ${359.4}^{+17.1}_{-17.1}$ & 7.47 \\
2.00 & ${0.9}^{+0.4}_{-0.4}$ & ${265.1}^{+8.5}_{-8.5}$ & 0.78
     & ${3.4}^{+0.9}_{-0.9}$ & ${338.3}^{+10.6}_{-10.6}$ & 8.25 \\

\midrule
\multicolumn{7}{c}{Filament 3} \\
\cmidrule{1-7}
{} & \multicolumn{3}{c}{Region 4}
   & \multicolumn{3}{c}{Region 5} \\
\cmidrule(lr){2-4} \cmidrule(l){5-7}
$\mu$ (-) & $\eta_2$ (-) & $B_0$ ($\mu$G) & $\chi^2_{\mt{red}}$
          & $\eta_2$ (-) & $B_0$ ($\mu$G) & $\chi^2_{\mt{red}}$ \\
\cmidrule{1-7}

0.00 & ${15.8}^{+156.8}_{-156.8}$ & ${656}^{+1346}_{-1346}$ & 33.84
     & ${17.1}^{+140.2}_{-140.2}$ & ${754}^{+1273}_{-1273}$ & 33.10 \\
0.33 & ${50.8}^{+472.1}_{-472.1}$ & ${795}^{+1625}_{-1625}$ & 19.24
     & ${46.4}^{+631.1}_{-631.1}$ & ${878}^{+2623}_{-2623}$ & 17.56 \\
0.50 & ${84.4}^{+869.8}_{-869.8}$ & ${864}^{+1998}_{-1998}$ & 14.24
     & ${78.0}^{+1126.0}_{-1126.0}$ & ${955}^{+3090}_{-3090}$ & 11.49 \\
1.00 & ${45.0}^{+62.4}_{-62.4}$ & ${683}^{+210}_{-210}$ & 9.72
     & ${334}^{+5515}_{-5515}$ & ${1213}^{+4660}_{-4660}$ & 1.12 \\
1.50 & ${7.6}^{+2.4}_{-2.4}$ & ${441.2}^{+23.6}_{-23.6}$ & 11.46
     & ${18.3}^{+9.2}_{-9.2}$ & ${579.3}^{+57.7}_{-57.7}$ & 0.97 \\
2.00 & ${5.3}^{+1.1}_{-1.1}$ & ${394.7}^{+11.6}_{-11.6}$ & 13.75
     & ${9.3}^{+2.4}_{-2.4}$ & ${478.6}^{+20.3}_{-20.3}$ & 1.93 \\

\midrule
\multicolumn{7}{c}{Filament 4} \\
\cmidrule{1-7}
{} & \multicolumn{3}{c}{Region 6\tablenotemark{a}}
   & \multicolumn{3}{c}{Region 7} \\
\cmidrule(lr){2-4} \cmidrule(l){5-7}
$\mu$ (-) & $\eta_2$ (-) & $B_0$ ($\mu$G) & $\chi^2_{\mt{red}}$
          & $\eta_2$ (-) & $B_0$ ($\mu$G) & $\chi^2_{\mt{red}}$ \\
\cmidrule{1-7}

0.00 & ${103.2}^{+64.3}_{-64.3}$ & ${677.6}^{+66.7}_{-66.7}$ & 5.51
     & ${16.1}^{+321.2}_{-321.2}$ & ${604}^{+2489}_{-2489}$ & 9.19 \\
0.33 & ${2455}^{+4100}_{-4100}$ & ${1166.9}^{+236.2}_{-236.2}$ & 5.64
     & ${47}^{+1069}_{-1069}$ & ${717.6}^{+3623.0}_{-3623.0}$ & 5.56 \\
0.50 & ${0.7}^{+0.6}_{-0.6}$ & ${256.6}^{+22.9}_{-22.9}$ & 6.09
     & ${51.1}^{+394.9}_{-394.9}$ & ${711.5}^{+1227}_{-1227}$ & 4.94 \\
1.00 & ${0.6}^{+0.4}_{-0.4}$ & ${243.7}^{+12.9}_{-12.9}$ & 6.34
     & ${4.8}^{+2.6}_{-2.6}$ & ${403.7}^{+37.4}_{-37.4}$ & 4.71 \\
1.50 & ${0.5}^{+0.3}_{-0.3}$ & ${237.0}^{+8.9}_{-8.9}$ & 6.59
     & ${3.2}^{+1.1}_{-1.1}$ & ${359.3}^{+17.2}_{-17.2}$ & 4.63 \\
2.00 & ${0.5}^{+0.3}_{-0.3}$ & ${232.8}^{+7.0}_{-7.0}$ & 6.82
     & ${2.9}^{+0.8}_{-0.8}$ & ${340.6}^{+11.1}_{-11.1}$ & 4.70 \\

\midrule
\multicolumn{7}{c}{Filament 5} \\
\cmidrule{1-7}
{} & \multicolumn{3}{c}{Region 8\tablenotemark{a}}
   & \multicolumn{3}{c}{Region 9} \\
\cmidrule(lr){2-4} \cmidrule(l){5-7}
$\mu$ (-) & $\eta_2$ (-) & $B_0$ ($\mu$G) & $\chi^2_{\mt{red}}$
          & $\eta_2$ (-) & $B_0$ ($\mu$G) & $\chi^2_{\mt{red}}$ \\
\cmidrule{1-7}

0.00 & ${20.6}^{+414.0}_{-414.0}$ & ${340}^{+1429}_{-1429}$ & 14.70
     & ${22.1}^{+245.8}_{-245.8}$ & ${600}^{+1384}_{-1384}$ & 71.05 \\
0.33 & ${64}^{+1198}_{-1198}$ & ${409}^{+1709}_{-1709}$ & 9.68
     & ${67.2}^{+834.4}_{-834.4}$ & ${717}^{+1966}_{-1966}$ & 56.36 \\
0.50 & ${106}^{+2186}_{-2186}$ & ${445}^{+2079}_{-2079}$ & 8.63
     & ${117}^{+1444}_{-1444}$ & ${787}^{+2189}_{-2189}$ & 50.51 \\
1.00 & ${7.0}^{+4.5}_{-4.5}$ & ${231.3}^{+26.1}_{-26.1}$ & 8.66
     & ${640.2}^{+6252}_{-6252}$ & ${1060}^{+2415}_{-2415}$ & 39.65 \\
1.50 & ${4.3}^{+1.7}_{-1.7}$ & ${201.9}^{+10.9}_{-10.9}$ & 8.97
     & ${75.1}^{+78.0}_{-78.0}$ & ${590.7}^{+136.4}_{-136.4}$ & 39.49 \\
2.00 & ${3.8}^{+1.1}_{-1.1}$ & ${190.2}^{+6.7}_{-6.7}$ & 9.44
     & ${19.3}^{+6.4}_{-6.4}$ & ${414.7}^{+24.9}_{-24.9}$ & 42.27 \\

\bottomrule
\end{tabular}
\tablenotetext{1}{Fits for regions 2, 6, 8 have 2 dof; rest have 3}
\tablenotetext{2}{Reported $\eta_2$ values of $0$ are $\lesssim 10^{-6}$.}

\end{table}

\begin{table*}[ht]
    \scriptsize
    \centering
    \caption{Filament averages of best-fit parameters for constituent regions.
    Capped and background subtracted FWHMs.}
    % Average of best-fit parameters for individual regions, for each filament
% Manually tabulated from Tycho best fits, no cap, no bkg removal
\begin{tabular}{@{}rr@{ $\pm$ }lr@{ $\pm$ }lr@{ $\pm$ }lr@{ $\pm$ }lr@{ $\pm$ }lr@{ $\pm$ }l@{}}
\toprule
{} & \multicolumn{4}{c}{Filament 1}
   & \multicolumn{4}{c}{Filament 2}
   & \multicolumn{4}{c}{Filament 3} \\
\cmidrule(lr){2-5} \cmidrule(lr){6-9} \cmidrule(lr){10-13}
$\mu$ (-) & \multicolumn{2}{c}{$\eta_2$ (-)} & \multicolumn{2}{c}{$B_0$ ($\mu$G)}
          & \multicolumn{2}{c}{$\eta_2$ (-)} & \multicolumn{2}{c}{$B_0$ ($\mu$G)}
          & \multicolumn{2}{c}{$\eta_2$ (-)} & \multicolumn{2}{c}{$B_0$ ($\mu$G)} \\
\cmidrule{1-13}
0.00 & $21$ & $1$ & $596$ & $42$
     & $11$ & $8$ & $492$ & $118$
     & $16$ & $1$ & $705$ & $49$ \\
0.33 & $49$ & $11$ & $653$ & $83$
     & $31$ & $30$ & $525$ & $214$
     & $49$ & $2$ & $837$ & $41$ \\
0.50 & $82$ & $20$ & $699$ & $102$
     & $81$ & $80$ & $595$ & $295$
     & $81$ & $3$ & $910$ & $46$ \\
1.00 & $1861$ & $1536$ & $1067$ & $297$
     & $3.7$ & $2.7$ & $346$ & $66$
     & $190$ & $145$ & $948$ & $265$ \\
1.50 & $32$ & $21$ & $454$ & $76$
     & $2.4$ & $1.5$ & $315$ & $44$
     & $13$ & $5$ & $510$ & $69$ \\
2.00 & $9.6$ & $3.1$ & $368$ & $36$
     & $2.2$ & $1.3$ & $302$ & $37$
     & $7.3$ & $2.0$ & $437$ & $42$ \\

\midrule
{} & \multicolumn{4}{c}{Filament 4}
   & \multicolumn{4}{c}{Filament 5} \\
\cmidrule(lr){2-5} \cmidrule(lr){6-9}
$\mu$ (-) & \multicolumn{2}{c}{$\eta_2$ (-)} & \multicolumn{2}{c}{$B_0$ ($\mu$G)}
          & \multicolumn{2}{c}{$\eta_2$ (-)} & \multicolumn{2}{c}{$B_0$ ($\mu$G)} \\
\cmidrule{1-9}
0.00 & $60$ & $44$ & $641$ & $37$
     & $21.4$ & $0.8$ & $470$ & $130$ \\
0.33 & $1251$ & $1204$ & $942$ & $225$
     & $65.4$ & $1.7$ & $563$ & $154$ \\
0.50 & $26$ & $25$ & $484$ & $227$
     & $111$ & $5$ & $616$ & $171$ \\
1.00 & $2.7$ & $2.1$ & $324$ & $80$
     & $324$ & $317$ & $646$ & $414$ \\
1.50 & $1.9$ & $1.3$ & $298$ & $61$
     & $40$ & $35$ & $396$ & $194$ \\
2.00 & $1.7$ & $1.2$ & $287$ & $54$
     & $12$ & $8$ & $302$ & $112$ \\

\bottomrule
\end{tabular}
\tablecomments{Reported errors are standard errors of the mean.  For all
filaments except Filament 1, there are only two samples to average.  Even the
data points in 1 are very susceptible to outliers, none of which have been
removed.}


\end{table*}

\begin{figure}[ht]
    \centering
    %\epsscale{0.45}
    \plottwo{figures/fits-sep19-capsubbkg/reg01.png}{figures/fits-sep19-capsubbkg/reg02.png} \\
    \plottwo{figures/fits-sep19-capsubbkg/reg03.png}{figures/fits-sep19-capsubbkg/reg04.png} \\
    \plottwo{figures/fits-sep19-capsubbkg/reg05.png}{figures/fits-sep19-capsubbkg/reg06.png} \\
    \plottwo{figures/fits-sep19-capsubbkg/reg07.png}{figures/fits-sep19-capsubbkg/reg08.png} \\
    \plottwo{figures/fits-sep19-capsubbkg/reg09.png}{figures/fits-sep19-capsubbkg/reg10.png} \\
    \plottwo{figures/fits-sep19-capsubbkg/reg11.png}{figures/fits-sep19-capsubbkg/reg12.png} \\
    \plottwo{figures/fits-sep19-capsubbkg/reg13.png}{figures/f0-box.png}
    \caption{Fitted full model widths as a function of energy for all regions,
    with measured data; best fit parameters are given in
    Tables~\ref{tab:fits},~\ref{tab:fits-pt2}.  Plots are ordered by region
    number.}
\end{figure}

\begin{table}[ht]
    \tiny
    \centering
    \caption{Simple model best fits for individual regions, Filaments 1 and 2.
    Same FWHM as in main text.}
    \begin{tabular}{@{}rllrllr@{}}
\toprule
\multicolumn{7}{c}{Filament 1} \\
\cmidrule{1-7}
{} & \multicolumn{3}{c}{Region 10}
   & \multicolumn{3}{c}{Region 11} \\
\cmidrule(lr){2-4} \cmidrule(l){5-7}
$\mu$ (-) & $\eta_2$ (-) & $B_0$ ($\mu$G) & $\chi^2_{\mt{red}}$
          & $\eta_2$ (-) & $B_0$ ($\mu$G) & $\chi^2_{\mt{red}}$ \\
\cmidrule{1-7}
0.00 & $\gg 2200$ & $\gg 2800$ & 23.87
     & $\gg 110$ & $\gg 1370$ & 29.74 \\
0.33 & $\gg 195$ & $\gg 1260$ & 18.99
     & ${12}^{+22}_{-6}$ & ${689}^{+226}_{-111}$ & 29.8 \\
0.50 & ${67}^{+811}_{-45}$ & ${899}^{+1124}_{-248}$ & 18.9
     & ${5.4}^{+4.4}_{-2.1}$ & ${569}^{+89}_{-58}$ & 30.0 \\
1.00 & ${4.9}^{+2.1}_{-1.4}$ & ${439}^{+37}_{-29}$ & 18.7
     & ${1.8}^{+0.6}_{-0.5}$ & ${454}^{+25}_{-21}$ & 30.6 \\
1.50 & ${2.2}^{+0.5}_{-0.4}$ & ${371}^{+15}_{-13}$ & 18.8
     & ${1.0}^{+0.3}_{-0.2}$ & ${416}^{+14}_{-12}$ & 31.6 \\
2.00 & ${1.4}^{+0.3}_{-0.2}$ & ${343}^{+9}_{-8}$ & 19.4
     & ${0.7}^{+0.2}_{-0.1}$ & ${397}^{+9}_{-9}$ & 32.9 \\

\cmidrule{1-7}
{} & \multicolumn{3}{c}{Region 12}
   & \multicolumn{3}{c}{Region 13} \\
\cmidrule(lr){2-4} \cmidrule(l){5-7}
$\mu$ (-) & $\eta_2$ (-) & $B_0$ ($\mu$G) & $\chi^2_{\mt{red}}$
          & $\eta_2$ (-) & $B_0$ ($\mu$G) & $\chi^2_{\mt{red}}$ \\
\cmidrule{1-7}
0.00 & $>0$ & $>4180$ & 35.2
     & $>1944$ & $>3970$ & 13.9 \\
0.33 & $>0$ & $>2770$ & 28.4
     & $>1590$ & $>3630$ & 6.4 \\
0.50 & ${1210}^{+\infty}_{-1140}$ & ${3254}^{+\infty}_{-1930}$ & 27.8
     & $>1090$ & $>3175$ & 3.9 \\
1.00 & ${7.07}^{+3.7}_{-2.2}$ & ${690}^{+76}_{-55}$ & 28.3
     & ${66}^{+791}_{-44}$ & ${1276}^{+1585}_{-350}$ & 1.8 \\
1.50 & ${2.67}^{+0.71}_{-0.54}$ & ${556}^{+26}_{-22}$ & 29.1
     & ${7.17}^{+4.15}_{-2.33}$ & ${686}^{+81}_{-58}$ & 1.7 \\
2.00 & ${1.56}^{+0.31}_{-0.25}$ & ${507}^{+14}_{-13}$ & 30.3
     & ${3.25}^{+1.04}_{-0.75}$ & ${574}^{+33}_{-27}$ & 1.7 \\

\midrule
\multicolumn{7}{c}{Filament 2} \\
\cmidrule{1-7}
{} & \multicolumn{3}{c}{Region 2\tablenotemark{a}}
   & \multicolumn{3}{c}{Region 3} \\
\cmidrule(lr){2-4} \cmidrule(l){5-7}
$\mu$ (-) & $\eta_2$ (-) & $B_0$ ($\mu$G) & $\chi^2_{\mt{red}}$
          & $\eta_2$ (-) & $B_0$ ($\mu$G) & $\chi^2_{\mt{red}}$ \\
\cmidrule{1-7}
0.00 & ${0.00}^{+0.09}_{-0.00}$ & ${320}^{+11}_{-4}$ & 26.1
     & $>360$ & $>2200$ & 9.3 \\
0.33 & ${0.00}^{+0.09}_{-0.00}$ & ${320}^{+10}_{-4}$ & 26.1
     & ${51}^{+7000}_{-38}$ & ${1180}^{+4540}_{-374}$ & 8.2 \\
0.50 & ${0.00}^{+0.09}_{-0.00}$ & ${320}^{+10}_{-4}$ & 26.1
     & ${12.2}^{+22.4}_{-6.2}$ & ${783}^{+263}_{-128}$ & 8.2 \\
1.00 & ${0.01}^{+0.08}_{-0.01}$ & ${322}^{+10}_{-6}$ & 26.1
     & ${2.74}^{+1.24}_{-0.80}$ & ${546}^{+43}_{-33}$ & 8.3 \\
1.50 & ${0.03}^{+0.07}_{-0.03}$ & ${324}^{+9}_{-7}$ & 26.0
     & ${1.44}^{+0.42}_{-0.33}$ & ${486}^{+20}_{-17}$ & 8.6 \\
2.00 & ${0.04}^{+0.06}_{-0.04}$ & ${325}^{+8}_{-7}$ & 25.7
     & ${0.96}^{+0.23}_{-0.19}$ & ${458}^{+13}_{-12}$ & 9.1 \\
\bottomrule
\end{tabular}
\tablecomments{Region 1 simple fit is presented in main text.}
\tablenotetext{1}{Fits for regions 2, 6, 8 have 2 dof; rest have 3}

\end{table}


\begin{table}[ht]
    \tiny
    \centering
    \caption{Simple model best fits for individual regions, Filaments 3--5.
    Same FWHM as in main text.}
    \begin{tabular}{@{}rllrllr@{}}
\toprule
\multicolumn{7}{c}{Filament 3} \\
\cmidrule{1-7}
{} & \multicolumn{3}{c}{Region 4}
   & \multicolumn{3}{c}{Region 5} \\
\cmidrule(lr){2-4} \cmidrule(l){5-7}
$\mu$ (-) & $\eta_2$ (-) & $B_0$ ($\mu$G) & $\chi^2_{\mt{red}}$
          & $\eta_2$ (-) & $B_0$ ($\mu$G) & $\chi^2_{\mt{red}}$ \\
\cmidrule{1-7}
0.00 & $>5670$ & $>5990$ & 38.9
     & $>2480$ & $>5500$ & 12.8 \\
0.33 & $>5040$ & $>5622$ & 18.4
     & $>1160$ & $>4200$ & 2.7 \\
0.50 & $>3830$ & $>5080$ & 11.5
     & $>210$ & $>2400$ & 1.1 \\
1.00 & ${67}^{+205}_{-38}$ & ${1352}^{+742}_{-294}$ & 4.7
     & ${7.6}^{+4.0}_{-2.3}$ & ${891}^{+102}_{-75}$ & 1.6 \\
1.50 & ${6.3}^{+2.0}_{-1.4}$ & ${690}^{+49}_{-39}$ & 3.4
     & ${2.6}^{+0.6}_{-0.5}$ & ${693}^{+31}_{-27}$ & 2.4 \\
2.00 & ${2.8}^{+0.5}_{-0.4}$ & ${574}^{+20}_{-18}$ & 2.5
     & ${1.5}^{+0.3}_{-0.2}$ & ${625}^{+17}_{-15}$ & 3.8 \\

\midrule
\multicolumn{7}{c}{Filament 4} \\
\cmidrule{1-7}
{} & \multicolumn{3}{c}{Region 6\tablenotemark{a}}
   & \multicolumn{3}{c}{Region 7} \\
\cmidrule(lr){2-4} \cmidrule(l){5-7}
$\mu$ (-) & $\eta_2$ (-) & $B_0$ ($\mu$G) & $\chi^2_{\mt{red}}$
          & $\eta_2$ (-) & $B_0$ ($\mu$G) & $\chi^2_{\mt{red}}$ \\
\cmidrule{1-7}
0.00 & ${0.00}^{+0.17}_{-0.00}$ & ${260}^{+16}_{-4}$ & 2.2
     & ${31.2}^{+\infty}_{-23.9}$ & ${1122}^{+\infty}_{-367}$ & 19.9 \\
0.33 & ${0.00}^{+0.11}_{-0.00}$ & ${260}^{+10}_{-4}$ & 2.2
     & ${3.44}^{+3.20}_{-1.44}$ & ${625}^{+103}_{-65}$ & 19.7 \\
0.50 & ${0.00}^{+0.09}_{-0.00}$ & ${260}^{+9}_{-4}$ & 2.2
     & ${2.21}^{+1.39}_{-0.78}$ & ${569}^{+60}_{-43}$ & 19.6 \\
1.00 & ${0.00}^{+0.06}_{-0.00}$ & ${260}^{+6}_{-4}$ & 2.2
     & ${1.03}^{+0.37}_{-0.27}$ & ${497}^{+24}_{-20}$ & 19.3 \\
1.50 & ${0.00}^{+0.04}_{-0.00}$ & ${260}^{+5}_{-4}$ & 2.2
     & ${0.66}^{+0.19}_{-0.15}$ & ${469}^{+15}_{-13}$ & 19.1 \\
2.00 & ${0.00}^{+0.03}_{-0.00}$ & ${260}^{+4}_{-4}$ & 2.2
     & ${0.48}^{+0.12}_{-0.10}$ & ${454}^{+11}_{-10}$ & 19.0 \\

\midrule
\multicolumn{7}{c}{Filament 5} \\
\cmidrule{1-7}
{} & \multicolumn{3}{c}{Region 8\tablenotemark{a}}
   & \multicolumn{3}{c}{Region 9} \\
\cmidrule(lr){2-4} \cmidrule(l){5-7}
$\mu$ (-) & $\eta_2$ (-) & $B_0$ ($\mu$G) & $\chi^2_{\mt{red}}$
          & $\eta_2$ (-) & $B_0$ ($\mu$G) & $\chi^2_{\mt{red}}$ \\
\cmidrule{1-7}
0.00 & $>87$ & $>636$ & 6.7
     & $>2460$ & $>3950$ & 72.5 \\
0.33 & ${12.5}^{+30.3}_{-6.9}$ & ${359}^{+147}_{-64}$ & 6.7
     & $>930$ & $>2820$ & 65.7 \\
0.50 & ${5.87}^{+5.94}_{-2.52}$ & ${296}^{+55}_{-34}$ & 6.6
     & $>164$ & $>1600$ & 64.7 \\
1.00 & ${2.01}^{+0.84}_{-0.57}$ & ${236}^{+15}_{-12}$ & 6.3
     & ${10.6}^{+8}_{-4}$ & ${714}^{+109}_{-73}$ & 64.7 \\
1.50 & ${1.17}^{+0.4}_{-0.3}$ & ${217}^{+8}_{-7}$ & 6.2
     & ${3.7}^{+1.17}_{-0.8}$ & ${555}^{+33}_{-27}$ & 64.8 \\
2.00 & ${0.8}^{+0.2}_{-0.2}$ & ${207}^{+5}_{-5}$ & 6.3
     & ${2.1}^{+0.5}_{-0.4}$ & ${499}^{+18}_{-16}$ & 65.0 \\
\bottomrule
\end{tabular}
\tablecomments{Region 1 simple fit is presented in main text.}
\tablenotetext{1}{Fits for regions 2, 6, 8 have 2 dof; rest have 3}

\end{table}

% -------------
% Region 5 data
% -------------

\begin{table*}[ht]
    \tiny
    \centering
    \caption{Measured FWHMs for all regions (22).}
    \begin{tabular}{@{}rlllllr@{ $\pm$ }lr@{ $\pm$ }lr@{ $\pm$ }lr@{ $\pm$ }l@{}}
\toprule
{} & \multicolumn{13}{c}{Filament 1} \\
\cmidrule(l){2-14}
{} & \multicolumn{5}{c}{FWHM (arcsec)} & \multicolumn{8}{c}{$\mE$ (-)} \\
\cmidrule(lr){2-6} \cmidrule(l){7-14}
Region & 0.7--1 keV & 1--1.7 keV & 2--3 keV & 3--4.5 keV & 4.5--7 keV
       & \multicolumn{2}{c}{1 keV} & \multicolumn{2}{c}{2 keV}
       & \multicolumn{2}{c}{3 keV} & \multicolumn{2}{c}{4.5 keV} \\
\cmidrule{1-14}
1 & {} & ${8.87}^{+0.18}_{-0.16}$ & ${6.90}^{+0.27}_{-0.21}$ & ${7.51}^{+0.30}_{-0.24}$ & ${5.76}^{+0.48}_{-0.42}$
  & \multicolumn{2}{c}{} & $-0.36$ & $0.02$ & $0.21$ & $0.03$ & $-0.65$ & $0.14$ \\
2 & {} & ${4.40}^{+0.12}_{-0.09}$ & ${2.49}^{+0.13}_{-0.10}$ & ${3.05}^{+0.17}_{-0.12}$ & ${4.11}^{+0.34}_{-0.30}$
  & \multicolumn{2}{c}{} & $-0.82$ & $0.06$ & $0.51$ & $0.08$ & $0.73$ & $0.16$ \\
3 & ${1.85}^{+0.14}_{-0.11}$ & ${2.57}^{+0.09}_{-0.08}$ & ${1.84}^{+0.09}_{-0.07}$ & ${2.14}^{+0.11}_{-0.11}$ & ${1.35}^{+0.11}_{-0.09}$
  & $0.91$ & $0.19$ & $-0.48$ & $0.04$ & $0.38$ & $0.06$ & $-1.14$ & $0.25$ \\

\midrule
{} & \multicolumn{13}{c}{Filament 2} \\
\cmidrule(l){2-14}
{} & \multicolumn{5}{c}{FWHM (arcsec)} & \multicolumn{8}{c}{$\mE$ (-)} \\
\cmidrule(lr){2-6} \cmidrule(l){7-14}
Region & 0.7--1 keV & 1--1.7 keV & 2--3 keV & 3--4.5 keV & 4.5--7 keV
       & \multicolumn{2}{c}{1 keV} & \multicolumn{2}{c}{2 keV}
       & \multicolumn{2}{c}{3 keV} & \multicolumn{2}{c}{4.5 keV} \\
\cmidrule{1-14}
4 & ${6.76}^{+0.41}_{-0.35}$ & ${4.48}^{+0.09}_{-0.09}$ & ${3.33}^{+0.11}_{-0.09}$ & ${3.69}^{+0.12}_{-0.11}$ & ${3.28}^{+0.21}_{-0.19}$
  & $-1.15$ & $0.19$ & $-0.43$ & $0.02$ & $0.25$ & $0.03$ & $-0.29$ & $0.05$ \\
5 & ${9.12}^{+0.45}_{-0.40}$ & ${4.69}^{+0.11}_{-0.12}$ & ${3.15}^{+0.12}_{-0.11}$ & ${3.31}^{+0.15}_{-0.13}$ & ${3.09}^{+0.22}_{-0.18}$
  & $-1.86$ & $0.28$ & $-0.57$ & $0.04$ & $0.12$ & $0.02$ & $-0.17$ & $0.03$ \\
6 & ${2.81}^{+0.20}_{-0.19}$ & ${2.40}^{+0.05}_{-0.06}$ & ${2.98}^{+0.11}_{-0.09}$ & ${2.10}^{+0.08}_{-0.09}$ & ${2.28}^{+0.15}_{-0.15}$
  & $-0.44$ & $0.09$ & $0.31$ & $0.02$ & $-0.86$ & $0.11$ & $0.20$ & $0.04$ \\
7 & ${3.04}^{+0.22}_{-0.18}$ & ${2.41}^{+0.05}_{-0.05}$ & ${2.31}^{+0.08}_{-0.08}$ & ${1.85}^{+0.09}_{-0.07}$ & ${1.85}^{+0.11}_{-0.08}$
  & $-0.66$ & $0.13$ & $-0.06$ & $0.00$ & $-0.55$ & $0.08$ & $-0.00$ & $0.00$ \\
8 & ${2.56}^{+0.23}_{-0.22}$ & ${2.79}^{+0.08}_{-0.08}$ & ${2.40}^{+0.10}_{-0.09}$ & ${2.11}^{+0.10}_{-0.09}$ & ${2.44}^{+0.20}_{-0.18}$
  & $0.24$ & $0.06$ & $-0.22$ & $0.02$ & $-0.32$ & $0.05$ & $0.36$ & $0.08$ \\
9 & ${2.45}^{+0.27}_{-0.26}$ & ${2.40}^{+0.07}_{-0.06}$ & ${2.54}^{+0.11}_{-0.11}$ & ${1.92}^{+0.09}_{-0.09}$ & ${2.25}^{+0.17}_{-0.17}$
  & $-0.06$ & $0.02$ & $0.08$ & $0.01$ & $-0.69$ & $0.11$ & $0.39$ & $0.08$ \\
10 & ${2.79}^{+0.28}_{-0.27}$ & ${2.06}^{+0.07}_{-0.06}$ & ${1.84}^{+0.10}_{-0.09}$ & ${1.62}^{+0.09}_{-0.08}$ & ${1.62}^{+0.14}_{-0.12}$
  & $-0.85$ & $0.25$ & $-0.16$ & $0.01$ & $-0.33$ & $0.06$ & $0.00$ & $0.00$ \\

\midrule
{} & \multicolumn{13}{c}{Filament 3} \\
\cmidrule(l){2-14}
{} & \multicolumn{5}{c}{FWHM (arcsec)} & \multicolumn{8}{c}{$\mE$ (-)} \\
\cmidrule(lr){2-6} \cmidrule(l){7-14}
Region & 0.7--1 keV & 1--1.7 keV & 2--3 keV & 3--4.5 keV & 4.5--7 keV
       & \multicolumn{2}{c}{1 keV} & \multicolumn{2}{c}{2 keV}
       & \multicolumn{2}{c}{3 keV} & \multicolumn{2}{c}{4.5 keV} \\
\cmidrule{1-14}
11 & {} & ${3.44}^{+0.15}_{-0.13}$ & ${2.57}^{+0.16}_{-0.14}$ & ${1.92}^{+0.14}_{-0.13}$ & ${3.23}^{+0.47}_{-0.39}$
  & \multicolumn{2}{c}{} & $-0.42$ & $0.04$ & $-0.72$ & $0.16$ & $1.28$ & $0.47$ \\
12 & {} & ${4.07}^{+0.17}_{-0.16}$ & ${2.67}^{+0.16}_{-0.13}$ & ${3.08}^{+0.23}_{-0.21}$ & ${2.23}^{+0.21}_{-0.17}$
  & \multicolumn{2}{c}{} & $-0.61$ & $0.06$ & $0.35$ & $0.08$ & $-0.79$ & $0.22$ \\
13 & ${3.69}^{+0.26}_{-0.19}$ & ${2.52}^{+0.05}_{-0.05}$ & ${2.42}^{+0.08}_{-0.06}$ & ${2.03}^{+0.09}_{-0.10}$ & ${1.87}^{+0.11}_{-0.14}$
  & $-1.06$ & $0.19$ & $-0.06$ & $0.00$ & $-0.44$ & $0.06$ & $-0.20$ & $0.04$ \\


\midrule
{} & \multicolumn{13}{c}{Filament 4} \\
\cmidrule(l){2-14}
{} & \multicolumn{5}{c}{FWHM (arcsec)} & \multicolumn{8}{c}{$\mE$ (-)} \\
\cmidrule(lr){2-6} \cmidrule(l){7-14}
Region & 0.7--1 keV & 1--1.7 keV & 2--3 keV & 3--4.5 keV & 4.5--7 keV
       & \multicolumn{2}{c}{1 keV} & \multicolumn{2}{c}{2 keV}
       & \multicolumn{2}{c}{3 keV} & \multicolumn{2}{c}{4.5 keV} \\
\cmidrule{1-14}
14 & ${2.99}^{+0.18}_{-0.16}$ & ${2.43}^{+0.06}_{-0.04}$ & ${2.23}^{+0.08}_{-0.07}$ & ${2.38}^{+0.10}_{-0.08}$ & ${2.19}^{+0.12}_{-0.10}$
  & $-0.58$ & $0.10$ & $-0.13$ & $0.01$ & $0.17$ & $0.02$ & $-0.20$ & $0.03$ \\
15 & ${2.76}^{+0.17}_{-0.16}$ & ${2.01}^{+0.05}_{-0.04}$ & ${1.80}^{+0.06}_{-0.05}$ & ${1.87}^{+0.07}_{-0.05}$ & ${1.52}^{+0.09}_{-0.08}$
  & $-0.90$ & $0.16$ & $-0.16$ & $0.01$ & $0.08$ & $0.01$ & $-0.51$ & $0.08$ \\
16 & ${1.98}^{+0.15}_{-0.13}$ & ${1.75}^{+0.04}_{-0.03}$ & ${1.54}^{+0.06}_{-0.05}$ & ${1.26}^{+0.06}_{-0.04}$ & ${1.23}^{+0.08}_{-0.06}$
  & $-0.34$ & $0.07$ & $-0.18$ & $0.01$ & $-0.51$ & $0.07$ & $-0.04$ & $0.01$ \\
17 & ${1.71}^{+0.13}_{-0.12}$ & ${1.95}^{+0.05}_{-0.05}$ & ${1.57}^{+0.06}_{-0.07}$ & ${1.45}^{+0.07}_{-0.06}$ & ${2.05}^{+0.16}_{-0.14}$
  & $0.36$ & $0.08$ & $-0.31$ & $0.02$ & $-0.19$ & $0.03$ & $0.85$ & $0.18$ \\

\midrule
{} & \multicolumn{13}{c}{Filament 5} \\
\cmidrule(l){2-14}
{} & \multicolumn{5}{c}{FWHM (arcsec)} & \multicolumn{8}{c}{$\mE$ (-)} \\
\cmidrule(lr){2-6} \cmidrule(l){7-14}
Region & 0.7--1 keV & 1--1.7 keV & 2--3 keV & 3--4.5 keV & 4.5--7 keV
       & \multicolumn{2}{c}{1 keV} & \multicolumn{2}{c}{2 keV}
       & \multicolumn{2}{c}{3 keV} & \multicolumn{2}{c}{4.5 keV} \\
\cmidrule{1-14}
18 & {} & ${4.91}^{+0.14}_{-0.13}$ & ${3.55}^{+0.18}_{-0.18}$ & ${3.12}^{+0.21}_{-0.20}$ & ${1.85}^{+0.23}_{-0.18}$
  & \multicolumn{2}{c}{} & $-0.47$ & $0.04$ & $-0.32$ & $0.06$ & $-1.29$ & $0.41$ \\
19 & {} & ${2.40}^{+0.08}_{-0.06}$ & ${2.38}^{+0.12}_{-0.09}$ & ${2.18}^{+0.12}_{-0.11}$ & ${1.64}^{+0.17}_{-0.14}$
  & \multicolumn{2}{c}{} & $-0.01$ & $0.00$ & $-0.22$ & $0.04$ & $-0.70$ & $0.19$ \\
20 & ${4.81}^{+0.31}_{-0.31}$ & ${1.88}^{+0.06}_{-0.03}$ & ${1.90}^{+0.09}_{-0.06}$ & ${1.57}^{+0.07}_{-0.06}$ & ${2.18}^{+0.23}_{-0.23}$
  & $-2.64$ & $0.51$ & $0.02$ & $0.00$ & $-0.47$ & $0.07$ & $0.81$ & $0.23$ \\
21 & ${4.27}^{+0.32}_{-0.22}$ & ${2.48}^{+0.07}_{-0.04}$ & ${2.63}^{+0.11}_{-0.07}$ & ${1.61}^{+0.07}_{-0.04}$ & ${2.67}^{+0.23}_{-0.19}$
  & $-1.53$ & $0.29$ & $0.09$ & $0.01$ & $-1.22$ & $0.15$ & $1.25$ & $0.26$ \\
22 & {} & ${2.96}^{+0.05}_{-0.09}$ & ${3.77}^{+0.11}_{-0.12}$ & ${2.16}^{+0.09}_{-0.10}$ & ${1.53}^{+0.19}_{-0.22}$
  & \multicolumn{2}{c}{} & $0.35$ & $0.02$ & $-1.38$ & $0.19$ & $-0.86$ & $0.30$ \\
\midrule
Mean & ${3.57}^{+0.52}_{-0.52}$ & ${3.18}^{+0.34}_{-0.34}$ & ${2.67}^{+0.24}_{-0.24}$ & ${2.45}^{+0.28}_{-0.28}$ & ${2.37}^{+0.22}_{-0.22}$
  & $-0.70$ & $0.23$ & $-0.21$ & $0.06$ & $-0.28$ & $0.11$ & $-0.04$ & $0.15$ \\
\bottomrule
\end{tabular}
\tablecomments{Mean values computed for all regions; mean $\mE$ values are
averages for region $\mE$ values (i.e., not computed from mean FWHMs).  Errors
on mean values are standard errors of the mean.  DRAFT ONLY: as noted in text,
$\mE$ values are not used in subsequent analysis!}

\end{table*}

\begin{table*}[ht]
    \tiny
    \centering
    \caption{Full model best fits for smaller regions, filaments 1--3.  Same
    FWHM as in main text.}
    \begin{tabular}{@{}r r@{ $\pm$ }lr@{ $\pm$ }lr
                     r@{ $\pm$ }lr@{ $\pm$ }lr
                     r@{ $\pm$ }lr@{ $\pm$ }lr
                     r@{ $\pm$ }lr@{ $\pm$ }lr@{}}

\toprule
\multicolumn{21}{c}{Filament 1} \\
\cmidrule{1-21}
{} & \multicolumn{5}{c}{Region 1}
   & \multicolumn{5}{c}{Region 2}
   & \multicolumn{5}{c}{Region 3} \\
\cmidrule(lr){2-6} \cmidrule(lr){7-11} \cmidrule(lr){12-16}
%$\mu$ (-) & $\eta_2$ (-) & $B_0$ ($\mu$G) & $\chi^2_{\mt{red}}$
%          & $\eta_2$ (-) & $B_0$ ($\mu$G) & $\chi^2_{\mt{red}}$
%          & $\eta_2$ (-) & $B_0$ ($\mu$G) & $\chi^2_{\mt{red}}$ \\
$\mu$ (-) & \multicolumn{2}{c}{$\eta_2$ (-)}
          & \multicolumn{2}{c}{$B_0$ ($\mu$G)} & $\chi^2_{\mt{red}}$
          & \multicolumn{2}{c}{$\eta_2$ (-)}
          & \multicolumn{2}{c}{$B_0$ ($\mu$G)} & $\chi^2_{\mt{red}}$
          & \multicolumn{2}{c}{$\eta_2$ (-)}
          & \multicolumn{2}{c}{$B_0$ ($\mu$G)} & $\chi^2_{\mt{red}}$ \\
\cmidrule{1-16}
0.00 & $20$ & $450$ & $329$ & $1572$ & 12.9
     & $0.9$ & $1.1$ & $324$ & $50$ & 38.5
     & $26$ & $199$ & $857$ & $1348$ & 23.5 \\
0.33 & $54$ & $1305$ & $386$ & $2057$ & 7.9
     & $0.8$ & $0.7$ & $309$ & $29$ & 37.9
     & $88$ & $689$ & $1054$ & $1815$ & 20.0 \\
0.50 & $69$ & $755$ & $395$ & $972$ & 6.9
     & $0.7$ & $0.5$ & $302$ & $23$ & 37.6
     & $246$ & $1042$ & $1290$ & $1214$ & 19.1 \\
1.00 & $7$ & $4$ & $223$ & $24$ & 6.5
     & $0.8$ & $0.5$ & $293$ & $15$ & 36.6
     & $17$ & $19$ & $658$ & $144$ & 19.8 \\
1.50 & $4.2$ & $1.6$ & $197$ & $10$ & 6.2
     & $0.8$ & $0.5$ & $288$ & $11$ & 35.6
     & $7$ & $3$ & $517$ & $39$ & 21.3 \\
2.00 & $3.8$ & $1.1$ & $186$ & $6$ & 6.2
     & $1.0$ & $0.5$ & $285$ & $9$ & 34.6
     & $4.8$ & $1.7$ & $472$ & $22$ & 22.8 \\

\midrule
\multicolumn{21}{c}{Filament 2} \\
\cmidrule{1-21}
{} & \multicolumn{5}{c}{Region 4}
   & \multicolumn{5}{c}{Region 5}
   & \multicolumn{5}{c}{Region 6}
   & \multicolumn{5}{c}{Region 7} \\
\cmidrule(lr){2-6} \cmidrule(lr){7-11} \cmidrule(lr){12-16} \cmidrule(l){17-21}
%$\mu$ (-) & $\eta_2$ (-) & $B_0$ ($\mu$G) & $\chi^2_{\mt{red}}$
%          & $\eta_2$ (-) & $B_0$ ($\mu$G) & $\chi^2_{\mt{red}}$
%          & $\eta_2$ (-) & $B_0$ ($\mu$G) & $\chi^2_{\mt{red}}$
%          & $\eta_2$ (-) & $B_0$ ($\mu$G) & $\chi^2_{\mt{red}}$ \\
$\mu$ (-) & \multicolumn{2}{c}{$\eta_2$ (-)}
          & \multicolumn{2}{c}{$B_0$ ($\mu$G)} & $\chi^2_{\mt{red}}$
          & \multicolumn{2}{c}{$\eta_2$ (-)}
          & \multicolumn{2}{c}{$B_0$ ($\mu$G)} & $\chi^2_{\mt{red}}$
          & \multicolumn{2}{c}{$\eta_2$ (-)}
          & \multicolumn{2}{c}{$B_0$ ($\mu$G)} & $\chi^2_{\mt{red}}$
          & \multicolumn{2}{c}{$\eta_2$ (-)}
          & \multicolumn{2}{c}{$B_0$ ($\mu$G)} & $\chi^2_{\mt{red}}$ \\
\cmidrule{1-21}
0.00 & $15$ & $118$ & $487$ & $777$ & 18.1
     & $0$ & $0$ & $253$ & $13$ & 31.7
     & $22$ & $279$ & $755$ & $1971$ & 45.3
     & $20$ & $307$ & $756$ & $2424$ & 11.1 \\
0.33 & $13$ & $22$ & $441$ & $162$ & 16.1
     & $0.3$ & $0.3$ & $265$ & $15$ & 30.2
     & $60$ & $1301$ & $878$ & $4205$ & 34.5
     & $56$ & $1064$ & $894$ & $3721$ & 6.0 \\
0.50 & $6$ & $6$ & $375$ & $64$ & 15.6
     & $0.3$ & $0.3$ & $262$ & $12$ & 30.0
     & $101$ & $2246$ & $957$ & $4757$ & 29.9
     & $88$ & $1859$ & $958$ & $4554$ & 4.6 \\
1.00 & $3$ & $1$ & $311$ & $20$ & 14.3
     & $0.4$ & $0.2$ & $256$ & $9$ & 29.4
     & $422$ & $11004$ & $1205$ & $7315$ & 19.8
     & $12$ & $9$ & $568$ & $84$ & 3.8 \\
1.50 & $3$ & $1$ & $290$ & $11$ & 13.1
     & $0.4$ & $0.2$ & $253$ & $7$ & 28.8
     & $5353$ & $41319$ & $1994$ & $3636$ & 16.1
     & $6$ & $2$ & $469$ & $28$ & 3.9 \\
2.00 & $3$ & $1$ & $280$ & $8$ & 12.1
     & $0.5$ & $0.2$ & $251$ & $6$ & 28.2
     & $34$ & $19$ & $572$ & $62$ & 17.0
     & $5$ & $1$ & $436$ & $16$ & 4.1 \\

%\midrule
%\multicolumn{13}{c}{Filament 2 (cont.)} \\
\cmidrule{1-21}
{} & \multicolumn{5}{c}{Region 8}
   & \multicolumn{5}{c}{Region 9}
   & \multicolumn{5}{c}{Region 10} \\
\cmidrule(lr){2-6} \cmidrule(lr){7-11} \cmidrule(lr){12-16}
%$\mu$ (-) & $\eta_2$ (-) & $B_0$ ($\mu$G) & $\chi^2_{\mt{red}}$
%          & $\eta_2$ (-) & $B_0$ ($\mu$G) & $\chi^2_{\mt{red}}$
%          & $\eta_2$ (-) & $B_0$ ($\mu$G) & $\chi^2_{\mt{red}}$ \\
$\mu$ (-) & \multicolumn{2}{c}{$\eta_2$ (-)}
          & \multicolumn{2}{c}{$B_0$ ($\mu$G)} & $\chi^2_{\mt{red}}$
          & \multicolumn{2}{c}{$\eta_2$ (-)}
          & \multicolumn{2}{c}{$B_0$ ($\mu$G)} & $\chi^2_{\mt{red}}$
          & \multicolumn{2}{c}{$\eta_2$ (-)}
          & \multicolumn{2}{c}{$B_0$ ($\mu$G)} & $\chi^2_{\mt{red}}$ \\
\cmidrule{1-16}
0.00 & $20$ & $426$ & $704$ & $3107$ & 9.6
     & $19$ & $493$ & $747$ & $3984$ & 17.1
     & $20$ & $422$ & $842$ & $3667$ & 4.8 \\
0.33 & $61$ & $1382$ & $847$ & $4196$ & 5.5
     & $60$ & $1452$ & $900$ & $4807$ & 11.5
     & $49$ & $1317$ & $966$ & $5659$ & 2.5 \\
0.50 & $97$ & $2640$ & $911$ & $5567$ & 4.2
     & $96$ & $2914$ & $969$ & $6617$ & 9.2
     & $71$ & $1332$ & $1018$ & $4258$ & 1.9 \\
1.00 & $31$ & $52$ & $647$ & $231$ & 2.9
     & $408$ & $13306$ & $1232$ & $9351$ & 5.2
     & $9$ & $8$ & $600$ & $106$ & 1.5 \\
1.50 & $9$ & $5$ & $469$ & $45$ & 3.0
     & $29$ & $29$ & $623$ & $127$ & 5.0
     & $5$ & $3$ & $512$ & $41$ & 1.3 \\
2.00 & $7$ & $2$ & $424$ & $23$ & 3.1
     & $13$ & $6$ & $497$ & $39$ & 5.4
     & $4$ & $2$ & $479$ & $25$ & 1.3 \\

\midrule
\multicolumn{21}{c}{Filament 3} \\
\cmidrule{1-21}
{} & \multicolumn{5}{c}{Region 11}
   & \multicolumn{5}{c}{Region 12}
   & \multicolumn{5}{c}{Region 13} \\
\cmidrule(lr){2-6} \cmidrule(lr){7-11} \cmidrule(lr){12-16}
$\mu$ (-) & \multicolumn{2}{c}{$\eta_2$ (-)}
          & \multicolumn{2}{c}{$B_0$ ($\mu$G)} & $\chi^2_{\mt{red}}$
          & \multicolumn{2}{c}{$\eta_2$ (-)}
          & \multicolumn{2}{c}{$B_0$ ($\mu$G)} & $\chi^2_{\mt{red}}$
          & \multicolumn{2}{c}{$\eta_2$ (-)}
          & \multicolumn{2}{c}{$B_0$ ($\mu$G)} & $\chi^2_{\mt{red}}$ \\
\cmidrule{1-16}
0.00 & $286$ & $338$ & $1020$ & $96$ & 7.0
     & $1.3$ & $2.3$ & $347$ & $91$ & 5.7
     & $20$ & $266$ & $722$ & $2031$ & 15.6 \\
0.33 & $0.4$ & $0.5$ & $333$ & $37$ & 6.8
     & $0.8$ & $1.0$ & $318$ & $43$ & 5.6
     & $54$ & $1053$ & $847$ & $3623$ & 10.6 \\
0.50 & $0.4$ & $0.5$ & $328$ & $31$ & 6.7
     & $0.8$ & $0.8$ & $310$ & $34$ & 5.5
     & $86$ & $1889$ & $910$ & $4496$ & 9.1 \\
1.00 & $0.4$ & $0.4$ & $320$ & $21$ & 6.6
     & $0.7$ & $0.5$ & $296$ & $21$ & 5.3
     & $13$ & $11$ & $551$ & $92$ & 8.2 \\
1.50 & $0.4$ & $0.4$ & $315$ & $16$ & 6.6
     & $0.6$ & $0.5$ & $289$ & $15$ & 5.2
     & $6$ & $2$ & $448$ & $29$ & 8.3 \\
2.00 & $0.4$ & $0.4$ & $312$ & $14$ & 6.5
     & $0.7$ & $0.4$ & $285$ & $12$ & 5.1
     & $5$ & $1$ & $414$ & $16$ & 8.6 \\
\bottomrule
\end{tabular}

\end{table*}

\begin{table*}[ht]
    \tiny
    \centering
    \caption{Full model best fits for smaller regions, filaments 4--5.  Same
    FWHM as in main text.}
    \begin{tabular}{@{}r r@{ $\pm$ }lr@{ $\pm$ }lr
                     r@{ $\pm$ }lr@{ $\pm$ }lr
                     r@{ $\pm$ }lr@{ $\pm$ }lr
                     r@{ $\pm$ }lr@{ $\pm$ }lr@{}}

\toprule
\multicolumn{21}{c}{Filament 4} \\
\cmidrule{1-21}
{} & \multicolumn{5}{c}{Region 14}
   & \multicolumn{5}{c}{Region 15}
   & \multicolumn{5}{c}{Region 16}
   & \multicolumn{5}{c}{Region 17} \\
\cmidrule(lr){2-6} \cmidrule(lr){7-11} \cmidrule(lr){12-16} \cmidrule(l){17-21}
$\mu$ (-) & \multicolumn{2}{c}{$\eta_2$ (-)}
          & \multicolumn{2}{c}{$B_0$ ($\mu$G)} & $\chi^2_{\mt{red}}$
          & \multicolumn{2}{c}{$\eta_2$ (-)}
          & \multicolumn{2}{c}{$B_0$ ($\mu$G)} & $\chi^2_{\mt{red}}$
          & \multicolumn{2}{c}{$\eta_2$ (-)}
          & \multicolumn{2}{c}{$B_0$ ($\mu$G)} & $\chi^2_{\mt{red}}$ \\
\cmidrule{1-21}
0.00 & $16$ & $219$ & $695$ & $1912$ & 32.0
     & $14$ & $232$ & $763$ & $2540$ & 21.9
     & $15$ & $241$ & $882$ & $2835$ & 3.9
     & $16$ & $321$ & $838$ & $3537$ & 17.8 \\
0.33 & $42$ & $761$ & $801$ & $3159$ & 20.5
     & $42$ & $782$ & $910$ & $3687$ & 13.5
     & $46$ & $817$ & $1054$ & $4137$ & 1.5
     & $44$ & $1126$ & $988$ & $5496$ & 13.0 \\
0.50 & $73$ & $1438$ & $876$ & $3883$ & 15.7
     & $71$ & $1365$ & $993$ & $4252$ & 10.5
     & $440$ & $549$ & $1695$ & $450$ & 1.4
     & $74$ & $2001$ & $1075$ & $6481$ & 11.3 \\
1.00 & $236$ & $3926$ & $1042$ & $4028$ & 6.0
     & $40$ & $69$ & $796$ & $306$ & 6.9
     & $3$ & $2$ & $555$ & $43$ & 1.5
     & $63$ & $205$ & $943$ & $690$ & 9.1 \\
1.50 & $36$ & $33$ & $622$ & $121$ & 3.4
     & $8$ & $4$ & $532$ & $41$ & 6.5
     & $2$ & $1$ & $505$ & $22$ & 1.6
     & $9$ & $6$ & $585$ & $64$ & 8.9 \\
2.00 & $13$ & $5$ & $473$ & $30$ & 2.7
     & $6$ & $2$ & $476$ & $21$ & 6.5
     & $2$ & $1$ & $483$ & $15$ & 1.9
     & $7$ & $3$ & $518$ & $31$ & 8.7 \\

\midrule
\multicolumn{21}{c}{Filament 5} \\
\cmidrule{1-21}
{} & \multicolumn{5}{c}{Region 18}
   & \multicolumn{5}{c}{Region 19}
   & \multicolumn{5}{c}{Region 20} \\
\cmidrule(lr){2-6} \cmidrule(lr){7-11} \cmidrule(lr){12-16}
$\mu$ (-) & \multicolumn{2}{c}{$\eta_2$ (-)}
          & \multicolumn{2}{c}{$B_0$ ($\mu$G)} & $\chi^2_{\mt{red}}$
          & \multicolumn{2}{c}{$\eta_2$ (-)}
          & \multicolumn{2}{c}{$B_0$ ($\mu$G)} & $\chi^2_{\mt{red}}$
          & \multicolumn{2}{c}{$\eta_2$ (-)}
          & \multicolumn{2}{c}{$B_0$ ($\mu$G)} & $\chi^2_{\mt{red}}$ \\
\cmidrule{1-16}
0.00 & $0$ & $0$ & $238$ & $15$ & 3.6
     & $19$ & $279$ & $730$ & $2219$ & 14.8
     & $17$ & $499$ & $836$ & $5079$ & 39.3 \\
0.33 & $0$ & $0$ & $238$ & $10$ & 3.5
     & $55$ & $1109$ & $867$ & $3833$ & 9.1
     & $43$ & $1251$ & $962$ & $6084$ & 34.9 \\
0.50 & $0$ & $0$ & $238$ & $8$ & 3.4
     & $91$ & $2171$ & $942$ & $5039$ & 7.0
     & $71$ & $2069$ & $1038$ & $6808$ & 33.3 \\
1.00 & $0$ & $0$ & $239$ & $5$ & 3.6
     & $2632$ & $9996$ & $1869$ & $1612$ & 3.9
     & $43$ & $104$ & $844$ & $445$ & 31.0 \\
1.50 & $0$ & $0$ & $239$ & $4$ & 3.6
     & $14$ & $12$ & $530$ & $87$ & 4.8
     & $9$ & $6$ & $571$ & $62$ & 30.3 \\
2.00 & $0$ & $0$ & $239$ & $4$ & 3.6
     & $8$ & $4$ & $449$ & $34$ & 5.9
     & $7$ & $3$ & $511$ & $30$ & 29.7 \\

%\midrule
%\multicolumn{13}{c}{Filament 5 (cont.)} \\
\cmidrule{1-21}
{} & \multicolumn{5}{c}{Region 21}
   & \multicolumn{5}{c}{Region 22} \\
\cmidrule(lr){2-6} \cmidrule(lr){7-11}
$\mu$ (-) & \multicolumn{2}{c}{$\eta_2$ (-)}
          & \multicolumn{2}{c}{$B_0$ ($\mu$G)} & $\chi^2_{\mt{red}}$
          & \multicolumn{2}{c}{$\eta_2$ (-)}
          & \multicolumn{2}{c}{$B_0$ ($\mu$G)} & $\chi^2_{\mt{red}}$ \\
\cmidrule{1-11}
0.00 & $15$ & $337$ & $683$ & $3269$ & 36.0
     & $17$ & $154$ & $610$ & $1111$ & 63.8 \\
0.33 & $3.1$ & $3.1$ & $476$ & $83$ & 35.8
     & $63$ & $476$ & $762$ & $1261$ & 58.3 \\
0.50 & $2.2$ & $1.6$ & $439$ & $50$ & 35.7
     & $310$ & $667$ & $1055$ & $495$ & 56.7 \\
1.00 & $1.4$ & $0.7$ & $393$ & $22$ & 35.6
     & $17$ & $24$ & $507$ & $148$ & 57.7 \\
1.50 & $1.3$ & $0.5$ & $375$ & $14$ & 35.5
     & $5$ & $3$ & $375$ & $30$ & 60.4 \\
2.00 & $1.3$ & $0.4$ & $365$ & $10$ & 35.5
     & $4$ & $1$ & $340$ & $16$ & 63.1 \\
\bottomrule
\end{tabular}


\end{table*}

\begin{table*}[ht]
    \scriptsize
    \centering
    \caption{Filament averages of best-fit parameters for constituent regions,
    with 22 instead of 13 regions.}
    \begin{tabular}{@{}rr@{ $\pm$ }lr@{ $\pm$ }lr@{ $\pm$ }lr@{ $\pm$ }lr@{ $\pm$ }lr@{ $\pm$ }l@{}}
\toprule
{} & \multicolumn{4}{c}{Filament 1}
   & \multicolumn{4}{c}{Filament 2}
   & \multicolumn{4}{c}{Filament 3} \\
\cmidrule(lr){2-5} \cmidrule(lr){6-9} \cmidrule(lr){10-13}
$\mu$ (-) & \multicolumn{2}{c}{$\eta_2$ (-)} & \multicolumn{2}{c}{$B_0$ ($\mu$G)}
          & \multicolumn{2}{c}{$\eta_2$ (-)} & \multicolumn{2}{c}{$B_0$ ($\mu$G)}
          & \multicolumn{2}{c}{$\eta_2$ (-)} & \multicolumn{2}{c}{$B_0$ ($\mu$G)} \\
\cmidrule{1-13}
0.00 & $15.6$ & $7.6$ & $503.5$ & $176.9$
     & $16.6$ & $2.9$ & $649.3$ & $78.0$
     & $102.3$ & $92.1$ & $696.3$ & $194.8$ \\
0.33 & $47.6$ & $25.4$ & $583.0$ & $236.5$
     & $42.8$ & $9.6$ & $741.5$ & $103.1$
     & $18.5$ & $17.8$ & $499.1$ & $174.1$ \\
0.50 & $105.3$ & $73.1$ & $662.5$ & $314.9$
     & $65.6$ & $16.5$ & $778.4$ & $120.1$
     & $29.0$ & $28.4$ & $516.2$ & $197.2$ \\
1.00 & $8.2$ & $4.8$ & $391.5$ & $134.9$
     & $126.5$ & $74.7$ & $688.3$ & $147.6$
     & $4.6$ & $4.1$ & $389.0$ & $81.1$ \\
1.50 & $3.9$ & $1.7$ & $333.8$ & $95.3$
     & $772.1$ & $763.6$ & $658.7$ & $227.8$
     & $2.3$ & $1.8$ & $350.6$ & $49.2$ \\
2.00 & $3.2$ & $1.2$ & $314.3$ & $84.0$
     & $9.3$ & $4.3$ & $419.9$ & $43.9$
     & $2.0$ & $1.4$ & $336.7$ & $39.3$ \\

\midrule
{} & \multicolumn{4}{c}{Filament 4}
   & \multicolumn{4}{c}{Filament 5} \\
\cmidrule(lr){2-5} \cmidrule(lr){6-9}
$\mu$ (-) & \multicolumn{2}{c}{$\eta_2$ (-)} & \multicolumn{2}{c}{$B_0$ ($\mu$G)}
          & \multicolumn{2}{c}{$\eta_2$ (-)} & \multicolumn{2}{c}{$B_0$ ($\mu$G)} \\
\cmidrule{1-9}
0.00 & $15.5$ & $0.4$ & $794.7$ & $41.2$
     & $13.6$ & $3.5$ & $619.6$ & $102.2$ \\
0.33 & $43.7$ & $0.8$ & $938.3$ & $54.3$
     & $33.0$ & $13.2$ & $660.9$ & $133.6$ \\
0.50 & $164.6$ & $91.9$ & $1159.5$ & $183.0$
     & $94.7$ & $56.7$ & $742.3$ & $169.1$ \\
1.00 & $85.6$ & $51.7$ & $834.1$ & $106.0$
     & $538.8$ & $523.4$ & $770.3$ & $292.0$ \\
1.50 & $14.0$ & $7.6$ & $561.0$ & $26.2$
     & $6.0$ & $2.6$ & $417.8$ & $59.9$ \\
2.00 & $6.9$ & $2.3$ & $487.3$ & $10.3$
     & $3.9$ & $1.5$ & $380.8$ & $46.8$ \\
\bottomrule
\end{tabular}
\tablecomments{Reported errors are standard errors of the mean.
Filaments 1, 3 have only 3 samples; Filament 2 has 7; Filament 4, 4; Filament
5, 5.}

\end{table*}

\begin{figure*}
    \centering
    \epsscale{0.7}
    \plotone{figures/f0-snr-reg5-inv.png}
    \caption{RGB image of Tycho with region selections overlaid.  Bands are
    0.7--1 keV (red), 1--2 keV (green) and 2--7 keV (blue).}
\end{figure*}


% --------------
% SN 1006 tables
% --------------
\clearpage
\section{Full model validation, SN 1006 (DRAFT ONLY)}

For comparison to \citetalias{ressler2014}, we reproduce Sean's full model fit
table (Table~\ref{tab:sean}) and present full model fits with 2 and 3 energy
bands (Tables~\ref{tab:sn1006-2band} and \ref{tab:sn1006-3band}, respectively).
We don't compare simple model results, as they are identical (only the error
calculations differ).
The procedures of Tables~\ref{tab:sean},~\ref{tab:sn1006-2band} are not quite
the same.  Sean used $\mE$ at 2 keV instead of the $1$--$2$ keV width for
fitting, whereas I used both widths.  But, it should not matter much as
the fit often has $\chi^2 \sim 0$, either way (depending on the exact
measurements and values of $\mu$).

Please take the errors with a grain of salt.  They are automatically
generated and \textbf{have not been manually validated}.  Some values may be
invalid, where my error-finding code failed and gave a best, conservative guess
of the error.

\begin{table}[ht]
    \tiny
    \centering
    \caption{Sean's SN 1006 best fit parameters \citepalias[Table 8]{ressler2014}.
    \label{tab:sean}}
    \begin{tabular}{@{} l c c c c c c @{}}
\toprule
{}&\multicolumn{2}{c}{Filament 1} & \multicolumn{2}{c}{Filament 2} & \multicolumn{2}{c}{Filament 3} \\
\midrule
$\mu$ &$\eta_{2}$ & $B_{0}$ &$\eta_{2}$ & $B_{0}$ & $\eta_{2}$ & $B_{0}$ \\
0 & 7.5 $\pm$ 2 & 142 $\pm$ 5 & - & - & $\lesssim$ 0.1& 77 $\pm$ .8\\
1/3 & 4 $\pm$ 1.3 & 120 $\pm$ 5 & - & - & $\lesssim$ 0.1 & 76 $\pm$ 1.4 \\
1/2 & 3 $\pm$ 1.1 & 112 $\pm$ 4 & - & - & $\lesssim$ 0.1 & 75 $\pm$ 1.0 \\
1 & 2 $\pm$ 1.0 & 100 $\pm$ 3 & 22 $\pm$ 3 & 214 $\pm$ 4 & $\lesssim$ 0.1 & 74 $\pm$ 1.1 \\
1.5 & 1.9 $\pm$ 1.2 & 95 $\pm$ 3 & 9 $\pm$ 1.2 & 167 $\pm$ 4 & $\lesssim$ 0.1 & 74 $\pm$ 1.2   \\
2 & 2 $\pm$ 1.0 & 92 $\pm$ 4  & 7 $\pm$ 1.1 & 152 $\pm$ 4 & $\lesssim$ 0.1 & 73 $\pm$ 1.2 \\
\midrule
& &\multicolumn{2}{c}{Filament 4} && \multicolumn{2}{c}{Filament 5} \\
\midrule
$\mu$ &&$\eta_{2}$ & $B_{0}$ &&$\eta_{2}$ & $B_{0}$\\
0 &&  $\lesssim$ 0.2 & 113 $\pm$ 2 && -& -\\
1/3 &&  $\lesssim$ 0.2 & 112 $\pm$2 && -& -\\
1/2 && $\lesssim$ 0.2  & 111 $\pm$2 && - & -\\
1 &&  $\lesssim$ 0.2 & 109 $\pm$ 2 && 80 $^{+\infty}_{-4}$   & 206 $\pm$ 3\\
1.5 && $\lesssim$ 0.2 & 108 $\pm$ 2 && 19 $\pm$ 2  & 140 $\pm$ 2\\
2 && $\lesssim$ 0.2 & 107 $\pm$ 2 && 12 $\pm$ 1.0  & 120 $\pm$ 2\\
\bottomrule
\end{tabular}

\end{table}

\begin{table*}[ht]
    \scriptsize
    \centering
    \caption{SN 1006 best fit parameters, 2 highest energy bands (full model).
    \label{tab:sn1006-2band}}
    %\renewcommand{\arraystretch}{1.5}
\begin{tabular}{@{}rllr llr llr @{}}

\toprule
{} & \multicolumn{3}{c}{Filament 1} & \multicolumn{3}{c}{Filament 2} &
     \multicolumn{3}{c}{Filament 3} \\
\cmidrule(lr){2-4}
\cmidrule(lr){5-7}
\cmidrule(lr){8-10}
$\mu$ (-) & $\eta_2$ (-) & $B_0$ ($\mu$G) & $\chi^2$
          & $\eta_2$ (-) & $B_0$ ($\mu$G) & $\chi^2$
          & $\eta_2$ (-) & $B_0$ ($\mu$G) & $\chi^2$ \\

\midrule
0.00 & ${21}^{\,+290}_{\,-20}$ & ${176}^{\,+108}_{\,-80}$ & 0.0086
     & ${21}^{\,+51}_{\,-14}$ & ${262}^{\,+77}_{\,-53}$ & 7.2668
     & ${0.004}^{\,+6800}_{\,-0.004}$ & ${74.75}^{\,+183}_{\,-0.77}$ & 0.8180\\[1.5pt]
0.33 & ${3.4}^{\,+100000}_{\,-3}$ & ${117}^{\,+552}_{\,-24}$ & 0.0000
     & ${68}^{\,+282}_{\,-56}$ & ${316}^{\,+136}_{\,-98}$ & 2.5848
     & ${0.02}^{\,+0.11}_{\,-0.02}$ & ${74.45}^{\,+1.04}_{\,-0.61}$ & 0.8037\\[1.5pt]
0.50 & ${2.5}^{\,+100000}_{\,-2.2}$ & ${109}^{\,+635}_{\,-17}$ & 0.0000
     & ${106}^{\,+792}_{\,-93}$ & ${337}^{\,+204}_{\,-122}$ & 1.0919
     & ${0.024}^{\,+0.11}_{\,-0.024}$ & ${74.16}^{\,+0.87}_{\,-0.53}$ & 0.7946\\[1.5pt]
1.00 & ${1.8}^{\,+6.6}_{\,-1.4}$ & ${98.4}^{\,+23}_{\,-9.2}$ & 0.0000
     & ${21}^{\,+100000}_{\,-15}$ & ${213}^{\,+1160}_{\,-42}$ & 0.0000
     & ${0.012}^{\,+0.11}_{\,-0.012}$ & ${73.68}^{\,+0.21}_{\,-0.88}$ & 0.5364\\[1.5pt]
1.50 & ${1.7}^{\,+3.6}_{\,-1.3}$ & ${94}^{\,+12}_{\,-6.4}$ & 0.0000
     & ${8.5}^{\,+11}_{\,-4.1}$ & ${166}^{\,+24}_{\,-14}$ & 0.0000
     & ${0.007}^{\,+0.13}_{\,-0.007}$ &
     $\left({73.4}^{\,+0}_{\,-1.2}\right)$\tablenotemark{a} & 0.5062\\[1.5pt]
2.00 & ${1.9}^{\,+3.1}_{\,-1.4}$ & ${91.6}^{\,+7.8}_{\,-4.9}$ & 0.0000
     & ${7.0}^{\,+4.9}_{\,-2.8}$ & ${152.4}^{\,+11}_{\,-7.7}$ & 0.0000
     & ${0.026}^{\,+0.10}_{\,-0.026}$ & ${72.21}^{\,+0.23}_{\,-0.53}$ & 0.4443\\

\midrule
{} & \multicolumn{3}{c}{Filament 4} & \multicolumn{3}{c}{Filament 5} \\
\cmidrule(lr){2-4}
\cmidrule(lr){5-7}
$\mu$ (-) & $\eta_2$ (-) & $B_0$ ($\mu$G) & $\chi^2$
          & $\eta_2$ (-) & $B_0$ ($\mu$G) & $\chi^2$ \\

\cmidrule(lr){1-7}
0.00 & ${4300}^{\,+2500}_{\,-4300}$ & ${374.7}^{\,+3.7}_{\,-266}$ & 0.1896
     & ${23}^{\,+33}_{\,-13}$ & ${192}^{\,+40}_{\,-31}$ & 17.9802\\[1.5pt]
0.33 & ${0.011}^{\,+0.16}_{\,-0.011}$ & ${109.2}^{\,+2.1}_{\,-0.9}$ & 0.2605
     & ${68}^{\,+167}_{\,-48}$ & ${230}^{\,+72}_{\,-54}$ & 8.7407\\[1.5pt]
0.50 & ${0.0025}^{\,+0.17}_{\,-0.0025}$ & ${109.2}^{\,+1.2}_{\,-1.3}$ & 0.1959
     & ${107}^{\,+416}_{\,-82}$ & ${246}^{\,+104}_{\,-67}$ & 5.3185\\[1.5pt]
1.00 & ${0.041}^{\,+0.15}_{\,-0.041}$ & ${107.48}^{\,+1.2}_{\,-0.72}$ & 0.2759
     & ${447}^{\,+16000}_{\,-424}$ & ${311}^{\,+389}_{\,-151}$ & 0.1152\\[1.5pt]
1.50 & ${0.023}^{\,+0.17}_{\,-0.023}$ & ${106.8}^{\,+0.54}_{\,-0.41}$ & 0.1114
     & ${20}^{\,+49}_{\,-10}$ & ${142}^{\,+42}_{\,-18}$ & 0.0000\\[1.5pt]
2.00 & ${0.013}^{\,+0.18}_{\,-0.013}$ & ${106.57}^{\,+0.33}_{\,-1.4}$ & 0.0457
     & ${12.1}^{\,+9.2}_{\,-4.7}$ & ${121.6}^{\,+12}_{\,-8.5}$ & 0.0000\\

\bottomrule
\end{tabular}
\tablenotetext{1}{Seems unlikely that error is $0$, probably bad calculation}

\end{table*}

\begin{table*}[ht]
    \scriptsize
    \centering
    \caption{SN 1006 best fit parameters, 3 energy bands (full model).
    \label{tab:sn1006-3band}}
    %\renewcommand{\arraystretch}{1.5}
\begin{tabular}{@{}rllr llr llr @{}}

\toprule
{} & \multicolumn{3}{c}{Filament 1} & \multicolumn{3}{c}{Filament 2} &
     \multicolumn{3}{c}{Filament 3} \\
\cmidrule(lr){2-4}
\cmidrule(lr){5-7}
\cmidrule(lr){8-10}
$\mu$ (-) & $\eta_2$ (-) & $B_0$ ($\mu$G) & $\chi^2$
          & $\eta_2$ (-) & $B_0$ ($\mu$G) & $\chi^2$
          & $\eta_2$ (-) & $B_0$ ($\mu$G) & $\chi^2$ \\

\midrule
0.00 & $25^{\,+170}_{\,-23}$ & ${183}^{\,+89}_{\,-70}$ & 0.1162
     & $\left({0.00}^{\,+0.26}_{\,-0.00}\right)$\tablenotemark{a} & ${132.18}^{\,+320}_{\,-0.72}$ & 53.8334
     & ${0.012}^{\,+0.18}_{\,-0.012}$ & ${75.25}^{\,+180}_{\,-0.9}$ & 1.7278\\[1.5pt]
0.33 & $730^{\,+6100}_{\,-730}$ & $350^{\,+130}_{\,-130}$ & 0.0102
     & ${0.082}^{\,+0.23}_{\,-0.082}$ & ${132.5}^{\,+3.5}_{\,-1}$ & 53.4559
     & ${0.01}^{\,+0.18}_{\,-0.01}$ & ${74.86}^{\,+1.4}_{\,-0.65}$ & 1.7621\\[1.5pt]
0.50 & $3.9^{\,+210}_{\,-2.9}$ & ${116}^{\,+630}_{\,-18}$ & 0.0741
     & ${0.043}^{\,+0.31}_{\,-0.043}$ & ${131.56}^{\,+3.8}_{\,-0.62}$ & 53.3622
     & ${0.0056}^{\,+0.16}_{\,-0.0056}$ & ${74.61}^{\,+0.72}_{\,-0.48}$ & 1.5513\\[1.5pt]
1.00 & ${2.6}^{\,+5.4}_{\,-1.7}$ & ${102.5}^{\,+18}_{\,-8.9}$ & 0.1420
     & ${0.18}^{\,+0.27}_{\,-0.17}$ & ${130.5}^{\,+2.3}_{\,-1}$ & 52.4123
     & ${0.016}^{\,+0.18}_{\,-0.016}$ & ${73.85}^{\,+0.72}_{\,-0.54}$ & 1.7703\\[1.5pt]
1.50 & ${2.5}^{\,+3.2}_{\,-1.5}$ & ${97.1}^{\,+9.1}_{\,-5.9}$ & 0.2221
     & ${0.2}^{\,+0.45}_{\,-0.18}$ & ${129.06}^{\,+2.6}_{\,-0.82}$ & 51.8975
     & ${0.01}^{\,+0.16}_{\,-0.01}$ & ${73.47}^{\,+0.68}_{\,-0.9}$ & 1.4024\\[1.5pt]
2.00 & ${2.7}^{\,+2.9}_{\,-1.6}$ & ${94.2}^{\,+6.2}_{\,-4.6}$ & 0.3025
     & ${0.53}^{\,+0.34}_{\,-0.44}$ & ${128.9}^{\,+1.6}_{\,-1.7}$ & 50.3784
     & ${0.015}^{\,+0.17}_{\,-0.015}$ & ${73.2741}^{\,+0.0092}_{\,-1.1}$ & 1.6535\\

\midrule
{} & \multicolumn{3}{c}{Filament 4} & \multicolumn{3}{c}{Filament 5} \\
\cmidrule(lr){2-4}
\cmidrule(lr){5-7}
$\mu$ (-) & $\eta_2$ (-) & $B_0$ ($\mu$G) & $\chi^2$
          & $\eta_2$ (-) & $B_0$ ($\mu$G) & $\chi^2$ \\

\cmidrule(lr){1-7}
0.00 & - & - & -
     & ${23}^{\,+34}_{\,-13}$ & ${192}^{\,+40}_{\,-31}$ & 18.1971 \\[1.5pt]
0.33 & - & - & -
     & ${61}^{\,+190}_{\,-43}$ & ${223}^{\,+80}_{\,-53}$ & 9.7081 \\[1.5pt]
0.50 & - & - & -
     & ${101}^{\,+470}_{\,-80}$ & ${242}^{\,+114}_{\,-71}$ & 6.8545 \\[1.5pt]
1.00 & - & - & -
     & ${35}^{\,+45000}_{\,-25}$ & ${174}^{\,+680}_{\,-39}$ & 3.7991 \\[1.5pt]
1.50 & - & - & -
     & ${10.9}^{\,+12}_{\,-4.7}$ & ${127}^{\,+18}_{\,-11}$ & 3.2257 \\[1.5pt]
2.00 & - & - & -
     & ${8.8}^{\,+5.1}_{\,-3}$ & ${115.4}^{\,+8.5}_{\,-6.3}$ & 2.7391 \\

\bottomrule
\end{tabular}
\tablenotetext{1}{Best fit value is formally $6.6\times10^{-6}$, close enough.}

\end{table*}

% ==========
% References
% ==========
\bibliographystyle{apj}  % AASTeX journal macros are supplied in ADS entries
\bibliography{refs-snr}

\end{document}
