% \documentclass[manuscript]{aastex}  % one-column, double-spaced GENERATE BIB
\documentclass[12pt,preprint]{aastex}  % one-column, single-spaced
% \documentclass[iop, apj, numberedappendix, twocolappendix]{emulateapj}
% \documentclass[iop, apj, numberedappendix]{emulateapj}


\shorttitle{Synchrotron Rims in Tycho's SNR}  % <~ 44 char
\shortauthors{XXX et al.}  % Max three
\slugcomment{Draft, \today}  % short title pg comment

%% ==================================================================== %%
%% README for track changes                                             %%
%% To add/remove text or add comments, use the following commands:      %%
%%                                                                      %%
%%       \note[editor]{The note}                                        %%
%%     \annote[editor]{Text to annotate}{The note}                      %%
%%        \add[editor]{Text to add}                                     %%
%%     \remove[editor]{Text to remove}                                  %%
%%     \change[editor]{Text to remove}{Text to add}                     %%
%%                                                                      %%
%% ==================================================================== %%

\usepackage[inline]{trackchanges}  % trackchanges.sourceforge.net
\addeditor{Rob}
\addeditor{Sean}
\addeditor{Steve}
\addeditor{Aaron}
\addeditor{Brian}

\usepackage{amsmath}  % amsthm, amssymb
% \usepackage{CJK}  % aas.org/authors/author-names-non-roman-alphabets
\usepackage{booktabs}
%\usepackage[labelfont=bf, labelsep=period]{caption}  % Custom float captions
%\usepackage{pdflscape}  % rotate pages (Texlive)
\usepackage{hyperref}

\newcommand*{\mt}{\mathrm}
\newcommand*{\unit}[1]{\;\mt{#1}}  % http://vemod.net/typesetting-units-in-latex
\newcommand*{\abt}{\mathord{\sim}} % http://tex.stackexchange.com/q/55701
\newcommand*{\ptl}{\partial}
\newcommand*{\del}{\nabla}

% This paper
\newcommand*{\tsup}{\textsuperscript}
\newcommand*{\Chandra}{\textit{Chandra}\ }
\newcommand*\mean[1]{\bar{#1}}
\defcitealias{ressler2014}{R14}

\begin{document}

\title{Energy Dependent X-Ray Rim Widths in Tycho's Supernova Remnant}

%\begin{CJK*}{UTF8}{gbsn}
\author{
Robert Petre\altaffilmark{1},
Sean M. Ressler\altaffilmark{2},
Stephen P. Reynolds\altaffilmark{3},
Aaron Tran\altaffilmark{1,4},
Brian J. Williams\altaffilmark{1,5}
}
%\end{CJK*}

\affil{
\tsup{1}NASA Goddard Space Flight Center, Greenbelt, MD 20771, USA \\
\tsup{2}Dept. Physics, University of California, Berkeley, CA 94720, USA \\
\tsup{3}Dept. Physics, North Carolina State University, Raleigh, NC 27695, USA
}

%\altaffiltext{1}{NASA Goddard Space Flight Center, Greenbelt, MD 20771, USA}
%\altaffiltext{2}{Dept. Physics, University of California,
%    Berkeley, CA 94720, USA}
%\altaffiltext{3}{Dept. Physics, North Carolina State University,
%    Raleigh, NC 27695, USA}
\altaffiltext{4}{CRESST, University of Maryland, College Park, MD 20742}
\altaffiltext{5}{NASA Postdoctoral Program Fellow}

\begin{abstract}
\note[Aaron]{Copy pasted from NASA abstract, not reviewed}
Young supernova remnants may exhibit thin (~1--10\% of shock radius) X-ray rims
of synchrotron radiation from forward shock-accelerated electrons that travel
downstream of the shock and quickly cease to radiate. Rim widths limited by
radiative energy losses should decrease with energy and require magnetic field
amplification $10$--$100\times$ that expected from adiabatic shock compression.
Damped magnetic fields behind rims may produce thin rims without strong field
amplification but require energy-independent rim widths. We measured rim widths
around Tycho's supernova remnant in 5 energy bands using a 750 ks \Chandra
observation. Rims narrow with increasing energy, favoring loss-limited
radiation over magnetic damping. Observed widths are best fit by electron
transport models requiring amplified magnetic fields $\abt0.1$--$1$ mG and
particle diffusion $\abt1$--$10\times$ Bohm values, consistent with prior work.
Inferred magnetic fields, diffusion coefficients, and diffusion-energy scaling
may constrain models for cosmic ray acceleration in supernova remnants and
plasma turbulence in astrophysical shocks.
\end{abstract}

% Six keywords, alphabetical order
\keywords{acceleration of particles ---
    ISM: individual objects (Tycho's SNR) ---
    ISM: magnetic fields ---
    ISM: supernova remnants ---
    shock waves ---
    X-rays: ISM}

% ============
% Introduction
% ============
\section{Introduction}

% -------------------
% Background / review
% -------------------
\subsection{What are thin rims?}

Define thin rims (hereafter, thin rims).
\note[Aaron]{most text below is copy-pasted from my own notes -- explanation
may be too simplistic}

Somewhere here, define upstream/downstream of shock (makes sense in reference
frame of shock).

Accelerating ISM/CSM particles in the forward shocks of supernovae emit
synchrotron radiation.  Emission is strong in the shock's immediate wake, but
dies off quickly downstream as particles radiate and lose energy. The radiation
from a spherical blast wave, then, is shell-like with bright X-ray and radio
rims/filaments due to line-of-sight projection.  These filaments have finite
and energy-dependent widths, set by:
(1) \emph{synchrotron losses}.  At the shock, high energy electrons efficiently
radiate harder x-ray photons.  As electrons are advected downstream while
radiating, they radiate at lower energies and lose energy more slowly.
Synchrotron losses depend on the initial electron energy distribution and the
gradual decrease of electron energies downstream of the shock.
(2) \emph{diffusion}.  Random motion with respect to bulk plasma advection.
Higher energy electrons diffuse and travel further downstream than would be
expected from pure advection; hence, higher energy radiation may be seen
farther downstream of the shock than expected. Higher energy particles may also
diffuse \emph{ahead} of the shock, possibly giving rise to a so-called
``cosmic-ray shock precursor''.  Note that this is distinct from radiative
precursors \citep[e.g.,][]{ghavamian2000}.

Mention relevance to: cosmic ray acceleration, turbulent field amplification,
(note assumptions in re proton/electron spectra).  Cut-off energy of injected
electrons?  Broadly, all astrophysical shock processes: AGN jets, galaxy
clusters \citep{van-weeren2010}, PWNe, et cetera.  Usual background,
utility spiel.

% ------------------------
% Measurements and meaning
% ------------------------
\subsection{Prior measurements and inferred physical parameters}

Models for particle transport.  Bohm, sub-Bohm diffusion.
Testable model predictions of magnetic damping \citep{pohl2005}.

Previous measurements of basic observables \citep[e.g.,][]{bamba2003,
bamba2005-hist, bamba2005-vela, parizot2006} and consequences.
Conclusion that magnetic fields are amplified.

Equivocal results by \citet{araya2010} for Cas A.
\citet{ressler2014} found rim narrowing in the remnant of SN 1006.

% -----------
% Tycho's SNR
% -----------
\subsection{Tycho's SNR}
Why Tycho's supernova remnant (hereafter, Tycho)?

Like SN 1006, Tycho has well defined synchrotron rims associated with expansion
into a low-density ISM \note[Aaron]{justification: high density ISM would
prevent the forward shock from running ahead of the thermal ejecta?}.
\citet{williams2013} favor mean pre-shock ISM density $\abt 0.2 \unit{cm^{-3}}$
from \textit{Spitzer} observations of shocked ISM dust emission,
\note[Aaron]{is this proton or electron density?  Or are they equal pre-shock?
Text mentions $n_e = 1.2 n_p$ \emph{post}-shock.}
consistent with Tycho's X-ray expansion \citep{katsuda2010}.
Compare w/ SN 1006 pre-shock density $n_p \abt 0.15$--$0.3 \unit{cm^{-3}}$
\citep{raymond2007, heng2007, winkler2013}.  \note[Aaron]{ISM gradients?}

Like all historical SNe, Tycho is close enough that we
can spatially resolve these rims with \Chandra and say something useful.

Previous estimates of Tycho fields/diffusion (something similar to Sean's table
for SN 1006?).  CR acceleration.

% -------------
% Paper roadmap
% -------------
\subsection{Our work (paper roadmap)}

We make measurements to distinguish between the two end member scenarios.
Our procedure, in both measurement and rim width modeling, follows that of
\citet{ressler2014} (hereafter, \citetalias{ressler2014}) with only slight
modifications.
We first select regions around Tycho's forward shock for analysis, measure
rim widths, and verify that rim spectra are free of thermal line emission.
Using two models for particle transport, we fit measured widths...

We discuss the implications of our fits and models for magnetic shock
amplification (lending credence to previous estimates, in disfavoring magnetic
field amplification), and discuss particle diffusion/acceleration at the shock.

Any further results (constraints on precursors, etc).

Consequences, implications, why is this worthwhile?

% =============================
% Transport models, observables
% =============================
\section{Transport models}

We briefly review our particle transport models and relevant equations for our
modeling; a fuller review and exposition is given by \citetalias{ressler2014}.
\note[Aaron]{is it necessary to give Sean's equation numbers?  How many of the
derived equations should we give?}

\note[Aaron]{Double check against Rettig/Pohl}
The electron distribution which gives rise to the observed synchrotron
radiation is injected at the forward shock and transported by advection
and plasma diffusion, following the advection-diffusion equation:
\[
  \frac{\ptl f}{\ptl t} + \del \cdot \left( f \vec{v} \right)
  = C + \del \cdot \left( D \del f \right)
\]
we obtain (1-D time-independent flow, neglect compressible flow?!,
space-independent diffusion coefficient (did we neglect the spherical
coordinate terms?)):
\[
    v \frac{\ptl f}{\ptl x} - D \frac{\ptl^2 f}{\ptl x^2} = C
\]
and by selection of the source/sink term $C$ we obtain either a
simplified model:
\begin{equation} \label{eq:simp-mod}
    v \frac{\ptl f}{\ptl x} - D \frac{\ptl^2 f}{\ptl x^2} +
    \frac{f}{\tau_{\mt{synch}}} = 0
\end{equation}
where this is equation (5) of \citetalias{ressler2014}; we also look at
\begin{equation} \label{eq:full-mod}
    v \frac{\ptl f}{\ptl x} - D \frac{\ptl^2 f}{\ptl x^2} =
    K_0 E^{-s} e^{-E/E_{\mt{cut}}} \delta(x) + \frac{\ptl}{\ptl E}
      \left(bB^2E^2f\right)
\end{equation}
which is equation (12) of \citetalias{ressler2014}.

Explain how the transport models constrain diffusion via $m_E$, define $m_E$.
Explain how we back out a filament shape/width from these models

\subsection{Energy dependent widths, diffusion}

\begin{itemize}
    \item Bohm diffusion: $D(E) = \eta C_d E / B$
    \item Generalized diffusion: $D(E) = \eta C_d E^\mu / B = \eta_h D(E_h)
        (E/E_h)^\mu$
\end{itemize}

Explain the use of fiducial energy -- this makes sense and eases discussion,
but linking $E_h$ and $\eta_2$ to $\eta$ and plain Bohm diffusion seems a bit
murky to me.

For the simple model, we have equation (6) of \citetalias{ressler2014}:
\[
    a = \frac{2 D / v_d}{ \sqrt{ 1 + \frac{4D}{v_d^2 \tau_{\mt{synch}}}} - 1}
\]
which we relate to full width half max by projection factor $\beta = 4.6$
so that $w = \beta a$, assuming (1) electron distribution exponential in radial
coordinate $x$, and (2) spherical source \note[Aaron]{shell?}
\citep{ballet2006}.

For the full model (equation~\eqref{eq:full-mod}), we numerically solve for
electron distribution $f(E,x)$ using solutions due to \citet{lerche1980} and
\citet{rettig2012};

\subsection{Tycho parameters}

Give numbers ($v_s$, $v_d$, $r_s$, etc.).  Give a definition of the cut-off
energy following Rettig and Pohl (again cite Sean's paper).
We use an electron spectrum cut-off energy of:
\[
    E_{\mt{cut}} =
        \left(8.3\unit{TeV}\right)^{2/(1+\mu)}
        \left(\frac{B_0}{100 \unit{\mu G}}\right)^{-1/(1+\mu)}
        \left(\frac{v_s}{10^8 \unit{cm/s}}\right)^{2/(1+\mu)}
    \left[
        \frac{E_h^{\mu - 1}}{\eta}
    \right]^{1 / (1 + \mu)}
\]
which is equation~(19) of \citetalias{ressler2014}.
\note[Aaron]{This differs from Sean's presentation -- I drew this from the
Fortran code and it looks consistent w/ Parizot et al., but I haven't vetted
the prefactor $8.3 \unit{TeV}$.  And, I have an extra exponent $\mu - 1$ on the
$E_h$ ?!  Come back to this shortly.}

\subsection{Magnetic damping}

We also consider a magnetically damped field of form
\[
    B(x) = B_{\mt{min}} + \left(B_0 + B_{\mt{min}}\right) \exp\left(-x / a_b\right) ,
\]
following \citetalias[Section 3.2]{ressler2014}.


% ============
% Observations
% ============
\section{Observations}

% TODO go through and double check tenses.
\note[Aaron]{inconsistent present/past tense usage}
We measured synchrotron rim widths from an archival \Chandra
ACIS-I observation of Tycho
(RA: 00\tsup{h}25\tsup{m}19\fs0, dec: +64\arcdeg08\arcmin10\farcs0; J2000)
between 2009 Apr 11 and 2009 May 5 (PI: John P. Hughes;
\dataset[ADS/Sa.CXO\#obs/10093--10097]{ObsIDs: 10093--10097},
\dataset[ADS/Sa.CXO\#obs/10902--10906]{10902--10906}).
The total exposure time was $734 \unit{ks}$.
Level 1 \Chandra data were reprocessed with CIAO 4.6 and CALDB 4.6.1.1 and kept
unbinned with ACIS spatial resolution $0.492\arcsec$.
Merged and corrected events were divided into five energy bands:
0.7--1 keV, 1--1.7 keV, 2--3 keV, 3--4.5 keV, and 4.5--7 keV;
we excluded 1.7--2 keV counts to avoid \ion{Si}{13} (He$\alpha$) line
emission, prevalent in the remnant's thermal ejecta, which could contaminate
our nonthermal profile measurements.

We selected 13 regions in 5 distinct filaments around Tycho's shock
(Figure~\ref{fig:snr}) based on the following criteria:
(1) filaments should be singular and localized; multiple filaments must not
overlap or completely overlap (rules out parts of NE limb);
(2) filaments should have clear FWHMs in all bands, compared to background
signal or downstream thermal emission (rules out some faint southern wisps);
(3) filaments should be clear of spatial plumes of thermal ejecta in \Chandra
images; this rules out, e.g., areas of strong nonthermal and thermal emission
on Tycho's eastern limb.

\begin{figure}
    %\plotone{figures/f0-snr.pdf}
    \centering
    \includegraphics[width=0.35\textwidth]{figures/f0-snr.pdf}
    \caption{RGB image of Tycho with region selections overlaid.  Bands are
    0.7--1 keV (red), 1--2 keV (green) and 2--7 keV (blue).
    \note[Aaron]{Temporary figure; regions need re-numbering.}}
    \label{fig:snr}
\end{figure}

Discuss point-spread function if that may be a point of contention.
Show that it is not a problem for us (simply say that all rims are at
least ~1 arcsec and hence are spatially resolved? Compare,
\citet{bamba2005-hist} measured upstream/downstream scale widths separately?)

% --------------------------
% FWHM measurement procedure
% --------------------------
\subsection{Filament width measurements}

We obtained radial intensity profiles by integrating along the shock in each
region.  Plotted and fitted profiles are reported in vignetting and
exposure-corrected \note[Aaron]{double check CIAO guides} intensity units;
error bars are computed from raw count data assuming Poisson statistics.
Intensity profiles peak sharply at/behind the shock, demarcating our
thin rims, \note[Aaron]{abt 2--3 arcsec; measure this or report $w_u$?}
and then fall off until thermal emission from ejecta picks up at Tycho's
``contact discontinuity'' \citep{warren2005}.

We fitted rim profiles, obtained by integrating intensity along the shock in
each region, to a piecewise two-exponential model:
\begin{equation} \label{eq:prof}
    h(x) =
    \begin{cases}
        A_u \exp \left(\frac{x_0 - x}{w_u}\right) + C_u, &x \geq x_0 \\
        A_d \exp \left(\frac{x - x_0}{w_d}\right) + C_d, &x < x_0
    \end{cases}
\end{equation}
where $h(x)$ is profile height and $x$ is radial distance from remnant center.
The rim model, which we emphasize is strictly empirical, has 6 free parameters
$A_u, x_0, w_u, w_d, C_u, C_d$ with $A_d = A_u + (C_u - C_d)$ enforcing
continuity at $x=x_0$. Our model is similar to that of \citet{bamba2003,
bamba2005-hist} and differs slightly from that of \citetalias{ressler2014}.
To fit only the nonthermal rim in each intensity profile, we smoothed profiles
with a 21-point ($\abt 10\arcsec$) Hanning window and bounded the fit
domain at the first local data minimum downstream of rim peak; the fit domain
extends upstream to the region's outer edge.

From the fitted profiles we extracted a full width at half maximum (FWHM) for
each region and each energy band.
\note[Aaron]{We did NOT subtract out $\min(C_u, C_d)$ before measuring FWHM
(fix this?).  May mention effect of using slightly different models for FWHMs}
We could not resolve a FWHM in regions 2, 6, 8, 9, and 11 \note[Aaron]{KEEP
UPDATED} at 0.7--1 keV (Table~\ref{tab:flmt2} \note[Aaron]{UPDATE depending on
how we present/tabulate FWHMs}); in these regions, the downstream FWHM bound
would extend outside the fit domain.  We were able to resolve FWHMs for all
regions at higher energy bands (1--7 keV).

To estimate FWHM uncertainties, we horizontally stretched each best-fit
profile by mapping radial coordinate $x$ to
$x'(x) = x (1 + \xi (x-x_0)/(50\arcsec-x))$ with $\xi$ an arbitrary stretching
parameter; this yields a new profile $h'(x) = h(x'(x))$.
We varied $\xi$ (and hence rim FWHM) to vary each profile fit $\chi^2$ by 2.7
and the stretched or compressed FWHMs as upper/lower bounds on our reported
FWHMs.

Finally, we took a geometric average of rim widths $w_i(E)$ in each filament,
with $i$ indexing each region, as
\[
    \mean{w}(E) = \left( \prod_{i=1}^{n} w_i(E) \right)^{1/n}
\]
Why?  My (Aaron's) justification:
\begin{itemize}
\item Arithmetic mean width is not representative of a filament's
    spatially varying width (in practice, projected rim emission likely
    does not arise from a spherical shell \citep[c.f.][]{hester1987}).
    Our measurements aren't randomly distributed.
\item I assume / guess that $m_E$ is more likely to be constant over a
    filament length.  Intuitively, I think of $m_E$ being a proxy for the
    relative strength of transport mechanisms, as opposed to being a function
    of the absolute strengths of two mechanisms.  But that's not rigorous,
    and it doesn't seem obvious $m_E$ should be more robust than width $w$
    over a filament's length (since we're taking a first derivative, which
    is usually more painful?).
\item The geometric mean of FWHMs gives an arithmetic mean of ``sampled''
    $m_E$ values by definition.  To show this explicitly (just a sanity check)
    from the equation $w(E) = \alpha E^{m_E}$, where $\alpha$ is a throwaway
    constant, we have $m_E = \ln(w(E)/\alpha) / \ln(E)$.  Then, $m_E$ computed
    from geometric mean width $\mean{w}(E)$ is:
    \begin{align*}
        \mean{m_E}(E)
            &= \frac{\ln\left[ \mean{w}(E) \right] - \ln\alpha}{\ln(E)} \\
            &= \frac{\ln\left[ \left( \prod_{i=1}^{n} w_i(E) \right)^{1/n}
                        \right] - \ln\alpha}{\ln(E)} \\
            &= \frac{\frac{1}{n}\left(
                         \sum_{i=1}^{n} \ln\left[w_i(E)\right]
                     \right) - \ln\alpha}{\ln(E)} \\
            &= \frac{\frac{1}{n} \sum_{i=1}^{n}
                     \left( \ln\left[w_i(E)\right] - \ln\alpha \right)}
                    {\ln(E)} \\
            &= \sum_{i=1}^{n} (m_E)_i (E)
    \end{align*}
    Or, we could just say that we're taking an arithmetic mean of
    $\ln(w_i(E))$.
\item The geometric mean is less sensitive to outliers than the arithmetic
    mean -- but this may not necessarily be desirable, if it hides
    real measurement signal.
\end{itemize}
The counterpoints are that (1) in practice, it may not make much difference
(especially as we only have sample size about $1$--$5$ at best), variations
from changing up calculations are hard to distinguish from plain old
measurement spread; (2) arithmetic mean is simpler to understand; (3)
assumption of roughly constant widths does not look terrible in Tycho -- but I
have not investigated this... (clearly, rims fade at edges, but at bright
center regions do they swell or vary in width?).

Errors were computed by taking standard errors on $\log w(E)$
(i.e., $\sigma(w(E)) / \sqrt(n)$, as the CLT applies in log space) and then
converting back to physical parameter space.  These errors are asymmetric; we
take the maximum error in either direction as error on our FWHMs for all fits
(but we report the asymmetric error otherwise).
\textbf{WARNING:} this procedure is incomplete.  Because our sample sizes are
tiny (typically, $n \sim 2$), to obtain a true 68.3\% confidence interval /
1-sigma error, we should compute error from the Student's $t$-distribution.
The net result is that our errors should be doubled in most cases.
\emph{I have not yet applied this correction.}

Some problems:
\begin{itemize}
    \item Errors are very large, as currently calculated -- they reflect the
        assumption that $m_E$ is normally distributed, and give a statistical
        uncertainty on population mean $m_E$.
        (or if we used the arithmetic mean, same problem for FWHMs)
        This throws out our very small width errors, and we want to take
        advantage of our nice photon statistics.
    \item Since geometric mean is strictly lower than arithmetic mean,
        this will bias our parameter values downwards -- it's not clear which
        approach would give more ``correct'' estimates
\end{itemize}

One approach may be to slice up the regions more thinly and accept a bit more
uncertainty in FWHM values.  This would make it eaiser to either (1) compute
fits to regions individually (then, errors are set by profile fitting
uncertainty alone -- I think our estimates are still a little too small, and we
should be using 1-sigma errors rather than 90\% CIs), and (2) average many
FWHMs, in whatever way, and get more robust / lower error filament values.


% ----------------
% Filament spectra
% ----------------
\subsection{Filament spectra}

We extracted and fitted spectra for all regions to confirm that our rim width
measurements are not contaminated by thermal line emission.
We divided each region into an upstream and downstream band.  The upstream band
is the smallest sub-region containing FWHM bounds from all energy bands.
The downstream band extends from the back of the rim (where the upstream band
ends) to the rim model fit's (equation~\eqref{eq:prof}) downstream limit.
Figure~\ref{fig:spec} plots an example profile (Region 1, 4.5--7 keV) with the
downstream/upstream selections highlighted in light blue/gray respectively.
We fitted spectra from each region to an absorbed power law between $0.5$ and $7$
keV in XSPEC.

Background spectra were extracted from circular regions around the remnant's
exterior; we matched each selected rim region to the nearest background region
and subtracted the background spectrum accordingly. \note[Aaron]{rewrite for
concision/clarity}.

\note[Aaron]{TODO?!} We also fit the spectra to absorbed power law models with
1 and 2 additional Gaussian components for \ion{Si}{13} and
\ion{S}{15} He$\alpha$ lines \note[Aaron]{?? I think I got this from AtomDB but
not certain of details} at $\abt 1.8$, $2.4$ keV.
We compute some equivalent width, or compare the goodness-of-fit with and
without the additional lines, to rule out the presence of a line
(but cf. \citet{protassov2002} on line detection statistics!).

% -------------
% Model fitting
% -------------
\subsection{Filament model fitting}

We fitted the two transport models given by Equations~\eqref{eq:simp-mod} and
\eqref{eq:full-mod} to the measured data.  Here we present only the full
model (equation~\eqref{eq:full-mod}) fits.  \note[Aaron]{Could we present the
full model alone, given that the numbers / qualitative trends for both models
are fairly similar?  In \citetalias{ressler2014} Sean used the simple model to
build intuition and gain physical insight, to great effect.  But, will we need
that here?}

Explain why we are fixing $\mu$.
Explain why we also give results with $\eta_2$ fixed at $1$.

To perform fits, we tabulated a large grid for fixed values of $\mu$ and shock
velocity $v_s$.  If any fit runs to $\eta_2 = 10^5$ or $B_0 = 10 \unit{mG}$,
we deem it effectively unconstrained (only a few fits do so, and only to the
$\eta_2$ limit).
Note that we do not allow / consider fits with $\eta_2$ negative (may be
trivial, just note parameter bounds).

A few remnant-specific parameters enter into the model calculations.  We take
electron spectral index $s = 2.3$ (from radio spectral index $\alpha = 0.65$,
\citet{kothes2006} \note[Aaron]{not using $0.58$ from Green's catalog?!}),
remnant distance $3 \unit{kpc}$ \citep[cf.][]{hayato2010}, and
shock radius $1.08 \times 10^{19} \unit{cm}$ from angular radius $240\arcsec$
\citep{green2009}.  Tycho's forward shock velocity varies with azimuth by up to
a factor of 2; we interpolate the data of \citet{williams2013} to estimate
individual shock velocities for each of our regions.

As the full continuous loss model must be numerically solved, its predicted
rim widths are subject to resolution error in the numerical integrals.  We
chose our integration resolutions such that the fractional error associated
with halving/doubling our resolution is less than $1\%$ for the
parameter space relevant to our filaments. \note[Aaron]{rewrite -- unclear.
Need to verify $1\%$ claim + check Pacholczyk table resolution...}

% =======================
% Results, FWHMs and fits
% =======================
\section{Results}

Explain a bit about how we divvy up / present the results.  What values are we
averaging, where do our errors come from?
For each region, give the relevant shock velocity value (and a range of
values?).
We do X to the FWHMs (geometric/arithmetic average) within each filament.

% --------------------
% FWHM results, tables
% --------------------
\subsection{Rim fits, spectra, and width measurements}

To perform fits, we took a geometric average of rim widths in each energy band,
for each filament; 
For FWHM errors, I use the standard error of the mean (with some
minor nuances; I am using sample standard deviation).
PUT IN METHODS SECTION


This contrasts with taking an arithmetic average of the rim widths.  The
geometric average does not represent the expected value of FWHM measurements,
along the rim.  However, filaments should vary in width along their lengths in
response to 


A key question is whether we wish to report an average w/ errors, corresponding
to our specific measurements (if we alter our profile fit function, or take a
new observation of Tycho with different counts, how will our measurement
change) -- or, if we wish to report an average w/ error that estimates the
population spread -- if we have huge resolution / collecting power, and could
slice up the filament in many locations and sample its FWHMs very well, what
would be the population spread on that sample?


If the FWHMs vary systematically within the filament -- an estimate of
population spread will give errors that are too large.

We might expect the FWHM energy dependence (our $m_E$) to be consistent across
the filament, as opposed to the actual FWHM lengths.
If we expect the FWHM energy dependence


But, both values are important -- the FWHM lengths tell us about the magnetic
field strength


In Table~\ref{tab:flmt2} we report averaged widths (data and fits for
individual filaments are given in \emph{appendix, supplement?}).


\begin{figure}
    \plotone{figures/f0-prfs.pdf}
    \caption{Profile fits for Region 1}
    \label{fig:profiles}
\end{figure}


\begin{table*}
\scriptsize
\centering
\caption{Filament 2 full width at half max measurements.
Exponent $m_E$ estimated point to point.\label{tab:flmt2}}
\begin{tabular}{@{}l ccccc r@{ $\pm$ }l r@{ $\pm$ }l r@{ $\pm$ }l r@{ $\pm$ }l @{}}

\toprule
{} & \multicolumn{5}{c}{FWHM (arcsec)} & \multicolumn{8}{c}{$\mE$ (-)} \\
\cmidrule(lr){2-6} \cmidrule(l){7-14}
Region & Band 1 & Band 2 & Band 3 & Band 4 & Band 5
       & \multicolumn{2}{c}{Bands 1--2} & \multicolumn{2}{c}{Bands 2--3}
       & \multicolumn{2}{c}{Bands 3--4} & \multicolumn{2}{r}{Bands 4--5} \\ [0.2em]
{} & (0.7--1 keV) & (1--1.7 keV) & (2--3 keV) & (3--4.5 keV) & (4.5--7 keV)
   & \multicolumn{2}{c}{(1 keV)} & \multicolumn{2}{c}{(2 keV)}
   & \multicolumn{2}{c}{(3 keV)} & \multicolumn{2}{r}{(4.5 keV)} \\
\midrule
1 & {} & ${8.80}^{+0.18}_{-0.15}$ & ${6.34}^{+0.26}_{-0.21}$ & ${7.40}^{+0.30}_{-0.23}$ & ${5.57}^{+0.47}_{-0.42}$
  & \multicolumn{2}{c}{} & $-0.47$ & $0.06$ & $0.38$ & $0.13$ & $-0.70$ & $0.22$ \\ [0.5em]
2 & {} & ${4.22}^{+0.12}_{-0.09}$ & ${2.36}^{+0.12}_{-0.09}$ & ${3.00}^{+0.16}_{-0.12}$ & ${4.11}^{+0.34}_{-0.30}$
  & \multicolumn{2}{c}{} & $-0.84$ & $0.08$ & $0.59$ & $0.16$ & $0.77$ & $0.23$ \\ [0.5em]
3 & {} & ${2.47}^{+0.08}_{-0.07}$ & ${1.78}^{+0.09}_{-0.07}$ & ${2.10}^{+0.11}_{-0.11}$ & ${1.32}^{+0.10}_{-0.09}$
  & \multicolumn{2}{c}{} & $-0.47$ & $0.08$ & $0.41$ & $0.17$ & $-1.15$ & $0.22$ \\

\cmidrule{1-14}
4 & ${5.85}^{+0.37}_{-0.33}$ & ${4.35}^{+0.09}_{-0.08}$ & ${3.26}^{+0.11}_{-0.09}$ & ${3.69}^{+0.12}_{-0.11}$ & ${3.20}^{+0.21}_{-0.18}$
  & $-0.83$ & $0.18$ & $-0.41$ & $0.05$ & $0.31$ & $0.11$ & $-0.35$ & $0.17$ \\ [0.5em]
5 & {} & ${4.52}^{+0.11}_{-0.12}$ & ${3.06}^{+0.11}_{-0.11}$ & ${3.25}^{+0.15}_{-0.13}$ & ${3.04}^{+0.21}_{-0.18}$
  & \multicolumn{2}{c}{} & $-0.56$ & $0.06$ & $0.15$ & $0.14$ & $-0.17$ & $0.19$ \\ [0.5em]
6 & ${2.48}^{+0.18}_{-0.18}$ & ${2.32}^{+0.05}_{-0.06}$ & ${2.98}^{+0.11}_{-0.09}$ & ${2.05}^{+0.08}_{-0.09}$ & ${2.21}^{+0.15}_{-0.14}$
  & $-0.19$ & $0.21$ & $0.36$ & $0.06$ & $-0.92$ & $0.13$ & $0.18$ & $0.19$ \\ [0.5em]
7 & ${2.69}^{+0.20}_{-0.17}$ & ${2.33}^{+0.05}_{-0.05}$ & ${2.31}^{+0.08}_{-0.08}$ & ${1.81}^{+0.09}_{-0.07}$ & ${1.83}^{+0.11}_{-0.08}$
  & $-0.39$ & $0.20$ & $-0.01$ & $0.06$ & $-0.60$ & $0.14$ & $0.02$ & $0.17$ \\ [0.5em]
8 & ${2.33}^{+0.21}_{-0.20}$ & ${2.72}^{+0.08}_{-0.08}$ & ${2.38}^{+0.10}_{-0.09}$ & ${2.10}^{+0.10}_{-0.09}$ & ${2.37}^{+0.20}_{-0.17}$
  & $0.43$ & $0.26$ & $-0.19$ & $0.07$ & $-0.30$ & $0.15$ & $0.29$ & $0.22$ \\ [0.5em]
9 & ${2.16}^{+0.24}_{-0.23}$ & ${2.35}^{+0.07}_{-0.06}$ & ${2.47}^{+0.11}_{-0.11}$ & ${1.91}^{+0.09}_{-0.09}$ & ${2.20}^{+0.17}_{-0.16}$
  & $0.24$ & $0.31$ & $0.07$ & $0.07$ & $-0.63$ & $0.16$ & $0.34$ & $0.22$ \\ [0.5em]
10 & ${2.38}^{+0.24}_{-0.23}$ & ${1.99}^{+0.07}_{-0.06}$ & ${1.76}^{+0.09}_{-0.08}$ & ${1.59}^{+0.09}_{-0.08}$ & ${1.58}^{+0.13}_{-0.12}$
  & $-0.50$ & $0.29$ & $-0.18$ & $0.08$ & $-0.24$ & $0.18$ & $-0.02$ & $0.23$ \\

\cmidrule{1-14}
11 & {} & ${3.23}^{+0.15}_{-0.13}$ & ${2.52}^{+0.16}_{-0.13}$ & ${1.90}^{+0.14}_{-0.13}$ & ${3.09}^{+0.45}_{-0.38}$
  & \multicolumn{2}{c}{} & $-0.36$ & $0.10$ & $-0.70$ & $0.22$ & $1.21$ & $0.37$ \\ [0.5em]
12 & {} & ${3.86}^{+0.17}_{-0.16}$ & ${2.61}^{+0.15}_{-0.13}$ & ${3.02}^{+0.22}_{-0.21}$ & ${2.23}^{+0.21}_{-0.17}$
  & \multicolumn{2}{c}{} & $-0.56$ & $0.10$ & $0.36$ & $0.22$ & $-0.74$ & $0.27$ \\ [0.5em]
13 & ${2.85}^{+0.22}_{-0.17}$ & ${2.43}^{+0.05}_{-0.05}$ & ${2.36}^{+0.08}_{-0.05}$ & ${1.95}^{+0.09}_{-0.10}$ & ${1.84}^{+0.11}_{-0.14}$
  & $-0.45$ & $0.20$ & $-0.04$ & $0.05$ & $-0.47$ & $0.13$ & $-0.15$ & $0.20$ \\

\cmidrule{1-14}
14 & ${2.86}^{+0.17}_{-0.16}$ & ${2.42}^{+0.06}_{-0.04}$ & ${2.23}^{+0.08}_{-0.07}$ & ${2.38}^{+0.10}_{-0.08}$ & ${2.19}^{+0.12}_{-0.10}$
  & $-0.47$ & $0.17$ & $-0.12$ & $0.06$ & $0.17$ & $0.12$ & $-0.20$ & $0.15$ \\ [0.5em]
15 & ${2.71}^{+0.17}_{-0.16}$ & ${1.99}^{+0.05}_{-0.04}$ & ${1.80}^{+0.06}_{-0.05}$ & ${1.87}^{+0.07}_{-0.05}$ & ${1.52}^{+0.09}_{-0.08}$
  & $-0.85$ & $0.18$ & $-0.15$ & $0.05$ & $0.09$ & $0.11$ & $-0.51$ & $0.16$ \\ [0.5em]
16 & ${1.87}^{+0.14}_{-0.13}$ & ${1.73}^{+0.04}_{-0.03}$ & ${1.52}^{+0.06}_{-0.05}$ & ${1.25}^{+0.06}_{-0.04}$ & ${1.23}^{+0.08}_{-0.06}$
  & $-0.22$ & $0.21$ & $-0.18$ & $0.06$ & $-0.49$ & $0.13$ & $-0.04$ & $0.17$ \\ [0.5em]
17 & ${1.65}^{+0.13}_{-0.12}$ & ${1.92}^{+0.05}_{-0.05}$ & ${1.54}^{+0.06}_{-0.07}$ & ${1.45}^{+0.07}_{-0.06}$ & ${2.05}^{+0.16}_{-0.14}$
  & $0.43$ & $0.22$ & $-0.31$ & $0.07$ & $-0.16$ & $0.15$ & $0.86$ & $0.21$ \\

\cmidrule{1-14}
18 & {} & ${4.45}^{+0.13}_{-0.12}$ & ${3.18}^{+0.17}_{-0.16}$ & ${2.96}^{+0.20}_{-0.19}$ & ${1.65}^{+0.21}_{-0.16}$
  & \multicolumn{2}{c}{} & $-0.49$ & $0.09$ & $-0.17$ & $0.21$ & $-1.45$ & $0.32$ \\ [0.5em]
19 & {} & ${2.30}^{+0.08}_{-0.06}$ & ${2.28}^{+0.11}_{-0.08}$ & ${2.16}^{+0.12}_{-0.11}$ & ${1.60}^{+0.17}_{-0.14}$
  & \multicolumn{2}{c}{} & $-0.02$ & $0.08$ & $-0.13$ & $0.17$ & $-0.74$ & $0.27$ \\ [0.5em]
20 & ${4.81}^{+0.31}_{-0.31}$ & ${1.84}^{+0.06}_{-0.03}$ & ${1.87}^{+0.08}_{-0.06}$ & ${1.56}^{+0.07}_{-0.06}$ & ${2.14}^{+0.23}_{-0.23}$
  & $-2.68$ & $0.19$ & $0.02$ & $0.07$ & $-0.44$ & $0.14$ & $0.77$ & $0.28$ \\

\midrule
Mean & $2.89 \pm 0.35$ & $3.11 \pm 0.37$ & $2.53 \pm 0.23$ & $2.47 \pm 0.30$ & $2.35 \pm 0.23$
  & $-0.46$ & $0.24$ & $-0.25$ & $0.06$ & $-0.14$ & $0.10$ & $-0.09$ & $0.15$ \\

\bottomrule
\end{tabular}
\tablecomments{Mean values computed for all regions; mean $\mE$ values are
averages for region $\mE$ values (i.e., not computed from mean FWHMs).  Errors
on mean values are standard errors of the mean.  Horizontal rules group
individual regions into filaments.}

\tablecomments{Mean FWHMs computed by arithmetic mean, errors from blah}
\end{table*}

Table~\ref{tab:flmt2} lists some numbers from Filament 2.  As previously noted
we could not resolve FWHMs for some regions in the 0.7--1 keV band.

Error estimates / magnitudes.

Mention the number of counts.  This might impact our decision to go back and
slice regions smaller (if we do that).

\begin{figure}
    \plotone{figures/f0-spec.pdf}
    \caption{Spectra from Region 1 (I think?)}
    \label{fig:spec}
\end{figure}

Here, mention any upper bounds on spectral lines (equivalent width or w/e).
Since this is not a ``result'', being ancillary to our work.

Table of best fit chi-squares, power law indices for spectral fits...

% -------------------------
% Model fit results, tables
% -------------------------
\subsection{Model fit results}

\subsubsection{Individual regions}

First, Table~\ref{tab:fits-flmt1} presents full model fit results for the regions
contained in Filament 1.  As currently presented (preliminary results), errors
are computed by varying free parameters $B_0$, $\eta_2$ to obtain a
$\Delta\chi^2 = 1$, corresponding to $1$-$\sigma$ confidence limits on our fit
parameters.

\begin{table*}[ht]
\scriptsize
\centering
\caption{Full model best fits for individual regions, Filament 1.
\label{tab:fits-flmt1}}
%\renewcommand{\arraystretch}{1.5}
\begin{tabular}{@{}rllr llr llr@{}}

\toprule
{} & \multicolumn{3}{c}{Region 1}
   & \multicolumn{3}{c}{Region 10}
   & \multicolumn{3}{c}{Region 11\tablenotemark{a}} \\
\cmidrule(lr){2-4} \cmidrule(lr){5-7} \cmidrule(lr){8-10}
$\mu$ (-) & $\eta_2$ (-) & $B_0$ ($\mu$G) & $\chi^2$
          & $\eta_2$ (-) & $B_0$ ($\mu$G) & $\chi^2$
          & $\eta_2$ (-) & $B_0$ ($\mu$G) & $\chi^2$ \\

\midrule
0.00 & ${19.1}^{\,+19}_{\,-7.1}$ & ${784}^{\,+120}_{\,-72}$ & 57.5590
     & ${19.5}^{\,+16}_{\,-6.2}$ & ${515}^{\,+71}_{\,-40}$ & 111.9722
     & $23^{\,+170}_{\,-170}$ & $670^{\,+1000}_{\,-1000}$ & 62.2861\\[1.5pt]
0.33 & ${56}^{\,+75}_{\,-29}$ & $930^{\,+190}_{\,-140}$ & 34.6006
     & ${58}^{\,+62}_{\,-29}$ & ${614}^{\,+110}_{\,-86}$ & 81.5166
     & $72^{\,+620}_{\,-620}$ & $808^{\,+1500}_{\,-1500}$ & 43.5822\\[1.5pt]
0.50 & ${92}^{\,+160}_{\,-54}$ & ${1010}^{\,+260}_{\,-180}$ & 25.5737
     & ${94}^{\,+140}_{\,-54}$ & $660^{\,+150}_{\,-120}$ & 70.4934
     & $130^{\,+1040}_{\,-1040}$ & $900^{\,+1600}_{\,-1600}$ & 37.8298\\[1.5pt]
1.00 & ${370}^{\,+3300}_{\,-310}$ & ${1260}^{\,+4500}_{\,-440}$ & 9.7750
     & ${68}^{\,+8000}_{\,-47}$ & $560^{\,+1100}_{\,-120}$ & 56.4783
     & $16000^{\,+21000}_{\,-21000}$ & $2450^{\,+670}_{\,-670}$ & 32.7315\\[1.5pt]
1.50 & ${23}^{\,+32}_{\,-10}$ & ${623}^{\,+130}_{\,-67}$ & 8.7937
     & ${10.8}^{\,+5.3}_{\,-3.1}$ & ${354}^{\,+27}_{\,-19}$ & 55.6460
     & ${6.9}^{\,+2.8}_{\,-2.8}$ & ${420}^{\,+28}_{\,-28}$ & 41.9271\\[1.5pt]
2.00 & ${11.4}^{\,+5.2}_{\,-3.3}$ & ${512}^{\,+34}_{\,-26}$ & 9.5862
     & ${7.6}^{\,+2.2}_{\,-1.6}$ & ${315}^{\,+12}_{\,-10}$ & 56.1258
     & ${4.9}^{\,+1.4}_{\,-1.4}$ & ${380}^{\,+15}_{\,-15}$ & 47.6997\\

\midrule
{} & \multicolumn{3}{c}{Region 12\tablenotemark{a}}
   & \multicolumn{3}{c}{Region 13\tablenotemark{a}} \\
\cmidrule(lr){2-4} \cmidrule(lr){5-7}
$\mu$ (-) & $\eta_2$ (-) & $B_0$ ($\mu$G) & $\chi^2$
          & $\eta_2$ (-) & $B_0$ ($\mu$G) & $\chi^2$ \\

\cmidrule(lr){1-7}
0.00 & $22^{\,+199}_{\,-199}$ & $762^{\,+1600}_{\,-1600}$ & 150.4259
     & $22^{\,+320}_{\,-320}$ & $780^{\,+2400}_{\,-2400}$ & 76.9297\\[1.5pt]
0.33 & $59^{\,+967}_{\,-967}$ & $891^{\,+3200}_{\,-3200}$ & 116.8732
     & $59^{\,+1400}_{\,-1400}$ & $900^{\,+4900}_{\,-4900}$ & 51.3875\\[1.5pt]
0.50 & $96^{\,+1800}_{\,-1800}$ & $963^{\,+4100}_{\,-4100}$ & 104.0160
     & $92^{\,+2600}_{\,-2600}$ & $965^{\,+6200}_{\,-6200}$ & 40.3597\\[1.5pt]
1.00 & $498^{\,+6600}_{\,-6600}$ & $1280^{\,+4000}_{\,-4000}$ & 83.815
     & $346^{\,+11700}_{\,-11700}$ & $1190^{\,+9300}_{\,-9300}$ & 15.7162\\[1.5pt]
1.50 & $16.9^{\,+8.4}_{\,-8.4}$ & $554^{\,+53}_{\,-53}$ & 84.7895
     & $716^{\,+10700}_{\,-10700}$ & $1280^{\,+4500}_{\,-4500}$ & 5.6257\\[1.5pt]
2.00 & $9.7^{\,+2.7}_{\,-2.7}$ & $472^{\,+21}_{\,-21}$ & 87.3368
     & $31^{\,+19}_{\,-19}$ & $581^{\,+68}_{\,-68}$ & 5.0051\\

\bottomrule
\end{tabular} 
\tablenotetext{1}{Reported errors are fit standard errors, not reliable!}

\end{table*}

\subsubsection{Combining regions, in various ways}

Now, we present fit results for averaged filament FWHMs, for sanity's sake.
Table~\ref{tab:fits-avg} presents full model fits to arithmetic average of
FWHMs in each filament.  The model fit predictions for each filament are
plotted in Figure~\ref{fig:fits-avg}.

\begin{table*}
\scriptsize
\centering
\caption{Best fits for each filament's averaged FWHMs.
\label{tab:fits-avg}}
% Best-fit parameters for arithmetic average of FWHMs, w/ std err of mean
\begin{tabular}{@{}rllr llr llr@{}}

\toprule
{} & \multicolumn{3}{c}{Filament 1}
   & \multicolumn{3}{c}{Filament 2}
   & \multicolumn{3}{c}{Filament 3} \\
\cmidrule(lr){2-4} \cmidrule(lr){5-7} \cmidrule(lr){8-10}
$\mu$ (-) & $\eta_2$ (-) & $B_0$ ($\mu$G) & $\chi^2$
          & $\eta_2$ (-) & $B_0$ ($\mu$G) & $\chi^2$
          & $\eta_2$ (-) & $B_0$ ($\mu$G) & $\chi^2$ \\

\midrule
0.00 & ${20}^{+68}_{-17}$ & ${630}^{+220}_{-200}$ & 3.8182
     & ${14}^{+89}_{-13}$ & ${625}^{+310}_{-240}$ & 2.4885
     & ${16.4}^{+68}_{-14.4}$ & ${811}^{+330.0}_{-275}$ & 4.2143\\
0.33 & ${55}^{+480}_{-51}$ & ${750.0}^{+450}_{-310}$ & 1.9923
     & ${30.6}^{+680}_{-29.3}$ & ${710}^{+650.0}_{-330.0}$ & 1.6504
     & ${42.6}^{+400}_{-39}$ & ${940}^{+610}_{-380}$ & 2.5354\\
0.50 & ${98}^{+1600}_{-94.0}$ & ${833}^{+690}_{-410}$ & 1.4907
     & ${34.8}^{+3100}_{-33.5}$ & ${720}^{+1100}_{-340}$ & 1.4014
     & ${74}^{+1000}_{-69}$ & ${1030}^{+830}_{-460}$ & 1.8403\\
1.00 & ${12}^{+100000}_{-9}$ & ${490}^{+3100}_{-120}$ & 1.3465
     & ${5.7}^{+100000}_{-4.7}$ & ${460}^{+3400}_{-100}$ & 1.0464
     & ${266}^{+98000}_{-261}$ & ${1260}^{+3200}_{-710}$ & 0.4332\\
1.50 & ${5.1}^{+17.5}_{-3.3}$ & ${400}^{+130}_{-55}$ & 1.5511
     & ${3.3}^{+23.5}_{-2.4}$ & ${397}^{+191}_{-56}$ & 0.7845
     & ${48}^{+100000}_{-44}$ & ${780}^{+4000}_{-280}$ & 0.0712\\
2.00 & ${3.9}^{+6.0}_{-2.3}$ & ${368}^{+62}_{-38}$ & 1.8270
     & ${2.7}^{+7.0}_{-1.8}$ & ${371}^{+80}_{-39}$ & 0.5856
     & ${15}^{+100000}_{-11}$ & ${570}^{+3800}_{-110}$ & 0.0214\\

\midrule
{} & \multicolumn{3}{c}{Filament 4}
   & \multicolumn{3}{c}{Filament 5} \\
\cmidrule(lr){2-4} \cmidrule(lr){5-7}
$\mu$ (-) & $\eta_2$ (-) & $B_0$ ($\mu$G) & $\chi^2$
          & $\eta_2$ (-) & $B_0$ ($\mu$G) & $\chi^2$ \\

\cmidrule(lr){1-7}
0.00 & ${1930}^{+2010}_{-1930}$ & ${953}^{+44}_{-690}$ & 0.2289
     & ${18}^{+4900}_{-18}$ & ${400}^{+380}_{-220}$ & 0.2070\\
0.33 & ${0.002}^{+1.3}_{-0.002}$ & ${274}^{+93}_{-12}$ & 0.2266
     & ${48}^{+69000}_{-48}$ & ${480}^{+1100}_{-300}$ & 0.1365\\
0.50 & ${0.002}^{+1.1}_{-0.002}$ & ${274}^{+77}_{-11}$ & 0.2263
     & ${72}^{+100000}_{-72}$ & ${510}^{+1500}_{-340}$ & 0.1208\\
1.00 & ${0.001}^{+0.77}_{-0.001}$ & ${273}^{+54}_{-14}$ & 0.2254
     & ${6}^{+100000}_{-6}$ & ${290}^{+2500}_{-110}$ & 0.1145\\
1.50 & ${0.01}^{+0.66}_{-0.01}$ & ${267}^{+48}_{-11}$ & 0.2244
     & ${4}^{+100000}_{-4}$ & ${251}^{+2600}_{-74}$ & 0.1132\\
2.00 & ${0.014}^{+0.62}_{-0.014}$ & ${265}^{+44}_{-11}$ & 0.2274
     & ${3}^{+100000}_{-3}$ & ${236}^{+2500}_{-61}$ & 0.1147\\

\bottomrule
\end{tabular} 
\tablecomments{Fitted FWHMs are arithmetic average of individual regions with
standard errors, hence the small values of $\chi^2$.  See table \emph{PUT LABEL
HERE}.  Fit errors correspond to $1$--$\sigma$ and are obtained by varying the
free fit parameters ($B_0$, $\eta_2$) to obtain $\Delta\chi^2 = 1$.}

\end{table*}

\begin{figure}
\centering
\plottwo{figures/f0-fits-avg-flmt1.png}{figures/f0-fits-avg-flmt2.png} \\
\plottwo{figures/f0-fits-avg-flmt3.png}{figures/f0-fits-avg-flmt4.png} \\
\includegraphics[width=0.25\textwidth]{figures/f0-fits-avg-flmt5.png}
\caption{Plots of best-fits to averaged FWHMs.\note[Aaron]{Update plots
for better subplot display in final version, if we want to keep these}}
\label{fig:fits-avg}
\end{figure}

Finally, Table~\ref{tab:fits-par-avg} shows averaged best-fit parameters.
Parameter errors are standard errors from averaging and may reflect the
presence of outliers (especially as the sample number is $1$--$5$).

\begin{table*}
\scriptsize
\centering
\caption{Filament-wide average of best-fit parameters for constituent regions.
\label{tab:fits-par-avg}}
% Average of best-fit parameters for individual regions, for each filament
\begin{tabular}{@{}rllr llr llr@{}}

\toprule
{} & \multicolumn{3}{c}{Filament 1}
   & \multicolumn{3}{c}{Filament 2}
   & \multicolumn{3}{c}{Filament 3} \\
\cmidrule(lr){2-4} \cmidrule(lr){5-7} \cmidrule(lr){8-10}
$\mu$ (-) & $\eta_2$ (-) & $B_0$ ($\mu$G) & $\chi^2$
          & $\eta_2$ (-) & $B_0$ ($\mu$G) & $\chi^2$
          & $\eta_2$ (-) & $B_0$ ($\mu$G) & $\chi^2$ \\

\midrule
0.00 & $21.1 \pm 0.7$ & $702 \pm 51$ & -
     & - & - & -
     & - & - & - \\
0.33 & $60.8 \pm 2.9$ & $829 \pm 57$ & -
     & - & - & -
     & - & - & - \\
0.50 & $100.8 \pm 7.3$ & $900. \pm 62$ & -
     & - & - & -
     & - & - & - \\
1.00 & $\left(3460 \pm 3140\right)$\tablenotemark{a} & $1348 \pm 306$ & -
     & - & - & -
     & - & - & - \\
1.50 & $\left(155 \pm 140\right)$\tablenotemark{a} & $646 \pm 165$ & -
     & - & - & -
     & - & - & - \\
2.00 & $12.9 \pm 4.6$ & $452 \pm 47$ & -
     & - & - & -
     & - & - & - \\

\midrule
{} & \multicolumn{3}{c}{Filament 4}
   & \multicolumn{3}{c}{Filament 5} \\
\cmidrule(lr){2-4} \cmidrule(lr){5-7}
$\mu$ (-) & $\eta_2$ (-) & $B_0$ ($\mu$G) & $\chi^2$
          & $\eta_2$ (-) & $B_0$ ($\mu$G) & $\chi^2$ \\

\cmidrule(lr){1-7}
0.00 & - & - & -
     & - & - & - \\
0.33 & - & - & -
     & - & - & - \\
0.50 & - & - & -
     & - & - & - \\
1.00 & - & - & -
     & - & - & - \\
1.50 & - & - & -
     & - & - & - \\
2.00 & - & - & -
     & - & - & - \\

\bottomrule
\end{tabular} 
\tablenotetext{1}{Here I included extreme outliers ($\eta_2 = 16000$ for Region
11, $\mu = 1$, other regions have $\eta_2 < 500$; $\eta_2 = 716$ for Region 13,
$\mu = 1.5$, other regions have $\eta_2 < 25$.  With 5 different regions there
can be some extreme scatter (for 2 regions we can barely even tell).  But,
qualitatively, best-fit parameters for Filament 1 regions appear to have
similar values (see stated errors, which are std error of mean
computed using sample std dev.).}

\end{table*}

We also computed best fits while holding $\eta_2 = 1$; i.e., requiring Bohm
diffusion at $2 \unit{keV}$.  The results (Table~\ref{tab:fits-avg-eta2-fix})
are relatively insensitive to $\mu$.

\begin{table*}
\scriptsize
\centering
\caption{Best fits with $\eta_2 = 1$ fixed, for filament averaged FWHMs.
\label{tab:fits-avg-eta2-fix}}
% Best fits with eta2 = 1 fixed, for arithmetic average of FWHMs...
\begin{tabular}{@{}rllr llr llr@{}}

\toprule
{} & \multicolumn{3}{c}{Filament 1}
   & \multicolumn{3}{c}{Filament 2}
   & \multicolumn{3}{c}{Filament 3} \\
\cmidrule(lr){2-4} \cmidrule(lr){5-7} \cmidrule(lr){8-10}
$\mu$ (-) & $\eta_2$ (-) & $B_0$ ($\mu$G) & $\chi^2$
          & $\eta_2$ (-) & $B_0$ ($\mu$G) & $\chi^2$
          & $\eta_2$ (-) & $B_0$ ($\mu$G) & $\chi^2$ \\

\midrule
0.00 & 1 & $363 \pm 14.0$ & 6.7047
     & 1 & $388 \pm 8$ & 3.5110
     & 1 & $489 \pm 24$ & 5.9779\\
0.33 & 1 & $351 \pm 13.0$ & 5.6331
     & 1 & $375 \pm 7$ & 2.8832
     & 1 & $468 \pm 23$ & 5.2188\\
0.50 & 1 & $346 \pm 13.0$ & 5.2477
     & 1 & $369 \pm 7$ & 2.6260
     & 1 & $467 \pm 23$ & 4.9331\\
1.00 & 1 & $334 \pm 12.0$ & 4.5444
     & 1 & $355 \pm 7$ & 2.0359
     & 1 & $439 \pm 22$ & 4.3860\\
1.50 & 1 & $325 \pm 12.0$ & 4.3048
     & 1 & $344 \pm 7$ & 1.6590
     & 1 & $424 \pm 21$ & 4.1910\\
2.00 & 1 & $319 \pm 12.0$ & 4.3274
     & 1 & $336 \pm 7$ & 1.4397
     & 1 & $415 \pm 21$ & 4.2184\\

\midrule
{} & \multicolumn{3}{c}{Filament 4}
   & \multicolumn{3}{c}{Filament 5} \\
\cmidrule(lr){2-4} \cmidrule(lr){5-7}
$\mu$ (-) & $\eta_2$ (-) & $B_0$ ($\mu$G) & $\chi^2$
          & $\eta_2$ (-) & $B_0$ ($\mu$G) & $\chi^2$ \\

\cmidrule(lr){1-7}
0.00 & 1 & $358 \pm 15$ & 0.8029
     & 1 & $239 \pm 37$ & 0.3256\\
0.33 & 1 & $350 \pm 15$ & 1.0612
     & 1 & $231 \pm 36$ & 0.2762\\
0.50 & 1 & $347 \pm 15$ & 1.1809
     & 1 & $228 \pm 36$ & 0.2578\\
1.00 & 1 & $338 \pm 15$ & 1.4792
     & 1 & $220 \pm 34$ & 0.2214\\
1.50 & 1 & $331 \pm 14$ & 1.6770
     & 1 & $215 \pm 34$ & 0.2046\\
2.00 & 1 & $325 \pm 14$ & 1.7851
     & 1 & $210 \pm 33$ & 0.1998\\

\bottomrule
\end{tabular} 
\tablecomments{With only one free parameter, it doesn't really make sense to
vary $B_0$ to get $\Delta\chi^2 = 1$.  Here I report fit standard
errors from the numerically estimated covariance matrix.}

\end{table*}

\subsubsection{How confidently can we reject B-damping?}

Everything hinges on our error bars -- this is most important.  Our rims do
thin consistently, in most regions (might be one or two places where they
don't), but (1) the errors in our individual FWHMs are larger than I'm
currently reporting because different profile fits will give a larger spread of
FWHMs (might need to consider reduced chi-squared here, but using chi-sqr to
select profile fit models is not physically meaningful)

% ==========
% Discussion
% ==========
\section{Discussion}

What do all these numbers with their errors mean?
Can we favor a value of $\mu$ (and hence a turbulent energy spectrum)?

How meaningful are our parameter estimates and errors?

Magnetic field amplification numbers -- how do they compare with previous
studies?  What is the azimuthal variation (and significance thereof)?

Diffusion -- sub-Bohm diffusion in Filament 4?

Constraints on diffusion / B field, if any.  What impact this has for models of
B amplification etc.


% ==========
% Conclusion
% ==========
\section{Conclusions}

In conclusion, crazy B field amplification is not that weird.
Magnetic damping can be ruled out and our result is robust throughout the remnant.

% ================
% Acknowledgements
% ================
\acknowledgments

The scientific results reported in this article are based on data obtained from
the \Chandra Data Archive.
This research has made use of NASA's Astrophysics Data System.

{\it Facilities:} \facility{CXO (ACIS-I)}

\clearpage

% ========
% Appendix
% ========
\appendix

\setcounter{table}{0}
\renewcommand{\thetable}{A\arabic{table}}
\setcounter{figure}{0}
\renewcommand{\thefigure}{A\arabic{figure}}

% --------------
% SN 1006 tables
% --------------
\section{Full model validation, SN 1006 (DRAFT ONLY)}

For comparison to \citetalias{ressler2014}, we reproduce Sean's table
(Table~\ref{tab:sean} and present full model fits with 2 and 3 energy bands
(Tables~\ref{tab:sn1006-2band} and \ref{tab:sn1006-3band}, respectively).
We don't compare simple model results, as they are identical (only the error
calculations differ).
The procedure of Tables~\ref{tab:sean},~\ref{tab:sn1006-2band} are not quite
the same.  Sean used $m_E$ at 2 keV instead of the $1$--$2$ keV width for
fitting, whereas I used both widths.  But, it should not matter much as
the fit often has $\chi^2 \sim 0$, either way (depending on the exact
measurements and values of $\mu$).

Please take the errors with a grain of salt.  They are automatically
generated and \textbf{have not been manually validated}.  Some values may be
invalid, where my error-finding code failed and gave a best, conservative guess
of the error.

\begin{table}[h]
\scriptsize
\centering
\caption{Sean's SN 1006 best fit parameters \citepalias[Table 8]{ressler2014}.
\label{tab:sean}}
\begin{tabular}{@{} l c c c c c c @{}}
\toprule
{}&\multicolumn{2}{c}{Filament 1} & \multicolumn{2}{c}{Filament 2} & \multicolumn{2}{c}{Filament 3} \\
\midrule
$\mu$ &$\eta_{2}$ & $B_{0}$ &$\eta_{2}$ & $B_{0}$ & $\eta_{2}$ & $B_{0}$ \\
0 & 7.5 $\pm$ 2 & 142 $\pm$ 5 & - & - & $\lesssim$ 0.1& 77 $\pm$ .8\\
1/3 & 4 $\pm$ 1.3 & 120 $\pm$ 5 & - & - & $\lesssim$ 0.1 & 76 $\pm$ 1.4 \\
1/2 & 3 $\pm$ 1.1 & 112 $\pm$ 4 & - & - & $\lesssim$ 0.1 & 75 $\pm$ 1.0 \\
1 & 2 $\pm$ 1.0 & 100 $\pm$ 3 & 22 $\pm$ 3 & 214 $\pm$ 4 & $\lesssim$ 0.1 & 74 $\pm$ 1.1 \\
1.5 & 1.9 $\pm$ 1.2 & 95 $\pm$ 3 & 9 $\pm$ 1.2 & 167 $\pm$ 4 & $\lesssim$ 0.1 & 74 $\pm$ 1.2   \\
2 & 2 $\pm$ 1.0 & 92 $\pm$ 4  & 7 $\pm$ 1.1 & 152 $\pm$ 4 & $\lesssim$ 0.1 & 73 $\pm$ 1.2 \\
\midrule
& &\multicolumn{2}{c}{Filament 4} && \multicolumn{2}{c}{Filament 5} \\
\midrule
$\mu$ &&$\eta_{2}$ & $B_{0}$ &&$\eta_{2}$ & $B_{0}$\\
0 &&  $\lesssim$ 0.2 & 113 $\pm$ 2 && -& -\\
1/3 &&  $\lesssim$ 0.2 & 112 $\pm$2 && -& -\\
1/2 && $\lesssim$ 0.2  & 111 $\pm$2 && - & -\\
1 &&  $\lesssim$ 0.2 & 109 $\pm$ 2 && 80 $^{+\infty}_{-4}$   & 206 $\pm$ 3\\
1.5 && $\lesssim$ 0.2 & 108 $\pm$ 2 && 19 $\pm$ 2  & 140 $\pm$ 2\\
2 && $\lesssim$ 0.2 & 107 $\pm$ 2 && 12 $\pm$ 1.0  & 120 $\pm$ 2\\
\bottomrule
\end{tabular}

\end{table}

\begin{table*}[h]
\scriptsize
\centering
\caption{SN 1006 best fit parameters, 2 highest energy bands (full model).
\label{tab:sn1006-2band}}
%\renewcommand{\arraystretch}{1.5}
\begin{tabular}{@{}rllr llr llr @{}}

\toprule
{} & \multicolumn{3}{c}{Filament 1} & \multicolumn{3}{c}{Filament 2} &
     \multicolumn{3}{c}{Filament 3} \\
\cmidrule(lr){2-4}
\cmidrule(lr){5-7}
\cmidrule(lr){8-10}
$\mu$ (-) & $\eta_2$ (-) & $B_0$ ($\mu$G) & $\chi^2$
          & $\eta_2$ (-) & $B_0$ ($\mu$G) & $\chi^2$
          & $\eta_2$ (-) & $B_0$ ($\mu$G) & $\chi^2$ \\

\midrule
0.00 & ${21}^{\,+290}_{\,-20}$ & ${176}^{\,+108}_{\,-80}$ & 0.0086
     & ${21}^{\,+51}_{\,-14}$ & ${262}^{\,+77}_{\,-53}$ & 7.2668
     & ${0.004}^{\,+6800}_{\,-0.004}$ & ${74.75}^{\,+183}_{\,-0.77}$ & 0.8180\\[1.5pt]
0.33 & ${3.4}^{\,+100000}_{\,-3}$ & ${117}^{\,+552}_{\,-24}$ & 0.0000
     & ${68}^{\,+282}_{\,-56}$ & ${316}^{\,+136}_{\,-98}$ & 2.5848
     & ${0.02}^{\,+0.11}_{\,-0.02}$ & ${74.45}^{\,+1.04}_{\,-0.61}$ & 0.8037\\[1.5pt]
0.50 & ${2.5}^{\,+100000}_{\,-2.2}$ & ${109}^{\,+635}_{\,-17}$ & 0.0000
     & ${106}^{\,+792}_{\,-93}$ & ${337}^{\,+204}_{\,-122}$ & 1.0919
     & ${0.024}^{\,+0.11}_{\,-0.024}$ & ${74.16}^{\,+0.87}_{\,-0.53}$ & 0.7946\\[1.5pt]
1.00 & ${1.8}^{\,+6.6}_{\,-1.4}$ & ${98.4}^{\,+23}_{\,-9.2}$ & 0.0000
     & ${21}^{\,+100000}_{\,-15}$ & ${213}^{\,+1160}_{\,-42}$ & 0.0000
     & ${0.012}^{\,+0.11}_{\,-0.012}$ & ${73.68}^{\,+0.21}_{\,-0.88}$ & 0.5364\\[1.5pt]
1.50 & ${1.7}^{\,+3.6}_{\,-1.3}$ & ${94}^{\,+12}_{\,-6.4}$ & 0.0000
     & ${8.5}^{\,+11}_{\,-4.1}$ & ${166}^{\,+24}_{\,-14}$ & 0.0000
     & ${0.007}^{\,+0.13}_{\,-0.007}$ &
     $\left({73.4}^{\,+0}_{\,-1.2}\right)$\tablenotemark{a} & 0.5062\\[1.5pt]
2.00 & ${1.9}^{\,+3.1}_{\,-1.4}$ & ${91.6}^{\,+7.8}_{\,-4.9}$ & 0.0000
     & ${7.0}^{\,+4.9}_{\,-2.8}$ & ${152.4}^{\,+11}_{\,-7.7}$ & 0.0000
     & ${0.026}^{\,+0.10}_{\,-0.026}$ & ${72.21}^{\,+0.23}_{\,-0.53}$ & 0.4443\\

\midrule
{} & \multicolumn{3}{c}{Filament 4} & \multicolumn{3}{c}{Filament 5} \\
\cmidrule(lr){2-4}
\cmidrule(lr){5-7}
$\mu$ (-) & $\eta_2$ (-) & $B_0$ ($\mu$G) & $\chi^2$
          & $\eta_2$ (-) & $B_0$ ($\mu$G) & $\chi^2$ \\

\cmidrule(lr){1-7}
0.00 & ${4300}^{\,+2500}_{\,-4300}$ & ${374.7}^{\,+3.7}_{\,-266}$ & 0.1896
     & ${23}^{\,+33}_{\,-13}$ & ${192}^{\,+40}_{\,-31}$ & 17.9802\\[1.5pt]
0.33 & ${0.011}^{\,+0.16}_{\,-0.011}$ & ${109.2}^{\,+2.1}_{\,-0.9}$ & 0.2605
     & ${68}^{\,+167}_{\,-48}$ & ${230}^{\,+72}_{\,-54}$ & 8.7407\\[1.5pt]
0.50 & ${0.0025}^{\,+0.17}_{\,-0.0025}$ & ${109.2}^{\,+1.2}_{\,-1.3}$ & 0.1959
     & ${107}^{\,+416}_{\,-82}$ & ${246}^{\,+104}_{\,-67}$ & 5.3185\\[1.5pt]
1.00 & ${0.041}^{\,+0.15}_{\,-0.041}$ & ${107.48}^{\,+1.2}_{\,-0.72}$ & 0.2759
     & ${447}^{\,+16000}_{\,-424}$ & ${311}^{\,+389}_{\,-151}$ & 0.1152\\[1.5pt]
1.50 & ${0.023}^{\,+0.17}_{\,-0.023}$ & ${106.8}^{\,+0.54}_{\,-0.41}$ & 0.1114
     & ${20}^{\,+49}_{\,-10}$ & ${142}^{\,+42}_{\,-18}$ & 0.0000\\[1.5pt]
2.00 & ${0.013}^{\,+0.18}_{\,-0.013}$ & ${106.57}^{\,+0.33}_{\,-1.4}$ & 0.0457
     & ${12.1}^{\,+9.2}_{\,-4.7}$ & ${121.6}^{\,+12}_{\,-8.5}$ & 0.0000\\

\bottomrule
\end{tabular}
\tablenotetext{1}{Seems unlikely that error is $0$, probably bad calculation}

\end{table*}

\begin{table*}[h]
\scriptsize
\centering
\caption{SN 1006 best fit parameters, 3 energy bands (full model).
\label{tab:sn1006-3band}}
%\renewcommand{\arraystretch}{1.5}
\begin{tabular}{@{}rllr llr llr @{}}

\toprule
{} & \multicolumn{3}{c}{Filament 1} & \multicolumn{3}{c}{Filament 2} &
     \multicolumn{3}{c}{Filament 3} \\
\cmidrule(lr){2-4}
\cmidrule(lr){5-7}
\cmidrule(lr){8-10}
$\mu$ (-) & $\eta_2$ (-) & $B_0$ ($\mu$G) & $\chi^2$
          & $\eta_2$ (-) & $B_0$ ($\mu$G) & $\chi^2$
          & $\eta_2$ (-) & $B_0$ ($\mu$G) & $\chi^2$ \\

\midrule
0.00 & $25^{\,+170}_{\,-23}$ & ${183}^{\,+89}_{\,-70}$ & 0.1162
     & $\left({0.00}^{\,+0.26}_{\,-0.00}\right)$\tablenotemark{a} & ${132.18}^{\,+320}_{\,-0.72}$ & 53.8334
     & ${0.012}^{\,+0.18}_{\,-0.012}$ & ${75.25}^{\,+180}_{\,-0.9}$ & 1.7278\\[1.5pt]
0.33 & $730^{\,+6100}_{\,-730}$ & $350^{\,+130}_{\,-130}$ & 0.0102
     & ${0.082}^{\,+0.23}_{\,-0.082}$ & ${132.5}^{\,+3.5}_{\,-1}$ & 53.4559
     & ${0.01}^{\,+0.18}_{\,-0.01}$ & ${74.86}^{\,+1.4}_{\,-0.65}$ & 1.7621\\[1.5pt]
0.50 & $3.9^{\,+210}_{\,-2.9}$ & ${116}^{\,+630}_{\,-18}$ & 0.0741
     & ${0.043}^{\,+0.31}_{\,-0.043}$ & ${131.56}^{\,+3.8}_{\,-0.62}$ & 53.3622
     & ${0.0056}^{\,+0.16}_{\,-0.0056}$ & ${74.61}^{\,+0.72}_{\,-0.48}$ & 1.5513\\[1.5pt]
1.00 & ${2.6}^{\,+5.4}_{\,-1.7}$ & ${102.5}^{\,+18}_{\,-8.9}$ & 0.1420
     & ${0.18}^{\,+0.27}_{\,-0.17}$ & ${130.5}^{\,+2.3}_{\,-1}$ & 52.4123
     & ${0.016}^{\,+0.18}_{\,-0.016}$ & ${73.85}^{\,+0.72}_{\,-0.54}$ & 1.7703\\[1.5pt]
1.50 & ${2.5}^{\,+3.2}_{\,-1.5}$ & ${97.1}^{\,+9.1}_{\,-5.9}$ & 0.2221
     & ${0.2}^{\,+0.45}_{\,-0.18}$ & ${129.06}^{\,+2.6}_{\,-0.82}$ & 51.8975
     & ${0.01}^{\,+0.16}_{\,-0.01}$ & ${73.47}^{\,+0.68}_{\,-0.9}$ & 1.4024\\[1.5pt]
2.00 & ${2.7}^{\,+2.9}_{\,-1.6}$ & ${94.2}^{\,+6.2}_{\,-4.6}$ & 0.3025
     & ${0.53}^{\,+0.34}_{\,-0.44}$ & ${128.9}^{\,+1.6}_{\,-1.7}$ & 50.3784
     & ${0.015}^{\,+0.17}_{\,-0.015}$ & ${73.2741}^{\,+0.0092}_{\,-1.1}$ & 1.6535\\

\midrule
{} & \multicolumn{3}{c}{Filament 4} & \multicolumn{3}{c}{Filament 5} \\
\cmidrule(lr){2-4}
\cmidrule(lr){5-7}
$\mu$ (-) & $\eta_2$ (-) & $B_0$ ($\mu$G) & $\chi^2$
          & $\eta_2$ (-) & $B_0$ ($\mu$G) & $\chi^2$ \\

\cmidrule(lr){1-7}
0.00 & - & - & -
     & ${23}^{\,+34}_{\,-13}$ & ${192}^{\,+40}_{\,-31}$ & 18.1971 \\[1.5pt]
0.33 & - & - & -
     & ${61}^{\,+190}_{\,-43}$ & ${223}^{\,+80}_{\,-53}$ & 9.7081 \\[1.5pt]
0.50 & - & - & -
     & ${101}^{\,+470}_{\,-80}$ & ${242}^{\,+114}_{\,-71}$ & 6.8545 \\[1.5pt]
1.00 & - & - & -
     & ${35}^{\,+45000}_{\,-25}$ & ${174}^{\,+680}_{\,-39}$ & 3.7991 \\[1.5pt]
1.50 & - & - & -
     & ${10.9}^{\,+12}_{\,-4.7}$ & ${127}^{\,+18}_{\,-11}$ & 3.2257 \\[1.5pt]
2.00 & - & - & -
     & ${8.8}^{\,+5.1}_{\,-3}$ & ${115.4}^{\,+8.5}_{\,-6.3}$ & 2.7391 \\

\bottomrule
\end{tabular}
\tablenotetext{1}{Best fit value is formally $6.6\times10^{-6}$, close enough.}

\end{table*}


% ------------------
% Sean's Cas A table
% ------------------
\section{Sean's Cas A comparison (DRAFT ONLY)}

This is dredged from the \LaTeX source for Sean's paper, available on arXiv.
I put this here just for my own reference.

A detailed application of our results to other SNRs such as Cas A will
require much more extensive analysis, but we can use the published
filament widths of Araya et al. (2010) for Cas A to get preliminary
estimates of the magnetic field strength and diffusion coefficient by
applying our model. In their data, it appears that the filaments in
Cas A shrink by a factor of $\sim 0.8$ between 0.3 and 3 keV, while the
filament widths appear to be energy-independent between 3 and 6 keV.
Qualitatively, this is consistent with the loss-limited model, as our
parameter $m_{\rm E}$ is predicted to decrease with energy.  For the
lower energy range of 0.3--3 keV, reproducing $m_{\rm E}\sim -0.1$
(equivalent to the factor of 0.8 drop in size) requires magnetic
fields on the order of 200-500 $\mu$G and diffusion coefficients about
$5 \times D_{\rm Bohm}$(3 keV), about an order of magnitude higher
than the values one obtains by neglecting the energy dependence. One
can also see directly from Figure~\textbf{?!} that $\mu < 1$ models
of the diffusion coefficient are excluded for $m_{\rm E}\sim -0.1$.

To quickly compare how our results will differ from those of other authors, we
applied the analytic approximation of Equation 22 and Equation 9 to the FWHMs
presented by Araya et al. (2010) for Cas A. Here we used $m_{\rm E} = \log(
{\rm FWHM}(3keV)/{\rm FWHM}(.3keV))/\log(10)$ and fit the filaments at a photon
energy of 3 keV.  Hence $\eta_3 \equiv D/D_{\rm Bohm}(3 keV)$. Using the same
parameters, our results are shown in Table~\ref{tab:araya}. To fit the energy
dependence of the FWHM (i.e. $m_{\rm E}$), we required higher diffusion
coefficients and, consequently, magnetic fields about an order of magnitude
higher than their results. We note here that values of $\mu<1$ were unable to
reproduce the data and that the diffusion coefficients cited by Araya et al.
(2010) are below the minimally allowed Bohm value ($\eta_3 = 1$).

We intended this comparison solely as a qualitative analysis, and thus did not
rigorously keep track of uncertainties or extend the comparison to the full
numerical calculation. The conclusion is clear, however, which is that
considering the measured value of $m_{\rm E}$ can have a dramatic effect.

\begin{table}[h]
\tiny
\centering
\caption{Best fit parameters for the filaments of Cas A based on data from
Araya et. al (2010) in varying values of $\mu$ (Approximate Analytic Results)
Dashes denote places where fits were unobtainable.
\label{tab:araya}}

\begin{tabular}{@{}c c c c c c c c c@{}}
\toprule
  &\multicolumn{2}{c}{$\mu = 1$} & \multicolumn{2}{c}{$\mu = 1.5$} &
    \multicolumn{2}{c}{$\mu = 2$} & \multicolumn{2}{c}{Araya et. al} \\
  & & & & & &  &\multicolumn{2}{c}{($\mu = 1$)} \\
\cmidrule(l){2-3}
\cmidrule(l){4-5}
\cmidrule(l){6-7}
\cmidrule(l){8-9}
Filament &$\eta_{3}$ & $B_{0}$ &$\eta_{3}$ & $B_{0}$ & $\eta_{3}$ & $B_{0}$ &
    $\eta_{3}$ & $B_{0}$ \\
\cmidrule(l){2-9}
1 & 6.5  & 750 $\mu$G & 2 & 567 $\mu$G & 1.1 & 506 $\mu$G & 0.12 & 72 $\mu$G \\
2 & 3.6 & 710 $\mu$G & 1.4 & 582 $\mu$G & - & - & 0.02 & 37 $\mu$G\\
3 & 3.2 &  710 $\mu$G& 1.3 & 589 $\mu$G& - & - & 0.02 & 53 $\mu$G\\
4 & 3.1& 515 $\mu$G & 1.3 & 430 $\mu$G & -& - & 0.02 & 40 $\mu$G \\
5 & 10.7 & 1163 $\mu$G & 2.5 & 809 $\mu$G & 1.3&  706 $\mu$G & 0.025 & 52 $\mu$G \\
6 & 5.7 & 844 $\mu$G & 1.9 & 650 $\mu$G  & 1 & 583 $\mu$G & 0.1 & 56 $\mu$G \\
7 & 8.3 & 872 $\mu$G & 2.2 & 635 $\mu$G& 1.2 & 560 $\mu$G & 0.15 & 66 $\mu$G\\
8 & 13 & 1010 $\mu$G & 2.7 & 738 $\mu$G & 1.4 & 639 $\mu$G & 0.02 & 35 $\mu$G\\
9 & 4.7 & 605 $\mu$G & 1.7 & 479 $\mu$G & - & -& 0.02 & 29 $\mu$G\\
\bottomrule
\end{tabular}

\end{table}


% ==========
% References
% ==========
\bibliographystyle{apj}  % AASTeX journal macros are supplied in ADS entries
\bibliography{refs-snr}

\end{document}
