%\documentclass[manuscript]{aastex}  % one-column, double-spaced GENERATE BIB
% \documentclass[12pt,preprint]{aastex}  % one-column, single-spaced
\documentclass[iop, apj, numberedappendix, twocolappendix]{emulateapj}
% \documentclass[iop, apj, numberedappendix]{emulateapj}


\shorttitle{Synchrotron Rims in Tycho's SNR}  % <~ 44 char
\shortauthors{XXX et al.}  % Max three
\slugcomment{Draft, \today}  % short title pg comment

%% ==================================================================== %%
%% README for track changes                                             %%
%% To add/remove text or add comments, use the following commands:      %%
%%                                                                      %%
%%       \note[editor]{The note}                                        %%
%%     \annote[editor]{Text to annotate}{The note}                      %%
%%        \add[editor]{Text to add}                                     %%
%%     \remove[editor]{Text to remove}                                  %%
%%     \change[editor]{Text to remove}{Text to add}                     %%
%%                                                                      %%
%% ==================================================================== %%

\usepackage[inline]{trackchanges}  % trackchanges.sourceforge.net
\addeditor{Rob}
\addeditor{Sean}
\addeditor{Steve}
\addeditor{Aaron}
\addeditor{Brian}

\usepackage{amsmath}  % amsthm, amssymb
% \usepackage{CJK}  % aas.org/authors/author-names-non-roman-alphabets
\usepackage{booktabs}
%\usepackage[labelfont=bf, labelsep=period]{caption}  % Custom float captions
%\usepackage{pdflscape}  % rotate pages (Texlive)
\usepackage{hyperref}

\newcommand*{\mt}{\mathrm}
\newcommand*{\unit}[1]{\;\mt{#1}}  % http://vemod.net/typesetting-units-in-latex
\newcommand*{\abt}{\mathord{\sim}} % http://tex.stackexchange.com/q/55701
\newcommand*{\ptl}{\partial}
\newcommand*{\del}{\nabla}

% This paper
\newcommand*{\tsup}{\textsuperscript}
\newcommand*{\Chandra}{\textit{Chandra}\ }
\newcommand*\mean[1]{\bar{#1}}
\newcommand*{\tsynch}{\tau_{\mt{synch}}}
\defcitealias{ressler2014}{R14}

\begin{document}

\title{Synchrotron X-Ray Rims in Tycho's Supernova Remnant are Energy Dependent}

%\begin{CJK*}{UTF8}{gbsn}
\author{
Robert Petre\altaffilmark{1},
Sean M. Ressler\altaffilmark{2},
Stephen P. Reynolds\altaffilmark{3},
Aaron Tran\altaffilmark{1,4},
Brian J. Williams\altaffilmark{1,5}
}
%\end{CJK*}

\affil{
\tsup{1}NASA Goddard Space Flight Center, Greenbelt, MD 20771, USA \\
\tsup{2}Dept. Physics, University of California, Berkeley, CA 94720, USA \\
\tsup{3}Dept. Physics, North Carolina State University, Raleigh, NC 27695, USA
}

%\altaffiltext{1}{NASA Goddard Space Flight Center, Greenbelt, MD 20771, USA}
%\altaffiltext{2}{Dept. Physics, University of California,
%    Berkeley, CA 94720, USA}
%\altaffiltext{3}{Dept. Physics, North Carolina State University,
%    Raleigh, NC 27695, USA}
\altaffiltext{4}{CRESST, University of Maryland, College Park, MD 20742}
\altaffiltext{5}{NASA Postdoctoral Program Fellow}

\begin{abstract}
\note[Aaron]{Copy pasted from NASA abstract, not reviewed} Young supernova
remnants may exhibit thin (~1--10\% of shock radius) X-ray rims of synchrotron
radiation from forward shock-accelerated electrons that travel downstream of
the shock and quickly cease to radiate. Rim widths limited by radiative energy
losses should decrease with energy and require magnetic field amplification
$10$--$100\times$ that expected from adiabatic shock compression.  Damped
magnetic fields behind rims may produce thin rims without strong field
amplification but require energy-independent rim widths. We measured rim widths
around Tycho's supernova remnant in 5 energy bands using a 750 ks \Chandra
observation. Rims narrow with increasing energy, favoring loss-limited
radiation over magnetic damping. Observed widths are best fit by electron
transport models requiring amplified magnetic fields $\abt0.1$--$1$ mG and
particle diffusion $\abt1$--$10\times$ Bohm values, consistent with prior work
on SN 1006. Inferred magnetic fields, diffusion coefficients, and
diffusion-energy scaling may constrain models for cosmic ray acceleration in
supernova remnants and plasma turbulence in astrophysical shocks.
\end{abstract}

% Six keywords, alphabetical order
\keywords{acceleration of particles ---
    ISM: individual objects (Tycho's SNR) ---
    ISM: magnetic fields ---
    ISM: supernova remnants ---
    shock waves ---
    X-rays: ISM}

% ============
% Introduction
% ============
\section{Introduction}

\note[Aaron]{this first draft of intro structure is mostly set (in terms of
material, number of paragraphs).}

% Intro -- shocks, DSA, thin rims, filament observations; other fields/work
Forward shock accelerated ISM/CSM particles in young supernova remnants emit
synchrotron radiation strongly in the shock's immediate wake but quickly turn
off downstream \citep{koyama1995, reynolds1996}, producing a shell-like
morphology with bright X-ray and radio rims/filaments due to line-of-sight
projection (hereafter, thin rims).
Seminal observations \& interpretations of these filaments, interpretation in
context of DSA.
People (such as the authors of this manuscript) are trying to model these
filaments and figure out what's going on.
Why do we care about these models?  They tell us something about what happens
behind the shock, and perhaps what's going on at the shock.  We get at the
plasma turbulence, magnetic field turbulence and structure, electron injection.
Applications to: cosmic ray acceleration, turbulent field amplification, (note
assumptions in re proton/electron spectra).  Cut-off energy of injected
electrons?  Relevant to, broadly, all astrophysical shocks: Earth's bow shock,
PWNe, SNe and/or stellar wind blown bubbles, starbursts \citep{heckman1990},
AGN jets, galaxy clusters \citep{van-weeren2010}, cosmological shocks?!
\citep{miniati2000, ryu2008}, other cool stuff.  See relevant reviews
\citep{blandford1987}.

% What makes these shocks?
The widths of these X-ray rims are thought to be controlled by a combination of
synchrotron losses, particle transport, and magnetic fields immediately
downstream of the shock.  Synchrotron losses depend on the initial electron
energy distribution and the gradual decrease of electron energies downstream of
the shock.  At the shock, high energy electrons efficiently radiate harder
synchrotron X-ray photons; as they are transported away from the shock, they
radiate at lower energies and lose energy more slowly.  The transport process
is driven by both bulk plasma advection away from the shock at downstream
velocity $v_d = v_s / r$, where $v_s$ is shock velocity and $r$ the compression
ratio, and diffusion ... (driven by collisionless particle interaction,
magnetic field turbulence, ????? fill in details) with respect to bulk
advection.  Energy-dependent diffusion (e.g., diffusion coefficient $D \propto
E$ for Bohm diffusion) allows higher energy electrons to diffuse further
upstream or downstream than would be expected from pure advection; in
particular, diffusion upstream of the shock may give rise to a shock precursor
\citep[e.g.,][]{ghavamian2000, wagner2009, laming2014}.  Finally, the magnetic
field strength may be quickly damped behind the shock, so that electrons may
not radiate as efficiently and thin rims would reflect the spatial structure of
the field, rather than efficient particle acceleration (and losses) and/or
efficient synchrotron cooling \citep{pohl2005}.  \note[Aaron]{I need to read up
on particle transport, diffusion, plasma physics!}

% Energy dependence.
Moreover, these controlling mechanisms predict different scalings for rim width
as a function of energy.  Loss-limited rims should scale as $w \propto
\nu^{-1/2}$.  Diffusion weakens this energy dependence by smearing out rims at
all energies (even for energy-independent diffusion), yielding a slower
drop-off in rim widths when diffusion dominates.  Magnetic damping, at a
length scale where it may be considered a relevant control on rim widths (i.e.,
length scale comparable to filament width) predicts energy-independent rim
widths -- intuitively, if the magnetic field turns off, synchrotron radiation
turns off regardless of electron energy.  \citet{ressler2014} give a much
fuller exposition of the relative effects of advection, diffusion, and magnetic
field damping.

% Which one is right?
Previous measurements of basic observables \citep[e.g.,][]{bamba2003,
bamba2005-hist, bamba2005-vela, parizot2006} and consequences.  Conclusion that
magnetic fields are amplified according to loss-limited model.  How could this
be plausible?  There are possible explanations -- field amplification from
plasma instabilities, evidence for stronger compression ratios?
\citep{blondin2001}.  Numerical simulations and theory on this matter
accounting for streaming instabilities, CR escape / energy losses, MHD
equations, all sorts of things I can't keep track of.  But, magnetic damping
model offers a tantalizing alternate explanation, and has not been tested.
Equivocal results by \citet{araya2010} for Cas A.
\citet{rettig2012} proposed discriminating based on filament spectra -- the
expectation is that damped spectra are softer, loss-limited harder.
\citet{ressler2014} developed the idea of investigating energy dependence and
found rim narrowing in the remnant of SN 1006, strongly favoring a loss-limited
model.

% Why Tycho's SNR
To test these models, we turn to the filaments of Tycho's supernova remnant
(hereafter, Tycho).
Like all historical SNe, Tycho is close enough that we can spatially
resolve these rims with \Chandra and say something useful.  Like SN 1006, Tycho
has well defined synchrotron rims associated with expansion into a low-density
ISM.  \citet{williams2013} favor mean pre-shock ISM density $\abt 0.1$--$0.2
\unit{cm^{-3}}$ from \textit{Spitzer} observations of shocked ISM dust
emission, consistent with Tycho's X-ray expansion rate \citep{katsuda2010}; SN
1006, by comparison, has pre-shock density $\abt 0.01$--$0.1 \unit{cm^{-3}}$ at
its nonthermal NE, SW limbs \citep{acero2007}.
\note[Aaron]{More disjointed}
Mention molecular cloud interaction \citep{reynoso1999}? Maybe below in region
selection as well.
Previous estimates of Tycho fields/diffusion (something similar to Sean's table
for SN 1006?).  CR acceleration (aside: \citet{eriksen2011}) evidence in Tycho,
if relevant.

% Paper roadmap
\note[Aaron]{roadmap}
We measure Tycho's rim widths, following the work of \citet{ressler2014}
(hereafter, \citetalias{ressler2014}) in the remnant of SN 1006.
We make measurements to distinguish between two main scenarios
(explain how diffusion/advection competition come into this).
Our procedure, in both measurement and rim width modeling, follows that of
\citetalias{ressler2014} with only slight modifications.
We first select regions around Tycho's forward shock for analysis, measure
rim widths, and verify that rim spectra are free of thermal line emission.
Using two models for particle transport, we fit measured widths...
We discuss the implications of our fits and models for magnetic shock
amplification (lending credence to previous estimates, in disfavoring magnetic
field amplification), and discuss particle diffusion/acceleration at the shock.
Any further misc. applications (constraints on precursors, etc).


% =============================
% Transport models, observables
% =============================
\section{Electron transport models}\label{sec:transport}

We briefly review relevant particle transport models that we use to fit our
measurements; a fuller review and exposition is given by \citetalias{ressler2014}.
\note[Aaron]{is it necessary to give Sean's equation numbers?  How many of the
derived equations should we give?}  In particular, the transport models
account for both advection and diffusion simultaneously, as opposed to
considering the lengthscales separately with advection dominating below a
critical energy and diffusion, above
\citep{bamba2003, vink2003, yamazaki2004, bamba2005-hist}
\note[Aaron]{clarify, be more specific}.
The effects of varying various parameters, especially in regards to filament
energy dependence, are explored in detail by \citetalias{ressler2014}.

% TODO Choose names for these models and refer to them consistently, throughout
% TODO define ALL symbols, synchrotron constants, etc (somewhere state that CGS
%      notation is being used)
\note[Aaron]{Double check against Rettig/Pohl.  Variable definitions will be
added when equations/exposition is closer to final state.}
The electron distribution which gives rise to the observed synchrotron
radiation is injected at the forward shock and transported by advection
and plasma diffusion, following the advection-diffusion equation:
\[
  \frac{\ptl f}{\ptl t} + \del \cdot \left( f \vec{v} \right)
  = C + \del \cdot \left( D \del f \right)
\]
By considering only steady-state transport and some other assumptions
(1-D flow (did we neglect the spherical coord terms?), neglect compressible flow?!,
space-independent diffusion coefficient):
\[
    v \frac{\ptl f}{\ptl x} - D \frac{\ptl^2 f}{\ptl x^2} = C .
\]
The source/sink term $C$ reflects assumptions about the injected electron
energy spectrum and its time/space evolution.
A first approximation is to treat electrons as having constant energy, but
dumping energy catastrophically after a synchrotron timescale
$\tsynch$ \note[Aaron]{define this -- either Section~\ref{sec:transport} before jumping into
transport models, or in the intro}, yielding:
\begin{equation} \label{eq:simp-mod}
    v \frac{\ptl f}{\ptl x} - D \frac{\ptl^2 f}{\ptl x^2} +
    \frac{f}{\tau_{\mt{synch}}} = 0
\end{equation}
where this is equation (5) of \citetalias{ressler2014}.
Equation~\eqref{eq:simp-mod} has a convenient analytic solution (note that
neglecting diffusion, this is $\ptl f/\ptl x \propto -f/(\tsynch v_d)$.
A more advanced model is given by accounting for continuous radiation energy
losses while electrons are transported downstream, given as:
\begin{equation} \label{eq:full-mod}
    v \frac{\ptl f}{\ptl x} - D \frac{\ptl^2 f}{\ptl x^2} =
    K_0 E^{-s} e^{-E/E_{\mt{cut}}} \delta(x) + \frac{\ptl}{\ptl E}
      \left(bB^2E^2f\right)
\end{equation}
which is equation (12) of \citetalias{ressler2014}.

Explain how the transport models constrain diffusion via $m_E$, define $m_E$.
Explain how we back out a filament shape/width from these models

\subsection{Energy dependent widths, diffusion}

Bohm diffusion: $D(E) = \eta C_d E / B$. Generalized diffusion writen as:
\[
    D(E) = \eta C_d E^\mu / B = \eta_h D(E_h) (E/E_h)^\mu
\]
Explain the use of fiducial energy -- this makes sense and eases discussion,
but linking $E_h$ and $\eta_2$ to $\eta$ and plain Bohm diffusion seems a bit
murky.

For the simple model, we have equation (6) of \citetalias{ressler2014}:
\begin{equation} \label{eq:simp-width}
    a = \frac{2 D / v_d}{ \sqrt{ 1 + \frac{4D}{v_d^2 \tau_{\mt{synch}}}} - 1}
\end{equation}
which we relate to full width half max by projection factor $\beta = 4.6$
so that $w = \beta a$, assuming (1) electron distribution exponential in radial
coordinate $x$, and (2) spherical source \note[Aaron]{shell?}
\citep{ballet2006}.

For the full model (equation~\eqref{eq:full-mod}), we numerically solve for
electron distribution $f(E,x)$ using solutions due to \citet{lerche1980} and
\citet{rettig2012}; DETAILS.
Ultimately, we integrate numerically computed electron distributions $f(E,x)$
over energy to obtain synchrotron emissivity as
\[
    j_{\nu}(x) \propto \int_0^\infty G(y) f(E,x) dE
\]
where $y \equiv \nu/(c_1 E^2 B)$ is a scaled synchrotron frequency and
$G(y) = y \int_y^\infty K_{5/3}(z) dz$ is the one-particle synchrotron
emissivity with $K_{5/3}(z)$ a modified Bessel function of the second kind
\citep{pacholczyk1970}.  Integrating emissivity over lines of sight yields
\begin{equation} \label{eq:intensity}
    I_{\nu}(r) = 2 \int_0^{\sqrt{r_s^2 - r}}
                    j_{\nu} \left( r_s - \sqrt{s^2 + r^2} \right) ds
\end{equation}
where $s$ is the line-of-sight coordinate.

\subsection{Tycho parameters}

Give numbers ($v_s$, $v_d$, $r_s$, etc.).  Give a definition of the cut-off
energy following Rettig and Pohl (again cite Sean's paper).
We use an electron spectrum cut-off energy of:
\begin{align*}
    E_{\mt{cut}} =
        &\left(8.3\unit{TeV}\right)^{2/(1+\mu)}
        \left(\frac{B_0}{100 \unit{\mu G}}\right)^{-1/(1+\mu)} \\
        &\times \left(\frac{v_s}{10^8 \unit{cm/s}}\right)^{2/(1+\mu)}
        \left[
            \frac{E_h^{\mu - 1}}{\eta}
        \right]^{1 / (1 + \mu)}
\end{align*}
which is equation~(19) of \citetalias{ressler2014}.
\note[Aaron]{This differs from Sean's presentation -- I drew this from the
Fortran code and it looks consistent w/ Parizot et al., but I haven't vetted
the prefactor $8.3 \unit{TeV}$.  And, I have an extra exponent $\mu - 1$ on the
$E_h$ ?!  Come back to this shortly.}

\subsection{Magnetic damping}

We also consider a magnetically damped field of form
\[
    B(x) = B_{\mt{min}} + \left(B_0 + B_{\mt{min}}\right) \exp\left(-x / a_b\right) ,
\]
following \citetalias[Section 3.2]{ressler2014}.


% ============
% Observations
% ============
\section{Observations}
\label{sec:observations}

% TODO go through and double check tenses.
We measured synchrotron rim full widths at half maximum (FWHMs) from an
archival \Chandra ACIS-I observation of Tycho
(RA: 00\tsup{h}25\tsup{m}19\fs0, dec: +64\arcdeg08\arcmin10\farcs0; J2000)
between 2009 Apr 11 and 2009 May 5 (PI: John P. Hughes;
\dataset[ADS/Sa.CXO\#obs/10093--10097]{ObsIDs: 10093--10097},
\dataset[ADS/Sa.CXO\#obs/10902--10906]{10902--10906}).
The total exposure time was $734 \unit{ks}$.
Level 1 \Chandra data were reprocessed with CIAO 4.6 and CALDB 4.6.1.1 and kept
unbinned with ACIS spatial resolution $0.492\arcsec$.
Merged and corrected events were divided into five energy bands:
0.7--1 keV, 1--1.7 keV, 2--3 keV, 3--4.5 keV, and 4.5--7 keV.
We excluded 1.7--2 keV energy range to avoid the \ion{Si}{13} (He$\alpha$)
emission line at 1.85 keV, prevalent in the remnant's thermal ejecta, which
\remove[Brian]{could} \remove[Aaron]{would} \add[Aaron]{may}
\note[Aaron]{since some profiles are practically uncontaminated?}
contaminate our nonthermal profile measurements.

We selected 13 regions in 5 distinct filaments around Tycho's shock
(Figure~\ref{fig:snr}) based on the following criteria:
(1) filaments should be clear of spatial plumes of thermal ejecta in \Chandra
images; this rules out, e.g., areas of strong thermal emission on Tycho's eastern limb.
(2) filaments should be singular and localized; that is, multiple filaments
should either not overlap or completely overlap (rules out parts of NE limb);
(3) filaments should have clear FWHMs bands; i.e., a peak should be evident
above the background signal or downstream thermal emission (rules out some
faint southern wisps) \note[Aaron]{wishy-washy qualitative}.

\begin{figure}
    \centering
    %\plotone{figures/f0-snr.pdf}
    \includegraphics[width=0.35\textwidth]{figures/f0-snr.pdf}
    \caption{RGB image of Tycho with region selections overlaid.  Bands are
    0.7--1 keV (red), 1--2 keV (green) and 2--7 keV (blue).
    \note[Aaron]{Temporary figure; regions need re-numbering.}
    \note[Aaron]{We could show the spectrum extraction regions on here too?
    Maybe too cluttered.}}
    \label{fig:snr}
\end{figure}

All measured rim widths are at least $\abt 1\arcsec$ and hence are resolved by
\textit{Chandra}'s point-spread function.
Looking at proposer's guide, Figure 4.13 -- the off-axis PSF should
still have FWHM of $\abt 1\arcsec$ at $4\arcmin$ (approx. Tycho's radius).
Is that good enough for us?

% ----------------
% Filament spectra
% ----------------
\subsection{Filament spectra}
\label{sec:spec}

We extracted spectra for all regions to check that rim width measurements are
not affected by contaminating thermal line emission.  In each region, we
selected upstream and downstream sections \note[Aaron]{any better word than
`section'?} from which to extract spectra.
The upstream section is the smallest sub-region that contains the measured FWHM
bounds from all energy bands (see Section~\ref{sec:fwhms}); i.e., this region
captures the thin rim.
The downstream section extends from the back of the rim (where the upstream
section ends) to the rim model fit's
(equation~\eqref{eq:prof}) downstream limit.
Figure~\ref{fig:spec} plots an example profile (4.5--7 keV) with the
downstream and upstream sections highlighted; the dividing line is set at the
\textbf{???--???} energy band's downstream FWHM limit (see Figure~\ref{fig:profiles}).
Background spectra were extracted from circular regions around the remnant's
exterior \note[Aaron]{give size in arcsec?}; each region's spectra used the
closest background region's spectrum \note[Aaron]{a bit awkward}.

% TODO must match fig:profiles!  Keep both figures' region numbers updated and
% consistent.
\begin{figure}
    \plotone{figures/f0-spec.pdf}
    \caption{Example spectra and fits from Region 1. Left: example profile
    ($4.5$--$7 \unit{keV}$) with highlighted downstream (blue) and upstream
    (grey) sections. These sections are fixed for Region 1, regardless of
    energy band (to be absolutely clear).  Middle: downstream spectrum with
    fitted absorbed power law; Si and S lines at $1.85$, $2.45 \unit{keV}$ are
    clearly visible.  Right: upstream spectrum with fitted absorbed power law
    shows that the thin rim is not contaminated by line emission.}
    \label{fig:spec}
\end{figure}

We fit $0.5$--$7 \unit{keV}$ spectra from each region's upstream and downstream
sections to an absorbed power law model (XSPEC 12.8.1, \texttt{phabs*po}) with
spectral index $n$, hydrogen column density $n_H$, and a normalization as free
parameters.  Table~\ref{tab:spec} lists best fit parameters and reduced
$\chi^2$s for all regions.  Spectra from thin rims (upstream sections) are
well-fit by the power law model alone; spectra from sections downstream of the
rims are generally poorly-fit, reflecting thermal contamination primarily from
\ion{Si}{13} and \ion{S}{15} He$\alpha$ line emission at $1.85$ and $2.45$ keV.
Although most downstream spectra show thermal contamination, some regions
(e.g., 3, 4 on Tycho's northwestern limb) are reasonably well-fit by an
absorbed power law as well due to a larger distance between the forward shock
and contact discontinuity.
Upstream spectra fits spectral indices between $2.5$--$3$ are consistent with
synchrotron radiation spectra for ???? population of electrons
\note[Aaron]{citation for e- distr. indices?}. Best fit column densities have
range ($0.61$--$0.78 \times 10^{22} \unit{cm^{-2}}$) consistent with previous
fits to Tycho's nonthermal rims \citep{hwang2002} but somewhat larger than
values reported from radio and optical? HI absorption line measurements
($\abt0.5 \times 10^{22}$ in radio \citep{albinson1986} per \citet{hwang2002};
$\abt0.35 \times 10^{22}$ by \citet{kothes2004}, $\abt 0.4 \times 10^{22}$ from
extinction--gas ratio (references in \citet{black1984})...).
% TODO keep stated ranges of spectral indices, column densities up to date

\note[Aaron]{On rob's suggestion of fitting thermal model -- can we get away
with just fitting line emission and constraining from there? or is the
whole NEI apparatus needed.  If we add thermal components -- report spectral
indices and show better fit / spectral softening (confirmation)?}

% TODO short script to build table from saved spectra fits
% TODO keep region numbers updated (discussion of good/bad downstream spectra)
\begin{table}
    \scriptsize
    \centering
    \caption{Absorbed power law fit parameters for all regions' upstream and
    downstream spectra.  The reduced $\chi^2$ values for upstream spectrum fits
    are consistent with a lack of thermal emission, whereas downstream spectra
    are often poorly-fit by an absorbed power law due to line
    emission (e.g., Figure~\ref{fig:spec}).\label{tab:spec}}
    \begin{tabular}{@{}lcccccr@{}}
\toprule
{} & \multicolumn{3}{c}{Downstream spectra}
   & \multicolumn{3}{c}{Upstream spectra} \\
\cmidrule(lr){2-4} \cmidrule(l){5-7}
Region & $n$ & $n_H$ & $\chi^2_{\mathrm{red}}$ (dof)
       & $n$ & $n_H$ & $\chi^2_{\mathrm{red}}$ (dof) \\
{} & (-) & ($\mt{cm}^{-2}$) & {}
   & (-) & ($\mt{cm}^{-2}$) & {} \\
\midrule
1 & 2.97 & 0.68 & 2.27 (272) & 2.77 & 0.78 & 0.92 (239) \\
2 & 2.91 & 0.64 & 3.55 (163) & 2.54 & 0.67 & 1.05 (232) \\
3 & 3.00 & 0.60 & 3.41 (181) & 2.75 & 0.66 & 1.13 (245) \\
4 & 2.94 & 0.45 & 1.35 (199) & 2.88 & 0.66 & 0.92 (224) \\
5 & 2.90 & 0.50 & 1.15 (224) & 2.83 & 0.68 & 0.96 (246) \\
6 & 2.85 & 0.43 & 1.87 (194) & 3.00 & 0.63 & 1.20 (222) \\
7 & 2.80 & 0.37 & 1.43 (100) & 2.79 & 0.61 & 1.11 (243) \\
8 & 2.79 & 0.55 & 2.36 (183) & 2.83 & 0.71 & 1.22 (285) \\
9 & 2.99 & 0.68 & 4.43 (239) & 2.77 & 0.70 & 1.01 (252) \\
10 & 2.89 & 0.58 & 1.31 (186) & 2.86 & 0.71 & 1.27 (301) \\
11 & 2.89 & 0.61 & 4.92 (231) & 2.91 & 0.72 & 1.16 (281) \\
12 & 3.03 & 0.65 & 1.02 (156) & 2.91 & 0.78 & 0.97 (271) \\
13 & 2.97 & 0.76 & 1.39 (181) & 2.72 & 0.75 & 0.96 (217) \\
\bottomrule
\end{tabular}

\end{table}

Table~\ref{tab:spec} confirms that our region selections are practically free
of thermal line emission, as already suggested by visual inspection
(Figure~\ref{fig:snr}).  Our exclusion of all $1.7$--$2
\unit{keV}$ photons should further limit any thermal contamination, as
the $1.85 \unit{keV}$ Si line emission is over a third of Tycho's thermal flux
as detected by \Chandra \citep{hwang2002}.  We may safely proceed to model our
measured rim widths as being due entirely to nonthermal radiation
\note[Aaron]{horrid phrasing}.

% --------------------------
% FWHM measurement procedure
% --------------------------
\subsection{Filament width measurements}
\label{sec:fwhms}

% TODO keep mentioned number of regions updated!
% TODO regions 2,3; 6,7 should be stretched forward a little more
%      just to be more consistent (just from eyeballing...)
We obtained radial intensity profiles from $\abt 10$--$20\arcsec$ behind the
shock to $\abt 2$--$8\arcsec$ in front for each of our 13 regions in all five
energy bands.  To increase signal-to-noise, we integrate along the shock ($\abt
8$--$24\arcsec$) in each region.  \note[Aaron]{Box dimensions are all over the
place} Plotted and fitted profiles are reported in vignetting and
exposure-corrected \note[Aaron]{double check CIAO guides} intensity units;
error bars are computed from raw count data assuming Poisson statistics.
Intensity profiles peak sharply (within $\abt 2$--$3\arcsec$) behind the shock,
demarcating our thin rims, and then fall off until thermal emission from ejecta
picks up at Tycho's contact discontinuity \citep{warren2005}.

We fitted rim profiles, obtained by integrating intensity along the shock in
each region, to a piecewise two-exponential model:
\begin{equation} \label{eq:prof}
    h(x) =
    \begin{cases}
        A_u \exp \left(\frac{x_0 - x}{w_u}\right) + C_u, &x \geq x_0 \\
        A_d \exp \left(\frac{x - x_0}{w_d}\right) + C_d, &x < x_0
    \end{cases}
\end{equation}
where $h(x)$ is profile height and $x$ is radial distance from remnant center.
The rim model, which we emphasize is strictly empirical, has 6 free parameters:
$A_u, x_0, w_u, w_d, C_u$, and $C_d$ with $A_d = A_u + (C_u - C_d)$ enforcing
continuity at $x=x_0$. Our model is similar to that of \citet{bamba2003,
bamba2005-hist} and differs slightly from that of \citetalias{ressler2014}.
To fit only the nonthermal rim in each intensity profile, we selected the fit
domain for each profile as follows.  The downstream bound was set at the first
local data minimum downstream of the rim peak, as identified by smoothing the
profiles with a 21-point ($\abt 10\arcsec$) Hanning window.  The upstream bound
was set at the profile's outer edge (i.e., no data were removed).

\begin{figure*}[ht]
    \plotone{figures/f0-prfs.pdf}
    \caption{Profile fits for Region 1}
    \label{fig:profiles}
\end{figure*}

% TODO do all the work, but with different FWHM calculation -- to confirm
%      that it doesn't make a big difference...
% TODO subtract $\min(C_u, C_d)$ from profiles before getting FWHM, then
%      come back and edit the text.
% TODO keep region numbers here updated!
From the fitted profiles we extracted a full width at half maximum (FWHM) for
each region and each energy band.  We could not resolve a FWHM in regions 2, 6,
8, 9, and 11 at 0.7--1 keV (Table~\ref{tab:fwhms}); in these regions, either
the downstream FWHM bound would extend outside the fit domain or we could not
find an acceptable fit to equation~\eqref{eq:prof}.  We were able to resolve
FWHMs for all regions at higher energy bands (1--7 keV).

To estimate FWHM uncertainties, we horizontally stretched each best-fit
profile by mapping radial coordinate $x$ to
$x'(x) = x (1 + \xi (x-x_0)/(50\arcsec-x_0))$ with $\xi$ an arbitrary stretching
parameter and $x_0$ the best-fit rim center from equation~\eqref{eq:prof};
this yields a new profile $h'(x) = h(x'(x))$.
We varied $\xi$ (and hence rim FWHM) to vary each profile fit $\chi^2$ by 2.7
and the stretched or compressed FWHMs as upper/lower bounds on our reported
FWHMs.

% -------------
% Model fitting
% -------------
\subsection{Filament model fitting}
\label{sec:fits}

% Fit data to models
We fitted the two transport models given by Equations~\eqref{eq:simp-mod} and
\eqref{eq:full-mod} to our measured rim widths as a function of energy.
Although our profiles are summed counts from a continuous energy range, we
assigned each width measurement to the lower energy limit of its band
(e.g., $0.7$--$1 \unit{keV}$ is assigned to $0.7 \unit{keV}$ and fitted to
model profile widths at $0.7 \unit{keV}$). \note[Aaron]{bad wording}
We also average the positive and negative error bars on each FWHM measurement
(Table~\ref{tab:fwhms}) for least squares fitting.
% Averaging/merging/synthesizing, somehow, our results
Rim widths from each region are fitted separately; we consider filament and
remnant-wide physical parameters by averaging best fit parameters for each
region in Section~\ref{sec:fit-results}.

% Model knobs and how we twiddled them
We have three parameters to vary (more if we consider magnetic damping).
They are diffusion-energy scaling exponent $\mu$, normalized diffusion
coefficient $\eta_2$, and magnetic field strength $B_0$.
Explain why we are fixing $\mu$ (forward reference best fit figures).
We consider $\mu = 0, 1/3, 1/2, 1, 3/2, 2$; recall that $\mu = 1/3$ and $1/2$
respectively correspond to predictions from Kolmogorov and Kraichnan turbulent
energy spectra.  Explain why we also give results with $\eta_2$ fixed at $1$.

A few remnant-specific parameters enter into the model calculations.  We take
electron spectral index $s = 2.3$ (from radio spectral index $\alpha = 0.65$,
\citet{kothes2006} \note[Aaron]{not using $0.58$ from Green's catalog?!}),
remnant distance $3 \unit{kpc}$ \citep[cf.][]{hayato2010}, and
shock radius $1.08 \times 10^{19} \unit{cm}$ from angular radius $240\arcsec$
\citep{green2009}.  Tycho's forward shock velocity varies with azimuth by up to
a factor of 2; we interpolate the data of \citet{williams2013} to estimate
individual shock velocities for each of our regions.
\note[Aaron]{this might belong in Section~\ref{sec:transport}?}

% How to do the fit
The simple model is readily fit from equation~\eqref{eq:simp-width}.
The full model is numerically computed following the procedure laid out in
\emph{section ?!}, yielding intensity profiles from equation~\eqref{eq:intensity} for each
energy band and hence a model full width at half maximum.
For least squares fitting purposes, the full model is treated as a black box
that takes in physical parameters
% TODO modify tables to match what is stated here -- in regards to parameters
% being effectively unbounded.
% TODO for extreme values in tables -- go back and double check chi^2 space
To perform full model fits, we tabulated a large grid for fixed values of $\mu$
and shock velocity $v_s$.  If any fit runs to $\eta_2 = 10^5$ or $B_0 = 10
\unit{mG}$, we deem it effectively unconstrained (only a few fits do so, and
only to the $\eta_2$ limit).  Note that we do not allow / consider fits with
$\eta_2$ negative (may be trivial, just note parameter bounds).  As the full
continuous loss model must be numerically solved, its predicted rim widths are
subject to resolution error in the numerical integrals.  We chose our
integration resolutions such that the fractional error associated with
halving/doubling our resolution is less than $1\%$ for the parameter space
relevant to our filaments. \note[Aaron]{rewrite -- unclear.  Need to verify
1\% claim and check Pacholczyk table resolution (and possibly Bessel
function numbers)...}

% Errors
Finally, we computed errors on our best fit parameters by varying each
parameter and obtaining a new best fit model, with one less degree of freedom,
to find the limit s.t. $\Delta \chi^2 = 1$ to obtain a roughly 1-sigma error.
Granted, this is not guaranteed correct for nonlinear fits (that $\Delta \chi^2
= 1$ corresponds to a $68.3$\% confidence limit), but should be good enough.

% =======================
% Results, FWHMs and fits
% =======================
\section{Results}

% --------------------
% FWHM results, tables
% --------------------
\subsection{Rim widths}
\label{sec:fwhm-results}

In Table~\ref{tab:fwhms} we report FWHM measurements for all of our regions.
As previously noted, we could not resolve FWHMs for some regions in the 0.7--1 keV band.
NOTE very low counts/numbers of regions for some averages (typically 1--2, see
Figure~\ref{fig:snr}).

Mention the number of counts in individual profiles?
May affect whether we go back and take smaller regions, or not.

% TODO give individual regions' FWHMs and m_E values
\begin{table*}[ht]
    \scriptsize
    \centering
    \caption{Averaged filaments NO UPDATE THIS\label{tab:fwhms}}
    \begin{tabular}{@{}rr@{ $\pm$ }lr@{ $\pm$ }lr@{ $\pm$ }lr@{ $\pm$ }lr@{ $\pm$ }l@{}}
\toprule
{} & \multicolumn{10}{c}{Averaged filament FWHMs (arcseconds), arithmetic mean} \\
\cmidrule(l){2-11}
 & \multicolumn{2}{c}{0.7-1kev} & \multicolumn{2}{c}{1-2kev} & \multicolumn{2}{c}{2-3kev} & \multicolumn{2}{c}{3-4.5kev} & \multicolumn{2}{c}{4.5-7kev}\\
\midrule
Filament 1, mean vs=5.09e+08 cm/s & $3.75$ & $0.80$ & $2.83$ & $0.38$ & $2.85$ & $0.28$ & $2.61$ & $0.45$ & $2.22$ & $0.20$\\
$m_E$ & {} & {} & $-0.79$ & $0.56$ & $0.01$ & $0.00$ & $-0.22$ & $0.11$ & $-0.39$ & $0.19$\\
Filament 2, mean vs=4.94e+08 cm/s & $3.69$ & $1.27$ & $3.42$ & $0.90$ & $2.34$ & $0.08$ & $2.19$ & $0.16$ & $2.14$ & $0.27$\\
$m_E$ & {} & {} & $-0.21$ & $0.26$ & $-0.54$ & $0.21$ & $-0.17$ & $0.03$ & $-0.05$ & $0.02$\\
Filament 3, mean vs=4.53e+08 cm/s & $2.10$ & $0.47$ & $1.90$ & $0.22$ & $1.79$ & $0.28$ & $1.77$ & $0.30$ & $1.76$ & $0.39$\\
$m_E$ & {} & {} & $-0.28$ & $0.19$ & $-0.08$ & $0.02$ & $-0.04$ & $0.02$ & $-0.01$ & $0.01$\\
Filament 4, mean vs=4.65e+08 cm/s & $5.36$ & $1.81$ & $3.59$ & $1.28$ & $2.83$ & $0.73$ & $2.40$ & $0.71$ & $1.87$ & $0.13$\\
$m_E$ & {} & {} & $-1.13$ & $1.55$ & $-0.34$ & $0.22$ & $-0.41$ & $0.40$ & $-0.61$ & $0.46$\\
Filament 5, mean vs=5.18e+08 cm/s & {} & {} & $6.07$ & $2.80$ & $4.47$ & $2.43$ & $5.16$ & $2.35$ & $4.12$ & $1.64$\\
$m_E$ & {} & {} & {} & {} & $-0.44$ & $0.45$ & $0.35$ & $0.62$ & $-0.55$ & $0.82$\\
\bottomrule
\end{tabular}

    \begin{tabular}{@{}rr@{ $\pm$ }lr@{ $\pm$ }lr@{ $\pm$ }lr@{ $\pm$ }lr@{ $\pm$ }l@{}}
\toprule
{} & \multicolumn{10}{c}{Averaged filament FWHMs (arcseconds), geometric mean} \\
\cmidrule(l){2-11}
 & \multicolumn{2}{c}{0.7-1kev} & \multicolumn{2}{c}{1-2kev} & \multicolumn{2}{c}{2-3kev} & \multicolumn{2}{c}{3-4.5kev} & \multicolumn{2}{c}{4.5-7kev}\\
\midrule
Filament 1, mean vs=5.09e+08 cm/s & $3.54$ & $0.74$ & $2.74$ & $0.35$ & $2.80$ & $0.28$ & $2.48$ & $0.41$ & $2.19$ & $0.20$\\
$m_E$ & {} & {} & $-0.72$ & $0.49$ & $0.03$ & $0.01$ & $-0.30$ & $0.14$ & $-0.30$ & $0.14$\\
Filament 2, mean vs=4.94e+08 cm/s & $3.69$ & $1.70$ & $3.30$ & $1.02$ & $2.34$ & $0.08$ & $2.18$ & $0.17$ & $2.12$ & $0.29$\\
$m_E$ & {} & {} & $-0.31$ & $0.49$ & $-0.49$ & $0.22$ & $-0.17$ & $0.04$ & $-0.07$ & $0.03$\\
Filament 3, mean vs=4.53e+08 cm/s & $2.05$ & $0.53$ & $1.89$ & $0.23$ & $1.77$ & $0.31$ & $1.74$ & $0.33$ & $1.72$ & $0.43$\\
$m_E$ & {} & {} & $-0.22$ & $0.18$ & $-0.09$ & $0.03$ & $-0.04$ & $0.03$ & $-0.04$ & $0.03$\\
Filament 4, mean vs=4.65e+08 cm/s & $5.36$ & $3.72$ & $3.35$ & $1.51$ & $2.73$ & $0.83$ & $2.29$ & $0.81$ & $1.86$ & $0.13$\\
$m_E$ & {} & {} & $-1.32$ & $3.06$ & $-0.29$ & $0.23$ & $-0.44$ & $0.51$ & $-0.51$ & $0.45$\\
Filament 5, mean vs=5.18e+08 cm/s & {} & {} & $5.39$ & $3.48$ & $3.75$ & $3.15$ & $4.59$ & $2.92$ & $3.79$ & $1.98$\\
$m_E$ & {} & {} & {} & {} & $-0.52$ & $0.80$ & $0.50$ & $1.30$ & $-0.48$ & $0.97$\\
\bottomrule
\end{tabular}

    \tablecomments{Errors computed as standard errors of mean, see text}
\end{table*}


% -------------------------
% Model fit results, tables
% -------------------------
\subsection{Model fit results}
\label{sec:fit-results}

% TODO keep updated -- give tables for ONE region with both simple, full fits
We find that the simpler catastrophic dump model (equation~\eqref{eq:simp-mod})
produces the same qualitative trends seen in the full continuous energy loss
model.
The first row of Table~\ref{tab:fits} compares simple and full model fits for
Region 1(?).
The key differences are \emph{blah}.  The errors are comparable, the parameters
are comparable but shifted by X amount; this is consistent in all filaments?
(details tbd).  Consequently, we present only the full model fits for all other
filaments in Table~\ref{tab:fits}.  However, the simple model is useful for
building intuition and we will refer to its results in the following discussion
(?); \citetalias{ressler2014} also give examples of simple model fits in SN
1006 \note[Aaron]{sentence maybe unnecessary}.

% Individual region fit results
Table~\ref{tab:fits} presents full model fit results for all regions
considered. Errors are computed by varying free parameters $B_0$,
$\eta_2$ to obtain $\Delta\chi^2 = 1$, corresponding to $1$-$\sigma$ confidence
limits on our fit parameters.

HERE we show two examples of fits -- one where the models just happen to fit
the data very nicely, and one where the models are way off (enormous $\chi^2$
and/or errors too small).  We specifically choose to highlight the most extreme
cases, to emphasize that our fits are hit-and-miss in practice.
We also show the simple model fits here to illustrate the similar qualitative
behavior between simple/full model fits.

% TODO Assemble table of regions
\begin{table*}[ht]
    \scriptsize
    \centering
    \caption{Full model best fits for individual regions, Filament 1.
    \label{tab:fits}}
    %\renewcommand{\arraystretch}{1.5}
\begin{tabular}{@{}rllr llr llr@{}}

\toprule
{} & \multicolumn{3}{c}{Region 1}
   & \multicolumn{3}{c}{Region 10}
   & \multicolumn{3}{c}{Region 11\tablenotemark{a}} \\
\cmidrule(lr){2-4} \cmidrule(lr){5-7} \cmidrule(lr){8-10}
$\mu$ (-) & $\eta_2$ (-) & $B_0$ ($\mu$G) & $\chi^2$
          & $\eta_2$ (-) & $B_0$ ($\mu$G) & $\chi^2$
          & $\eta_2$ (-) & $B_0$ ($\mu$G) & $\chi^2$ \\

\midrule
0.00 & ${19.1}^{\,+19}_{\,-7.1}$ & ${784}^{\,+120}_{\,-72}$ & 57.5590
     & ${19.5}^{\,+16}_{\,-6.2}$ & ${515}^{\,+71}_{\,-40}$ & 111.9722
     & $23^{\,+170}_{\,-170}$ & $670^{\,+1000}_{\,-1000}$ & 62.2861\\[1.5pt]
0.33 & ${56}^{\,+75}_{\,-29}$ & $930^{\,+190}_{\,-140}$ & 34.6006
     & ${58}^{\,+62}_{\,-29}$ & ${614}^{\,+110}_{\,-86}$ & 81.5166
     & $72^{\,+620}_{\,-620}$ & $808^{\,+1500}_{\,-1500}$ & 43.5822\\[1.5pt]
0.50 & ${92}^{\,+160}_{\,-54}$ & ${1010}^{\,+260}_{\,-180}$ & 25.5737
     & ${94}^{\,+140}_{\,-54}$ & $660^{\,+150}_{\,-120}$ & 70.4934
     & $130^{\,+1040}_{\,-1040}$ & $900^{\,+1600}_{\,-1600}$ & 37.8298\\[1.5pt]
1.00 & ${370}^{\,+3300}_{\,-310}$ & ${1260}^{\,+4500}_{\,-440}$ & 9.7750
     & ${68}^{\,+8000}_{\,-47}$ & $560^{\,+1100}_{\,-120}$ & 56.4783
     & $16000^{\,+21000}_{\,-21000}$ & $2450^{\,+670}_{\,-670}$ & 32.7315\\[1.5pt]
1.50 & ${23}^{\,+32}_{\,-10}$ & ${623}^{\,+130}_{\,-67}$ & 8.7937
     & ${10.8}^{\,+5.3}_{\,-3.1}$ & ${354}^{\,+27}_{\,-19}$ & 55.6460
     & ${6.9}^{\,+2.8}_{\,-2.8}$ & ${420}^{\,+28}_{\,-28}$ & 41.9271\\[1.5pt]
2.00 & ${11.4}^{\,+5.2}_{\,-3.3}$ & ${512}^{\,+34}_{\,-26}$ & 9.5862
     & ${7.6}^{\,+2.2}_{\,-1.6}$ & ${315}^{\,+12}_{\,-10}$ & 56.1258
     & ${4.9}^{\,+1.4}_{\,-1.4}$ & ${380}^{\,+15}_{\,-15}$ & 47.6997\\

\midrule
{} & \multicolumn{3}{c}{Region 12\tablenotemark{a}}
   & \multicolumn{3}{c}{Region 13\tablenotemark{a}} \\
\cmidrule(lr){2-4} \cmidrule(lr){5-7}
$\mu$ (-) & $\eta_2$ (-) & $B_0$ ($\mu$G) & $\chi^2$
          & $\eta_2$ (-) & $B_0$ ($\mu$G) & $\chi^2$ \\

\cmidrule(lr){1-7}
0.00 & $22^{\,+199}_{\,-199}$ & $762^{\,+1600}_{\,-1600}$ & 150.4259
     & $22^{\,+320}_{\,-320}$ & $780^{\,+2400}_{\,-2400}$ & 76.9297\\[1.5pt]
0.33 & $59^{\,+967}_{\,-967}$ & $891^{\,+3200}_{\,-3200}$ & 116.8732
     & $59^{\,+1400}_{\,-1400}$ & $900^{\,+4900}_{\,-4900}$ & 51.3875\\[1.5pt]
0.50 & $96^{\,+1800}_{\,-1800}$ & $963^{\,+4100}_{\,-4100}$ & 104.0160
     & $92^{\,+2600}_{\,-2600}$ & $965^{\,+6200}_{\,-6200}$ & 40.3597\\[1.5pt]
1.00 & $498^{\,+6600}_{\,-6600}$ & $1280^{\,+4000}_{\,-4000}$ & 83.815
     & $346^{\,+11700}_{\,-11700}$ & $1190^{\,+9300}_{\,-9300}$ & 15.7162\\[1.5pt]
1.50 & $16.9^{\,+8.4}_{\,-8.4}$ & $554^{\,+53}_{\,-53}$ & 84.7895
     & $716^{\,+10700}_{\,-10700}$ & $1280^{\,+4500}_{\,-4500}$ & 5.6257\\[1.5pt]
2.00 & $9.7^{\,+2.7}_{\,-2.7}$ & $472^{\,+21}_{\,-21}$ & 87.3368
     & $31^{\,+19}_{\,-19}$ & $581^{\,+68}_{\,-68}$ & 5.0051\\

\bottomrule
\end{tabular} 
\tablenotetext{1}{Reported errors are fit standard errors, not reliable!}
  % TODO CHANGE/UPDATE THIS
\end{table*}

In Figure~\ref{fig:fits} we plot the full model fits for the same regions
tabulated in Table~\ref{tab:fits}

% TODO Assemble all the individual plots.
\begin{figure}[ht]
    \centering
    \plottwo{figures/f0-fits-avg-flmt1.png}{figures/f0-fits-avg-flmt2.png} \\
    \plottwo{figures/f0-fits-avg-flmt3.png}{figures/f0-fits-avg-flmt4.png} \\
    \plottwo{figures/f0-fits-avg-flmt5.png}{figures/f0-box.png}
    \plottwo{figures/f0-fits-avg-flmt1-geom.png}{figures/f0-fits-avg-flmt2-geom.png} \\
    \plottwo{figures/f0-fits-avg-flmt3-geom.png}{figures/f0-fits-avg-flmt4-geom.png} \\
    \plottwo{figures/f0-fits-avg-flmt5-geom.png}{figures/f0-box.png}
    \caption{Plots of best-fits to averaged FWHMs (arithmetic, geometric avg).
    \note[Aaron]{Need better subplot display in final version, if we keep these}}
    \label{fig:fits-all}
\end{figure}

% Individual region fits with eta2=1 fixed
We also computed best fits while holding $\eta_2 = 1$; i.e., requiring Bohm
diffusion at $2 \unit{keV}$.  We found that the best fit $B_0$ is relatively
insensitive to $\mu$; for all regions, $B_0$ varies by no more than XXX ($\abt 10$)\% from
the value at $\mu = 1$ (with larger $B_0$ at smaller $\mu$ and vice versa).
\note[Aaron]{compute/verify this}.
Therefore, we show only best fits with $\mu = 1$ and $\eta_2 = 1$ fixed in
Table~\ref{tab:fits-avg-eta2-fix}).

\begin{table*}
    \scriptsize
    \centering
    \caption{Best fits with $\eta_2 = 1$ fixed, for filament averaged FWHMs.
    \label{tab:fits-avg-eta2-fix}}
    % Best fits with eta2 = 1 fixed, for arithmetic average of FWHMs...
\begin{tabular}{@{}rllr llr llr@{}}

\toprule
{} & \multicolumn{3}{c}{Filament 1}
   & \multicolumn{3}{c}{Filament 2}
   & \multicolumn{3}{c}{Filament 3} \\
\cmidrule(lr){2-4} \cmidrule(lr){5-7} \cmidrule(lr){8-10}
$\mu$ (-) & $\eta_2$ (-) & $B_0$ ($\mu$G) & $\chi^2$
          & $\eta_2$ (-) & $B_0$ ($\mu$G) & $\chi^2$
          & $\eta_2$ (-) & $B_0$ ($\mu$G) & $\chi^2$ \\

\midrule
0.00 & 1 & $363 \pm 14.0$ & 6.7047
     & 1 & $388 \pm 8$ & 3.5110
     & 1 & $489 \pm 24$ & 5.9779\\
0.33 & 1 & $351 \pm 13.0$ & 5.6331
     & 1 & $375 \pm 7$ & 2.8832
     & 1 & $468 \pm 23$ & 5.2188\\
0.50 & 1 & $346 \pm 13.0$ & 5.2477
     & 1 & $369 \pm 7$ & 2.6260
     & 1 & $467 \pm 23$ & 4.9331\\
1.00 & 1 & $334 \pm 12.0$ & 4.5444
     & 1 & $355 \pm 7$ & 2.0359
     & 1 & $439 \pm 22$ & 4.3860\\
1.50 & 1 & $325 \pm 12.0$ & 4.3048
     & 1 & $344 \pm 7$ & 1.6590
     & 1 & $424 \pm 21$ & 4.1910\\
2.00 & 1 & $319 \pm 12.0$ & 4.3274
     & 1 & $336 \pm 7$ & 1.4397
     & 1 & $415 \pm 21$ & 4.2184\\

\midrule
{} & \multicolumn{3}{c}{Filament 4}
   & \multicolumn{3}{c}{Filament 5} \\
\cmidrule(lr){2-4} \cmidrule(lr){5-7}
$\mu$ (-) & $\eta_2$ (-) & $B_0$ ($\mu$G) & $\chi^2$
          & $\eta_2$ (-) & $B_0$ ($\mu$G) & $\chi^2$ \\

\cmidrule(lr){1-7}
0.00 & 1 & $358 \pm 15$ & 0.8029
     & 1 & $239 \pm 37$ & 0.3256\\
0.33 & 1 & $350 \pm 15$ & 1.0612
     & 1 & $231 \pm 36$ & 0.2762\\
0.50 & 1 & $347 \pm 15$ & 1.1809
     & 1 & $228 \pm 36$ & 0.2578\\
1.00 & 1 & $338 \pm 15$ & 1.4792
     & 1 & $220 \pm 34$ & 0.2214\\
1.50 & 1 & $331 \pm 14$ & 1.6770
     & 1 & $215 \pm 34$ & 0.2046\\
2.00 & 1 & $325 \pm 14$ & 1.7851
     & 1 & $210 \pm 33$ & 0.1998\\

\bottomrule
\end{tabular} 
\tablecomments{With only one free parameter, it doesn't really make sense to
vary $B_0$ to get $\Delta\chi^2 = 1$.  Here I report fit standard
errors from the numerically estimated covariance matrix.}

\end{table*}

% Averaged fit results
In each filament, we averaged best fit parameters $B_0$, $\eta_2$ from each
constituent region to obtain filament-wide estimates on $B_0$ and $\eta_2$,
reported in Table~\ref{tab:fits-avg}.  Stated uncertainties are standard errors
of the mean \note[Aaron]{may need to scale by $\abt 2 \times$ to get 1-sigma
confidence interval since our sample sizes are so small}.  We note that both
averages and errors are strongly affected by outliers from extreme best fit
values, as can be seen in Table~\ref{tab:fits-all} -- especially as the typical
number of regions per filament (i.e., our sample size) is quite small.

\begin{table*}
    \scriptsize
    \centering
    \caption{Filament-wide average of best-fit parameters for constituent regions.
    \label{tab:fits-avg}}
    % Average of best-fit parameters for individual regions, for each filament
\begin{tabular}{@{}rllr llr llr@{}}

\toprule
{} & \multicolumn{3}{c}{Filament 1}
   & \multicolumn{3}{c}{Filament 2}
   & \multicolumn{3}{c}{Filament 3} \\
\cmidrule(lr){2-4} \cmidrule(lr){5-7} \cmidrule(lr){8-10}
$\mu$ (-) & $\eta_2$ (-) & $B_0$ ($\mu$G) & $\chi^2$
          & $\eta_2$ (-) & $B_0$ ($\mu$G) & $\chi^2$
          & $\eta_2$ (-) & $B_0$ ($\mu$G) & $\chi^2$ \\

\midrule
0.00 & $21.1 \pm 0.7$ & $702 \pm 51$ & -
     & - & - & -
     & - & - & - \\
0.33 & $60.8 \pm 2.9$ & $829 \pm 57$ & -
     & - & - & -
     & - & - & - \\
0.50 & $100.8 \pm 7.3$ & $900. \pm 62$ & -
     & - & - & -
     & - & - & - \\
1.00 & $\left(3460 \pm 3140\right)$\tablenotemark{a} & $1348 \pm 306$ & -
     & - & - & -
     & - & - & - \\
1.50 & $\left(155 \pm 140\right)$\tablenotemark{a} & $646 \pm 165$ & -
     & - & - & -
     & - & - & - \\
2.00 & $12.9 \pm 4.6$ & $452 \pm 47$ & -
     & - & - & -
     & - & - & - \\

\midrule
{} & \multicolumn{3}{c}{Filament 4}
   & \multicolumn{3}{c}{Filament 5} \\
\cmidrule(lr){2-4} \cmidrule(lr){5-7}
$\mu$ (-) & $\eta_2$ (-) & $B_0$ ($\mu$G) & $\chi^2$
          & $\eta_2$ (-) & $B_0$ ($\mu$G) & $\chi^2$ \\

\cmidrule(lr){1-7}
0.00 & - & - & -
     & - & - & - \\
0.33 & - & - & -
     & - & - & - \\
0.50 & - & - & -
     & - & - & - \\
1.00 & - & - & -
     & - & - & - \\
1.50 & - & - & -
     & - & - & - \\
2.00 & - & - & -
     & - & - & - \\

\bottomrule
\end{tabular} 
\tablenotetext{1}{Here I included extreme outliers ($\eta_2 = 16000$ for Region
11, $\mu = 1$, other regions have $\eta_2 < 500$; $\eta_2 = 716$ for Region 13,
$\mu = 1.5$, other regions have $\eta_2 < 25$.  With 5 different regions there
can be some extreme scatter (for 2 regions we can barely even tell).  But,
qualitatively, best-fit parameters for Filament 1 regions appear to have
similar values (see stated errors, which are std error of mean
computed using sample std dev.).}

\end{table*}

% Global fits...
Finally, we report a global average of best fit parameters from all regions.
We could toss this into Table~\ref{tab:fits-avg}.  Reported errors are
again standard errors of the mean (not CI, use CI? I dunno), which gives an
idea of the spread of our measurements.

Plot of best fit parameters with $\mu=1$ as a function of azimuth angle, with
error bars or something similar?

% ==========
% Discussion
% ==========
\section{Discussion}

\subsection{How confidently can we reject B-damping?}

Everything hinges on our error bars -- this is most important.  Our rims do
thin consistently, in most regions (might be one or two places where they
don't), but (1) the errors in our individual FWHMs are larger than I'm
currently reporting because different profile fits will give a larger spread of
FWHMs (might need to consider reduced chi-squared here, but using chi-sqr to
select profile fit models is not physically meaningful)

It would be helpful to average $m_E$ values together, under the
assumption that they should be somewhat consistent within a given filament.
Then report the values and errors observed.

Recall that our FWHMs (Table~\ref{tab:fwhms}) have a large range of values,
IF magnetic damping is relevant in controlling filament widths, we might, e.g.,
expect a maximum filament width at which $m_E$ sharply levels off, being now
restricted by the magnetic field turn off.  We simply do not observe this in
our data between $0.7$--$7 \unit{keV}$.  We must conclude that, as in SN 1006
\citepalias{ressler2014}, magnetic damping cannot be relevant to our
observations.

What is the smallest damping scale length that is (roughly) consistent with all
of our observations?

\subsection{Interpretation of fit parameters}

Magnetic field $B_0$: despite the large range of fit quality, we can give
fairly robust lower bounds.  Give lowest bound?  Consistent with previous
studies, Tycho requires strong magnetic field amplification -- up to ??? times
the expected value of $\abt 10 \unit{\mu G}$ from a strong adiabatic shock with
compression ratio $r=4$ and typical galactic magnetic field strength of $\abt
2$--$3 \unit{\mu G}$ \citep{lyne1989, han2006}.

Diffusion coefficient $\eta_2$, on the other hand, is a bit of a mess.

Can we favor a value of $\mu$ (and hence a turbulent energy spectrum)?

\subsection{Bohm diffusion}

Fixing $\mu = 1$ and $\eta_2 = 1$ in Table~\ref{tab:fits-avg} is equivalent to
assuming Bohm diffusion, as most previous studies have done.
We should find/show that our result is very similar to theirs, in terms of
$B_0$ estimates, and this may be the easiest way to back out any estimates of
azimuthal magnetic field variation.

Magnetic field amplification numbers -- how do they compare with previous
studies?  What is the azimuthal variation (and significance thereof)?

Diffusion -- sub-Bohm diffusion in Filament 4?

\subsection{Other things}

Precursors, rim position variation (none obvious from preliminary look; see
Figure~\ref{fig:peak-pos}), steepness of the filament rise?

\begin{figure}
    \centering
    \plotone{figures/f0-peak-pos.pdf}
    \caption{Best-fit rim peak positions ($x_0$ in equation~\eqref{eq:prof})
        for all regions as a function of energy band, normalized to the $2
        \unit{keV}$ peak position.  Red line plots best linear fit to all data
        with slope of $-0.025 \unit{arcsec/keV}$.
        \label{fig:peak-pos}}
\end{figure}

% ==========
% Conclusion
% ==========
\section{Conclusions}

In conclusion, crazy B field amplification is not that weird.
Magnetic damping can be ruled out and our result is robust throughout the remnant.

% ================
% Acknowledgements
% ================
\acknowledgments

The scientific results reported in this article are based on data obtained from
the \Chandra Data Archive.
This research has made use of NASA's Astrophysics Data System.

{\it Facilities:} \facility{CXO (ACIS-I)}

\clearpage

% ========
% Appendix
% ========
\appendix

\setcounter{table}{0}
\renewcommand{\thetable}{A\arabic{table}}
\setcounter{figure}{0}
\renewcommand{\thefigure}{A\arabic{figure}}

% --------------
% SN 1006 tables
% --------------
\section{Full model validation, SN 1006 (DRAFT ONLY)}

For comparison to \citetalias{ressler2014}, we reproduce Sean's table
(Table~\ref{tab:sean} and present full model fits with 2 and 3 energy bands
(Tables~\ref{tab:sn1006-2band} and \ref{tab:sn1006-3band}, respectively).
We don't compare simple model results, as they are identical (only the error
calculations differ).
The procedure of Tables~\ref{tab:sean},~\ref{tab:sn1006-2band} are not quite
the same.  Sean used $m_E$ at 2 keV instead of the $1$--$2$ keV width for
fitting, whereas I used both widths.  But, it should not matter much as
the fit often has $\chi^2 \sim 0$, either way (depending on the exact
measurements and values of $\mu$).

Please take the errors with a grain of salt.  They are automatically
generated and \textbf{have not been manually validated}.  Some values may be
invalid, where my error-finding code failed and gave a best, conservative guess
of the error.

\begin{table}[ht]
    \tiny
    \centering
    \caption{Sean's SN 1006 best fit parameters \citepalias[Table 8]{ressler2014}.
    \label{tab:sean}}
    \begin{tabular}{@{} l c c c c c c @{}}
\toprule
{}&\multicolumn{2}{c}{Filament 1} & \multicolumn{2}{c}{Filament 2} & \multicolumn{2}{c}{Filament 3} \\
\midrule
$\mu$ &$\eta_{2}$ & $B_{0}$ &$\eta_{2}$ & $B_{0}$ & $\eta_{2}$ & $B_{0}$ \\
0 & 7.5 $\pm$ 2 & 142 $\pm$ 5 & - & - & $\lesssim$ 0.1& 77 $\pm$ .8\\
1/3 & 4 $\pm$ 1.3 & 120 $\pm$ 5 & - & - & $\lesssim$ 0.1 & 76 $\pm$ 1.4 \\
1/2 & 3 $\pm$ 1.1 & 112 $\pm$ 4 & - & - & $\lesssim$ 0.1 & 75 $\pm$ 1.0 \\
1 & 2 $\pm$ 1.0 & 100 $\pm$ 3 & 22 $\pm$ 3 & 214 $\pm$ 4 & $\lesssim$ 0.1 & 74 $\pm$ 1.1 \\
1.5 & 1.9 $\pm$ 1.2 & 95 $\pm$ 3 & 9 $\pm$ 1.2 & 167 $\pm$ 4 & $\lesssim$ 0.1 & 74 $\pm$ 1.2   \\
2 & 2 $\pm$ 1.0 & 92 $\pm$ 4  & 7 $\pm$ 1.1 & 152 $\pm$ 4 & $\lesssim$ 0.1 & 73 $\pm$ 1.2 \\
\midrule
& &\multicolumn{2}{c}{Filament 4} && \multicolumn{2}{c}{Filament 5} \\
\midrule
$\mu$ &&$\eta_{2}$ & $B_{0}$ &&$\eta_{2}$ & $B_{0}$\\
0 &&  $\lesssim$ 0.2 & 113 $\pm$ 2 && -& -\\
1/3 &&  $\lesssim$ 0.2 & 112 $\pm$2 && -& -\\
1/2 && $\lesssim$ 0.2  & 111 $\pm$2 && - & -\\
1 &&  $\lesssim$ 0.2 & 109 $\pm$ 2 && 80 $^{+\infty}_{-4}$   & 206 $\pm$ 3\\
1.5 && $\lesssim$ 0.2 & 108 $\pm$ 2 && 19 $\pm$ 2  & 140 $\pm$ 2\\
2 && $\lesssim$ 0.2 & 107 $\pm$ 2 && 12 $\pm$ 1.0  & 120 $\pm$ 2\\
\bottomrule
\end{tabular}

\end{table}

\begin{table*}[ht]
    \tiny
    \centering
    \caption{SN 1006 best fit parameters, 2 highest energy bands (full model).
    \label{tab:sn1006-2band}}
    %\renewcommand{\arraystretch}{1.5}
\begin{tabular}{@{}rllr llr llr @{}}

\toprule
{} & \multicolumn{3}{c}{Filament 1} & \multicolumn{3}{c}{Filament 2} &
     \multicolumn{3}{c}{Filament 3} \\
\cmidrule(lr){2-4}
\cmidrule(lr){5-7}
\cmidrule(lr){8-10}
$\mu$ (-) & $\eta_2$ (-) & $B_0$ ($\mu$G) & $\chi^2$
          & $\eta_2$ (-) & $B_0$ ($\mu$G) & $\chi^2$
          & $\eta_2$ (-) & $B_0$ ($\mu$G) & $\chi^2$ \\

\midrule
0.00 & ${21}^{\,+290}_{\,-20}$ & ${176}^{\,+108}_{\,-80}$ & 0.0086
     & ${21}^{\,+51}_{\,-14}$ & ${262}^{\,+77}_{\,-53}$ & 7.2668
     & ${0.004}^{\,+6800}_{\,-0.004}$ & ${74.75}^{\,+183}_{\,-0.77}$ & 0.8180\\[1.5pt]
0.33 & ${3.4}^{\,+100000}_{\,-3}$ & ${117}^{\,+552}_{\,-24}$ & 0.0000
     & ${68}^{\,+282}_{\,-56}$ & ${316}^{\,+136}_{\,-98}$ & 2.5848
     & ${0.02}^{\,+0.11}_{\,-0.02}$ & ${74.45}^{\,+1.04}_{\,-0.61}$ & 0.8037\\[1.5pt]
0.50 & ${2.5}^{\,+100000}_{\,-2.2}$ & ${109}^{\,+635}_{\,-17}$ & 0.0000
     & ${106}^{\,+792}_{\,-93}$ & ${337}^{\,+204}_{\,-122}$ & 1.0919
     & ${0.024}^{\,+0.11}_{\,-0.024}$ & ${74.16}^{\,+0.87}_{\,-0.53}$ & 0.7946\\[1.5pt]
1.00 & ${1.8}^{\,+6.6}_{\,-1.4}$ & ${98.4}^{\,+23}_{\,-9.2}$ & 0.0000
     & ${21}^{\,+100000}_{\,-15}$ & ${213}^{\,+1160}_{\,-42}$ & 0.0000
     & ${0.012}^{\,+0.11}_{\,-0.012}$ & ${73.68}^{\,+0.21}_{\,-0.88}$ & 0.5364\\[1.5pt]
1.50 & ${1.7}^{\,+3.6}_{\,-1.3}$ & ${94}^{\,+12}_{\,-6.4}$ & 0.0000
     & ${8.5}^{\,+11}_{\,-4.1}$ & ${166}^{\,+24}_{\,-14}$ & 0.0000
     & ${0.007}^{\,+0.13}_{\,-0.007}$ &
     $\left({73.4}^{\,+0}_{\,-1.2}\right)$\tablenotemark{a} & 0.5062\\[1.5pt]
2.00 & ${1.9}^{\,+3.1}_{\,-1.4}$ & ${91.6}^{\,+7.8}_{\,-4.9}$ & 0.0000
     & ${7.0}^{\,+4.9}_{\,-2.8}$ & ${152.4}^{\,+11}_{\,-7.7}$ & 0.0000
     & ${0.026}^{\,+0.10}_{\,-0.026}$ & ${72.21}^{\,+0.23}_{\,-0.53}$ & 0.4443\\

\midrule
{} & \multicolumn{3}{c}{Filament 4} & \multicolumn{3}{c}{Filament 5} \\
\cmidrule(lr){2-4}
\cmidrule(lr){5-7}
$\mu$ (-) & $\eta_2$ (-) & $B_0$ ($\mu$G) & $\chi^2$
          & $\eta_2$ (-) & $B_0$ ($\mu$G) & $\chi^2$ \\

\cmidrule(lr){1-7}
0.00 & ${4300}^{\,+2500}_{\,-4300}$ & ${374.7}^{\,+3.7}_{\,-266}$ & 0.1896
     & ${23}^{\,+33}_{\,-13}$ & ${192}^{\,+40}_{\,-31}$ & 17.9802\\[1.5pt]
0.33 & ${0.011}^{\,+0.16}_{\,-0.011}$ & ${109.2}^{\,+2.1}_{\,-0.9}$ & 0.2605
     & ${68}^{\,+167}_{\,-48}$ & ${230}^{\,+72}_{\,-54}$ & 8.7407\\[1.5pt]
0.50 & ${0.0025}^{\,+0.17}_{\,-0.0025}$ & ${109.2}^{\,+1.2}_{\,-1.3}$ & 0.1959
     & ${107}^{\,+416}_{\,-82}$ & ${246}^{\,+104}_{\,-67}$ & 5.3185\\[1.5pt]
1.00 & ${0.041}^{\,+0.15}_{\,-0.041}$ & ${107.48}^{\,+1.2}_{\,-0.72}$ & 0.2759
     & ${447}^{\,+16000}_{\,-424}$ & ${311}^{\,+389}_{\,-151}$ & 0.1152\\[1.5pt]
1.50 & ${0.023}^{\,+0.17}_{\,-0.023}$ & ${106.8}^{\,+0.54}_{\,-0.41}$ & 0.1114
     & ${20}^{\,+49}_{\,-10}$ & ${142}^{\,+42}_{\,-18}$ & 0.0000\\[1.5pt]
2.00 & ${0.013}^{\,+0.18}_{\,-0.013}$ & ${106.57}^{\,+0.33}_{\,-1.4}$ & 0.0457
     & ${12.1}^{\,+9.2}_{\,-4.7}$ & ${121.6}^{\,+12}_{\,-8.5}$ & 0.0000\\

\bottomrule
\end{tabular}
\tablenotetext{1}{Seems unlikely that error is $0$, probably bad calculation}

\end{table*}

\begin{table*}[ht]
    \tiny
    \centering
    \caption{SN 1006 best fit parameters, 3 energy bands (full model).
    \label{tab:sn1006-3band}}
    %\renewcommand{\arraystretch}{1.5}
\begin{tabular}{@{}rllr llr llr @{}}

\toprule
{} & \multicolumn{3}{c}{Filament 1} & \multicolumn{3}{c}{Filament 2} &
     \multicolumn{3}{c}{Filament 3} \\
\cmidrule(lr){2-4}
\cmidrule(lr){5-7}
\cmidrule(lr){8-10}
$\mu$ (-) & $\eta_2$ (-) & $B_0$ ($\mu$G) & $\chi^2$
          & $\eta_2$ (-) & $B_0$ ($\mu$G) & $\chi^2$
          & $\eta_2$ (-) & $B_0$ ($\mu$G) & $\chi^2$ \\

\midrule
0.00 & $25^{\,+170}_{\,-23}$ & ${183}^{\,+89}_{\,-70}$ & 0.1162
     & $\left({0.00}^{\,+0.26}_{\,-0.00}\right)$\tablenotemark{a} & ${132.18}^{\,+320}_{\,-0.72}$ & 53.8334
     & ${0.012}^{\,+0.18}_{\,-0.012}$ & ${75.25}^{\,+180}_{\,-0.9}$ & 1.7278\\[1.5pt]
0.33 & $730^{\,+6100}_{\,-730}$ & $350^{\,+130}_{\,-130}$ & 0.0102
     & ${0.082}^{\,+0.23}_{\,-0.082}$ & ${132.5}^{\,+3.5}_{\,-1}$ & 53.4559
     & ${0.01}^{\,+0.18}_{\,-0.01}$ & ${74.86}^{\,+1.4}_{\,-0.65}$ & 1.7621\\[1.5pt]
0.50 & $3.9^{\,+210}_{\,-2.9}$ & ${116}^{\,+630}_{\,-18}$ & 0.0741
     & ${0.043}^{\,+0.31}_{\,-0.043}$ & ${131.56}^{\,+3.8}_{\,-0.62}$ & 53.3622
     & ${0.0056}^{\,+0.16}_{\,-0.0056}$ & ${74.61}^{\,+0.72}_{\,-0.48}$ & 1.5513\\[1.5pt]
1.00 & ${2.6}^{\,+5.4}_{\,-1.7}$ & ${102.5}^{\,+18}_{\,-8.9}$ & 0.1420
     & ${0.18}^{\,+0.27}_{\,-0.17}$ & ${130.5}^{\,+2.3}_{\,-1}$ & 52.4123
     & ${0.016}^{\,+0.18}_{\,-0.016}$ & ${73.85}^{\,+0.72}_{\,-0.54}$ & 1.7703\\[1.5pt]
1.50 & ${2.5}^{\,+3.2}_{\,-1.5}$ & ${97.1}^{\,+9.1}_{\,-5.9}$ & 0.2221
     & ${0.2}^{\,+0.45}_{\,-0.18}$ & ${129.06}^{\,+2.6}_{\,-0.82}$ & 51.8975
     & ${0.01}^{\,+0.16}_{\,-0.01}$ & ${73.47}^{\,+0.68}_{\,-0.9}$ & 1.4024\\[1.5pt]
2.00 & ${2.7}^{\,+2.9}_{\,-1.6}$ & ${94.2}^{\,+6.2}_{\,-4.6}$ & 0.3025
     & ${0.53}^{\,+0.34}_{\,-0.44}$ & ${128.9}^{\,+1.6}_{\,-1.7}$ & 50.3784
     & ${0.015}^{\,+0.17}_{\,-0.015}$ & ${73.2741}^{\,+0.0092}_{\,-1.1}$ & 1.6535\\

\midrule
{} & \multicolumn{3}{c}{Filament 4} & \multicolumn{3}{c}{Filament 5} \\
\cmidrule(lr){2-4}
\cmidrule(lr){5-7}
$\mu$ (-) & $\eta_2$ (-) & $B_0$ ($\mu$G) & $\chi^2$
          & $\eta_2$ (-) & $B_0$ ($\mu$G) & $\chi^2$ \\

\cmidrule(lr){1-7}
0.00 & - & - & -
     & ${23}^{\,+34}_{\,-13}$ & ${192}^{\,+40}_{\,-31}$ & 18.1971 \\[1.5pt]
0.33 & - & - & -
     & ${61}^{\,+190}_{\,-43}$ & ${223}^{\,+80}_{\,-53}$ & 9.7081 \\[1.5pt]
0.50 & - & - & -
     & ${101}^{\,+470}_{\,-80}$ & ${242}^{\,+114}_{\,-71}$ & 6.8545 \\[1.5pt]
1.00 & - & - & -
     & ${35}^{\,+45000}_{\,-25}$ & ${174}^{\,+680}_{\,-39}$ & 3.7991 \\[1.5pt]
1.50 & - & - & -
     & ${10.9}^{\,+12}_{\,-4.7}$ & ${127}^{\,+18}_{\,-11}$ & 3.2257 \\[1.5pt]
2.00 & - & - & -
     & ${8.8}^{\,+5.1}_{\,-3}$ & ${115.4}^{\,+8.5}_{\,-6.3}$ & 2.7391 \\

\bottomrule
\end{tabular}
\tablenotetext{1}{Best fit value is formally $6.6\times10^{-6}$, close enough.}

\end{table*}


% ------------------
% Sean's Cas A table
% ------------------
\section{Sean's Cas A comparison (DRAFT ONLY)}

This is dredged from the \LaTeX source for Sean's paper, available on arXiv.
I put this here just for my own reference.

A detailed application of our results to other SNRs such as Cas A will
require much more extensive analysis, but we can use the published
filament widths of Araya et al. (2010) for Cas A to get preliminary
estimates of the magnetic field strength and diffusion coefficient by
applying our model. In their data, it appears that the filaments in
Cas A shrink by a factor of $\sim 0.8$ between 0.3 and 3 keV, while the
filament widths appear to be energy-independent between 3 and 6 keV.
Qualitatively, this is consistent with the loss-limited model, as our
parameter $m_{\rm E}$ is predicted to decrease with energy.  For the
lower energy range of 0.3--3 keV, reproducing $m_{\rm E}\sim -0.1$
(equivalent to the factor of 0.8 drop in size) requires magnetic
fields on the order of 200-500 $\mu$G and diffusion coefficients about
$5 \times D_{\rm Bohm}$(3 keV), about an order of magnitude higher
than the values one obtains by neglecting the energy dependence. One
can also see directly from Figure~\textbf{?!} that $\mu < 1$ models
of the diffusion coefficient are excluded for $m_{\rm E}\sim -0.1$.

To quickly compare how our results will differ from those of other authors, we
applied the analytic approximation of Equation 22 and Equation 9 to the FWHMs
presented by Araya et al. (2010) for Cas A. Here we used $m_{\rm E} = \log(
{\rm FWHM}(3keV)/{\rm FWHM}(.3keV))/\log(10)$ and fit the filaments at a photon
energy of 3 keV.  Hence $\eta_3 \equiv D/D_{\rm Bohm}(3 keV)$. Using the same
parameters, our results are shown in Table~\ref{tab:araya}. To fit the energy
dependence of the FWHM (i.e. $m_{\rm E}$), we required higher diffusion
coefficients and, consequently, magnetic fields about an order of magnitude
higher than their results. We note here that values of $\mu<1$ were unable to
reproduce the data and that the diffusion coefficients cited by Araya et al.
(2010) are below the minimally allowed Bohm value ($\eta_3 = 1$).

We intended this comparison solely as a qualitative analysis, and thus did not
rigorously keep track of uncertainties or extend the comparison to the full
numerical calculation. The conclusion is clear, however, which is that
considering the measured value of $m_{\rm E}$ can have a dramatic effect.

\begin{table}[h]
\tiny
\centering
\caption{Best fit parameters for the filaments of Cas A based on data from
Araya et. al (2010) in varying values of $\mu$ (Approximate Analytic Results)
Dashes denote places where fits were unobtainable.
\label{tab:araya}}

\begin{tabular}{@{}c c c c c c c c c@{}}
\toprule
  &\multicolumn{2}{c}{$\mu = 1$} & \multicolumn{2}{c}{$\mu = 1.5$} &
    \multicolumn{2}{c}{$\mu = 2$} & \multicolumn{2}{c}{Araya et. al} \\
  & & & & & &  &\multicolumn{2}{c}{($\mu = 1$)} \\
\cmidrule(l){2-3}
\cmidrule(l){4-5}
\cmidrule(l){6-7}
\cmidrule(l){8-9}
Filament &$\eta_{3}$ & $B_{0}$ &$\eta_{3}$ & $B_{0}$ & $\eta_{3}$ & $B_{0}$ &
    $\eta_{3}$ & $B_{0}$ \\
\cmidrule(l){2-9}
1 & 6.5  & 750 $\mu$G & 2 & 567 $\mu$G & 1.1 & 506 $\mu$G & 0.12 & 72 $\mu$G \\
2 & 3.6 & 710 $\mu$G & 1.4 & 582 $\mu$G & - & - & 0.02 & 37 $\mu$G\\
3 & 3.2 &  710 $\mu$G& 1.3 & 589 $\mu$G& - & - & 0.02 & 53 $\mu$G\\
4 & 3.1& 515 $\mu$G & 1.3 & 430 $\mu$G & -& - & 0.02 & 40 $\mu$G \\
5 & 10.7 & 1163 $\mu$G & 2.5 & 809 $\mu$G & 1.3&  706 $\mu$G & 0.025 & 52 $\mu$G \\
6 & 5.7 & 844 $\mu$G & 1.9 & 650 $\mu$G  & 1 & 583 $\mu$G & 0.1 & 56 $\mu$G \\
7 & 8.3 & 872 $\mu$G & 2.2 & 635 $\mu$G& 1.2 & 560 $\mu$G & 0.15 & 66 $\mu$G\\
8 & 13 & 1010 $\mu$G & 2.7 & 738 $\mu$G & 1.4 & 639 $\mu$G & 0.02 & 35 $\mu$G\\
9 & 4.7 & 605 $\mu$G & 1.7 & 479 $\mu$G & - & -& 0.02 & 29 $\mu$G\\
\bottomrule
\end{tabular}

\end{table}


% ==========
% References
% ==========
\bibliographystyle{apj}  % AASTeX journal macros are supplied in ADS entries
\bibliography{refs-snr}

\end{document}
