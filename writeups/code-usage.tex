\documentclass[10pt]{article}

\usepackage[margin=1in, letterpaper]{geometry}
\usepackage{parskip}

\usepackage{amsthm, amsmath, amssymb}
% \usepackage{gensymb}  % For use of degree symbol
\usepackage[pdftex]{graphicx}
\usepackage{hyperref}

\usepackage{enumerate} % For use of (a), (b), et cetera
\usepackage{booktabs} % Tables
\usepackage[margin=20pt, labelfont=bf, labelsep=period,
justification=justified]{caption} % Captions in figure floats

% The following metadata will show up in the PDF properties
\hypersetup{
	colorlinks = true,
	urlcolor = blue,
	pdfauthor = {Aaron Tran},
	pdfkeywords = {x-ray, SNR, notes},
	pdftitle = {Supernova remnants and stuff - \today},
%	pdfsubject = {},
	pdfpagemode = UseNone
}

% Don't indent paragraphs
\setlength\parindent{0em}

% ===============
% Useful commands
% ===============
\newcommand{\mt}{\mathrm}
\newcommand{\unit}[1]{\; \mt{#1}} % vemod.net/typesetting-units-in-latex
\newcommand{\abt}{\mathord{\sim}} % tex.stackexchange.com/q/55701
\newcommand{\del}{\nabla}
\newcommand{\ptl}{\partial} % Laziness

\newcommand{\Chandra}{\textit{Chandra}}

% ===============
% Document proper
% ===============
\begin{document}

\begin{center}
    \Large{Guide to Sean's model codes}

    \normalsize{Aaron Tran}\\
    \today \\
\end{center}

% ============
% Introduction
% ============
\section{Introduction}

Here I give an informal explanation of Sean's model code: the code inputs and
outputs, and fitting code outputs to FWHM measurements. I do not explain the
physics or the internal calculations of the code; please see the accompanying
code review for that.

Table, figure, and equation references are from the version of record at
\href{http://dx.doi.org/10.1088/0004-637X/790/2/85}
{doi:10.1088/0004-637X/790/2/85}.

% ============================
% Synchrotron rim width models
% ============================
\section{Filament width model codes}

Two codes are used to model filament widths, which I refer to as \emph{simple}
and \emph{complex}; they are defined by transport equations (5) and (12)
respectively.  Equations (6--11), (13), (23) describe results from the simple
model; equations (12), (14--21) describe the complex model implementation.

Both models are loss-limited; Sean does not present \emph{fitting} results for
magnetic damping models (which would add more free parameters).

Both codes take inputs $B_0$, $\eta_2$, and $\mu$.
$B_0$ is downstream (amplified) magnetic field strength, $\eta_2$ is diffusion
coefficient constant at $2$ keV, and $\mu$ is the scaling exponent for
energy-dependence of the diffusion coefficient ($D \propto E^\mu$).

Both codes output \textbf{predicted filament FWHM values} as a function of
energy.  We seek values of $B_0$, $\eta_2$, and $\mu$ that give best-fit
filament FWHM predictions to our FWHM measurements.

Rephrased, we have:
\[
    w_{\mt{model}} = w(\nu; B_0, \eta_2, \mu)
\]
and we want to minimize
\[
    \chi^2 = \sum \left( \frac{w_{\mt{obs}}(\nu) - w_{\mt{model}}(\nu)}
                       { \delta w_{\mt{obs}} } \right)^2
\]

First, I explain a little more about how predicted FWHMs are calculated and
fitted to measured FWHMs, for each model.
Then, I explain how we are attempting to extend Sean's model fitting.

% --------------------------------
% Simple (catastrophic dump) model
% --------------------------------
\subsection{Simple model fitting}

The simple model gives an analytic expression for $w_{\mt{model}}(\nu;\ldots)$.
This is equation (6) times a projection factor $\beta = 4.6$.

Fitting is easy, in principle.  We can simply find a best fit curve
$w_{\mt{model}}(\nu)$, with 3 free parameters, to the measured FWHMs.
But, SN 1006 had only 3 FWHM measurements -- and, as I explain next, Sean used
only 2 of the FWHM measurements to fit the complex model.

Therefore, Sean used several fixed values of $\mu$: $0, 1/3, 1/2, 1, 1.5, 2$,
and computed best fit $B_0$ and $\eta_2$ values for the 3 FWHM measurements.
Table 7 summarizes the results of these fits.

I'm not sure why $\mu$ was fixed -- in linear least-squares, 3 free parameters
would exactly hit 3 measurements; for non-linear least-squares, the parameters
are not entirely independent so it'd be a little messier.  I could imagine that
(1) the values are just not well-constrained in such a fit, and/or
(2) Sean wanted to be consistent across both models.

% ---------------------------------
% Complex (continuous E loss) model
% ---------------------------------
\subsection{Complex model fitting}

The complex model numerically computes predicted intensity profiles
for $w_{\mt{model}}(\nu; \ldots)$, by solving equations (20--21)
FWHM values are manually extracted from the predicted profiles.

For SN 1006, the code predicts FWHM values at 0.7, 1, and 2 keV.  At some
point, Sean tried fitting as done for the simple model -- i.e., perform a
non-linear fit in free parameters $B_0$, $\eta_2$ to obtain best-fit
FWHM predictions for 3 FWHM measurements.  But, this was not implemented.

Section 4.2 describes the fitting procedure used.  Sean threw away the FWHM at
0.7 keV and computed the approximate scaling exponent
\[
    m_E(\text{2 keV}) = \frac{ \ln( \mathrm{FWHM}(\text{2 keV}) /
                                    \mathrm{FWHM}(\text{1 keV}) ) }
                             { \ln( \text{2 keV} / \text{1 keV} ) }
\]
This $m_E$ was computed for measured and predicted FWHMs at 1, 2 keV.

Sean varied $B_0$ and $\eta_2$ by hand for fixed values of $\mu$ to find best
fit values.  In practice, he chose $\eta_2$ to best fit the $m_E$ value, then
varied $B_0$ to fit the FWHM (as $m_E$ is relatively insensitive to $B_0$).
Table 8 summarizes the results of these fits.

Here, fitting $\{m_E(\text{2 kev})$, $\mathrm{FWHM}(\text{2 keV})\}$ should be
equivalent to fitting $\{\mathrm{FWHM}(\text{1 keV})$,
$\mathrm{FWHM}(\text{2 keV})\}$ instead.  We are just changing variables, and
it turns out the manual fitting is easier (predicted $m_E$ being insensitive to
fit parameter $B_0$, as noted above).  Note: this FWHM / $m_E$ matter was
clarified by email exchange with Sean on 2014 July 3.

\subsection{Results}

Just looking at Tables 7, 8, we can see that the simple model is pretty robust.
Fits weren't given for all $\mu$ values in Table 8 (Filaments 2, 5 appear to
have small $m_E$ values at 2 keV).

What do the parameters mean, or how do they affect predicted FWHM widths?
I still need to mess with this.  Roughly, per Sean's observations (to be
checked): $B_0$ controls overall FWHM widths, but doesn't affect width-energy
scaling much.  Diffusion parameters $\eta_2$ and $\mu$ are the main controls on
the width-energy scaling, as you might expect.

The effect of $\mu$ on width-energy scaling is shown in Figure 1.
The effects of both $\mu$ and $\eta_2 \sim D/D_{\mt{bohm}}$ on width-energy
scaling are shown in Figure 3.  Both decreasing $\mu$ and increasing $\eta_2$
will weaken the width-energy scaling, as the filaments are all smeared out by
diffusion, and smaller $\mu$ means diffusion is less sensitive to energy.

% ==========
% Next steps
% ==========
\section{Next steps}

The simple model is implemented for Tycho, though I have not messed with the
fits / implemented variable shock speed / etc. yet.  The fits give $B_0$
of order $100$ to $1000 \unit{\mu G}$, and $\eta_2$ of order $0.1$ to $10$ --
though the fits are not great.

As for the complex model: I reviewed Sean's code and hit/asked about all issues
I could find.
I verified the calculations against equations in Sean's paper, excepting one
calculation not mentioned in the paper which needs to be rederived.
I have not re-derived the equations in Sean's paper, or checked them against
the source papers (namely Lerche \& Schlickeiser, 1980).
The code appears to work and is giving reasonable output values.

I am porting the complex model code to Python at this time, and will attempt to
implement the non-linear fitting as done for the simple model code (and as Sean
previously attempted).

\end{document}
